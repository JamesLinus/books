% vim: ft=tex ff=unix ts=4 sw=4 et wm=8 tw=0

\chapter{* Exceptions}

There are two ways of dealing with error conditions that are detected at
run-time.  The first is for the operation which detected the error to
return an error code of some sort.  This is what happens in the @io@
library with predicates that return an @io__result@.  Return codes
generally have to be dealt with as soon as the operation returns; in
some situations this can mean that the underlying algorithm ends up
hidden beneath a slew of error handling code.

The other option is to \emph{throw an exception} (sometimes called
\emph{raising} an exception).  An exception causes execution to
immediately return to the nearest \emph{exception handler} in the call
stack.  The exception handler must decide what to do from that point on.
Exceptions are useful when the best place to handle an error (or rather,
an exceptional situation) is at some higher level of abstraction -- that
is, further up the call stack -- rather than having to worry about
unlikely events at every point in the program.

Exception based code in Mercury makes use of the modules @exception@ and
@std_util@, the latter because exceptions are ``transmitted'' as values
of type @univ@.

\section{An Example}

\XXX{I use an IO based example.  Perhaps it would be better to use a
non-IO example and afterwards mention @try\_io@ and relatives.}

In this example we want to ``translate'' one file into another.  The
obvious code using error code handling is unfortunately ugly:
\begin{verbatim}
:- import_module io, string, list.

main(!IO) :-
    io__open_input("input_file", OpenInputResult, !IO),
    (
        OpenInputResult = error(_),
        report_error("Couldn't open input_file", !IO)
    ;
        OpenInputResult = ok(Input),
        io__open_output("output_file", OpenOutputResult, !IO),
        (
            OpenOutputResult = error(_),
            io__close_input(Input, !IO),
            report_error("Couldn't open output_file", !IO)
        ;
            OpenOutputResult = ok(Output),
            translate(Input, Output, TranslateResult, !IO),
            io__close_output(Output, !IO),
            io__close_input(Input, !IO),
            (
                TranslateResult = error(_),
                report_error("Something went wrong in translation", !IO)
            ;
                TranslateResult = ok
            )
        )
    ).

:- pred report_error(string, io, io).
:- mode report_error(in,     di, uo) is det.

report_error(ErrMsg, !IO) :-
    io__stderr_stream(StdErr, !IO),
    io__format(, StdErr, "%s\n", [s(ErrMsg)], !IO),
    io__set_exit_status(1, !IO).

:- pred translate(input_stream, output_stream, io__res, io, io).
:- mode translate(in,           in,            out,     di, uo) is det.

translate(Input, Output, Result, !IO) :- ...
\end{verbatim}
We can write more concise code that handles all the possible errors in
one place by using exceptions:
\begin{verbatim}
:- import_module io, string, list, unit, exception, univ.

:- pred main(io, io).
:- mode main(di, uo) is cc_multi.

main(!IO) :-
    try_io(open_files_and_translate, Result, !IO),
    (
        Result = succeeded(_)
    ;
        Result = exception(Exception),

            % Here we use mode univ(out) = in is semidet.
            %
        ( if   Exception = univ(ErrMsg)
          then report_error(ErrMsg, !IO)
          else rethrow(Result)          % These are not the droids
                                        % we're looking for.
        )
    ).

:- pred open_files_and_translate(unit, io, io).
:- mode open_files_and_translate(out,  di, uo) is det.

open_files_and_translate(unit, !IO) :-
    open_in("input_file", Input, !IO),
    open_out("output_file", Input, Output, !IO),
    translate(Input, Output, !IO),
    io__close_output(Output),
    io__close_input(Input).

:- pred open_in(string, input_stream, io, io).
:- mode open_in(in,     in,           di, uo) is det.

open_in(File, Input, !IO) :-
    io__open_input(File, Result, !IO),
    (
        Result = ok(Input)
    ;
        Result = error(_),
        throw("Couldn't open " ++ File)
    ).

:- pred open_out(string, input_stream, output_stream, io, io).
:- mode open_out(in,     in,           out,           di, uo) is det.

open_out(File, Output, !IO) :-
    io__open_output(File, Result, !IO),
    (
        Result = ok(Output)
    ;
        Result = error(_),
        io__close_input(Input),
        throw("Couldn't open " ++ File)
    ).
\end{verbatim}
We can see that the code in the version that uses exceptions has
better structure, although the exception handler (the @try_io@ code) is
forced to use some advanced mechanisms, most notably @univ@ values.  The
code here \emph{is} slightly longer, but that is mainly an artefact of
using a toy example.  In a real application we would expect the
situation to be reversed (otherwise there would be little point in using
exceptions.)

It is important to observe that every exception-throwing site is
responsible for performing clean-up operations that are not dealt with
by the exception handler.  In this case we have to pass the @Input@
stream handle to the @open_out@ predicate so that we can close it if we
fail to successfully open the @Output@ stream (and, similarly, the
@translate@ predicate should close both streams before throwing an
exception.)



\section{Throwing Exceptions}

An exception @X@ is thrown by calling @throw(X)@ from the @exception@
module (@throw/1@ exists in both predicate and function versions and, in
the normal course of things, the compiler will be able to work out from
context which is meant.)

@throw/1@ has determinism @erroneous@, indicating to the compiler that
this predicate does not terminate normally (@erroneous@ can also be used
for predicates that do not return at all), hence it is not necessary to
tie-up all loose ends.  For instance, in
\begin{verbatim}
    p(A, X0, X1),
    (
        A = ok,
        q(X1, X2)
    ;
        A = ouch,
        throw(...)
    ),
    r(X2, X)
\end{verbatim}
we do not have to include the unification @X2 = X1@ in the second
disjunct of the switch, as would be required if @throw/1@ did not have
determinism @erroneous@ since @X2@ is required by the call to @r/2@.

Any value at all can be thrown as an exception; it is up to the
enclosing exception handler to decide if and how to handle matters.

If, in an exception handler, one has to pass the exception back up to an
exception handler even further up the call stack then one should use
@rethrow@.  The reason for this will be explained in the next section
\XXX{}.

\section{Handling Exceptions}

Exception handling works via a call to @try@ or one of its variants:
\begin{verbatim}
    % try(Goal, Result)
    %
:- pred try(pred(T),                exception_result(T)).
:- mode try(pred(out) is det,       out(cannot_fail)) is cc_multi.
:- mode try(pred(out) is semidet,   out             ) is cc_multi.
:- mode try(pred(out) is cc_multi,  out(cannot_fail)) is cc_multi.
:- mode try(pred(out) is cc_nondet, out             ) is cc_multi.
\end{verbatim}
where
\begin{verbatim}
:- type exception_result(T)
    --->    succeeded(T)
    ;       failed
    ;       exception(univ).

:- inst cannot_fail
    --->    succeeded(ground)
    ;       exception(ground).
\end{verbatim}
and @univ@ is the universal type defined in the @std_util@ standard
library module.

What happens operationally with a call @try(Goal, Result)@ is that the
closure @Goal@ is executed and @Result@ instantiated according to what
happened:
if @Goal@ succeeded, returning @X@, then @Result = succeeded(X)@;
if @Goal@ failed then @Result = failed@;
otherwise @Goal@ must have thrown an exception @E@, in which case
@Result = exception(U)@ where @U = univ(E)@ (i.e. the encapsulation of
the abitrary type of @E@ in the universal type @univ@.)

The code that follows the call to @try@ must decide what to do depending
upon @Result@.

If @Result = succeeded(X)@ or @Result = failed@ then we can proceed as
if we had simply called @Goal@.  Note that if @Goal@ has determinism
@det@ or @cc_multi@ then it cannot fail, in which case the argument mode
of @Result@ is @out(cannot_fail)@ which, in turn, means that we do not
need to test for @Result = failed@.

If @Result = exception(U)@ then we need to handle the exception.
Sometimes this can be as simple as performing some clean-up operations
and calling @rethrow(Result)@.  Usually, however, we are interested in
exactly what exception was thrown, which requires a checked, dynamic
(i.e.  run-time) \emph{cast} of @U@ from type @univ@ to the expected
type for @E@.

The @univ@ function in @std_util@ has the following signature:
\begin{verbatim}
:- func univ(T  ) = univ.
:- mode univ(in ) = out is det.
:- mode univ(out) = in is semidet.
\end{verbatim}
Going in the ``forwards'' direction, any value of any type can be
converted in a value of type @univ@, as one might expect.  In the
``reverse'' direction, however, a value of type @univ@ may or may not be
convertible into a value of some given type @T@.

So the code for handling an exception looks like this:
\begin{verbatim}
    try(Goal, Result),
    (
        Result = succeeded(X), ...
    ;
        Result = failed, ...
    ;
        Result = exception(U),
        ( if U = univ(E) then
                % E is the value of the exception in whatever
                % type which deal_with_exception/1 below expects.
                %
            deal_with_exception(E),
            ...
          else
                % Otherwise the exception is of a different type
                % to that which we were prepared to handle, so the
                % exception cannot be intended for us to handle.
                % Therefore we pass it on the the next exception
                % handler up the call stack.
                %
            rethrow(Result)
        )
    )
\end{verbatim}
Note that our if-then-else goal could have several arms if we are
prepared to deal with exceptions of several different types, but we must
still be prepared to @rethrow@ a result which we cannot handle -- simply
dropping such things on the floor means that higher-level exception
handlers cannot do their job.  This is particularly important for
higher order code that deals with exceptions.

\section{Effect On Determinism}

To throw an exception is to effect an abnormal exit from a procedure.
Thanks to the halting problem \XXX{} it is impossible for a compiler to
decide whether an arbitrary piece of code will throw an exception or
even what exceptions it may throw.  The only way to find out in general
is to run the code and see what happens.  As a consequence, the
determinism of @try@ is @cc_multi@: it always returns a result, but we
have no way of knowing, analytically, whether the result will be an
exception or a ``normal'' return.  Either way, we only get the one
result and cannot backtrack into the computation.

\section{Er, Something}

\XXX{There was something I was going to say here\ldots}

\section{IO States and Stores}

The @io@ state and @store@ types play a special role in Mercury in as
much as they are the main exploiters of uniqueness.  \XXX{What a
horrible way to put it.}

The usual exception handling method is not sufficient for handling
computations that manipulate @io@ states and @store@s.  The problem is
that we do not want an exception to cause us to lose the @io@ state or
@store@ -- the former because then our program could no longer perform
any IO at all (and hence not report the results of any computation) and
the latter because we do not want to lose access to the information
in the @store@.

One might imagine that these problems could be remedied by including the
@io@ state or @store@ as part of the exception result -- for example
\begin{verbatim}
:- type excn_with_io ---> excn_with_io(string, io).

p(!IO) :-
    do_something(X, !IO),
    if   something_went_wrong(X)
    then throw(excn_with_io("Aieee!", !.IO))
    else carry_on_as_usual(X, !IO).
\end{verbatim}
The first difficulty here is that the argument to @throw@ has mode @in@,
which means the @io@ state contained in the @excn_with_io@ will not be
seen as @unique@ by the exception handler.

The second objection is that the predicate that throws the exception may
not be in posession of the @io@ state or @store@.  Higher order code is
most likely to have this problem:
\begin{verbatim}
p(P, !IO) :-
    P(X),       % P may throw an exception, but
                % does not have the IO state.
    ...
\end{verbatim}
while we could conceivably place our own exception handler around the
call to @P@, catch any exception thrown by P and wrap that up in
another exception that included the @io@ state, we then have the
problem that any exception handler further up the call stack looking
for exceptions thrown by @P@ will not recognise them because of our
wrapper.  One could probably find a solution to the conundrum, but
it seems unlikely that it would be elegant.

The third problem is that the current version of the Mercury compiler
does not yet handle nested unique objects -- that is, even if we could
place an @io@ state or @store@ inside an exception value \emph{and}
arrange for a mode of @try@ that preserved its @inst@ \emph{and}
came up with a means of transmitting the requisite @inst@
information to the exception handler, we would still have to wait for
a version of the compiler to come along that handled nested @unique@
objects.  \XXX{To our shame!}

Unfortunately, no general solution to all these problems presents
itself -- there remains research to be done.  It should be recognised
that these problems are not specific to Mercury; they exist in all
exception systems that have code that performs destructive update.
The problems are merely highlighted by Mercury's strict type and mode
systems, which the designers do not consider worth compromising for
the sake of a friendlier exception system.  \XXX{Could probably put
that more diplomatically.}

The pragmatic solution currently adopted is to supply two extra
versions of @try@:
\begin{verbatim}
:- pred try_io(pred(T,   io, io),             exception_result(T),
            io, io).
:- mode try_io(pred(out, di, uo) is det,      out(cannot_fail),
            di, uo) is cc_multi.
:- mode try_io(pred(out, di, uo) is cc_multi, out(cannot_fail),
            di, uo) is cc_multi.

:- pred try_store(pred(T,   store, store),       exception_result(T),
            store, store).
:- mode try_store(pred(out, di, uo) is det,      out(cannot_fail),
            di,    uo) is cc_multi.
:- mode try_store(pred(out, di, uo) is cc_multi, out(cannot_fail),
            di,    uo) is cc_multi.
\end{verbatim}
These work very much like @try@, except that in this case the goal
argument takes two extra arguments for the @io@ state or @store@ as
appropriate and that, of course, the @io@ state or @store@ is
preserved across the exception.

So, say we have a @io@ based predicate @parse_input@ that throws
an exception on detecting an error rather than returning an error
code, we could handle it like this:
\begin{verbatim}
:- pred parse_input(parse_result, io, io).
:- mode parse_input(out,          di, uo) is det.
...
    try_io(parse_input, Result, !IO),
    (
        Result = succeeded(Parse),
        io__format("Parsing succeeded\n", [], !IO),
        ...
    ;
        Result = exception(U),
        io__format("Parsing exception\n", [], !IO),
        ( if   U = univ(E)
          then handle_parser_exception(E)
          else rethrow(Result)
        )
    )
...
\end{verbatim}
As pointed out earlier, were it not for @try_io@ returning the @io@
state after catching the exception, we could not make the call to
@io__format@ in the exception handler.

It is important to be aware that it is the programmer's
responsibility to ensure that @store@s and various aspects of the
@io@ state are consistent with the program's logic after catching
an exception.  That is, they will be consistent as far as Mercury
is concerned, but various program-specific invariants concerning
them may or may not be preserved.  This is a difficult problem
and is what makes exception handling less attractive than it would
at first seem.

\section{All Solutions Predicates}

Occasionally one may want to enumerate all the solutions of a
nondeterministic predicate up to the point that all solutions
have been generated or the predicate in question throws an
exception.  This is what @try_all@ is for:
\begin{verbatim}
:- pred try_all(pred(T),              list(exception_result(T))).
:- mode try_all(pred(out) is det,     out(try_all_det))     is cc_multi.
:- mode try_all(pred(out) is semidet, out(try_all_semidet)) is cc_multi.
:- mode try_all(pred(out) is multi,   out(try_all_multi))   is cc_multi.
:- mode try_all(pred(out) is nondet,  out(try_all_nondet))  is cc_multi.
\end{verbatim}
the output argument is a list of the @exception_result@s and the last
element in the list will contain the exception, if any was thrown.
Otherwise, this predicate is very similar to the @builtin@
predicate @unsorted_solutions@.

Just as @unsorted_solutions@ has a companion, @unsorted_aggregate@,
that supports interleaving processing of the results of the
nondeterministic predicate, @try_all@ has a companion,
@incremental_try_all@:
\begin{verbatim}
:- pred incremental_try_all(
            pred(T1),
            pred(exception_result(T1), T2, T2),
            T2, T2).
:- mode incremental_try_all(
            pred(out) is nondet,
            pred(in, di, uo) is det,
            di, uo) is cc_multi.
:- mode incremental_try_all(
            pred(out) is nondet,
            pred(in, in, out) is det,
            in, out) is cc_multi.
\end{verbatim}

\XXX{Do I need more examples in this section?}

\section{The Down Side}

From the discussion above, it should be clear that exceptions are not
silver bullets.  The cases where they \emph{may} simplify your code are
those where there is some distance between the point where an exceptional
case may be detected and the point where it is best handled.  Even then,
one must be careful to ensure that the programs ``state'' (in the sense
of unique values such as @store@s and the @io@ state) are consistent with
the invariants of the program before leaving the exception handler.

In many cases, it turns out to be simpler to return error codes and
handle problems as locally as possible.

As always, it takes experience and a little experimentation to work out
which is the best solution in each particular situation.

