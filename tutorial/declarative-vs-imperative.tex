\section{Declarative vs Imperative Programming}

\subsection{Definitions}

\begin{description}
\item{Imperative:} a sequence of instructions for transforming state.
\item{Declarative:} a specification of \emph{what} is to be computed.
\end{description}

Declarative languages typically have a simple translation into
conventional mathematical logic, which is how the meaning of a program
is defined.

Purely declarative programming languages exhibit referential
transparency.  In a nutshell, this means that anywhere you see a
reference to a name, you can replace it with the body of the
definition of the name and it will make no difference.

In more formal language,
\[
(\text{let}\ x = e\ \text{in}\ M)  \equiv  M[e/x]
\]
This is clearly not true of imperative languages.  Consider
the following C program:

\begin{verbatim}
int g = 0;

int f(int x)
{
    g = g + 1;
    return x + g;
}

void main(int argc, char **argv)
{
    int tmp = f(1);
    int a   = tmp  + tmp;
    int b   = f(1) + f(1);

    if(a == b)
        printf("equivalent\n");
    else
        printf("not equivalent\n");
}
\end{verbatim}

According to C semantics, this program should print out "not
equivalent", proving that the expression @f(1)@ is not equal to
itself!  This sort of thing is great for writing buggy, hard to
maintain code.\footnote{One can give a reasonably simple
operational semantics to C whereby we repeatedly substitute the
body of @f(1)@ into a sequence of \emph{instructions}, but as we
see here, this does not necessarily support the more intuitive,
declarative reading one might hope for.}

Since referential transparency means no side-effects, you can't have
variables that change their state as the program evolves.  Instead, the
term \emph{variable} in a declarative programming language refers to a
label given to a value or the result of a computation.

The nearest Mercury equivalent to the function @f()@ in the C
program above is

\begin{verbatim}
:- pred f(int, int, int, int).
:- mode f(in,  out, in,  out) is det.

f(X, Result, Old_Value_of_G, New_Value_of_G) :-
    New_Value_of_G = Old_Value_of_G + 1,
    Result         = New_Value_of_G + X.
\end{verbatim}

Since there are no variables in Mercury (and hence no global
variables), any ``global'' state has to be explicitly passed
around wherever it is needed.  In this case @f/4@ takes the old
``state'' as its third argument and returns the new ``state'' in
its fourth argument.

\subsection{Benefits Of Declarative Style}

While the lack of mutable state might seem like a serious
drawback to programmers raised on imperative languages, in
practice it turns out to be something of a boon.  Experienced
imperative programmers acknowledge that fewer globals and less
state means clearer, more maintainable, more reusable code
with fewer bugs.

The philosophy underlying declarative languages is to simplify the
\emph{writing} of bug free programs and to leave the tedious
business of identifying programming errors, run-time book
keeping such as memory management, and non-algorithmic
optimization to the compiler.  Referential transparency means
that more optimizations can be applied in more places in a
program than is the case with imperative programs, simply
because proving that it is safe to apply a particular
optimization is so much easier in the absence of side effects.

Mercury was designed as a purely declarative, industrial
strength programming language aimed at the rapid development
of medium and large scale systems with the emphasis on
producing fast, correct programs.

To this end, Mercury doesn't support a corner-cutting
programming style: you \emph{have} to get the types right, check
return codes and so forth -- the quick-and-dirty fix is rarely
an option.\footnote{Mercury does have support for impure code
via calls to another language, but one has to label each such
call explicitly; in the end it is often easier just to do the
right thing.}  Very often, Mercury programmers find that once
a program is accepted by the compiler, it also does exactly
what was intended; in the author's long experience this almost
never happens with imperative languages, even for small
programs.

\XXX{Place to mention types, garbage collection, polymorphism,
pattern matching, etc etc?}

\subsection{Pragmatism}

\begin{quote}
``A foolish consistency is the hobgoblin of small minds.'' \\
\hfill --- Ralph Waldo Emmerson
\end{quote}

Purity is all very well, but the fact remains that
occasionally one has to interoperate with code written in
impure languages and that (very rarely and usually when that
last drop of speed is required) some low-level algorithms may
be best expressed as impure constructs.

For the former, Mercury has a simple and well developed
foreign language interface allowing the programmer to write
foreign code in-line with the Mercury program (provided the
Mercury compiler has an appropriate back-end for the foreign
language in question -- the alternative is to use C as the
lingua franca in the traditional style.)  It is the
programmer's responsiblility to supply the appropriate purity
declarations for predicates defined in terms of foreign code.

The latter is handled using purity annotations.  Impure code
(written in a foreign language) must be labelled as such.
These labels also have to be applied to all predicates that
use the impure code.  At some point we hope that a pure
interface can be presented to the programmer, in which case
the program must include a \emph{promise} to the compiler that
impurity annotations are not required for users of the
top-level predicate with an impure definition.

\subsection{Mercury Philosophy}

\XXX{I think I've already covered this one.  Tyson suggests an
implementation philosophy section (\eg no distributed fat) in
a much later section.}



