\section{Input and Output}

One unfortunate consequence of being a pure declarative language
is that IO becomes somewhat more complex than is the case for
imperative languages.

\subsection{IO \emph{Is} a Side-Effect}

One problem is that performing IO necessarily has an effect on the
outside world that cannot be backtracked over or undone -- there is
no returning to an earlier state of the world!  This is in
contrast to the mathematically pure world that Mercury inhabits,
where there is no concept of a value (such as the state of the
world) ``changing'', only one of new such values being computed.

\subsection{Order is Important}

Another problem is that since logically there is no difference
between the goal @(G1, G2)@ and the goal @(G2, G1)@, we also need to
find some mechanism for ensuring that IO operations happen in the
intended order and are not mixed up as a consequence of the
compiler reordering goals for optimization purposes and so forth.

\subsection{Uniqueness}

A number of solutions to the IO problem have been adopted by
the pure, declarative languages, the main contenders being the
monadic approach (as exemplified by Haskell) and the
uniqueness approach (as exemplified by Clean and Mercury.)

The uniqueness approach works like this: we view a Mercury
program as a function from world states to world states.  The
top-level @main/2@ predicate of a Mercury program takes the
current world state as an input and computes a new world state
as its result.  Every IO operation does the same thing: takes
a world state as input and produces a new world state as
output -- notionally updated with the effects of the IO
operation (and the actions of the world at large between IO
operations).  The so-called IO state is opaque to the Mercury
program; it can only be queried via the operations defined in
the io module.\footnote{Of course, the Mercury program doesn't actually
pass the state of the world around in fact -- the IO state
abstraction serves both to ensure the properties we require
and to give a semantics to IO in Mercury.}

Example (we eschew state variable notation here for clarity of
exposition):
\begin{verbatim}
:- pred main(io, io).
:- mode main(di, uo) is det.

main(IO0, IO) :-
    io__write_string("pi = ", IO0, IO1),
    io__write_float(math__pi, IO1, IO2),
    io__nl(                   IO2, IO ).
\end{verbatim}
The type of the IO state is called just io and is defined
as an abstract type in the @io@ library module.  The
top-level predicate @main/2@ takes the initial IO state as a
@di@ mode argument (\emph{destructive input}), and produces another as a
@uo@ mode result (\emph{unique output}).  The three IO operations in the body
show how the initial IO state, @IO0@, is transformed into
the final IO state, @IO@.  So, @io__write_string/3@ destroys
@IO0@ and produces @IO1@ as a unique output.  Next,
@io__write_float/3@ destroys @IO1@ and produces @IO2@ as a
unique output.  Finally, @io__nl@ (which writes out a
newline) destroys @IO2@ and produces @IO@ as a unique output.

In order to ensure that old versions of the world state cannot
be accessed after an IO operation, the IO state is \emph{unique} --
this means that there can only ever be one live reference to
it.\footnote{A live reference is one that will be used
later on in computation.}  The old version of the IO state
is said to be clobbered by an IO operation -- the compiler will
report an error if the old version is still live after the IO
operation in question.  Similarly, it is impossible to make a
copy the IO state.\footnote{It is possible to ``fork'' the IO
state, this is necessary to support concurrency.  Concurrency
is dealt with in a later chapter. \XXX{}}

Example:
\begin{verbatim}
:- pred main(io, io).
:- mode main(di, uo) is det.

main(IO0, IO) :-
    io__write_string("Hello, ",  IO0, IO1),
    io__write_string("world!\n", IO0, IO ).
\end{verbatim}
Here @IO0@ is used twice; the compiler spots the bug and
rejects the program with
\begin{verbatim}
In clause for `main(di, uo)':
  in argument 2 of call to predicate `io:write_string/3':
  unique-mode error: the called procedure would clobber
  its argument, but variable `IO0' is still live.
\end{verbatim}
The uniqueness constraint is sufficient to ensure that IO
operations happen in a strict order, specified by the
programmer, and that it is impossible to backtrack over IO or
refer to a dead IO state.

Note that uniqueness is not a property reserved for IO states:
it is used to implement destructively updated arrays, the
store data type which allows the construction of arbitrary
pointer graphs, hash tables and so forth.  Uniqueness allows
the compiler to use a safe form of destructive update of data
structures: there is no reason why a dead unique object cannot
be reused to create a new live unique object (since the old
value can never be accessed), which is exactly what happens
for the data types just mentioned.

\subsection{Stylistic Considerations}

Since passing the IO state around everywhere is a little
cumbersome and also quite restrictive (it can only be passed
into det predicates), Mercury programmers naturally find
themselves separating applications into two parts: the part
that handles all the IO and the part that handles all the
interesting processing.  This is good style in general, and
although one might find it slightly annoying not to be able to
insert print statements willy-nilly as is the case with impure
languages, one soon finds that the discipline imposed pays
real dividends in terms of reusability, maintainability, ease
of debugging and so forth.

\subsection{* Determinism Restrictions}

Since IO operations cannot be backtracked across, the IO state
(and, indeed, unique objects in general) cannot be passed to
non-deterministic predicates -- that is, only deterministic
predicates can take unique IO states as arguments.\footnote{This is not strictly true; there is another
determinism, cc-multi, which is compatible with uniqueness.}

\subsection{* State Variable Syntax}

Having to name and pass two variables around for every IO
operation quickly becomes tiresome.  Mercury has a special
\emph{state variable} syntax for just this purpose.  The idea is to
write code that looks a little more like what one would write
in an imperative language, but which is transformed by the
compiler into pure Mercury.  A state variable argument @!X@
stands for two real arguments, @!.X@ and @!:X@, which in turn
stand for the ``current'' and ``next'' values of the state variable
@X@, respectively.  Occurrences of @!.X@ and @!:X@ are converted by
the compiler into appropriately numbered variables.

For example, the following code
\begin{verbatim}
    % Writes out a list of strings, separated by
    % commas.
    %
:- pred write_strings(list(string), io, io).
:- mode write_strings(in,           di, uo) is det.

write_strings([],            !IO).

write_strings([S1],          !IO) :-
    io__write_string(S1,     !IO).

write_strings([S1, S2 | Ss], !IO) :-
    io__write_string(S1,     !IO),
    io__write_string(", ",   !IO),
    write_strings(Ss,        !IO).
\end{verbatim}
is transformed by the compiler into\footnote{Note that the pred and mode declarations reflect
the fact that @!IO@ is actually two arguments, not one.}
\begin{verbatim}
    % Writes out a list of strings, separated by
    % commas.
    %
:- pred write_strings(list(string), io, io).
:- mode write_strings(in,           di, uo) is det.

write_strings([],            IO0, IO) :-
    IO = IO0.

write_strings([S1],          IO0, IO) :-
    io__write_string(S1,     IO0, IO).

write_strings([S1, S2 | Ss], IO0, IO) :-
    io__write_string(S1,     IO0, IO1),
    io__write_string(", ",   IO1, IO2),
    write_strings(Ss,        IO2, IO ).
\end{verbatim}
Henceforth we will use state variable syntax rather than
explicitly numbered IO states.

\subsection{Common IO Operations}

The @io@ library module defines a plethora of useful IO
operations and as usual with libraries, the reader is
encouraged to take some time to peruse the interface section.
Here we will present some basic IO operations to help get the
ball rolling.

\subsubsection{Output}

Output is generally simpler to deal with than input,
because, by and large, there are no error codes to deal
with.

The @io@ library module includes predicates for the output
of the basic types:
\begin{verbatim}
:- pred io__write_string(string, io, io).
:- mode io__write_string(in,     di, uo) is det.

:- pred io__write_char(char, io, io).
:- mode io__write_char(in,   di, uo) is det.

:- pred io__write_int(int, io, io).
:- mode io__write_int(in,  di, uo) is det.

:- pred io__write_float(float, io, io).
:- mode io__write_float(in,    di, uo) is det.
\end{verbatim}
However, it's typically easier to use the more general
formatted output predicate:
\begin{verbatim}
:- pred io__format(string, list(poly_type), io, io).
:- mode io__format(in,     in,              di, uo) is det.
\end{verbatim}
The @string@ argument describes how the output is to be formatted, very
similar in spirit and detail to what one would pass to C's @printf()@.
The @list@ argument is a type safe means of passing the parameters to be
formatted.

Using @io__format/4@ one might write
\begin{verbatim}
:- pred write_record(string, int, float, io, io).
:- mode write_record(in,     in,  in,    di, uo) is det.

write_record(Name, Age, Children, !IO) :-
    io__format("%s is %d years old and has %f children.\n",
               [s(Name), i(Age), f(Children)], !IO).
\end{verbatim}
and then we could call
\begin{verbatim}
    write_record("Joe Bloggs", 43, 2.4, !IO)
\end{verbatim}
and the program would write out
\begin{verbatim}
Joe Bloggs is 43 years old and has 2.4 children.
\end{verbatim}
The implementation of @write_record/5@ is much simpler than the
functionally equivalent
\begin{verbatim}
:- pred write_record(string, int, float, io, io).
:- mode write_record(in,     in,  in,    di, uo) is det.

write_record(Name, Age, Children,           !IO) :-
    io__write_string(Name,                  !IO),
    io__write_string(" is ",                !IO),
    io__write_int(Age,                      !IO),
    io__write_string(" years old and has ", !IO),
    io__write_float(Children,               !IO),
    io__write_string(" children.\n",        !IO).
\end{verbatim}

In fact, @io__format/4@ is quite a bit more powerful, in
that the @%@ formatting specifications can include
details as to the style of formatting, precision,
justification and so forth.

\XXX{I'm not sure I should mention @io\_\_print/3@ this early.}
Another useful predicate the @io@ library module provides is
\begin{verbatim}
:- pred io__print(T,  io, io).
:- mode io__print(in, di, uo) is det.
\end{verbatim}
@io__print/3@ is used to print a representation of arbitrary
Mercury values.  Be aware, though, that if you try to
print the results of expressions, the compiler may ask you
to supply more type information to help resolve what
exactly it is you are printing (\ie the expression or the
value of the expression.)  This is subtle stuff and will
be dealt with in a later chapter. \XXX{}

\subsubsection{Input}

Input is marginally more complex than output since
typically on of three things can happen:
\begin{enumerate}
\item we successfully read a value of the required type from
  the input stream;
\item we hit the end-of-file;
\item an error of some kind occurs (\eg the input is
  malformed or the input source has gone away unexpectedly.)
\end{enumerate}

To this end the @io@ library module defines the following
type to report the results of input operations:
\begin{verbatim}
:- type io__result(T) ---> ok(T)
                      ;    eof
                      ;    error(io__error).
\end{verbatim}
In order, @ok(X)@ is returned if the input operation
succeeded, reading @X@; @eof@ is returned if the end of file
has been reached; and @error(ErrorCode)@ is returned if
something went wrong (the function @io__error_message/1@ can
be used to turn @ErrorCode@ into a printable error message.)

This arrangement forces the programmer to handle the error
cases.  There is still lively debate over whether error
codes or throwing exceptions is the best way to handle
errors for things like this.  A genuine advantage of the
error code approach is that you have to consider the error
cases from the outset, which, while requiring a little
more initial thought from the programmer, usually pays
real dividends later on.
\XXX{is this the right place to say this?  Should I
enlarge on the debate?  Probably no and yes
respectively...}

Two very basic input predicates are
\begin{verbatim}
:- pred io__read_char(io__result(char), io, io).
:- mode io__read_char(in,               di, uo) is det.

:- pred io__read_line_as_string(io__result(string), io, io).
:- mode io__read_line_as_string(in,                 di, uo)
            is det.
\end{verbatim}

The input predicates are also less comprehensive in the
sense that there are no predicates @io__read_int/3@,
@io__read_float/3@ or @io__read_string/3@.  The problem is
that it's not clear exactly what should be allowed to
terminate the input stream for an @int@, @float@ or @string@.
Instead, the library leaves issues of parsing up to
applications (there are several programs in the Mercury
extras distribution to help, including a lexer and parser
generators.)

Here's a simple program echoes the capitalised version of
letters in the input stream:
\begin{verbatim}
:- module capitalise.
:- interface.
:- import_module io.

:- pred main(io, io).
:- mode main(di, uo) is det.

:- implementation.
:- import_module char, exception.

main(!IO) :-
    io__read_char(R, !IO),
    (
        R = ok(C),
        io__write_char(char__to_upper(C), !IO),
        main(!IO)
    ;
        R = eof
    ;
        R = error(E),
        exception__throw(E)
    ).
\end{verbatim}



