% vim: ft=tex ff=unix ts=4 sw=4 et wm=8 tw=0

\chapter{The Mode and Determinism Systems}

One of the things that distinguishes Mercury from other languages
is its strong mode system.  In casual use, the term \emph{mode}
encompasses both direction of data flow and determinism.

\section{Direction of Data Flow}

In logic, there is no difference between the goal @(A, B)@ and
the goal @(B, A)@.  However, the compiler must convert the
logical specification that is a Mercury program into an
imperative language that can be executed on a computer.  This
constraint means that it matters very much for efficiency and,
in some cases, correctness, if @A@ ``produces'' values that
are ``consumed'' by @B@, or vice versa.

The mode system is what allows Mercury to compile logic
programs so efficiently.  The body of each procedure of a
predicate is reordered so that the producer of a value is
guaranteed to be executed before any consumers.

This scheme also allows the compiler to identify programming
errors such as using an unbound variable as an argument to a
procedure expecting an input value.

Before we can discuss modes properly, however, we must first
examine the notion of \emph{insts}.

\subsection{Insts}

An \emph{inst} describes the instantiation state of a variable
at any point in a Mercury program.  Broadly speaking, a
variable's inst will be one of the following:
\begin{description}
\item[@free@] where the variable has not been bound to anything yet;
\item[@bound@] to a value (the top-level
  data constructor of the value is fixed, but each argument has its own
  inst);
  \XXX{should the tutorial mention partially instantiated
  values at all?}
\item[@unique@] where the variable is the only live (\ie non-@dead@)
  reference to its binding;
\item[@dead@] where the value bound to the variable will never be
  examined again, even on backtracking.
\end{description}

A variable's inst tells us which possible states it can be
in at some point in the program.

The need for higher order insts (which we will come to)
and the potential for manipulating partially instantiated
values means that insts cannot be restricted to the set
@{free, ground, unique, dead}@.  However, it is sufficient
to restrict ourselves to the @{free, ground}@ subset of
possible insts in order to understand the basic ideas.

\XXX{Need to include the syntax etc. for inst
declarations.  Also mention parametric insts.}

On top of the built-in insts (@free@, @ground@, @unique@, @dead@
and the higher order insts), it is possible to define your
own insts.  \XXX{Finish this off...}

\subsection{Modes}

A \emph{mode} describes how a variable's inst changes over the
course of a call.  In the following, I is an inst
variable and @(BeforeInst >> AfterInst)@ is a mode, stating
that before the call the argument must have inst
@BeforeInst@ and afterwards it will have inst @AfterInst@.  As
with types, we can give names to modes and even give them
parameters.

@AfterInst@ must be a refinement of @BeforeInst@ in the sense
that instantiated (\ie non-@free@) parts of the value in
question must remain the same, but @free@ parts may become
bound.  \XXX{This isn't strictly true -- the @laziness@
module in extras uses a force implementation that changes
the constructor of the lazy value.  Of course, this means
lazy values require user-defined equality and so forth,
but...}

Let's take a look at some examples (these are taken from
the built-in Mercury definitions):
\begin{verbatim}
:- mode in(I)  == (I    >> I).
:- mode out(I) == (free >> I).

:- mode in     == in(ground).
:- mode out    == out(ground).

    % The inst parameter for uo/1 and ui/1 should be
    % unique.
    %
:- mode di(I)  == (I    >> dead).
:- mode uo(I)  == (free >> I   ).
:- mode ui(I)  == (I    >> I   ).

:- mode di     == di(unique).
:- mode uo     == uo(unique).
:- mode ui     == ui(unique).
\end{verbatim}

So, a procedure parameter with mode @in(I)@ must have inst @I@
before the call and will also have inst @I@ afterwards.  The
most common case of this is the plain in mode, in which
case the parameter must be ground beforehand and will be
ground afterwards.

A parameter with with mode @out(I)@ must be free before the
call and will have inst @I@ after the call.  The common case
is the plain out mode, in which case the parameter will
be ground after the call.

The three uniqueness modes, @di@, @uo@ and @ui@ (\emph{destructive
input}, \emph{unique output} and \emph{unique input}, respectively)
are similar, except that in the di case the parameter is
dead after the call (hence no further references may be
made to that variable in the program, even on
backtracking), @uo@ produces a @unique@ (@ground@) value and @ui@
preserves the @unique@ (@ground@) inst of the parameter across
the call.

\section{Determinism Categories}

Every Mercury procedure has a determinism.  The set of
possible determinisms and their meanings is given in the
following table:

\begin{tabular}{lll}
\textbf{Determinism} & \multicolumn{2}{l}{\textbf{Number of Solutions}} \\
                     & \textbf{Min} & \textbf{Max} \\
\hline \\
@failure@            & 0            & 0 \\
@semidet@            & 0            & 1 \\
@nondet@             & 0            & 1 or more \\
@det@                & 1            & 1 \\
@multi@              & 1            & 1 or more \\
@cc_nondet@          & 0            & 1 \\
@cc_multi@           & 1            & 1 \\
@erroneous@          & never returns \\
\end{tabular}

Each determinism says something about the number of possible
solutions a procedure can have on backtracking.  This table
includes a number of categories we haven't seen before.

@failure@ applies to procedures such as @false@ that have no
solutions.

@semidet@ applies to procedures such as @(>)/2@ or @map__search/3@
that have at most one solution.

@nondet@ applies to procedures such as @member/2@ that can have
any number of solutions, including zero.

@det@ applies to procedures such as @(+)/2@ that have precisely
one solution.

@multi@ applies to procedures such as @nat/1@ below that may have
several solutions, but which always have at least one:
\begin{verbatim}
    % nat(N) enumerates the natural numbers in N.
    %
:- pred nat(int).
:- mode nat(out) is multi.

nat(0).
nat(N + 1) :- nat(N).
\end{verbatim}
@cc_nondet@ and @cc_multi@ are \emph{committed choice} versions of
@nondet@ and @multi@, respectively.  Committed choice
non-determinism is described in more detail in the next
section, suffice to say that these procedures will only
produce (at most) one solution.

Finally, @erroneous@ applies to predicates that either never
return and/or always terminate by throwing an exception (we
will deal with exceptions in a later chapter \XXX{}).

Determinisms are useful for three main reasons:
\begin{enumerate}
\item determinism declarations act both as useful documentation
  to the reader of a program and as information for the
  compiler;
\item the compiler will check that a procedure does indeed have
  the declared determinism and can give helpful error messages
  when this is not the case; and
\item the compiler needs to know the determinism of
\end{enumerate}

\subsection{Modes and Determinisms}

As one might expect, there is a tight relationship between
modes and determinisms.  Essentially, for each possible
set of argument modes, a procedure should have at most one
possible determinism.

\XXX{My brain is too foggy to say more about this right
now.}

\XXX{Clean this up and check it (esp. the cc and e entries):
There is a partial order on determinisms (copy it from the
reference manual?).}

\begin{verbatim}
Disjunction:
    min-solns(@P ; Q@) = max(1, min-solns(@P@) (+) min-solns(@Q@))
    max-solns(@P ; Q@) =        max-solns(@P@) (+) max-solns(@Q@)

where

     (+) |   0   |   1   |  > 1
     ----+----------------------
      0  |   0   |   1   |  > 1
      1  |   1   |  > 1  |  > 1
     > 1 |  > 1  |  > 1  |  > 1
\end{verbatim}

\begin{verbatim}
Conjunction:
    min-solutions(P, Q) = max(1, min-solns(P) (*) min-solns(Q))
    max-solutions(P, Q) =        max-solns(P) (*) max-solns(Q)

where

     (*) |   0   |   1   |  > 1
     ----+----------------------
      0  |   0   |   0   |   0
      1  |   0   |   1   |  > 1
     > 1 |   0   |  > 1  |  > 1
\end{verbatim}

[XXX Perhaps an example or two would be better?]

\section{Committed Choice Non-Determinism}

Committed choice non-determinism is useful when you only
require a single solution from a predicate that may have many.
For instance, a theorem prover only needs a single proof
(among all the possible proofs it could generate) for each
sub-goal. \XXX{Too abstract an example?}

It is also useful for describing the semantics of exceptions
(we will discuss this in more detail later \XXX{}, but briefly
since it is impossible in general to decide whether a
procedure will always throw an exception, an exception handler
is deemed to be a committed choice goal since the outcome --
normal return or exception -- is unknown and it can only happen
once.)

\XXX{Mention relationship to Hilbert's choice operator and how cc
declarations add axioms to the program concerning the single
solution.}

It is an error if a program attempts to backtrack over a
committed choice goal and the compiler will detect and reject
such programs.

\XXX{Examples}
\XXX{Probably more to say.}

\section{* Using Insts for Subtyping}

Insts can be used to define limited subtypes, where a subtype
takes on a subset of the possible values of the supertype.

Examples:
\begin{verbatim}

:- inst sign  ---> -1 ; 0 ; 1.

:- func sign(int) = sign.
:- mode sign(in)  = out(sign) is det.

sign(X) = ( if      X < 0 then -1
            else if X = 0 then  0
            else                1 ).

:- inst digit ---> 0 ; 1 ; 2 ; 3 ; 4 ; 5 ; 6 ; 7 ; 8 ; 9.

:- func digit_to_char(int) = char.
:- mode digit_to_char(in(digit)) = out is det.

digit_to_char(0) = '0'.
digit_to_char(1) = '1'.
...
digit_to_char(9) = '9'.

:- inst non_empty_list(I) ---> [I | list(I)].
:- inst non_empty_list    ==   non_empty_list(ground).

:- pred head(list(T),            T  ).
:- mode head(in,                 out) is semidet.
:- mode head(in(non_empty_list), out) is det.

head([X | _], X).

:- type abc ---> a ; b ; c.
:- inst ab  ---> a ; b.
\end{verbatim}
and so forth\ldots

\XXX{Should probably add some explanation for the above.}

\section{Higher Order Insts and Modes}



\section{The Standard Function Mode}

\XXX{NB I've already mentioned this one.}



\section{Higher Order Mode Declarations}

It is possible to specify the mode of a procedure using the
following syntax:
\begin{verbatim}
:- inst test == ( pred(in) is semidet ).

:- pred even(int).
:- mode even `with_inst` test.

even(X) :- X `mod` 2 = 0.
\end{verbatim}
The mode declaration is equivalent to writing
\begin{verbatim}
:- mode even(in) is semidet.
\end{verbatim}

\XXX{This isn't very convincing since the longhand is more concise.}

Judicious use of @with_inst@ declarations can improve the
readability and maintainability of a program.

\section{* Uniqueness}
\subsection{Determinism Restrictions}

\XXX{I've said something of this earlier on just before
Determinism.}

\subsection{* Mostly Unique Insts}

\XXX{This is definitely more advanced stuff.}
\begin{itemize}
\item Mostly unique $Leftrightarrow$ destructive update undone on
  backtracking.
\item Trailing and cost of trailing.
\item Possibility of optimizing trailing via FFI and impurity.
\item Distributed fat: implementation costs everywhere you can
  backtrack and may slurp up a general purpose register or
  two.
\end{itemize}

\section{Higher Order Insts and Modes}

\begin{itemize}
\item The type system says what values a variable may take on.
\item The mode system dictates the direction the data flows.
\item Higher order values are \emph{procedures} (not predicates), hence
  their modes also need to tell us the direction of data flow
  and the associated determinism.  \XXX{Fill in how HO insts are
  described and used.}
\item For the sake of brevity, the mode @in(pred(...) is <detism>)@
  can be abbreviated to just @pred(...) is <detism>@
\end{itemize}

\subsection{The Standard Function Inst}

\XXX{Said something about this earlier.}

\begin{itemize}
\item The set of insts @func(in, in, ..., in) = out is det@ is
  subsumed by the inst @ground@.  That is, you do not need to
  use a complex inst to pass around and use values with this
  sort of inst.
\item (Any of the @in@ arg modes may be replaced with an higher
  order inst.)
\item Since this is the most common type of higher order inst,
  higher order programming is thereby simplified.
\item There is a consequence: it is an error for a program to try
  to convert an higher order value with a different sort of
  inst to @ground@ (\eg by passing it to something expecting a
  @ground@ value).  This is necessary because otherwise there
  would be no way of knowing whether a ground value with a
  func type had the standard @func@ inst or not.
\end{itemize}

\section{Polymorphic Modes}

\XXX{Look this up.}



