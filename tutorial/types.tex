\section{Types}

Mercury is strongly typed.  Every value has a type and every
predicate associates a particular type with each argument.

\subsection{Primitive Types}

The primitive Mercury types are @int@, @float@, @char@ and @string@ --
examples are @123@, @3.141@, @('x')@, and @"foo"@,
respectively.\footnote{Mercury also includes tuples, such as @\{"Route",
66\}@, and higher order values, namely predicates and functions, among
its primitive types; however, we can ignore these for now.}

While primitive types are recognised by the compiler, no
operations are.  Operations on the primitive types are defined in
modules of the same name in the Mercury standard library.

\subsection{Algebraic Types}

You can define new types in Mercury:
\begin{verbatim}
:- type primary_colour
    --->    red
    ;       green
    ;       blue.

:- type rgb_colour
    --->    rgb(float, float, float).   % red, green, blue.
\end{verbatim}

Here @primary_colour@ is the name of the new type and its
\emph{data constructors} (possible values) are @red@, @green@ and @blue@.

Similarly @rgb_colour@ is a type with only one constructor,
@rgb/3@, which in this case takes three float arguments.\footnote{Some languages require you to add empty parentheses
after function symbols, such as @red@, @green@, and @blue@ above,
that have no arguments; Mercury, on the other hand, requires
that the parentheses be omitted alltogether if the functor has
no arguments.}

You can define parametric types
\begin{verbatim}
:- type binary_tree(T)
    --->    leaf
    ;       branch(T, binary_tree(T), binary_tree(T)).
\end{verbatim}

@T@ is a \emph{type variable} that can stand for any type at all.
The type @binary_tree(T)@ has two constructors: @leaf/0@ and
@branch/3@ which takes a value of type @T@ (for any choice of
type), a @binary_tree(T)@ and a @binary_tree(T)@ as arguments
respectively.

We can have instances of @binary_tree(T)@ for any type @T@ -- for
example, @binary_tree(int)@, @binary_tree(string)@ or even
@binary_tree(binary_tree(...))@.

\begin{itemize}
\item @T@ here is said to be a type variable.
\item Variables of any kind in Mercury are distinguished by
  starting with a capital letter.
\item A type may have several type parameters.
\item Any variables that appear in the constructor definitions
  must also appear in the head of the type declaration.
\item All type variables in the head must be distinct.
\end{itemize}

You can use other types in your definitions.  For example, we
could define a more general tree type with
\begin{verbatim}
:- type tree(T)
    --->    tree(T, list(tree(T))).
\end{verbatim}
So the type @tree(T)@ has a single constructor @tree/2@ which
takes a @T@ and a @list(tree(T))@ as arguments, respectively
(@list(T)@ is defined in the @list@ module in the standard
library.)

(Note that Mercury doesn't get confused if you use the same
name for both type and data constructor.)

\XXX{Tyson suggests that diagrams would make this section
easier to understand.}

\subsection{Pattern Matching}

Pattern matching is the primary means of identifying which
constructor is used for a particular type value.  The
following function definition illustrates:\footnote{Functions are a useful shorthand for predicates
that have a single output and can therefore appear as
sub-expressions.}
\begin{verbatim}
:- func insert(binary_tree(T), T) = tree(T).
:- mode insert(in, in) = out is det.

insert(leaf, X) =
    branch(X, leaf, leaf).

insert(branch(A, L, R), X) =
    ( if      X < A then branch(A, insert(L, X), R)
      else if A < X then branch(A, L, insert(R, X))
      else               branch(X, L, R)
    ).
\end{verbatim}
\XXX{This is wrong -- we should either use @compare/3@ (ugh) or make @<@
etc. synonyms for @compare/2@ closures in the language.}

The first clause matches when the first argument is a @leaf/0@
and the second clause matches when the first argument is a
@branch/3@, unifying @A@, @L@, and @R@ respectively with the
corresponding arguments of the @branch/3@ value.

(Here we using if-then-else expressions where what follows the
@then@ or @else@ is the result of the whole if-then-else
expression.)

Note that all function clauses must compute results of the
same type.  Similarly, the results of the then- and else-arms
of an if-then-else expression must have the same type.

Another way of writing @insert/2@ is to use explicit unification
to find out what data constructor was used for a particular
argument:
\begin{verbatim}
:- func insert(binary_tree(T), T) = tree(T).
:- mode insert(in, in) = out is det.

insert(T0, X) = T :-
    (
        T0 = leaf,           T  = branch(X, leaf, leaf)
    ;
        T0 = branch(A, L, R),
        ( if      X < A then T = branch(A, insert(L, X), R)
          else if A < X then T = branch(A, L, insert(R, X))
          else               T = branch(X, L, R)
        )
    ).
\end{verbatim}
Here we follow the function head with a @:-@ and a \emph{goal} that
computes the result.  (In fact, this is exactly how the
compiler would see the original implementation of @insert/2@.)

The goal in this case is a special kind of disjunction
(``choice'') called a \emph{switch}.  In this case, the argument @T0@
is unified with @leaf/0@ in the first disjunct and @branch/3@ in
the second.  Since @T0@ is an @in@ mode argument, the compiler
knows that only one of these disjuncts will succeed (if two or
more disjuncts could succeed at the same time then this
wouldn't be a switch.)  Switches are also very common choice
structures in Mercury.\footnote{The convention is to name a sequence of related
values as @X0@, @X1@, @X2@, \ldots, @X@.}

(Note that in this version we are using if-then-else \emph{goals},
as opposed to if-then-else \emph{expressions}.)

The third way we could implement insert/2 is to use an
if-then-else to decide what to do:
\begin{verbatim}
:- func insert(binary_tree(T), T) = tree(T).
:- mode insert(in, in) = out is det.

insert(T0, X) = T :-
    ( if T0 = branch(A, L, R) then
        ( if      X < A then T = branch(A, insert(L, X), R)
          else if A < X then T = branch(A, L, insert(R, X))
          else               T = branch(X, L, R)
        )
      else                   T = branch(X, leaf, leaf)
    ).
\end{verbatim}
This is also fine, although where possible it is often a
better idea to use pattern matching or switches, since then
the compiler generally has better scope for optimisation.

\subsection{Algebraic Types with Fields}
Just as with many languages, it is possible to give names to
the fields in data constructors, which can be used to access
just those fields.\footnote{Not all fields have to be named, although one has
to use pattern matching or unification to access unnamed
fields.}
\begin{verbatim}
:- type rgb_colour
    --->    rgb(
                red   :: float,
                green :: float,
                blue  :: float
            ).
\end{verbatim}
Now, say @X@ is a value of type @rgb_colour@.  Then @X ^ red@,
@X ^ green@ and @X ^ blue@ are expressions that return the values
of the corresponding fields.  If we have
\begin{verbatim}
    X = rgb(0.1, 0.2, 0.3)
\end{verbatim}
then
\begin{verbatim}
    X ^ red   = 0.1,
    X ^ green = 0.2,
    X ^ blue  = 0.3
\end{verbatim}
A field can be ``updated'' in the following way:
\begin{verbatim}
    Y = ( X ^ red := 0.4 )
\end{verbatim}
results in
\begin{verbatim}
    Y = rgb(0.4, 0.2, 0.3)
\end{verbatim}
What we're actually doing here is applying a substitution to
a field and obtaining a new value -- @X@ is unaffected by the
substitution.

We can chain several substitutions together like this:
\begin{verbatim}
    Y = ((( X ^ red   := 0.4 )
              ^ green := 0.5 )
              ^ blue  := 0.6 )
\end{verbatim}
Fields can also nest:
\begin{verbatim}
:- type rgb_point
    --->    xyrgb(
                x   :: float,
                y   :: float,
                rgb :: rgb_colour
            ).
\end{verbatim}
Now, if we have
\begin{verbatim}
    X = xyrgb(3.0, 2.0, rgb(0.1, 0.2, 0.3))
\end{verbatim}
then
\begin{verbatim}
    X ^ rgb ^ red = 0.1
\end{verbatim}
and
\begin{verbatim}
    Y = ( X ^ rgb ^ red := 0.4 )
\end{verbatim}
results in
\begin{verbatim}
    Y = xyrgb(3.0, 2.0, rgb(0.4, 0.2, 0.3))
\end{verbatim}
Note that it is an error for the same field name to be used in
data constructors for different \emph{types} defined in the same
module.  However, different \emph{constructors} of a type may have
fields with the same name.  For example:\footnote{@foo@, @bar@, @baz@ and @quux@ are traditional programming
names used in much the same way as Tom, Dick and Harry are often
used to name arbitrary individuals.}
\begin{verbatim}
:- type foo ---> foo(a :: int).

:- type bar ---> bar(a :: int).
\end{verbatim}
would lead to the compiler flagging an error since the types
@foo@ and @bar@ are both defined in the same scope and both have
constructors with a common field name.  The following is
acceptable, however, since all the data constructors with the
same field name belong to the same type:
\begin{verbatim}
:- type baz
    --->    constructor1(a :: int, ...)
    ;       constructor2(a :: int, ...)
    ;       constructor3(a :: int, ...).
\end{verbatim}
Field access syntax is just syntactic sugar for functions that
are automatically created by the compiler.  It is also
possible to write your own functions that can be used as field
names (for example, if you wanted to compute a value rather
than storing it in the data constructor, but make it look like
a field.)  We will go into this in more detail later on \XXX{}.\footnote{\emph{Syntactic sugar}: syntax that makes life easier,
without otherwise adding anything new to the language.}

\subsection{Equivalence Types}

For documentation and brevity it is also useful to be able to
declare type synonyms:
\begin{verbatim}
    :- type int_tree == tree(int).
\end{verbatim}
Mercury makes no distinction between @int_tree@ and
@tree(int)@.

Equivalence types can also take parameters:
\begin{verbatim}
            :- type list_tree(T) == tree(list(T)).
            :- type ints_tree    == list_tree(int).
\end{verbatim}

\subsection{Abstract Types}

Often one wants to export a type from a module, but hide its
definition so that users of the module can only manipulate
values of the type in question via the predicates exported by
the module interface.

Abstract types are used for this purpose:\footnote{Mercury supports \emph{overloading}, where the same name can
be used for several different things; the compiler uses type and
contextual information to work out which is really meant.  This is how
we can use the functor @(+)/2@ for addition for @int@s, @float@s,
complex numbers etc., despite the fact that these operations have
different implementations.}
\begin{verbatim}
:- module complex.
:- interface.
:- import_module float.

:- type complex.                        % Abstract type.

    % Constructor functions for complex numbers.
    %
:- func rect(float, float) = complex.   % re, im.
:- func polar(float, float) = complex.  % r, theta.

    % Operations on complex numbers.
    %
:- func complex + complex = complex.
:- func complex * complex = complex.
...

:- implementation.

    % This implementation uses a single, rectangular
    % representation.
    %
:- type complex                         % Concrete type.
    --->    re_im(float, float).
...
\end{verbatim}
The abstract type declaration in the module interface must be
matched by a concrete type definition (an algebraic or
equivalence type) in the implementation section.

The definition of the abstract type is hidden from users of
the complex module.

Finally, abstract types may also take parameters:
\begin{verbatim}
:- module set.
:- interface.

:- type set(T).                     % Abstract type.

:- func empty = set(T).
:- func union(set(T), set(T)) = set(T).
...

:- implementation.
:- import_module list.

    % This implementation uses lists to represent sets.
    %
:- type set(T) == list(T).          % Concrete type.

empty       = [].
union(A, B) = A ++ B.
...
\end{verbatim}

\subsection{* Types with Programmer-Defined Equality}

\XXX{Defer explanation until later.}

\subsection{Explicit Type Constraints}

Occasionally the compiler needs the programmer to specify what the type
of a particular variable is in order to resolve an ambiguous situation.
For example, the predicate @io__print/3@ endeavours to write out a human
readable form of its first argument, regardless of its type (this is
made possible through the magic of run-time type information, or
\emph{RTTI}, which we will discuss later \XXX{}.) Normally the compiler
has no problem working out the type of the argument, but occasionally
the programmer has to state the type explicitly.
\begin{verbatim}
main(!IO) :-
    io__print(123, !IO),
    io__print(length("abcd"), !IO).
\end{verbatim}
The compiler complains about the second call to @io__print/3@
with
\begin{verbatim}
foo.m:028: In clause for predicate `foo:main/2':
foo.m:028:   error: ambiguous overloading causes type ambiguity.
foo.m:028:   Possible type assignments include:
foo.m:028: V_6 :: (pred int) or int
\end{verbatim}
where @V_6@ is the type of the variable that the compiler has
introduced to unify with @length("abcd")@ and then pass to
@io__print/3@.

The problem is that (for hysterical raisins \XXX{}) the Mercury
standard library @string@ module includes both a function of
arity one and a predicate with the same name of arity two:
\begin{verbatim}
:- func length(string) = int.

:- pred length(string, int).
:- mode length(in,     out) is det.
\end{verbatim}
Both perform the same task, but are called in different ways.
The problem that the expression @length("abcd")@ presents the
compiler, in the absence of other typing information, is that
it could be either the result of function application, in
which case the type of @V_6@ would be int, or the closure formed
by applying the predicate @length/2@ to a single argument (this
is called \emph{currying} and happens all the time in higher order
programming; higher order programming is discussed later
\XXX{}), in which case @V_6@ would have the higher-order type
@pred(int)@.

One way to solve the problem is to add an explicit type
constraint to the program:
\begin{verbatim}
main(!IO) :-
    io__print(123, !IO),
    io__print(length("abcd") `with_type` int, !IO).
\end{verbatim}
The compiler can now deduce that the programmer means
@length("abcd")@ to be a function application rather than a
closure and can compile this program without further
assistance.

%\XXX{I'd like to have @with_type@ replaced with @:@ in the language
%before this goes to print...}



