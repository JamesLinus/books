% vim: ft=tex ts=4 sw=4 et



\chapter{Run-Time Type Information}

Mercury knows the type of every variable in a program and makes this
information available to the programmer.  This is called run-time type
information or RTTI for short.

RTTI has two main uses: the first is to allow dynamic type casting,
where a value of an unknown (i.e. polymorphic) type is tested to see
whether or not it belongs to a particular known type, in which case it
can then be treated as such.  This is the sort of thing that happens all
the time in Java programs (and is a source of some inefficiency
for Java programs, by the way.)

The second use is to support type-independent reading and writing of
values; for instance, the @io.read@ and @io.write@ predicates.  The
advantage of RTTI is that one module can contain all the code necessary
to read and write values of any type whatsoever.  Without RTTI, separate
@read@ and @write@ predicates would have to be written for every type in
a program.  \XXX{Is it worth mentioning the relationship with dynamic
type class casts, if we had them?  And what about the limits of type
class based approaches in general for this sort of thing.}

An example of both kinds of use can be found in the standard
library module, @pprint@, which provides the means to pretty print
(attractively display) arbitrary values.  The RTTI system allows the
pretty printer to examine an arbitrary value and identify its concrete
type and the name and arguments of the data constructor comprising the
value.  The latter information alone is sufficient to be able to
recursively construct a textual representation of a value.  The pretty
printer goes one step further: types such as @list@s, @array@s,
@tuple@s and so forth are handled specially since they have more
pleasing alternative representations.  For instance, it is better to
display a list as @[1, 2, 3]@ than @[|](1, [|](2, [|](3, [])))@.  In
order to perform this feat it is necessary to attempt dynamic casts of
the value in question into each of the types that have special
representations.

Often the run-time type information for a variable can be decided at
compile time; however, for polymorphically typed arguments the compiler
transparently arranges for extra arguments to be introduced to the
various functions and predicates in which the associated run-time type
information is passed by the callers of the predicate in question.

This chapter describes the key aspects of the Mercury RTTI system.  The
RTTI interface is provided in the @std_util@ standard library module.

\section{Type Descriptors and Type Constructor Descriptors}

A @type_desc@ (short for \emph{type descriptor}) is a representation of
the concerete type of a value.  The function @type_of@ returns the
@type_desc@ for its argument.  For example, @type_of([1, 2, 3])@ is the
@type_desc@ representing @list(int)@:
\begin{verbatim}
:- func type_of(T) = type_desc.
\end{verbatim}

Sometimes one is only interested in the ``top-level'' of a value's type,
otherwise known as the top-level \emph{type constructor}.  For instance,
one may be interested in finding out whether a value is a @list@ of some
kind, regardless of the type of the list members.  A @type_ctor_desc@
(short for \emph{type constructor descriptor}) is a representation of
the top-level type constructor of a value.  The function @type_ctor@
returns the @type_ctor_desc@ for its argument.  For example,
@type_ctor([1, 2, 3])@ is the @type_ctor_desc@ representing @list/1@.
\begin{verbatim}
:- func type_ctor(T) = type_ctor_desc.
\end{verbatim}

There is an inverse operation to @type_of@, called @has_type@:
\begin{verbatim}
:- some [T] pred      T `has_type` type_desc_desc.
\end{verbatim}
This looks slightly odd, but is nevertheless very useful!  The
expression @X `has_type` TypeDesc@ notionally unifies @X@ with a value
whose type descriptor is represented by @TypeDesc@.  Since the type of
@X@ is existentially quantified, all we know about it is that
@type_of(X) = TypeDesc@.  We typically use @has_type@ not to use @X@,
per se, but rather to work out the type represented by @TypeDesc@ --
this is explained in the section below in @Dynamic Casts@.
\XXX{Too scary too early?}

\section{Deconstructing Types}

Given a @type_desc@ one can identify its @type_ctor_desc@ and the
@type_desc@s of its arguments:
\begin{verbatim}
:- pred type_ctor_and_args(type_desc, type_ctor_desc, list(type_desc)).
:- mode type_ctor_and_args(in,        out,            out            ) is det.

:- func type_ctor(type_desc) = type_ctor_desc.

:- func type_args(type_desc) = list(type_desc).
\end{verbatim}

Given a @type_ctor_desc@ one can identify its name, arity, and the name
of its defining module:
\begin{verbatim}
    % type_ctor_name_and_arity(TypeCtorDesc, ModuleName, TypeName, TypeArity).
:- pred type_ctor_name_and_arity(type_ctor_desc, string, string, int).
:- mode type_ctor_name_and_arity(in,             out,    out,    out) is det.

:- func type_ctor_name(type_ctor_desc)        = string.

:- func type_ctor_arity(type_ctor_desc)       = int.

:- func type_ctor_module_name(type_ctor_desc) = string.
\end{verbatim}

\section{Constructing Values}

\section{Dynamic Casts}

Basic type casting is provided by the @dynamic_cast/2@ predicate:
\begin{verbatim}
:- pred dynamic_cast(T1, T2).
:- mode dynamic_cast(in, out) is semidet.
\end{verbatim}
with the definition
\begin{verbatim}
dynamic_cast(X, Y) :- univ(X) = univ(Y).
\end{verbatim}
using @mode univ(in) = out is det@ and @mode univ(out) = in is semidet@
respectively.  Values of type @univ@ need to carry around the
@type_desc@ associated with the value they encapsulate; a dynamic cast,
therefore, only needs to ensure that the @type_desc@s for @X@ and @Y@
are the same.  \XXX{What about subtypes?}  A dynamic cast does not affect
the value in @X@; instead, if it succeeds it merely binds the value to
the variable @Y@ whose type (in most normal uses) \emph{is} known ahead
of time.

Unfortunately, @dynamic_cast@ alone is not sufficient to handle cases
such as casting into @list/1@, where the type parameter is not of
interest: @dynamic_cast@ requires that the target type be known.  To
solve this problem we can use the following idiom:
\begin{verbatim}
:- some [T2] pred dynamic_cast_to_list(T1, list(T2)).
:-           mode dynamic_cast_to_list(in, out     ) is semidet.

dynamic_cast_to_list(X, L) :-
    [ArgTypeDesc] = type_args(type_of(X)),
    (_ `with_type` ArgType) `has_type` ArgTypeDesc,
    dynamic_cast(X, L `with_type` list(ArgType)).
\end{verbatim}
This is a complicated piece of code which bears some explanation.
