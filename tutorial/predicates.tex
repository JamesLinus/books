% vim: ft=tex ff=unix ts=4 sw=4 et wm=8 tw=0

\chapter{Predicates}

The key computational unit in Mercury is the predicate.  A
predicate describes a relationship between its arguments; it is
also possible to provide a procedural view of Mercury predicates,
which is how they can be used in a programming language.

Predicates cover not only tests and functions, as found in other
languages, but also procedures with multiple return values and
even non-deterministic relationships, in which there may be
several different solutions for a given set of input arguments.

\section{Introduction}

A predicate is defined by a set of \emph{clauses}, where each
clause takes the form
\begin{verbatim}
Head :- Body.
\end{verbatim}
This should be read as saying ``@Head@ is true if @Body@ is true'' with
the set of clauses forming an exhaustive specification of the
predicate.

If the body of a predicate is empty (taken to mean just @true@), one can
just write
\begin{verbatim}
    Head.
\end{verbatim}
Clauses like this are called \emph{facts}.

The body of a predicate clause is a \emph{goal} -- a logical formula
that must be satisfied in order for the head to hold.

A very simple example of a predicate is
\begin{verbatim}
:- pred max(int, int, int).         % A, B, Max.
:- mode max(in,  in,  out) is det.

max(A, B, Max) :-
    ( if B < A then Max = A else Max = B ).
\end{verbatim}
This predicate takes two integers, @A@ and @B@, and succeeds
binding @Max@ to @A@ if @A@ is greater than @B@ or binding @Max@ to @B@
otherwise.  The part of the mode declaration @is det@ means
that this is a deterministic predicate: it will always succeed
and has just one solution for any given pair of inputs.

(This sort of deterministic predicate with a single output is
called a \emph{function}.  Functions are so common that Mercury
includes special syntax and conventions to simplify working
with them.  More of this in a later section \XXX{}.)

For a more interesting example, consider the following small
genealogical database of people defined by the @person@ type:
\begin{verbatim}
:- type person
    --->    arthur  ; bill      ; carl
    ;       alice   ; betty     ; cissy.
\end{verbatim}
We start by defining predicates that we can use to decide if a
particular person is male or female.  These predicates are
labelled semidet because they have at most one solution
(success) for a given argument, but might also fail.
\begin{verbatim}
:- pred male(person).
:- mode male(in) is semidet.

male(arthur).
male(bill).
male(carl).

:- pred female(person).
:- mode female(in) is semidet.

female(Person) :- not male(Person).
\end{verbatim}
The definition for @female/1@ states that @female(Person)@ succeeds
iff\footnote{\emph{Iff}: if and only if} @male(Person)@ fails.  (Recall
that variables in Mercury are distinguished by starting with a capital
letter.)
\begin{verbatim}
:- pred father(person, person).     % Child, Father.
:- mode father(in,     out) is semidet.

father(betty, arthur).
father(carl,  bill).
father(cissy, bill).

:- pred mother(person, person).     % Child, Mother.
:- mode mother(in,     out) is semidet.

mother(bill,  alice).
mother(carl,  betty).
mother(cissy, betty).
\end{verbatim}
The predicates @father/2@ and @mother/2@ take their first argument
as an input and return a result in the second.  The
determinism is semidet again because they are not exhaustive:
some inputs can cause them to fail (\eg there is no solution
for @father(arthur, X)@.)
\begin{verbatim}
:- pred parent(person, person).     % Child, Parent.
:- mode parent(in,     out) is nondet.

parent(Child, Parent) :- father(Child, Parent).
parent(Child, Parent) :- mother(Child, Parent).
\end{verbatim}
This simply says that the parent of a child is either the
father or the mother.  The determinism is nondet because for a
given @Child@ argument this predicate may fail or may have more
than one solution:
\begin{itemize}
\item both @father/2@ and @mother/2@ can fail
(\eg if @Child = arthur@);
\item just one of @father/2@ or @mother/2@ could succeed
(\eg if @Child = bill@);
\item both @father/2@ and @mother/2@ could succeed
(\eg if @Child = cissy@).
\end{itemize}

(How failure and multiple solutions affect the execution of a
Mercury program will be explained below.)
\begin{verbatim}
:- pred ancestor(person, person).   % Person, Ancestor.
:- mode ancestor(in,     out) is nondet.

ancestor(Person, Ancestor) :-
    parent(Person, Ancestor).

ancestor(Person, Ancestor) :-
    parent(Person, Parent),
    ancestor(Parent, Ancestor).
\end{verbatim}
We can now define an ancestor of a Person as being either
\begin{itemize}
\item a parent of that Person or
\item an ancestor of a parent of that Person.
\end{itemize}

Since @parent/2@ is nondet, so is @ancestor/2@.

This table explains the difference between the number of
solutions a predicate can have for a given determinism:

\begin{tabular}{lll}
\textbf{Determinism}       & \multicolumn{2}{l}{\textbf{Number of Solutions}} \\
            & \textbf{Min} & \textbf{Max} \\
\hline \\
@semidet@   & 0            & 1 \\
@nondet@    & 0            & 1 or more \\
@det@       & 1            & 1 \\
@multi@     & 1            & 1 or more \\
\end{tabular}

(There are a few other determinisms, but we don't need to
consider them just yet \XXX{}.)

The compiler will check that your determinism declarations are
correct.

One interesting thing about this database is that there's no
reason why it shouldn't be run in ``reverse''.  That is, for any
father or mother, we should be able to deduce who their
children are and for any ancestor we should be able to work
out who their descendants are.

To gain this extra functionality we have only to add the
required extra mode declarations; there is no need to touch
the definitions themselves!  The extra mode declarations are
\begin{verbatim}
:- mode father(out, in) is nondet.   % Infer Child from Father.
:- mode mother(out, in) is nondet.   % Infer Child from Mother.
:- mode parent(out, in) is nondet.   % Infer Child from Parent.
:- mode ancestor(out, in) is nondet. % Infer Person from Ancestor.
\end{verbatim}
Notice that @father/2@ and @mother/2@ are nondet in this mode
rather than semidet: looking at the definitions we see that
@father(Child, bill)@ has multiple solutions @Child = carl@ and
@Child = sissy@, while @mother(Child, betty)@ also has solutions
@Child = carl@ and @Child = sissy@.

There is no reason why a predicate cannot have several output
arguments or even no input arguments.  For example, we
might go on to add
\begin{verbatim}
:- pred parents(person, person, person).    % Child, Mother, Father.
:- mode parents(in,     out,    out   ) is semidet.
:- mode parents(out,    in,     in    ) is nondet.

parents(Child, Mother, Father) :-
    mother(Child, Mother),
    father(Child, Father).
\end{verbatim}
The first mode of @parents/3@ is semidet because both @mother/2@
and @father/2@ are semidet when called with @Child@ as an input
and both must succeed for @parents/3@ to succeed.

The second mode is nondet for similar reasons, but exhibits an
interesting property.  At first glance you might think that
something will go wrong here: when @parents/3@ is called in the
second mode, initially @Mother@ and @Father@ are inputs and @Child@
is an output.  However, if the call to @mother/2@ succeeds, then
@Child@ will end up being bound to the identifier for some
person.  This appears that we would then be trying to call
@father/2@ with \emph{both} arguments as inputs, but there is no such
mode declaration for @father/2@.  There is no need to worry,
however -- the compiler will recognise the situation and treat
the definition of @parents/3@ like this as far as the second
mode is concerned:\footnote{The compiler isn't doing any special analysis here;
this is just a natural consequence of conversion to
super-homogeneous normal form, which will be explained later
on \XXX{}.}
\begin{verbatim}
:- mode parents(out, in, in) is nondet.

parents(Child, Mother, Father) :-
    mother(Child, Mother),
    father(X,     Father),
    X = Child.
\end{verbatim}
So here @father/2@ is being called in mode @(out, in) is nondet@ and
@parents/3@ succeeds if the result @X@ is bound to the same value
as @Child@.

In effect the compiler has deduced that @father/2@ has the
implied mode
\begin{verbatim}
:- mode father(in, in) is semidet.
\end{verbatim}
In general, Mercury will reorder each clause body for each mode
declaration so that results are generated before they are needed -- each
mode of a predicate is referred to as a \emph{procedure}.

\section{Execution}
This section explains Mercury's execution model in more
detail.  In particular, the notions of failure, backtracking
and non-determinism are addressed.

Let's start by looking at some of the code for our
genealogical database again -- this time we're going to label
the clauses to help us see how programs execute in Mercury.
\begin{verbatim}

    :- pred father(person, person).     % Child, Father.
    :- mode father(in,     out   ) is semidet.
    :- mode father(out,    in    ) is nondet.

f1  father(betty, arthur).
f2  father(carl,  bill).
f3  father(cissy, bill).

    :- pred mother(person, person).     % Child, Mother.
    :- mode mother(in,     out   ) is semidet.
    :- mode mother(out,    in    ) is nondet.

m1  mother(bill,  alice).
m2  mother(carl,  betty).
m3  mother(cissy, betty).

    :- pred parent(person, person).     % Child, Parent.
    :- mode parent(in,     out   ) is nondet.
    :- mode parent(out,    in    ) is nondet.

p1  parent(Child, Parent) :- father(Child, Parent).
p2  parent(Child, Parent) :- mother(Child, Parent).

    :- pred ancestor(person, person).   % Person, Ancestor.
    :- mode ancestor(in,     out   ) is nondet.

a1  ancestor(Person, Ancestor) :-
        parent(Person, Ancestor).

a2  ancestor(Person, Ancestor) :-
        parent(Person, Parent),
        ancestor(Parent, Ancestor).

    :- pred parents(person, person, person).    % Child, Mother, Father.
    :- mode parents(in,     out,    out   ) is semidet.
    :- mode parents(out,    in,     in    ) is nondet.

s1  parents(Child, Mother, Father) :-
        mother(Child, Mother),
        father(Child, Father).
\end{verbatim}
First off, we'll consider the goal @parent(cissy, Parent)@.
Every time we see a goal we try expanding it according to each
clause definition in turn (this is what being referentially
transparent is all about).  If we get to a dead end we have to
\emph{backtrack} to the last point where we had a choice and try a
different clause (when we backtrack we also forget any
variable bindings we might have made one the way from the
/choice point/.)
\begin{verbatim}
        parent(cissy, Parent)
(by p1)     father(cissy, Parent)
(by f1)         false because betty \= cissy
(by f2)         false because carl  \= cissy
(by f3)         Parent = bill
\end{verbatim}
So one solution to @parent(cissy, Parent)@ is @Parent = bill@.  We
obtained this by first expanding the goal according to p1 and
then expanding the resulting goal @father(cissy, Parent)@
according to each of @f1@, @f2@ and @f3@ until we found one that
succeeded.

If the program later has to backtrack over this goal then we
have to forget the binding @Parent = bill@ and try the next
clause for @parent/2@ (since there are no more clauses for
@father/2@):
\begin{verbatim}
        parent(cissy, Parent)
(by p1)     mother(cissy, Parent)
(by m1)         false because bill \= cissy
(by m2)         false because carl \= cissy
(by m3)         Parent = betty
\end{verbatim}
Now, say we wanted to ask the question ``which ancestor of
@carl@, if any, is also a parent of @bill@?''  Here's how things
would proceed:\footnote{We have to supply fresh local vars in each
 expansion -- for instance, @Ancestor@ in the expansion of @a2@ has
 been replaced with a new variable @Z@.}
\begin{verbatim}
        ancestor(carl, X), parent(bill, Y), X = Y
(by a1)     parent(carl, X), parent(bill, Y), X = Y
(by p1)         father(carl, X), parent(bill, Y), X = Y
(by f1)             false because betty \= carl
(by f2)             X = bill, parent(bill, Y), X = Y
(  =  )             parent(bill, Y), bill = Y
(by p1)                 father(bill, Y), bill = Y
(by f1)                     false because bill \= betty
(by f2)                     false because bill \= carl
(by f3)                     false because bill \= cissy
(by p2)                 mother(bill, Y), bill = Y
(by m1)                     Y = alice, bill = Y
(  =  )                     bill = alice
(  =  )                     false because bill \= alice
(by m2)                     false because bill \= carl
(by m3)                     false because bill \= cissy
(by p2)         mother(carl, X), parent(bill, Y), X = Y
(by m1)             false because carl \= bill
(by m2)             X = betty, parent(bill, Y), X = Y
(  =  )             parent(bill, Y), betty = Y
(by p1)                 father(bill, Y), betty = Y
(by f1)                     false because bill \= betty
(by f2)                     false because bill \= carl
(by f3)                     false because bill \= cissy
(by p2)                 mother(bill, Y), betty = Y
(by m1)                     Y = alice, betty = Y
(  =  )                     betty = alice
(  =  )                     false because alice \= betty
(by m2)                     false because bill \= carl
(by m3)                     false because bill \= cissy
(by a2)     parent(carl, Z), ancestor(Z, X), parent(bill, Y), X = Y
(by p1)         father(carl, Z), ancestor(Z, X), parent(bill, Y), X = Y
(by f1)             false because carl \= betty
(by f2)             Z = bill, ancestor(Z, X), parent(bill, Y), X = Y
(  =  )             ancestor(bill, X), parent(bill, Y), X = Y
(by a1)                 parent(bill, X), parent(bill, Y), X = Y
(by p1)                     father(bill, X), parent(bill, Y), X = Y
(by f1)                         false because bill \= betty
(by f2)                         false because bill \= carl
(by f3)                         false because bill \= cissy
(by p2)                     mother(bill, X), parent(bill, Y), X = Y
(by m1)                         X = alice, parent(bill, Y), X = Y
(  =  )                         parent(bill, Y), alice = Y
(by p1)                             father(bill, Y), alice = Y
(by f1)                                 false because bill \= betty
(by f2)                                 false because bill \= carl
(by f3)                                 false because bill \= cissy
(by p2)                             mother(bill, Y), alice = Y
(by m1)                                 Y = alice, alice = Y
(  =  )                                 alice = alice
(  =  )                                 true
\end{verbatim}
So our program concludes that a solution to
\begin{verbatim}
    ancestor(carl, X), parent(bill, Y), X = Y
\end{verbatim}
is
\begin{verbatim}
    X = alice, Y = alice
\end{verbatim}
As we indicated earlier in the definition of @parents/3@, in
practice we'd be more likely to write the goal as just
\begin{verbatim}
    ancestor(carl, X), parent(bill, X)
\end{verbatim}
and let Mercury work out where the implicit unification should
go.

\Aside{The above example might give the impression that Mercury is a term
rewriting system.  This is not true (although conceivably a very
slow Mercury implementation might use such a technique\ldots)  What
happens behind the scenes is much more efficient, albeit
computationally equivalent.}

In practice, ninety per cent or so of the code one writes is
deterministic.  However, there are times (as in the example above) when
non-determinism can be used to write very clear, concise programs.
Support for backtracking and unification is what distinguishes Mercury
from purely functional languages.

\section{Recursion}

Imperative languages like C and Java provide a whole slew of
mechanisms for supporting iterative (looping) control flow --
while loops, repeat-until loops, for loops and so forth.

Declarative languages typically provide but one mechanism for
such things: \emph{recursion}.  A recursive predicate is simply one
defined in terms of itself \XXX{Don't forget to mention mutual
recursion}.  (Later on we will discover \emph{higher order} predicates
and observe that the standard library supplies predicates that stand in
for the common iterative mechanisms found in imperative languages.)

Some simple examples:
\begin{verbatim}
:- func factorial(int) = int.

factorial(N) =
    ( if N  = 0 then 1 else N * factorial(N - 1) ).


:- func fibonacci(int) = int.

fibonacci(N) =
    ( if N =< 2 then 1 else fibonacci(N - 1) + fibonacci(N - 2) ).
\end{verbatim}
These are examples of what is sometimes called ``middle
recursion'' where the recursive call is \emph{not} the last thing
that the predicate (or function) does when looping.  Here,
calls to @factorial/1@ finishes with a multiplication and calls
to @fibonacci/1@ finish with an addition.

Although middle recursion is easy to read, it incurrs some
performance penalty in that each iteration has to record
something extra on the stack in order to finish up the
computation.\footnote{That said, the compiler can in fact turn some
middle recursive predicates into equivalent tail recursive
predicates that do not incurr a stack overhead \XXX{}.}

\XXX{Include a side-bar or something explaining stack frames.}

\emph{Tail recursion} describes the situation where the last thing a
predicate does is call itself.  Since there is nothing left to
do after the call, the compiler can reuse the current call's
stack frame for the recursive call, allowing the predicate to
execute using only a fixed amount of stack space.  For
example, tail recursive implementations of the above
functions are
\begin{verbatim}
factorial(N) = fac(N, 1).

fac(N, X) =
    ( if N = 0 then X     else fac(N - 1, N * X) ).


fibonacci(N) = fib(1, 1, N).

fib(FN_2, FN_1, N) =
    ( if N =< 2 then FN_1 else fib(FN_1, FN_1 + FN_2, N - 1) ).
\end{verbatim}
Tail recursive code like this should execute just as quickly
and efficiently as a for or while loop in an imperative
language.

\section{Unifications}

Mercury has but two basic atomic (\ie indivisible) types of
goal: unifications and calls.

A unification is written @X = Y@.  A unification can fail if @X@
and @Y@ are not unifiable.

Two values are unifiable if they are ``structurally similar'' --
that is, where you see a data constructor in one, you either
see the same data constructor in the other (and the
corresponding arguments are also structurally similar) or a
variable, which will end up being bound to the corresponding
term on the other side if the unification is successful.\footnote{Unlike Prolog, Mercury forbids the aliasing of
variables whereby a partially instantiated data structure
may contain the same unbound variable in two different places.
This will be explained fully in a later chapter.  \XXX{}}

Unifications, therefore, can be used to bind variables to
values, test to see if a variable is bound to a particular
constructor, unpack the arguments of a constructor or all of
the above.

In the following examples we assume that variables are
initially unbound:

\begin{tabular}{rcll}
         @123@ & @=@ & @123@ &
                -- Succeeds \\
           @X@ & @=@ & @123@ &
                -- Binds @X = 123@ \\
         @123@ & @=@ & @234@ &
                -- Fails \\
           @X@ & @=@ & @foo(1, 2, 3)@ &
                -- Binds @X = foo(1, 2, 3)@ \\
@foo(X, Y, Z)@ & @=@ & @foo(1, 2, 3)@ &
                -- Binds @X = 1@, @Y = 2@, @Z = 3@ \\
@foo(X, Y, 4)@ & @=@ & @foo(1, 2, 3)@ &
                -- Fails (@4 \= 3@) \\
@foo(X, 2, 3)@ & @=@ & @foo(1, 2, Z)@ &
                -- Binds @X = 1@, @Z = 3@ \\
\end{tabular}

The complex unifications are most easily understood by first
converting them into \emph{super homogeneous normal form} (which is
what the compiler does.)  In SHNF, the left hand side of each
unification is a variable and the right hand side is either
another variable or a functor \XXX{have I explained this term?}, all of
whose arguments are variables.  For example

\begin{verbatim}
    foo(X, 2, 3) = foo(1, 2, Z)
\end{verbatim}
becomes
\begin{verbatim}
    V_1 = X, V_2 = 2, V_3 = 3, V_4 = foo(V_1, V_2, V_3),
    V_5 = 1, V_6 = 2, V_7 = Z, V_8 = foo(V_5, V_6, V_7),
    V_4 = V_8
\end{verbatim}
where @V_1@\ldots@V_8@ are all temporary variables
introduced in the conversion to SHNF.

\section{Calls}
The other kind of primitive goal supported by Mercury is the
predicate call, @p(X1, ..., Xn)@, which we have already seen in
the examples above.

One way to picture the evaluation of a call is to think of it
as expanding into the different clause bodies for the
predicate definition while looking for solutions.

Two relatively important built-in predicates are @true@ and
@false@ \XXX{}.  @true@ always succeeds and @false@ always fails.\footnote{@false@ is often written as @fail@, which is a hangover
from Mercury's Prolog roots where it was sometimes more useful
to think in procedural terms.}

\section{Conjunction}

A goal of the form @G1, G2, ..., Gn@ is called a \emph{conjunction}
with the separating commas read as ``and''.  A conjunction
succeeds iff a consistent solution to each of the sub-goals @G1@,
@G2@, \ldots, @Gn@ can be found by the program.

(The compiler may have to reorder the sequence of goals in a
clause in order to satisfy the mode constraints.  Although one
can set a flag to force the compiler to do no more reordering
than is necessary, in general this will mean that certain
optimizations will not be possible.  The upshot of this is
that one should avoid writing code that assumes a particular
evaluation order other than that dictated by the mode
constraints.)

A conjunction executes by trying each of the goals in order.
If a goal fails then the program backtracks to the nearest
preceding choice-point (\ie non-deterministic goal that may
have other solutions).

\section{Negation}

A goal of the form not @G@ succeeds iff @G@ has no solutions.  @G@
may be a compound goal, in which case it should be enclosed in
parentheses to avoid syntactic precedence problems.

The sub-goal @G@ is said to be in a \emph{negated context} and as
such cannot bind any variables visible outside the negation
(since the only way not @G@ can succeed is if @G@ fails, in which
case it will not produce any variable bindings.)

Note that not not @G@ is equivalent to @G@ and hence may bind
variables if it succeeds (while not not @G@ would be an odd
thing to write, some of the code transformations the compiler
performs can generate such things; getting the modes right for
such things requires that the compiler observe this
simplification rule.)

\section{If-Then-Else Goals}

Mercury's if-then-else construct looks like this:
\begin{verbatim}
    ( if ConditionGoal then YesGoal else NoGoal )
\end{verbatim}
Note that the else part is \emph{not} optional.  \XXX{Except in DCG
code...  But we probably don't need to mention this.}

You may also see if-then-elses written as
\begin{verbatim}
    ( ConditionGoal -> YesGoal ; NoGoal )
\end{verbatim}
although the author finds this style less appealing.

While the parentheses are not always required, it is a very
good idea to include them in order to avoid confusing
syntactic precedence errors.  One common exception is a chain
of if-then-elses where only the top level of parentheses are
necessary:
\begin{verbatim}
    ( if      ConditionGoal1 then YesGoal1
      else if ConditionGoal2 then YesGoal2
      else if ConditionGoal3 then YesGoal3
      ...
      else                        NoGoal
    )
\end{verbatim}
Extra parentheses are not required even if any of the @ConditionGoal@s,
@YesGoal@s or the @NoGoal@ are compound goals.

The if-then-else goal
\begin{verbatim}
    ( if ConditionGoal then YesGoal else NoGoal )
\end{verbatim}
is semantically equivalent to the disjunction
\begin{verbatim}
    ( ConditionGoal, YesGoal ; not ConditionGoal, NoGoal )
\end{verbatim}
but will be implemented more efficiently by the compiler (if
@ConditionGoal@ produces no solutions in the first disjunct then
there's no point in checking again that it has none in the
second disjunct.)

It's worth looking at a few examples to really understand how
if-then-else goals work.  Again, we assume that all variables
are initially unbound:
\begin{verbatim}
    ( if ( X = 1 ; X = 2 ) then ( X = 2 ; X = 4 ) else X = 5 )
\end{verbatim}
The above goal is @nondet@ (the condition is @multi@
and the then-goal is @nondet@, since @X@ will be bound at this point),
but has the single solution @X = 2@.
\begin{verbatim}
    ( if ( X = 1 ; X = 2 ) then ( X = 3 ; X = 4 ) else X = 5 )
\end{verbatim}
The above goal is also @nondet@ for the same reason, but in fact has no
solutions (the compiler can only be expected to perform a certain amount
of program analysis and will sometimes not be completely precise --
mode inference is actually undecidable in general, although this is not
a problem in practice.  \XXX{Check this is true!})
\begin{verbatim}
    ( if 1 = 2 then ( X = 3 ; X = 4 ) else ( X = 5 ; X = 6 ) )
\end{verbatim}
The above goal is @nondet@ because the condition is @semidet@ and the
then- and else-goals are
@multi@.  (Since the condition will fail, the
else-goal is evaluated, with solutions @X = 5@ and @X = 6@.)
\begin{verbatim}
    ( if 1 = 1 then ( X = 3 ; X = 4 ) else ( X = 5 ; X = 6 ) )
\end{verbatim}
Similary, the above goal is @nondet@ because the condition is @semidet@
and the then- and else-goals are
@multi@.  (Since the condition will succeed, the
else-goal is evaluated, with solutions @X = 3@ and @X = 4@.)

That said, in the vast majority of cases where the
condition-goal is semidet and the then- and else-goals are
deterministic, if-then-else goals will act in very much the
same way as similar structures in other programming languages.

Since the condition-goal is in a negated context in the else-arm
of the disjunctive form of an if-then-else, it cannot
produce any outputs that would be used in @NoGoal@ or anything
outside the if-then-else as a whole.  It can, however, produce
outputs that are only used in the @YesGoal@.  (The reason for
this restriction is slightly subtle and will be explained in
more detail later.  \XXX{It's to do with having mode
independent semantics.})

For example, the following somewhat contrived code violates
the constraint because @S@ is bound inside the condition and is
also visible outside the if-then-else goal.
\begin{verbatim}

:- pred prime_divisor(int, int).
:- mode prime_divisor(in,  out) is nondet.
...

:- pred prime_divisor_or_zero(int, int).
:- mode prime_divisor_or_zero(in,  out) is multi.

prime_divisor_or_zero(N, S) :-
    ( if   prime_divisor(N, S)
      then true
      else S = 0
    ).

\end{verbatim}
The correct way to write @prime_divisor_or_zero/2@ is
\begin{verbatim}
prime_divisor_or_zero(N, S) :-
    ( if   prime_divisor(N, D)
      then S = D
      else S = 0
    ).
\end{verbatim}
Note that if-then-else \emph{expressions} are slightly different;
the following is perfectly legal:
\begin{verbatim}
prime_divisor_or_zero(N, S) :-
    S = ( if prime_divisor(N, D) then D else 0 ).
\end{verbatim}

\section{Disjunction}

Just as conjunction lets you use ``and'' to construct goals,
disjunction lets you use ``or''.  A \emph{disjunctive} goal takes the
form @(G1 ; G2 ; ... ; Gn)@ with the separating semicolons being
read as ``or''.

A disjunction succeeds iff any of its disjunct sub-goals
succeeds.  A disjunction has as many solutions as all of its
disjuncts put together: if one disjunct fails or backtracking
exhausts all the solutions for one disjunct then execution
proceeds with another disjunct.  Again, the compiler is
generally free to reorder disjuncts, although this should not
have a visible impact on programs.  Disjunctions are typically
non-deterministic, although switches, mentioned earlier, are a
special case.

\XXX{Need examples?}

The clausal notation we have been using in the examples above
is in fact convenient syntactic sugar for writing top-level
disjunctions.  For example, the @ancestor/2@ predicate we
defined earlier
\begin{verbatim}

ancestor(Person, Ancestor) :-
    parent(Person, Ancestor).

ancestor(Person, Ancestor) :-
    parent(Person, Parent),
    ancestor(Parent, Ancestor).
\end{verbatim}
could equivalently be written as
\begin{verbatim}
ancestor(Person, Ancestor) :-
    (
        parent(Person, Ancestor)
    ;
        parent(Person, Parent),
        ancestor(Parent, Ancestor)
    ).
\end{verbatim}
(Indeed, this is how the compiler sees multi-clause
definitions.)

In general it is better style to use clausal form for
top-level disjunctions.

\section{Switches}
Mercury recognises particular forms of (@semi-@)@det@
disjunction which it can compile very efficiently.

The @string@ library module defines the following type:
\begin{verbatim}
:- type poly_type
    --->    f(float)
    ;       i(int)
    ;       s(string)
    ;       c(char).
\end{verbatim}
which can be used to form heterogeneous collections of the
primitive types (this is useful, amongst other things, for
supplying argument lists for formatted output.)

Say we wanted to write a predicate that would convert any
@poly_type@ value into a @string@.  Here's how we might write the
code (in practice we would use a function; here we use a
predicate for the purposes of illustration):
\begin{verbatim}
:- pred poly_type_to_string(poly_type, string).
:- mode poly_type_to_string(in, out) is det.

poly_type_to_string(f(F), S) :- float_to_string(F, S)
poly_type_to_string(i(I), S) :- int_to_string(I, S)
poly_type_to_string(s(S), S).
poly_type_to_string(c(C), S) :- char_to_string(C, S)
\end{verbatim}
this is equivalent to the single clause definition
\begin{verbatim}
poly_type_to_string(Poly, S) :-
    (   Poly = f(F), float_to_string(F, S)
    ;   Poly = i(I), int_to_string(I, S)
    ;   Poly = s(S)
    ;   Poly = c(C), char_to_string(C, S)
    ).
\end{verbatim}

The compiler knows that since @Poly@ is an input variable it
must be bound when the disjunction is evaluated.  The compiler
also sees that each arm of the disjunction unifies @Poly@
against a different data constructor.  The compiler therefore
deduces that at most one disjunct can succeed on a particular
call (and, since all @poly_type@ data constructors are tested
for, \emph{exactly} one must succeed.)  The compiler generates very
efficient code for so-called \emph{switch} constructs such as this.\footnote{The name \emph{switch} is used because of its similarity
to the C language construct of the same name.}

Switches are an elegant way of describing conditions based on
unification tests and are typically more efficient than the
equivalent if-then-else chains.

\section{Existential Quantification}

Sometimes we only need to know whether a solution exists, but
are not interested in the result.  For this we use existential
quantification, which looks like this:
\begin{verbatim}
    (some [X, Y, Z] G)
\end{verbatim}
A goal of this form will succeed iff there is a solution to @G@,
but any bindings for @X@, @Y@ and @Z@ will not be visible outside
the quantification -- it's rather like saying that @X@, @Y@ and @Z@ 
are local variables for the goal @G@.

Mercury has an rule that any variables in a clause that do not
also appear in the head are implicitly existentially
quantified, which means you never actually need to use
explicit existential quantification in your programs.

\section{Universal Quantification}

On the other hand, we may wish to know whether a particular
property holds for all solutions to a particular goal.  This
is where universal quantification is useful.

The goal
\begin{verbatim}
    (all [X, Y, Z] G)
\end{verbatim}
is equivalent to writing
\begin{verbatim}
    not (some [X, Y, Z] not G)
\end{verbatim}

\section{Implication}

Mercury has three types of goal for describing implicative
relationships between goals.\footnote{The translations are given by de Morgan's laws.}
\begin{itemize}
\item @(G1 => G2)@ is shorthand for @(not G1 ; G2)@;
\item @(G1 <= G2)@ is shorthand for @(G2 => G1)@; and
\item @(G1 <=> G2)@ is shorthand for @((G1 => G2), (G1 <= G2))@.
\end{itemize}

Note that parentheses are required around @G1@ and @G2@ if they
are not atomic goals; it is a good idea to also put
parentheses around the implication as a whole to avoid
ambiguity in the scope of the implication.)

Implication is most often used with universal quantification
to test for some general property.

\subsection{Examples}

This example uses the predicate @member(X, Xs)@ to non-deterministically
project members @X@ from the @list@ @Xs@ and the semidet predicate
@even(X)@ which succeeds iff @X@ is even.\footnote{The convention is to
name a @list@ of items, @X@, as @Xs@.}
\begin{verbatim}
:- pred all_even(list(int)).
:- mode all_even(in) is semidet.

all_even(Xs) :-
    all [X] ( member(X, Xs) => even(X) ).

\end{verbatim}

\begin{verbatim}

    % Two list can be interpreted as equivalent sets if
    % each contains the same members as the other.
    %
:- pred equivalent_sets(list(T), list(T)).
:- mode equivalent_sets(in, in) is semidet.

equivalent_sets(Xs, Ys) :-
    all [Z] ( member(Z, Xs) <=> member(Z, Ys) ).
\end{verbatim}
The auxiliary predicates @member/2@ and @even/1@ are defined as\footnote{The Mercury parser views anything in @`@backquotes@`@
as an infix operator.  This is the same as the rule used in Haskell.}
\begin{verbatim}
:- pred member(T, list(T)).
:- mode member(out, in) is nondet.

    % X is a member of a list if it is either the head of
    % that list or a member of the tail.
    %
member(X, [X | _ ]).
member(X, [_ | Xs]) :- member(X, Xs).


:- pred even(int).
:- mode even(in) is semidet.

even(X) :- X `mod` 2 = 0.
\end{verbatim}

\section{Higher Order Application}

\XXX{I'll talk about this later.}

\section{Anonymous and Singleton Variables}

Often one is not interested in a particular output variable
from a call or unification.  In these cases you can use the
special variable named @_@ (a single underscore) which stands
for a different anonymous or ``don't care'' variable every time
it appears.

Sometimes, however, it makes programs easier to read if you do
name don't care variables.  Since variables that only appear
once in a clause are usually the result of typographical
error, the compiler will issue a warning when it sees such
things.  To get around this problem, giving a variable a
name that starts with an underscore (\eg @_X@) tells the compiler that
you know this is a named don't care variable and that there's
no need to issue a warning.



