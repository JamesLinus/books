%------------------------------------------------------------------------------%
% book.tex
% Ralph Becket <rafe@cs.mu.oz.au>
% Mon Jul 15 12:11:53 EST 2002
% vim: ft=tex ff=unix ts=4 sw=4 et wm=8 tw=0
%
%------------------------------------------------------------------------------%



%- Preamble -------------------------------------------------------------------%

\documentclass[a4paper,11pt,notitlepage,onecolumn]{book}
    %
    % [options]
    %   (10|11|12)pt            -- (default 10pt)
    %   (a4|letter)paper        -- (default letterpaper)
    %   fleqn                   -- (default centred) left-align formulae as
    %   leqno                   -- (default right) number formulae on the left
    %   [no]titlepage           -- [do not] start new page after the title
    %   (one|two)column         -- (default onecolumn)
    %   (one|two)side
    %   open(right|any)         -- open chapters on (right|any) pages only
    % {class}
    %   (article|report|book|slides)
    %                           -- consider FoilTeX instead of slides

\usepackage{doc}              % -- I want \MakeShortVerb
\usepackage{amsmath}          % -- I want align*
\usepackage{verbatim}         % -- It's better than the default, apparently
\usepackage{epsfig}           % -- I want \epsfbox
\usepackage{multirow}         % -- I want \multirow for tables

\pagestyle{headings}
    %
    % {style}
    %   (plain|headings|empty)  -- (default plain)
    %                           -- use \thispagestyle{} for new styles

% vim: ft=tex ff=unix ts=4 sw=4 et wm=8 tw=0

%- Definitions and Customization ----------------------------------------------%

    % \newenvironment{name}[numargs]{begincmds}{endcmds}

    % \newcommand{name}[numargs]{definition}
    %                           -- (default numargs 0)
    %                           -- refer to args as #1, #2 etc. in definition
    %                           -- use % at EOL if definition not finished
    %                           -- space after a cmd is ignored unless
    %                           --  preceded by {}

\newcommand{\eg} {e.g.\@ }
\newcommand{\ie} {i.e.\@ }
\newcommand{\XXX}[1] {{\small\textbf{XXX} \emph{#1}}}

\newcommand{\Aside}[1]%
{{\small{\begin{description}\item{\textbf{Aside:}} #1\end{description}}}}

\newcommand{\BoxedPar}[1]%
{{\center{\fbox{\parbox{\linewidth}{#1}}}}}

\newcommand{\True} {\top}
\newcommand{\False} {\perp}
\newcommand{\Not}[1] {\neg{}#1}
\newcommand{\Conj} {\wedge}
\newcommand{\Disj} {\vee}
\newcommand{\Imp} {\Rightarrow}
\newcommand{\Bimp} {\Leftarrow}
\newcommand{\Eqv} {\Leftrightarrow}
\newcommand{\All}[2] {\forall\ #1.\ #2}
\newcommand{\Some}[2] {\exists\ #1.\ #2}

\newcommand{\Union} {\cup}
\newcommand{\Intersection} {\cap}
\newcommand{\Excluding} {\backslash}

\newcommand{\Odd} {\text{odd}}
\newcommand{\FV} {\text{FV}}
\newcommand{\Wolf} {\text{wolf}}
\newcommand{\Fox} {\text{fox}}
\newcommand{\Bird} {\text{bird}}
\newcommand{\Caterpillar} {\text{caterpillar}}
\newcommand{\Snail} {\text{snail}}
\newcommand{\Animal} {\text{animal}}
\newcommand{\Herbivorous} {\text{herbivorous}}
\newcommand{\Carnivorous} {\text{carnivorous}}
\newcommand{\Eats} {\text{eats}}
\newcommand{\Plant} {\text{plant}}
\newcommand{\Grain} {\text{grain}}
\newcommand{\BiggerThan} {\text{bigger-than}}


%- Start of Document ----------------------------------------------------------%

\begin{document}

\setlength{\parindent}{0pt}
\setlength{\parskip}{\baselineskip}
% \setlength{\hoffset}{}        % -- left margin is this + 1in
% \setlength{\voffset}{}        % -- top margin is this + 1in
% \setlength{\textheight}{}
% \setlength{\textwidth}{}
% \setlength{\marginparwidth}{}
    %
    % -- or can use \addtolength{parameter}{length}
    % -- or can use \settoheight{parameter}{text}
    % -- or can use \settodepth{parameter}{text}
    % -- or can use \settowidth{parameter}{text}

\title{The Art of Mercury}
\author{Ralph Becket \\ \texttt{\small rafe@cs.mu.oz.au}}
\date{15 October 2001}

% \maketitle

% \begin{abstract}
%
% ...
%
% \end{abstract}

%\tableofcontents

%- Body -----------------------------------------------------------------------%

\MakeShortVerb{\@}            % -- @...@ is now shorthand for \verb@...@
\MakeShortVerb{\#}            % -- #...# is now shorthand for \verb#...#

    % \text(rm|tt|sf|bf|it|sc|sl|up|md|normal){} \emph{}

    % \math(rm|tt|sf|bf|it|sc|sl|up|md|normal){}

    % \tiny \scriptsize \footnotesize \small \normalsize
    % \large \Large \LARGE \huge \Huge
    %                           -- also work as environments

    % ~                         -- small, fixed, nonbreaking space
    % \hspace{size}             -- soft space (may be lost at SOL or EOL)
    % \hspace*{size}            -- hard space
    %                           -- size may be \stretch{factor}
    % \hfill                    -- same as \hspace{\fill}

    % \vspace{size}             -- soft space (may be lost at TOP or BOP)
    % \vspace*{size}            -- hard space
    %                           -- size may be \stretch{factor}
    %                           --  of \smallskip or \bigskip

    % \\                        -- linebreak
    % \\*                       -- linebreak but prohibit page break
    % \newpage

    % \rule[lift]{width}{height}

    % \parbox[(c|t|b)]{width}{text}
    % \begin{minipage}[(c|t|b)]{width} text \end{minipage}
    % \mbox{text}               -- prevents word breaking

    % \begin{(flushleft|flushright|center|quote|verse)}
    % text
    % \end{(flushleft|flushright|center|quote|verse)}

    % \verb@verbatim text@
    % \begin{verbatim}
    % verbatim text
    % \end{verbatim}

    % \begin{tabular}{(l|r|c|p{width}|<bar>|@{colsep})...}
    % datum & datum & ... \
    % datum & datum & ... \
    % \hline
    % datum & datum & ... \
    % \cline{1-2}   & ... \
    % \multicolumn{2}{(l|c|r)}{wide datum} & ... \
    % \end{tabular}

    % \begin{(figure|table)}[[!](h|t|b|p)...]
    % ...
    % \caption{caption text}
    % \end{(figure|table)}

    % \label{marker}            -- set a marker
    % \ref{marker}              -- section containing marker
    % \pageref{marker}          -- page number of marker

    % $ ... $                   -- inline mathematics
    % \[ ... \]                 -- display mathematics
    % \begin{equation}          -- align* suppresses numbering
    % ...
    % \end{equation}
    % \begin{array}{...}        -- as tabular
    % ...
    % \end{array}
    % \begin{eqnarray}          -- as {array}{rcl} but with numbering
    % ...                       -- use \nonumber to suppress numbering of a row
    % \end{eqnarray}
    % \, \; \<spc> \quad \qquad -- math-mode spacing
    % \(over|under)line{...}
    % \(over|under)brace{...}_{...}
    % \wide(tilde|hat)
    % \overrightarrow
    % \frac{top}{bottom}
    % {... \choose ...}         -- adds parentheses
    % {... \atop ...}           -- no parentheses
    % \stackrel{topsym}{linesym}
    % \left<brasym> ... \right<ketsym>
    %                           -- use \right. for no <keysym>
    %                           -- empty lines etc. forbidden in math mode

% \include{}
    %
    % -- \includeonly takes a list of file names and filters any \includes
    %    (can only appear in the preamble.)
    % -- \include takes a single file name, starts a new page
    % -- \input does not start a new page
    % -- omit the .tex suffix from the file names

% % vim: ft=tex ff=unix ts=4 sw=4 et wm=8 tw=0

\chapter{Introduction}

\XXX{Check consistency of hyphenation for non- sub- etc.}




% % vim: ft=tex ff=unix ts=4 sw=4 et tw=76

\chapter{Mercury By Example}

This chapter aims to convey through examples a basic feeling for how
Mercury works.  Pedagogical efficiency at this stage requires some glossing
over of finer details and a certain lattitude in precision; these
deficiencies are remedied in later chapters.  The approach taken here is to
start by presenting the ``obvious'' solution to each problem and then
introduce features of Mercury that allow for more elegant or efficient
programs (although we are not necessarily trying to write the most efficient
programs possible at this point!)

By convention, we use a @fixed width typeface@ for source code, code
fragments, and output from programs.  Examples of program output are
presented as if run from a standard Unix command line shell. \XXX{Is it okay
to be Unix-centric in a book?}  A @>@ character in the first column
indicates input from the user; anything else is output from the program.  It
should be obvious from context which parts of the text are program output
and which are Mercury code.



\section{Hello, World!}

It is slightly unfortunate that the ``Hello, World!'' program introduces
no less than three advanced Mercury concepts, but since tradition
dictates that any tutorial text start with ``Hello, World!'' we'll just
have to jump straight in with the knowledge that things will get easier
thereafter.

We define a \emph{predicate} called @main@ with two parameters which
we've named @IO_in@ and @IO_out@:
\begin{myverbatim}
main(IO_in, IO_out) :-
    io.write_string("Hello, World!\n", IO_in, IO_out).
\end{myverbatim}
Throughout this book we use the notation @name/arity@ to refer to symbols
(the \emph{arity} is the number of arguments), except when the arity is
zero, in which case we'll just write @name@.  Consequently we use @main/2@
when talking about this predicate.  Every Mercury program must include a
@main/2@ predicate that is called when the program is run.

Syntactically, \emph{variables} start with a capital letter, while
everything else (things that start with lower case letters, numbers or
non-alphabetic characters) is a \emph{functor}.

A predicate is defined by one or more \emph{clauses} in which the
\emph{head} is separated from the \emph{body} by a @:-@ symbol.  Here
the body is simply a call to @io.write_string/3@ -- this is a
\emph{module qualified name} and refers to @write_string/3@ as defined
in the @io@ module.  Module qualification of names is only
\emph{necessary} when there may be ambiguity, although it is often a
good idea to do so anyway for software engineering reasons.

Back to the source code: strings in Mercury are delimited by @"@double
quote marks@"@ and appear much the same as they do in C or Java
programs.  Character escapes are introduced with a preceding backslash
@\@ and have the usual meanings: @\n@, for example, denotes the newline
character.

All predicates that perform input/output (IO) operations have to include
two @io.state@ parameters denoting respectively the ``state of the
world'' before and after the IO operation.  In order to ensure that IO
operations occur in a definite order, @io.state@s are \emph{unique}:
they cannot be copied and after being used once an @io.state@ becomes
``dead'' and is no longer available.

In Mercury, every predicate must be accompanied by a @pred@ declaration,
giving the types of its parameters, and @mode@ declarations, indicating
which parameters are inputs and which are outputs.  We'll return to
these topics in later chapters, but for now we'll simply write down the
declarations for @main/2@:
\begin{myverbatim}
:- pred main(io.state, io.state).
:- mode main(di,       uo      ) is det.
\end{myverbatim}
The first declaration says that @main@ is a predicate with two
arguments, both of which have type @io.state@.  The second declaration
says that a call to @main/2@ requires that the first argument be a
``destructive input'' (@di@) which is unique on input and is dead
after the call, and that the second argument is a ``unique
output'' (\ie after the call it will be instantiated with a unique value.)
The @is det@ part of the @mode@ declaration states that @main/2@ is
\emph{deterministic} -- that is, it will always succeed and will have
only one possible \emph{solution}.  (We will see later that some
predicates can fail and some can have multiple solutions.)

Note that declarations always start with a @:-@ and that declarations
and clauses always finish with a single full stop.

Since we refer to names defined in the @io@ module (which is part of the
Mercury standard library), we must also have an @import_module@
declaration in our program:
\begin{myverbatim}
:- import_module io.
\end{myverbatim}

All that is left to do is to wrap up our clauses and declarations in a
\emph{module}, which is the unit of compilation for Mercury programs.  A
Mercury module has two parts: the \emph{interface} section containing the
declarations for the things that are \emph{exported} by the module; and the
\emph{implementation} section which contains predicate definitions and so
forth, the details of which are hidden from other users of the module.

We will call our module @hello@; the Mercury tools will expect us to put
the source code in a file called @hello.m@:
\begin{myverbatim}
:- module hello.
:- interface.
:- import_module io.

:- pred main(io.state, io.state).
:- mode main(di,       uo      ) is det.

:- implementation.

main(IO_in, IO_out) :-
    io.write_string("Hello, World!\n", IO_in, IO_out).
\end{myverbatim}
A Mercury module starts with a @module@ declaration giving its name,
followed immediately by an @interface@ declaration.  What follows
thereafter, up to the @implementation@ declaration, is deemed to be
\emph{exported} by the module.  We put the declarations (but not the
clauses) for @main/2@ here.  Since the declarations for @main/2@ use
names exported by the @io@ module, we have to include the appropriate
@import_module@ declaration and convention dictates we place this at the
top of the interface section.  After the @implementation@ declaration we
write our predicate definitions -- these are \emph{private} and anything
defined here, but not declared in the interface section, will not be
visible to other users of the module.

We build and run the @hello@ program like this:
\begin{myverbatim}
> mmc --make hello
  Making hello.int3
  Making hello.c
  Making hello.o
  Making hello
> ./hello
  Hello, World!
\end{myverbatim}
Et voil\`a!

\subsection{Goals and Conjunctions}

To illustrate a point, let's define @main/2@ in a slightly more
long-winded way:
\begin{myverbatim}
main(IO_in, IO_out) :-
    io.write_string("Hello, ", IO_in, IO1),
    io.write_string("World!",  IO1, IO2),
    io.nl(IO2, IO_out).
\end{myverbatim}
(The predicate @io.nl/2@ simply writes out a newline character.)

The body of a clause is also what's known as a \emph{goal}.  An \emph{atomic
goal} is either a call, such as @io.nl(IO2, IO)@, or a \emph{unification} of
the form @X = Y@.  A sequence of goals separated by commas (read as ``and'')
forms a compound goal called a \emph{conjunction}.

\subsection{IO and State Variables}

Since Mercury only allows a variable to be bound once, we have to use a
different variable name for each @io.state@.  As demonstrated in the
definition of @main/2@ above, this quickly becomes tedious.  So the language
provides some syntactic sugar -- a notational convenience that has nothing
to do with the language semantics -- in the guise of \emph{state variables}.
A state variable @!X@ actually stands for a \emph{pair} of ordinary
variables, renumbered as required in situations like this.  Hence 
@main/2@ above is more likely to be written as
\begin{myverbatim}
main(!IO) :-
    io.write_string("Hello, ", !IO),
    io.write_string("World!",  !IO),
    io.nl(!IO).
\end{myverbatim}
The Mercury compiler will transform this code into something just like the
version in which we numbered each @io.state@ by hand.  State variables are
explained in detail in chapter \XXX{}.



\section{Rot13}

The simplest form of Caesar cypher is @rot13@, in which each letter in the
plaintext is replaced with the letter thirteen letters ahead in the alphabet
(wrapping around from @z@ to @a@) in the cyphertext.  That is, @a@ becomes
@n@, @b@ becomes @o@, \ldots and @z@ becomes @m@.  There being twenty six
letters in the Roman alphabet, a second application of @rot13@ to the
cyphertext will reveal the original plaintext message.

A session with a @rot13@ program might look like this:
\begin{myverbatim}
> I shot an arrow in the air, she fell to Earth in Berkely Square.
  V fubg na neebj va gur nve, fur sryy gb Rnegu va Orexryl Fdhner.
> V fubg na neebj va gur nve, fur sryy gb Rnegu va Orexryl Fdhner.
  I shot an arrow in the air, she fell to Earth in Berkely Square.
\end{myverbatim}

It is the custom in some public fora, newsgroups for instance, to use
@rot13@ to encode things such as plot spoilers in messages discussing film
endings so as not to spoil the surprise for those who haven't yet seen the
film.  Those wishing to view the unencrypted text can simply filter the
message through a @rot13@ program.

A @rot13@ program is a simple text filter, reading in a character at a time
from the input stream and writing the converted character to the output
stream.

The predicate we will use to read in each character, @io.read_char/3@,
returns a result which is either @ok(C)@ where @C@ is the character just
read in, @eof@ for the end-of-file condition, or @error(E)@ where @E@ is a
code indicating the nature of the error.  Here is our definition for
@main/2@:
\begin{myverbatim}
main(!IO) :-
    io.read_char(Result, !IO),
    (
        Result = ok(C),
        io.write_byte(rot13(C), !IO),
        main(!IO)
    ;
        Result = eof
    ;
        Result = error(_),
        exception.throw(Result)
    ).
\end{myverbatim}
The call to @io.read_char/3@ instantiates @Result@ with one of the
possible outcomes of the input operation.  To decide what to do next we
use a \emph{disjunction}: a sequence of goals separated by
semicolons (read as ``or'') inside parentheses.  Each
\emph{arm} of the disjunction starts by attempting to unify @Result@
with one of the possible outcomes.  (This type of disjunction is called a
\emph{switch} because it's rather like the @switch@ statement found in C and
Java.)

If @Result@ matches @ok(C)@ -- unifying @C@ with the character just read
in -- then we encrypt the character using a function @rot13/1@, which we'll
come to next, write it out and make a recursive call to @main/2@ to process
the rest of the input.  If @Result@ matches @eof@ then we're finished.
Finally, if @Result@ matches @error(_)@ (the underscore @_@ stands for any
``don't care'' value) then we simply \emph{throw} it as an exception, which
will cause execution to halt with an error message.

(Note that a functor with no arguments, such as @eof@, \emph{is not}
followed by empty parentheses, as one would expect in a language like C or
Java.)

We implement @rot13/1@ as a function.  A function is just syntactic sugar
for a deterministic predicate with a single output that, unlike ordinary
predicates, can be used in expressions.  Functions are almost always the
right stylistic choice in such cases:
\begin{myverbatim}
:- func rot13(int) = int.
:- mode rot13(in ) = out is det.

rot13(X) = Z :-
    ( if   encrypt_letter(X, Y)
      then Z = Y
      else Z = X
    ).
\end{myverbatim}
(The @( if _ then  _ else )@ construct is called a \emph{conditional goal};
the @else@ arm is \emph{not} optional.)

In this case, if @encrypt_letter(X, Y)@ succeeds (unifying @Y@ with the
encrypted form of @X@) then the result, @Z@, is @Y@, otherwise @X@ can't
be a letter and we just return it unchanged in @Z@.

We're on the home strait.  All we need to do now is implement the 
translation of letters:
\begin{myverbatim}
:- pred encrypt_letter(int, int).
:- mode encrypt_letter(in,  out) is semidet.

encrypt_letter(X, Y) :-
    (   X = 'A', Y = 'N'
    ;   X = 'B', Y = 'O'
    ...
    ;   X = 'Z', Y = 'M'
    ;   X = 'a', Y = 'n'
    ;   X = 'b', Y = 'o'
    ...
    ;   X = 'z', Y = 'm'
    ).
\end{myverbatim}
Once again, we use a switch to decide what to do: if @X@ matches @'A'@
then we return @'N'@ in @Y@, and so forth.
Because this switch does not cover all possible values that @X@ might
have (the digit characters, for instance)
the switch can \emph{fail}.  Don't worry too much about what
this means for now, other than that failure of a call can be detected
using a conditional goal, as in the code for @rot13/1@ above.
Since @encrypt_letter/2@ has at most one solution for any given @X@, but can
fail, it is said to be \emph{semi-deterministic}, hence the @semidet@
determinism annotation on the mode declaration.

To turn the above code into something we can compile, we need only wrap
it up with the appropriate module declarations as before, remembering to
include
\begin{myverbatim}
:- import_module exception.
\end{myverbatim}
at the start of the implementation section since we use
@exception.throw/1@ in @main/2@.

\subsection{Functions and Conditional Expressions}

Function mode declarations like the following
common,
\begin{myverbatim}
:- mode f(in, in, ..., in) = out is det.
\end{myverbatim}
are so common that Mercury allows us to omit them altogether for functions
that have just one mode and it is of this form.

Second, just as Mercury has conditional goals, it also has \emph{conditional
expressions}.  Thus we can shorten the definition of @rot13/1@ to
\begin{myverbatim}
rot13(X) = ( if encrypt_letter(X, Y) then Y else X ) :-
    true.
\end{myverbatim}
where @true@ is a built-in predicate that always succeeds, but does
nothing else.  The @then@ and @else@ arms of a conditional expression
contain expressions, rather than goals.

Third, any clause in which the body is just @true@ can be written as
just the head part on its own, hence we can simplify @rot13/1@ still
further to:
\begin{myverbatim}
rot13(X) = ( if encrypt_letter(X, Y) then Y else X ).
\end{myverbatim}
This rule works for predicates as well as functions.

\subsection{Predicates and Disjunctions}

If the body of a clause is a disjunction, it can often be described more
clearly using multiple clauses.  That is, we could rewrite
@encrypt_letter/2@ as
\begin{myverbatim}
encrypt_letter(X, Y) :- X = 'A', Y = 'N'.
encrypt_letter(X, Y) :- X = 'B', Y = 'O'.
...
encrypt_letter(X, Y) :- X = 'y', Y = 'l'.
encrypt_letter(X, Y) :- X = 'z', Y = 'm'.
\end{myverbatim}
which will be transparently converted by the Mercury compiler into our
original definition using a disjunction.  So far we've actually made
life harder on our fingers\ldots but!  Mercury has some more
syntactic sugar that, together with clausal notation, really does pay
off.

If the Mercury compiler sees an expression rather than a variable in an
head argument, as in
\begin{myverbatim}
p(expr1, expr2, ...) :- ...
\end{myverbatim}
it will transparently convert this into
\begin{myverbatim}
p(X1, X2, ...) :- X1 = expr1, X2 = expr2, ...
\end{myverbatim}
where @X1@ and @X2@ and so on are chosen so as not to coincide with
anything in the original definition of the clause.

Consequently we can write @encrypt_letter/2@ in a quite perspicuous
fashion:
\begin{myverbatim}
encrypt_letter('A', 'N').
encrypt_letter('B', 'O').
...
encrypt_letter('y', 'l').
encrypt_letter('z', 'm').
\end{myverbatim}

\subsection{Using Arithmetic Instead}

If we know that our program is only going to be run on ASCII data (or
some other character set encoding in which the byte codes for the
letters of the alphabet are contiguous), we could rewrite @rot13/1@ to
do away with the need for @encrypt_letter/2@ altogether:
\begin{myverbatim}
rot13(X) = Y :-
    char.to_int(X, XCode),
    YCode = ( if      0'A =< X, X =< 0'Z
              then    ((13 + X - 0'A) `mod` 26) + 0'A
              else if 0'a =< X, X =< 0'z
              then    ((13 + X - 0'a) `mod` 26) + 0'a
              else                            X
            )
    char.to_int(Y, YCode).
\end{myverbatim}
Several new things are illustrated in this definition.  Let's start off by
looking at syntax.  The first thing to notice is that the integer code for
the character @'A'@ is written as @0'A@.  The second is that while the names
@=<@ (less than or equal to), @+@ and @-@ can all appear as infix operators
as one would expect, it is also possible to use any other name as an infix
operator simply by enclosing it in @`@backquotes@`@.  This is what we've
done with @mod/2@ above.  The Mercury Reference Manual \XXX{} contains a
list of which names can be used as prefix, infix or postfix operators.  Note
that it makes no difference to Mercury if we write @+(13, X)@ rather than
@13 + X@ or @mod(Y, 26)@ rather than @Y `mod` 26@.  The latter are simply
easier to read.

The integer operators are all defined in the @int@ module in the Mercury
standard library, which \emph{must} be imported before they can be used --
unlike most languages, basic arithmetic is not built-in to Mercury.

The other new thing here is the use of the same predicate in different
``directions''.  The predicate @char.to_int/2@ converts between characters
and integer character codes.  The first such call,
@char.to_int(X, XCode)@ uses the mode
\begin{myverbatim}
:- mode char.to_int(in, out) is det.
\end{myverbatim}
to convert @X@ into its character code, @XCode@.  The second call,
@char.to_int(Y, YCode)@ uses the mode
\begin{myverbatim}
:- mode char.to_int(out, in) is semidet.
\end{myverbatim}
to convert @YCode@ into the corresponding character, @Y@ (this ``direction''
is semi-deterministic because not all integers correspond to characters.)

\subsection{If-Then-Else Chains vs. Switches}

Where possible, it is not only clearer, on the whole, to use a switch in
preference to an @if-then-else@ chain, but the compiler can often do a
better job of optimizing switches (implementing them with jump tables or
hash
tables for instance.)



\section{A Spelling Checker}

In this section we write a simple spelling checker that will take two
files as arguments, a dictionary file and a file to be checked for spelling
mistakes, and print out any words in the latter that do not appear in the
former:
\begin{myverbatim}
> ./spell_check my_dictionary_file my_text_file
  doog
  kat
  mowse
\end{myverbatim}
(On Unix systems one can usually find a dictionary file in
@/usr/dict/words@.)
Our first approach will simply compute the set of words in each file
and use the set difference function to decide which words in the text file
have been misspelled.

First, let's write @main/2@: it has to retrieve the file names of the
dictionary and source file from the command line arguments, read
and compute the set of words in each file, and print out any words in
the text file set that do not appear in the dictionary file set:
\begin{myverbatim}
main(!IO) :-

    io.command_line_arguments(Args, !IO),

    ( if Args = [DictFileName, TextFileName] then

            % Read in the list of words in each file.
            %
        file_to_word_set(DictFileName, DictWordSet, !IO),
        file_to_word_set(TextFileName, TextWordSet, !IO),

            % Find the list of misspelt words.
            %
        ErrorWordSet = TextWordSet `set.difference` DictWordSet,
        Errors       = set.to_sorted_list(ErrorWordSet),

            % Write out the misspelt words.
            %
        write_words_in_list(Errors, !IO)

      else

        exception.throw("usage: spell_check <dictfile> <textfile>")
    ).
\end{myverbatim}
This definition introduces several new things.  Let's start with the new
piece of syntax: comments begin with a per cent @%@ symbol and extend to the
end of the line.

@main/2@ begins by calling @io.command_line_arguments/3@ to obtain the
list of command line arguments in @Args@.  If @Args@ is a two member
list (instantiating @DictFileName@ and @TextFileName@ respectively) then we
enter the spelling checker part of the algorithm, otherwise we report a
usage error by throwing an exception.

Having obtained the dictionary and text file names, @file_to_word_set/4@
is called to get the set of words in each file.  The predicate
@file_to_word_set/4@ is described below.

We use the @set.difference/2@ function to obtain the set of words in the
text file, but not in the dictionary, and then call
@set.to_sorted_list/1@ to turn the error set into a list.

Finally, we call @write_words_in_list/3@ (also described below) to print out
each member of @ErrorWords@.

Returning to @file_to_word_set/4@, this predicate has to open the given
file, read its contents, and work out the set of words therein.  Here's
one possible implementation:
\begin{myverbatim}
:- pred file_to_word_set(string, set(string), io, io).
:- mode file_to_word_set(in,     out,         di, uo) is det.

file_to_word_set(FileName, WordSet, !IO) :-

    io.open_input(FileName, OpenResult, !IO),
    (
        OpenResult = ok(InputStream),
        io.read_file_as_string(InputStream, ReadResult, !IO),
        (
            ReadResult = ok(String),
            Words      = string.words(String),
            WordSet    = set.list_to_set(Words)
        ;
            ReadResult = error(_, _),
            exception.throw(ReadResult)
        )
    ;
        OpenResult = error(_),
        exception.throw(OpenResult)
    ).
\end{myverbatim}
The first thing to observe is that we've used @io@ rather than
@io.state@ in the @pred@ declaration; this is a common convention in
modern Mercury programs: the @io@ library defines @io.io@ as a
synonym for @io.state@ and @io@ is the unqualified form of @io.io@.  It
simply looks better on the page.

The next new thing is the type @set(string)@.  The @set/1@ type, defined
in the @set@ module in the Mercury standard library, is
\emph{parameterised}.  That is, we can have @set(int)@, @set(string)@,
@set(set(int))@ and so forth and Mercury will ensure that we never get
different types of set mixed up in our programs.  The chapter on types
\XXX{} explains parametric types in detail.

The call to @io.open_input@ tries to open an input stream for the file
named in @FileName@.  \XXX{Do I need to explain ``input stream''?}
The only possible outcomes are @ok/1@ or @error/1@, which
we test for in a switch.

If all is well, @io.read_file_as_string/4@ attempts to read all the data
from the file referred to by @InputStream@.  The result will be of the form
@ok/1@ for success or @error/2@ if there was a problem (this particular
error result includes any data that was read up to the point the error
occurred along with the error code.)  Again, we use a switch to decide what
to do.

The function @string.words/1@ is used to break the resulting string along
contiguous sequences of whitespace into a list of strings.

Finally, we turn the list of words into a @set(string)@ using
@set.list_to_set/1@.

There's only one more thing to do: define @write_words_in_list/3@.
This is easy to do: if we have the empty list then we don't write
anything at all; otherwise we write out the first string in the list
(the \emph{head}) followed by a newline, and then repeat for the rest of
the list (the \emph{tail}):
\begin{myverbatim}
:- pred write_words_in_list(list(string), io, io).
:- mode write_words_in_list(in,           di, uo) is det.

write_words_in_list([],             !IO).

write_words_in_list([Word | Words], !IO) :-
    io.write_string(Word, !IO),
    io.nl(!IO),
    write_words_in_list(Words, !IO).
\end{myverbatim}
The functor @[]@ matches for the empty list, while @[Word | Words]@
unifies @Word@ and @Words@ with the head and tail respectively of a
non-empty list.  Since the first argument must be one or the other, our two
clauses constitute a complete switch on the first argument.

The recursive call to @write_words_in_list/3@ handles printing the words
in the tail of the non-empty list.  Recursion is the only mechanism used for
iteration in Mercury (unlike other languages with special constructs such as
@for@ and @while@ loops.)

Now we can wrap up the code into a module as usual (this time importing
@string@, @list@, @set@ and @exception@ in the implementation section)
and\ldots \emph{hey presto!}

\subsection{Handling Punctuation}

Our spelling checker is rather literal minded.  One issue is that any
immediately adjacent punctuation is assumed to be part of a word as
well.  For example,
\begin{myverbatim}
    string.words("Panic stations!") = ["Panic", "stations!"]
\end{myverbatim}
The dictionary file is unlikely to include @stations!@ as an entry.

We can change our program to treat anything that \emph{isn't} a letter
as a word separator by replacing the following line in
@file_to_word_set/4@
\begin{myverbatim}
    Words = string.words(String),
\end{myverbatim}
with
\begin{myverbatim}
    Words = string.words(isnt_a_letter, String),
\end{myverbatim}
and adding the following predicate to the program:
\begin{myverbatim}
:- pred isnt_a_letter(char).
:- mode isnt_a_letter(in  ) is semidet.

isnt_a_letter(Char) :-
    not char.is_alpha(Char).
\end{myverbatim}
This change introduces three new points: overloading, higher order
programming and negation.

\emph{Overloading} refers to using the same name for more than one thing.
Here we use @string.words/2@ rather than @string.words/1@.  The former takes
as its first argument a predicate that decides whether a particular
character is something that separates words.  We've
used a predicate of our own, @isnt_a_letter/1@ that does the job we
want (we'll come to @isnt_a_letter/1@ in a moment.)  Passing predicates as
arguments is part of \emph{higher order programming}.

The definition of @isnt_a_letter/1@ uses a new logical connective: @not@,
the mercury symbol for \emph{logical negation}.  The goal
@not char.is_alpha(Char)@ \emph{fails} if the goal @char.is_alpha(Char)@
succeeds (which it will if @Char@ is a letter) and \emph{succeeds} if
@char.is_alpha(Char)@ fails.

(Note that since we use a predicate from the @char@ module, we have to add
this to our list of imports at the top of the implementation section.)

\subsection{Handling Capitalisation}

Another problem with our spelling checker is that two strings that are
identical, except for capitalisation, do not compare the same -- so @"Dog"@
and @"dog"@ are considered different by our program.  An easy remedy for
this is to first convert every word to lower case.  To achieve this, all we
need do is change the following lines in @file_to_word_set/4@
\begin{myverbatim}
    Words   = string.words(String),
    WordSet = set.list_to_set(Words)
\end{myverbatim}
to
\begin{myverbatim}
    Words   = string.words(String),
    LCWords = words_to_lower_case(Words),
    WordSet = set.list_to_set(LCWords)
\end{myverbatim}
and add the function
\begin{myverbatim}
:- func words_to_lower_case(list(string)) = list(string).

words_to_lower_case([]) = [].

words_to_lower_case([Word | Words]) =
    [string.to_lower(Word) | words_to_lower_case(Words)].
\end{myverbatim}
where the function @string.to_lower/1@ returns its argument string with any
upper case letters converted to lower case.

The the list constructors, @[]@ and @[|]@, \emph{deconstruct} the function
argument in the clause heads, whereas in the body they happen to
\emph{construct} the result (it all depends upon whether the term being
unified with is an input or an output, respectively.)

The pattern used in @words_to_lower_case/1@ is so common that the @list@
module abstracts it with the function @list.map/2@.  A black belt Mercury
programmer probably wouldn't bother writing @words_to_lower_case/1@ at all,
but would say @list.map(string.to_lower, Words)@ instead -- it amounts to
exactly the same thing.  Don't worry if this seems a little confusing right
now.  The mysteries of higher order programming, such as they are, will be
made clear in chapter \XXX{}.

\subsection{Writing Lists the Easy Way}

Here's another example of how higher order programming can save time and
effort: writing out each item in a list is a common task.  The @io@ library
provides the predicate @io.write_list/5@ to save you the trouble of having
to write your own routine each time.

Instead of using @write_words_in_list/3@ in @main/2@, we could have written
\begin{myverbatim}
    io.write_list(Errors, "\n", io.write_string, !IO),
    io.nl(!IO)
\end{myverbatim}
This call to @io.write_list/5@ applies @io.write_string@ to each member of
@Errors@ (along with the @!IO@ pair), and writes out @"\n"@ \emph{between}
each call (so we need to add a call to @io.nl/2@ afterwards.)



\section{A Simple Data Base}

Our final example demonstrates how some predicates can have
multiple solutions and can even be used in more than one ``direction''.

To represent the information in the data base clearly, we need to introduce
some new types.  We define the type @gender@ which has \emph{data
constructors} (\ie possible values) @male@ and @female@.  We also
define @date@ with a single data constructor, @date/3@ (another
example of overloading), with three named fields for year, month and day.
And we add the type @name@ as a synonym for @string@:
\begin{myverbatim}
:- type gender ---> male ; female.
:- type date   ---> date(year :: int, month :: int, day ::int).
:- type name   ==   string.
\end{myverbatim}
Next we have a @person/3@ relation (another word for predicate) which
records against each person's name their gender and date of birth:
\begin{myverbatim}
:- pred person(name, gender, date).
:- mode person(in,   out,    out ) is semidet.
:- mode person(out,  out,    out ) is multi.

    % Great-grandparents.
    %
person("Edith Sorrel",      female, date(1897,  1,  3)).

    % Grandparents.
    %
person("George Becket",     male,   date(1925,  7, 18)).
person("Faith Becket",      female, date(1923,  2, 23)).
person("William Allen",     male,   date(1917, 11, 14)).
person("Gertrude Allen",    female, date(1921,  8, 29)).

    % Parents.
    %
person("William Becket",    male,   date(1945,  6, 12)).
person("Patricia Becket",   female, date(1943,  3,  3)).
person("Tom Allen",         male,   date(1944,  4, 30)).
person("Ina Allen",         female, date(1950, 12, 12)).

    % Most recent generation.
    %
person("Ralph Becket",      male,   date(1970, 12, 10)).
person("Kieran Becket",     male,   date(1972,  7, 17)).
person("Simon Allen",       male,   date(1970, 11, 14)).
person("Suzanne Allen",     female, date(1969,  4,  1)).
\end{myverbatim}
``What's going on here?'' I hear you cry, ``@person/3@ has \emph{two} mode
declarations!''  In Mercury it is perfectly reasonable for a predicate to
have more than one mode declaration, indicating that the predicate can be
called in more than one way.  The first mode declaration,
\begin{myverbatim}
:- mode person(in, out, out) is semidet.
\end{myverbatim}
says that a goal @person(Name, Gender, Date)@ for which @Name@ is an input
(\ie has already been bound) is semi-deterministic: there is at most one
solution (\ie the goal \emph{may} fail) instantiating @Gender@ and @Date@.
On the other hand,
\begin{myverbatim}
:- mode person(out, out, out) is multi.
\end{myverbatim}
says that a goal @person(Name, Gender, Date)@ for which none of the
arguments are inputs has \emph{one or more} solutions instantiating @Name@,
@Gender@ and @Date@.  This is what the @multi@ determinism category stands
for.  We'll come to what happens with predicates that have have more than
one solution shortly.

Moving on to the other key relation in our data base, @is_child_of/2@
records which person is a child of which other person:
\begin{myverbatim}
:- pred name `is_child_of` name.
:- mode in   `is_child_of` in  is semidet.
:- mode in   `is_child_of` out is nondet.
:- mode out  `is_child_of` in  is nondet.

"Faith Becket"      `is_child_of`   "Edith Sorrel".
"William Becket"    `is_child_of`   "George Becket".
"William Becket"    `is_child_of`   "Faith Becket".
"Patricia Becket"   `is_child_of`   "William Allen".
"Patricia Becket"   `is_child_of`   "Gertrude Allen".
"Tom Allen"         `is_child_of`   "William Allen".
"Tom Allen"         `is_child_of`   "Gertrude Allen".
"Ralph Becket"      `is_child_of`   "William Becket".
"Ralph Becket"      `is_child_of`   "Patricia Becket".
"Kieran Becket"     `is_child_of`   "William Becket".
"Kieran Becket"     `is_child_of`   "Patricia Becket".
"Simon Allen"       `is_child_of`   "Tom Allen".
"Simon Allen"       `is_child_of`   "Ina Allen".
"Suzanne Allen"     `is_child_of`   "Tom Allen".
"Suzanne Allen"     `is_child_of`   "Ina Allen".
\end{myverbatim}
\XXX{Change names etc.  It's part of my family tree with wild guesses about
most of the dates.}
The first mode declaration states that given inputs @Child@ and @Parent@,
the goal @Child `is_child_of` Parent@ will either succeed or fail -- that
is, it can be used to test whether one person is the offspring of another.

The second mode declaration states that a goal
@Child `is_child_of` Parent@, for which @Child@ is an input, is
non-deterministic (hence the @nondet@ determinism category).  In other
words, the goal may fail or have one or more solutions for @Parent@.
The goal,
\begin{myverbatim}
    "Edith Sorrel" `is_child_of` Parent
\end{myverbatim}
for instance, has no solutions.  The goal
\begin{myverbatim}
    "Faith Becket" `is_child_of` Parent
\end{myverbatim}
has one solution, namely
\begin{myverbatim}
    Parent = "Edith Sorrel"
\end{myverbatim}
while
\begin{myverbatim}
    "William Becket" `is_child_of` Parent
\end{myverbatim}
has both
\begin{myverbatim}
    Parent = "George Becket"
\end{myverbatim}
and
\begin{myverbatim}
    Parent = "Faith Becket"
\end{myverbatim}
as solutions.

The third mode declaration states that a goal @Child `is_child_of` Parent@,
in the case where @Parent@ is an input, is similarly non-deterministic.

Now let's introduce some new relations (if you'll pardon the pun).  The
simplest one, @is_parent_of/2@, is just the converse of @is_child_of/2@:
\begin{myverbatim}
:- pred name `is_parent_of` name.
:- mode in   `is_parent_of` in  is semidet.
:- mode in   `is_parent_of` out is nondet.
:- mode out  `is_parent_of` in  is nondet.

Parent `is_parent_of` Child :- Child `is_child_of` Parent.
\end{myverbatim}
That was easy.  What about mothers and fathers?
\begin{myverbatim}
:- pred name `is_mother_of` name.
:- mode in   `is_mother_of` in  is semidet.
:- mode in   `is_mother_of` out is nondet.
:- mode out  `is_mother_of` in  is nondet.

Mother `is_mother_of` Child :-
    Mother `is_parent_of` Child,
    person(Mother, Gender, _),
    Gender = female.
\end{myverbatim}
(Recall that an underscore @_@ on its own means ``don't care''.) So @Mother@
is the mother of @Child@ if @Mother@ is a parent of @Child@ and the
@person/3@ record for @Mother@ lists @Mother@ as female.

The careful reader will be wondering about the third mode declaration for
@is_mother_of/2@ -- surely each @Child@ can have no more than one mother,
making the correct mode @semidet@ rather than @nondet@.  While this is true
for the genealogical domain, Mercury has no idea that this is actually a
genealogical data base and that this is therefore a necessary domain
constraint.  Rather, it reasons that if @Child@ is an input, then the call
to @is_parent_of/2@ can have more than one possible solution for @Mother@,
hence @is_mother_of/2@ may have more than one possible solution.  (There
\emph{is} a way to express this domain constraint, but doing so requires
black belt level techniques which we aren't going to get into this early in
the book.)

The code for @is_father_of/2@ is very similar:
\begin{myverbatim}
:- pred name `is_father_of` name.
:- mode in   `is_father_of` in  is semidet.
:- mode in   `is_father_of` out is nondet.
:- mode out  `is_father_of` in  is nondet.

Father `is_father_of` Child :-
    Father `is_parent_of` Child,
    person(Father, Gender, _),
    Gender = male.
\end{myverbatim}
Now, Mercury will replace a predicate call with a non-variable term as an
argument with one in which the argument \emph{is} a variable and add a
unification goal connecting the new variable and the term.  That is, a call
\begin{myverbatim}
    p(<expr1>, <expr2>, ...)
\end{myverbatim}
is transparently converted to
\begin{myverbatim}
    X1 = <expr1>,
    X2 = <expr2>,
    ...
    p(X1, X2, ...)
\end{myverbatim}
(and a similar thing happens for goals containing function applications.)
This means we can rewrite the clause for @is_father_of/2@ (and similarly for
@is_mother_of/2@) as
\begin{myverbatim}
Father `is_father_of` Child :-
    Father `is_parent_of` Child,
    person(Father, male, _).
\end{myverbatim}
which is easier to both read and maintain.

Let's return briefly to the question of what it means for a predicate
to have more than one possible solution.  Consider the goal
\begin{myverbatim}
    Mother `is_mother_of` "Ralph Becket"
\end{myverbatim}
The definition of @is_mother_of/2@ says we must therefore find a solution to
\begin{myverbatim}
    Mother `is_parent_of` "Ralph Becket",
    person(Mother, female, _)
\end{myverbatim}
and hence, from the definition of @is_parent_of/2@, a solution to
\begin{myverbatim}
    "Ralph Becket" `is_child_of` Mother,
    person(Mother, female, _)
\end{myverbatim}
Mercury works its way through the clauses of @is_child_of/2@ looking for a
solution to the first subgoal.  The first one it comes
to is
\begin{myverbatim}
    Mother = "William Becket"
\end{myverbatim}
Execution proceeds to the second subgoal,
\begin{myverbatim}
    person("William Becket", female, _)
\end{myverbatim}
Since there are no matching clauses for this goal, execution
\emph{backtracks} (or unwinds) to the
last non-deterministic goal (the call to @is_child_of/2@) to look for a
different solution.  The next solution to the @is_child_of/2@ subgoal is
\begin{myverbatim}
    Mother = "Patricia Becket"
\end{myverbatim}
As before, execution proceeds to the second subgoal
\begin{myverbatim}
    person("Patricia Becket", female, _)
\end{myverbatim}
which this time \emph{does} have a matching clause.  So now we know that
\begin{myverbatim}
    Mother = "Patricia Becket"
\end{myverbatim}
is a solution to the goal
\begin{myverbatim}
    Mother `is_mother_of` "Ralph Becket"
\end{myverbatim}
Hurrah!

To sum up, Mercury's execution model for non-deterministic
predicates is to search systematically through all possibilities while
looking for solutions, backtracking to the most recent non-exhausted
non-deterministic \emph{choice point} whenever a subgoal fails.  One final
word on the subject: Mercury is allowed to reorder the goals in a
conjunction as necessary to ensure mode correctness (\ie before a variable
can be used as an input it must be instantiated either from a unification or
as an output argument to a call.)

Now that we have a feeling for how non-determinism works, let's finish up by
writing down some more interesting relations.  To save space we won't bother
including the mode declarations (working out the possible modes for each is
left as a light exercise for the reader).

First, a grandparent is defined as the parent of a parent:
\begin{myverbatim}
Grandparent `is_grandparent_of` Child :-
    Parent `is_parent_of` Child,
    Grandparent `is_parent_of` Parent.
\end{myverbatim}
More generally, an ancestor of a person is \emph{either} a parent 
\emph{or} an ancestor of a parent of the person:
\begin{myverbatim}
Ancestor `is_ancestor_of` Person :-
    Ancestor `is_parent_of` Person.

Ancestor `is_ancestor_of` Person :-
    Parent `is_parent_of` Person,
    Ancestor `is_ancestor_of` Parent.
\end{myverbatim}
Two people are siblings if they have a parent in common and are different
(one isn't usually considered to be one's own sibling):
\begin{myverbatim}
SiblingA `is_sibling_of` SiblingB :-
    Parent `is_parent_of` SiblingA,
    Parent `is_parent_of` SiblingB,
    SiblingA \= SiblingB.
\end{myverbatim}
The goal @SiblingA \= SiblingB@ is syntactic sugar for
@not SiblingA = SiblingB@.

Two people are cousins if their parents are siblings:
\begin{myverbatim}
CousinA `is_cousin_of` CousinB :-
    ParentA `is_parent_of` CousinA,
    ParentB `is_parent_of` CousinB,
    ParentA `is_sibling_of` ParentB.
\end{myverbatim}
Here's how we define niece-or-nephewdom:
\begin{myverbatim}
NieceOrNephew `is_niece_or_nephew_of` AuntOrUncle :-
    Parent `is_sibling_of` AuntOrUncle,
    NieceOrNephew `is_child_of` Parent.
\end{myverbatim}
It's all remarkably straightforward!

\XXX{Are there too many examples here?  Do they need more explanation?
What could I do with the date information?  What about getting the solutions
to these predicates?}

\XXX{I've ditched the maze searching example as too complex to do right this
early in the book.}

\XXX{Overuse of ``simple'', ``much''...}

% \section{Declarative vs Imperative Programming}

\subsection{Definitions}

\begin{description}
\item{Imperative:} a sequence of instructions for transforming state.
\item{Declarative:} a specification of \emph{what} is to be computed.
\end{description}

Declarative languages typically have a simple translation into
conventional mathematical logic, which is how the meaning of a program
is defined.

Purely declarative programming languages exhibit referential
transparency.  In a nutshell, this means that anywhere you see a
reference to a name, you can replace it with the body of the
definition of the name and it will make no difference.

In more formal language,
\[
(\text{let}\ x = e\ \text{in}\ M)  \equiv  M[e/x]
\]
This is clearly not true of imperative languages.  Consider
the following C program:

\begin{verbatim}
int g = 0;

int f(int x)
{
    g = g + 1;
    return x + g;
}

void main(int argc, char **argv)
{
    int tmp = f(1);
    int a   = tmp  + tmp;
    int b   = f(1) + f(1);

    if(a == b)
        printf("equivalent\n");
    else
        printf("not equivalent\n");
}
\end{verbatim}

According to C semantics, this program should print out "not
equivalent", proving that the expression @f(1)@ is not equal to
itself!  This sort of thing is great for writing buggy, hard to
maintain code.\footnote{One can give a reasonably simple
operational semantics to C whereby we repeatedly substitute the
body of @f(1)@ into a sequence of \emph{instructions}, but as we
see here, this does not necessarily support the more intuitive,
declarative reading one might hope for.}

Since referential transparency means no side-effects, you can't have
variables that change their state as the program evolves.  Instead, the
term \emph{variable} in a declarative programming language refers to a
label given to a value or the result of a computation.

The nearest Mercury equivalent to the function @f()@ in the C
program above is

\begin{verbatim}
:- pred f(int, int, int, int).
:- mode f(in,  out, in,  out) is det.

f(X, Result, Old_Value_of_G, New_Value_of_G) :-
    New_Value_of_G = Old_Value_of_G + 1,
    Result         = New_Value_of_G + X.
\end{verbatim}

Since there are no variables in Mercury (and hence no global
variables), any ``global'' state has to be explicitly passed
around wherever it is needed.  In this case @f/4@ takes the old
``state'' as its third argument and returns the new ``state'' in
its fourth argument.

\subsection{Benefits Of Declarative Style}

While the lack of mutable state might seem like a serious
drawback to programmers raised on imperative languages, in
practice it turns out to be something of a boon.  Experienced
imperative programmers acknowledge that fewer globals and less
state means clearer, more maintainable, more reusable code
with fewer bugs.

The philosophy underlying declarative languages is to simplify the
\emph{writing} of bug free programs and to leave the tedious
business of identifying programming errors, run-time book
keeping such as memory management, and non-algorithmic
optimization to the compiler.  Referential transparency means
that more optimizations can be applied in more places in a
program than is the case with imperative programs, simply
because proving that it is safe to apply a particular
optimization is so much easier in the absence of side effects.

Mercury was designed as a purely declarative, industrial
strength programming language aimed at the rapid development
of medium and large scale systems with the emphasis on
producing fast, correct programs.

To this end, Mercury doesn't support a corner-cutting
programming style: you \emph{have} to get the types right, check
return codes and so forth -- the quick-and-dirty fix is rarely
an option.\footnote{Mercury does have support for impure code
via calls to another language, but one has to label each such
call explicitly; in the end it is often easier just to do the
right thing.}  Very often, Mercury programmers find that once
a program is accepted by the compiler, it also does exactly
what was intended; in the author's long experience this almost
never happens with imperative languages, even for small
programs.

\XXX{Place to mention types, garbage collection, polymorphism,
pattern matching, etc etc?}

\subsection{Pragmatism}

\begin{quote}
``A foolish consistency is the hobgoblin of small minds.'' \\
\hfill --- Ralph Waldo Emmerson
\end{quote}

Purity is all very well, but the fact remains that
occasionally one has to interoperate with code written in
impure languages and that (very rarely and usually when that
last drop of speed is required) some low-level algorithms may
be best expressed as impure constructs.

For the former, Mercury has a simple and well developed
foreign language interface allowing the programmer to write
foreign code in-line with the Mercury program (provided the
Mercury compiler has an appropriate back-end for the foreign
language in question -- the alternative is to use C as the
lingua franca in the traditional style.)  It is the
programmer's responsiblility to supply the appropriate purity
declarations for predicates defined in terms of foreign code.

The latter is handled using purity annotations.  Impure code
(written in a foreign language) must be labelled as such.
These labels also have to be applied to all predicates that
use the impure code.  At some point we hope that a pure
interface can be presented to the programmer, in which case
the program must include a \emph{promise} to the compiler that
impurity annotations are not required for users of the
top-level predicate with an impure definition.

\subsection{Mercury Philosophy}

\XXX{I think I've already covered this one.  Tyson suggests an
implementation philosophy section (\eg no distributed fat) in
a much later section.}




% % vim: ft=tex ff=unix ts=4 sw=4 et wm=8 tw=0

\chapter{Logic and Logic Programming}

\XXX{Convert all logic to tt and Mercury syntax.  Add comments in the
intro to make it clear what it would look like in a logic text book.}

While programming in a purely functional style is a fairly intuitive
notion, the notion of programming in logic might at first seem a little
odd.  Nevertheless, in this section we shall demonstrate that there is
an easily understood logic-based programming paradigm.  A firm grasp of
logic is a great help in understanding Mercury, so we start with a
refresher course on basic logic.

\section{The Abstract Stuff}

To quote Douglas Adams, \emph{Don't Panic!}

\subsection{What's the Point?}

Logics help us to reason about the world.  Ordinary, natural language is
generally too verbose and too imprecise to use effectively in complex
situations.  When thinking about computer programs the problem is
exacerbated because computers take everything we say to them literally
and they have absolutely no imagination.

A logic allows us to say simple things clearly and concisely and gives
us an unambiguous mechanism for deducing more complex things thereafter.

The simplicity of logic can be deceptive in two ways.  The first is it
often seems \emph{too} simple to be of any use.  The second is that one
is often tempted to say of a particular statement, ``Why do we need
logic here?  This is clearly correct.''  The answer in both cases is
that you'd be amazed at how often these points of view are wrong.

Finally, when we use logic to help us think about and solve problems on
paper, we can very often turn that effort directly into a reliable,
working computer program with the minimum of effort.

\subsection{What Is a Logic?}

In the most abstract sense, a \emph{logic} consists of
\being{itemize}
\item a \emph{language} which specifies the set of well formed
\emph{sentences}, and
\item a set of \emph{inference rules} which allow us to deduce new
sentences from a given set of sentences.
\end{itemize}
A sentence is often also referred to as a \emph{formula}.  We will use
both terms interchangably throughout the book.

\subsection{What is a Theory?}

A theory is the set of sentences that can be obtained with respect to a
particular logic where each sentence is either
\begin{itemize}
\item a member of a particular starting set of \emph{axioms}, which are
taken to be ``true without proof'', or
\item can be \emph{proved} from other sentences in the theory.
\end{itemize}
A sentence in a theory is referred to as a \emph{theorem}.

\subsection{What is a Proof?}

A \emph{proof} is a sequence of sentences where each step is either
\begin{itemize}
\item an axiom or
\item a sentence obtained by application of an inference rule to one or
more of the preceding sentences in the proof.
\end{itemize}

\subsection{What is an Interpretation?}

An \emph{interpretation} can be viewed as the set of sentences that are
true in a particular problem domain, all other sentences being deemed to
be false.  One way of looking at an interpretaion is as a way of
relating sentences in the logic to things in the problem domain.

\subsection{Consistency, Completeness and Correctness}

A theory is \emph{consistent} if it does not contain any mutually
contradictory theorems.

A theory is \emph{complete} with respect to an interpretation if the
interpretation is a subset of the theory (otherwise the interpretation
contains some sentences that are true but are not part of the theory.)

A theory is \emph{correct} with respect to an interpretation if it is a
subset of the interpretation (otherwise the theory claims some theorems
are true that are in fact false under the given interpretation.)

\subsection{Putting it All Together}

So, for any particular problem, we need the following if we are going to
reason about it logically:
\begin{itemize}
\item a language defining the sentences we can consider;
\item a set of rules allowing us to deduce new sentences from old;
\item a set of axioms defining our starting point;
\item an interpretation connecting the results of our abstract, logical
efforts to the problem domain we are interested in.
\end{itemize}

\section{Propositional Logic}

The simplest useful logic is the \emph{propositional calculus}.
The propositional calculus is used to reason about logical formulae
composed of atomic propositions -- that is, simple statements that are
either true or false.

In what follows we use @P@ and @Q@ to stand for arbitrary sentences.

\subsection{Language}

\begin{itemize}
\item We use @word@s or letters, @a@, @b@, @c@, \ldots to stand for
\emph{atomic} propositions;
\item we usually include @true@ as a proposition that is true in all
interpretations and @false@ as a proposition that is false in all
interpretations;
\item @not P@ stands for the \emph{negation} ``@P@ is false'';
\item @(P, Q)@ stands for the \emph{conjunction} ``@P@ and @Q@'' meaning
\emph{both} @P@ and @Q@ are true;
\item @(P ; Q)@ stands for the \emph{disjunction} ``@P@ or @Q@'' meaning
\emph{at least one of} @P@ and @Q@ is true;
\item @(P => Q)@ stands for the \emph{implication} ``if @P@ then @Q@''
meaning if @P@ is true, then @Q@ must be true (it says nothing about @Q@
in the case where @P@ is false).  @P@ is referred to as the
\emph{antecedent} and @Q@ as the \emph{consequent} of the implication.
\end{itemize}

We may occasionally use @(Q <= P)@ as a convenient notation for
@(P => Q)@ and @(P <=> Q)@ for @((P => Q), (P <= Q))@ (the latter type
of formula is called an \emph{equivalence}).

\textbf{Note on syntax:} we use Mercury style syntax throughout this
chapter.  Books on logic will tend to use slightly different symbols for
the logical connectives:
\begin{itemize}
\item @not P@ is conventionally written as $\Not$@P@;
\item @(P, Q)@ is conventionally written as @P@$\Conj$@Q@;
\item @(P ; Q)@ is conventionally written as @P@$\Disj$@Q@;
\item @(P => Q)@ is conventionally written as @P@$\Imp$@Q@.

\subsection{Rules of Inference}

We present rules of inference using \emph{sequent notation}.  Sequents
can be read thus: given sentences matching what appears above the line,
we may deduce the sentence below the line.

We can always infer @true@, which is the negation of @false@.
The double negation of a proposition is the same as asserting that
proposition.
\begin{verbatim}
                                not not P       P
-----           ----------      ----------      ----------
true            not false       P               not not P
\end{verbatim}
(If we omit the double negation rule then we have an intuitionistic
logic.)

We can form the conjunction of any two true propositions.
We can reorder conjunctions.
We can project a conjunct from a conjunction.
Conjunction with any false proposition is also false.
\begin{verbatim}
P
Q               P, Q            P, Q            not P
-----           -----           -----           ----------
P, Q            Q, P            P               not (P, Q)
\end{verbatim}
Conjunction is therefore commutative and associative.

We can form a disjunction of any other proposition with a true
proposition.
We can reorder disjunctions.
If a disjunction is false then so are its disjuncts.
\begin{verbatim}
                                                not P
P               P ; Q           not (P ; Q)     not Q
------          ------          ------------    ------------
P ; Q           Q ; P           not P           not (P ; Q)
\end{verbatim}
Disjunction is therefore commutative and associative.

If the antecedent of an implication is true then the consequent must
also be true (\emph{modus ponens}).
If the consequent of an implication is false then the antecedent must
also be false (\emph{modus tollens}).
We can introduce an implication using any proposition at all for a true
consequent or a false antecedent:
\begin{verbatim}
P               not Q
P => Q          P => Q          Q               not P
-------         -------         -------         -------
Q               not P           P => Q          P => Q
\end{verbatim}

\subsection{On Implication}

The rules for implication might seem a little odd.  One should be
careful to avoid confusing logical implication with causation.  With
causation, the consequent must follow ``because'' of the antecedent.
Logical implication, on the other hand, need not make any kind sense.
It is perfectly all right to put any proposition at all on either side
of a logical implication; whether the implication is justified or not is
another matter entirely and is dictated by its context within a theory
or proof.

Another common source of confusion is that implication says nothing
about what happens if the antecedent is false.  The idea is that if we
have @(P => Q)@ and also @not P@ then the implication isn't
``triggered'' and hence cannot make any false claim.  The only situation
where an implication @(P => Q)@ is false is if @P@ is true, but @Q@ is
\emph{false}.  In this case the implication is ``triggered'', but
contradicts @not Q@ which has already been established.  Consequently,
if we have @(P => Q)@ and we know that @Q@ is false then the only
consistent deduction is that @P@ is also false.

\subsection{Some Basic Proofs}

It is instructive to see some simple proofs.  When constructing a proof
we work bottom-up: we start with the theorem we wish to prove and then
use the inference rules in reverse to get back to whatever we've allowed
ourselves to take as given.

First off, let's show that from @(false ; P)@ we can infer @P@:
\begin{verbatim}
 1      (false ; P)             given

 2      not false               from the negation rules
 3      P                       resolution of 2 and 1
\end{verbatim}

Next, let's show that @not (false, P)@ is always true:
\begin{verbatim}
 1      not false               from the negation rules
 2      not (false, P)          from the conjunction rules and 1
\end{verbatim}

The following relationship is extremely useful: @(P => Q)@ is equivalent
to @(not P ; Q)@.  We prove this in two stages, first starting with the
assumption @(not P ; Q)@.
Since we start with a disjunction, it is sufficient to show that if each
disjunct independently supports the conclusion then the disjunction as a
whole must support the conclusion.
\begin{verbatim}
 1      (not P ; Q)             given

 2.L    not P                   assuming the left hand side of 1
 3.L    (P => Q)                from the implication rules and 2.L

 2.R    Q                       assuming the right hand side of 1
 3.R    (P => Q)                from the implication rules and 2.R

 4      (P => Q)                from 1, 3.L and 3.R
\end{verbatim}
To finish, we start with the assumption @(P => Q)@.  Since we start with
an implication, it is sufficient to show that we can prove @(not P ; Q)@
if either the antecedent is false or the consequent is true.
\begin{verbatim}
 1      (P => Q)                given

 2.L    not P                   assuming the antecedent of 1 is false
 3.L    (not P ; Q)             from the disjunction rules and 2.L

 2.R    Q                       assuming the consequent of 1
 3.R    (not P ; Q)             from the disjunction rules and 2.R

 4      (not P ; Q)             from 1, 3.L and 3.R
\end{verbatim}

Augustus De Morgan, the 19th century logician, showed that @not (P, Q)@
is equivalent to @(not P ; not Q)@, which is the same as saying
@not (P ; Q)@ is equivalent to @(not P, not Q)@.  We prove this in two
stages, first starting with the assumption @not (P, Q)@.
\begin{verbatim}
 1      not (P, Q)              given

 2.L    not P                   assuming the left hand side of 1 is false
 3.L    (not P ; not Q)         from the disjunction rules and 2.L

 2.R    not P                   assuming the right hand side of 1 is false
 3.R    (not P ; not Q)         from the disjunction rules and 2.R

 4      (not P ; not Q)         from 1, 3.L and 3.R
\end{verbatim}
Going the other way, we assume @not (P ; Q)@.
\begin{verbatim}
 1      not (P ; Q)             given

 2      not P                   from the disjunction rules and 1
 3      not Q                   from the disjunction rules and 1
 4      (not P, not Q)          from the conjunction of 2 and 3
\end{verbatim}

The \emph{resolution} rule for disjunctions says that if one disjunct is
false then the other must be true:
\begin{verbatim}
 1      (P ; Q)                 given
 2      not P                   given

 3      (not P => Q)            equivalent to 1
 4      Q                       modus ponens over 2 and 3
\end{verbatim}
As a special case of this rule we see that @P@ must follow from
@(false ; P)@.

Aristotle's \emph{law of the excluded middle} states that the
disjunction of a proposition with itself must be true:
\begin{verbatim}
 1      (P ; not P)             given

 2      true                    trivial
 3      (P => true)             from the implication rules and 2
 4      (not P => true)         from the implication rules and 2

 5.L    P                       assuming the left hand side of 1
 6.L    true                    modus ponens over 5.L and 3

 5.R    not P                   assuming the right hand side of 1
 6.R    true                    modus ponens over 5.R and 4

 7      true                    from 1, 6.L and 6.R
\end{verbatim}

Implication is transitive:
\begin{verbatim}
 1      (P => Q)                given
 2      (Q => R)                given

 3.L    not P                   assuming antecedent of 1 is false
 4.L    (P => R)                from implication rules and 3.L

 3.R    Q                       assuming consequent of 1 is true
 4.R    R                       modus ponens over 3.R and 2
 5.R    (P => R)                from implication rules and 4.R

 6      (P => R)                from 1, 2, 4.L and 5.R
\end{verbatim}

\subsection{Cavilling Vilely}

\XXX{Check this isn't copyright!  It's been doing the Cambridge Tripos
rounds for years, at least.}

Consider the following somewhat contorted problem statement:
\begin{quote}
If Anna can cancan or Kant can't cant, then Greville will cavil vilely.
If Greville will cavil vilely, Will will want.
But Will \emph{won't} want.
So can Kant cant?
\end{quote}
How can we use propositional logic to decide the truth of the question?
Here's how.  We start by labelling the various propositions:
\begin{description}
\item let @a@ stand for ``Anna can cancan'';
\item let @k@ stand for ``Kant can cant'';
\item let @g@ stand for ``Greville will cavil vilely'';
\item and let @w@ stand for ``Will will want''.
\end{description}

Our axioms are obtained by translating the statement of the problem into
sentences in propositional logic:
\begin{verbatim}
 1      ((a ; not k) => g)      given
 2      (g => w)                given
 3      not w                   given
\end{verbatim}
Our \emph{goal} is to find a proof of either @k@ or @not k@; we can
obtain a proof of the former as follows (and since the problem statement
and logic are consistent, this means we can't prove the latter):
\begin{verbatim}
 4      not g                   from the implication rules and 3 and 2
 5      not (a ; not k)         from the implication rules and 4 and 1
 6      (not a, k)              De Morgan's law applied to 5
 7      k                       projecting the right hand side of 6
\end{verbatim}
So Kant \emph{can} cant after all!

\subsection{On Obtaining Proofs}

Finding a proof or disproof of a formula is not straightforward.  Since
this book only introduces logic, an in-depth treatment of proof
procedures is beyond its scope.  However, it behooves us to say a few
words on the subject.

The key idea in obtaining a proof is to work backwards from the goal.
One has to look at the inference rules and the axioms of the problem
domain and search for steps that will break down the goal into simpler
subgoals.  This procedure continues until the subgoals are either
trivial (deducible without axioms) or are themselves in the set of
axioms.  The tricky part is that one may need to handle disjunctions (it
is sometimes simpler to convert implications into their equivalent
disjunctive forms) by proving that the goal is provable regardless of
which disjunct is true.  Finding proofs is not easy and takes some
practice before one becomes proficient.

Note that if you don't \emph{know} that a particular formula is true or
false then one has to simultaneously look for proofs of both
possibilities: in a consistent theory you cannot find a disproof of a
theorem or a proof of a non-theorem!

While complete proof algorithms exist for deciding the truth of
arbitrary formulae in the propositional calculus (see the
\XXX{Davis-Putnam method}, for instance), unfortunately its more
expressive cousin, the predicate calculus, does not admit this property
(that is, there is no way to construct an algorithm to decide whether an
arbitrary predicate calculus formula is true or false.)  Mercury can be
viewed as implementing a correct, but incomplete, proof procedure for
first order logic.  But don't worry about that for now.

\subsection{Informal Shortcuts}

We often want to sketch out a proof quickly before writing it down in a
formal fashion (if, indeed, we decide we need this level of rigour.)  In
such situations it is acceptable to take a ``term rewriting'' approach:
given @P <=> Q@ we can always substitute @P@ for @Q@ and vice versa;
given @P => Q@ we may substitute @P@ for @Q@ in the antecedent of
implications and @Q@ for @P@ in the consequent of implications.  Below
is a handy selection of ``rewriting'' rules:
\begin{verbatim}
(P => Q)             <=>  (not Q => not P)
(P => Q)             <=>  (not P ; Q)
((P, Q) => R)        <=>  (P => (Q => R))

not true             <=>  false
not false            <=>  true
not not P            <=>  P

(P, not P)           <=>  false
(P, false)           <=>  false
(P, true)            <=>  P

(P ; not P)          <=>  true
(P ; true)           <=>  true
(P ; false)          <=>  P

((P => Q), (Q => R))  =>  (P => R)
(P, Q)                =>  P
(not P, (P, Q))       =>  Q
\end{verbatim}

\subsection{An Interesting Observation}

The reader may find it interesting to note that one can construct all
the rules of inference of propositional logic using just implication and
falsity:
\begin{verbatim}
not P                <=>  (P => false)
(P ; Q)              <=>  (not P => Q)
(P, Q)               <=>  ((P => not Q) => false)
\end{verbatim}

\section{Predicate Logic}

The expressive power of propositional logic is rather limited.  The key
problem is that it makes no provision for making general statements such
as, ``The square of an odd number is also an odd number.''  Instead one
is forced into writing down an (in this case) infinite number of
propositions of the form ``$1$ is an odd number'', ``$1^2$ is an odd
number'', ``$3$ is an odd number'', ``$3^2$ is an odd number'' and so
forth.  The relationship between odd numbers and this property of their
squares is not explicitly represented, other than as a correspondence
amongst the set of axioms.

Mercury programs are restricted predicate calculus theories; it is
actually very helpful to bear this in mind when reasoning about them.

\subsection{Language}

The language of predicate calculus is slightly more complicated than
that for propositional logic, introducing as it does logical variables,
terms and parameterised predicates.
\begin{itemize}
\item \emph{Terms} represent objects in the problem domain, such as
numbers, people, and so forth.  Terms can be structured -- a list of
terms would itself be a perfectly good term.
\item \emph{Logical variables} range over terms and appear as parameters
to predicates.
\item \emph{Predicates} represent properties of terms.  Unlike
propositions, predicates may be parameterized.  For example, we might
have a predicate @odd(X)@ to denote @X@ being an odd number, or a
predicate @parent(A, B)@ to denote @A@ being a parent of @B@.  We
normally assume the usual equality relation, @=@.
\end{itemize}
Sticking with Mercury syntax, predicates and term names start with
@lower case@ letters (unless we are using symbols for conveience, such
as @123@ or @*@ or list notation and such like.)  Variable names start
with @Upper Case@ letters.

Formulae are composed in much the same way as for the propositional
calculus, with the addition of two ways of introducing logical
variables.  Taking @P@ and @Q@ to stand for arbitrary logical formulae:
\begin{itemize}
\item @not P@ stands for the \emph{negation} ``@P@ is false'';
\item @(P, Q)@ stands for the \emph{conjunction} ``@P@ and @Q@'' meaning
\emph{both} @P@ and @Q@ are true;
\item @(P ; Q)@ stands for the \emph{disjunction} ``@P@ or @Q@'' meaning
\emph{at least one of} @P@ and @Q@ is true;
\item @(P => Q)@ stands for the \emph{implication} ``if @P@ then @Q@'';
\item @(some [X] P)@ stands for the \emph{existential
quantification} ``for some value of @X@ @P@'';
\item @(all [X] P)@ stands for the \emph{universal
quantification} ``for all values of @X@ @P@'';
\end{itemize}

We allow several variables to appear in the list of a quantification
with the understanding that
\begin{verbatim}
(some [X, Y, Z] P)
\end{verbatim}
really means
\begin{verbatim}
(some [X] (some [Y] (some [Z] P)))
\end{verbatim}

We will allow ourselves to use a pattern matching shorthand in
definitions (@<=>@) and implications (@=>@ and @<=@), so rather than
writing
\begin{verbatim}
(all [X] (p(X) <= X = 42))
\end{verbatim}
we simply write
\begin{verbatim}
p(42)
\end{verbatim}
to mean the same thing.

Returning briefly to our ``square numbers'' example, we would write that
as
\begin{verbatim}
(all [X, Y] ((odd(X), square(X, Y)) => odd(Y)))
\end{verbatim}
which reads as ``for all values of @X@ and @Y@, if @X@ is odd and @Y@ is
the square of @X@, then @Y@ is odd.''

\subsection{Rules of Inference}

For the most part, the rules of inference for the predicate calculus
appear identical to those for the propositional calculus.  The important
difference is the rules that dictate how variables are handled.

We can always infer @true@, which is the negation of @false@.
The double negation of a proposition is the same as asserting that
proposition.
\begin{verbatim}
                                not not P       P
-----           ----------      ----------      ----------
true            not false       P               not not P
\end{verbatim}
(If we omit the double negation rule then we have an intuitionistic
logic.)

We can form the conjunction of any two true propositions.
We can reorder conjunctions.
We can project a conjunct from a conjunction.
Conjunction with any false proposition is also false.
\begin{verbatim}
P
Q               P, Q            P, Q            not P
-----           -----           -----           ----------
P, Q            Q, P            P               not (P, Q)
\end{verbatim}

We can form a disjunction of any other proposition with a true
proposition.
We can reorder disjunctions.
If a disjunction is false then so are its disjuncts.
\begin{verbatim}
                                                not P
P               P ; Q           not (P ; Q)     not Q
------          ------          ------------    ------------
P ; Q           Q ; P           not P           not (P ; Q)
\end{verbatim}

If the antecedent of an implication is true then the consequent must
also be true (\emph{modus ponens}).
If the consequent of an implication is false then the antecedent must
also be false (\emph{modus tollens}).
We can introduce an implication using any proposition at all for a true
consequent or a false antecedent:
\begin{verbatim}
P               not Q
P => Q          P => Q          Q               not P
-------         -------         -------         -------
Q               not P           P => Q          P => Q
\end{verbatim}

We can substitute any term at all for a universally quantified
variable.
The dual of universal quantification is existential quantification.
\begin{verbatim}
all [X] P       some [X] P
----------      --------------------
P[a/X]          not (all [X] not P)

P[a/X]          not (all [X] not P)
-----------     --------------------
some [X] P      some [X] P
\end{verbatim}
where @P[a/X]@ means @P@ with the term @a@ replacing all free
occurrences of @X@ therein.  A free variable is one not in the scope of
any enclosing quantification.  For instance, @X@ is free in @odd(X)@,
but not in @some [X] odd(X)@.

\XXX{Check the above sequents.  I think there's a bug.}

\subsection{Finding the Length of a List}

Here's an example of how we can use logic to work out the length of a
list.
\begin{verbatim}
all [X, Y]
    length(Xs, N) <=>
        (
            (Xs = [], N = 0)
        ;
            some [X0, Xs0, N0]
                (   Xs = [X0 | Xs0],
                    length(Xs0, N0),
                    N  = N0 + 1
                )
        )
\end{verbatim}
which, incidentally, corresponds to the Mercury definition
\begin{verbatim}
length([],         0     ).
length([X0 | Xs0], N0 + 1) :- length(Xs0, N0).
\end{verbatim}

First, let's see how we can prove @length([a, b], 2)@ (remember, we
construct this proof bottom-up):
\begin{verbatim}
 1      [] = []
 2      0  = 0
 3      ([] = [], 0 = 0)
 4      (([] = [], 0 = 0) ; ...)
 5      length([], 0)

 6      [b]    = [b | []]
 7      0 + 1  = 0 + 1
 8      ([b] = [b | []], length([], 0), 0 + 1 = 0 + 1)
 9      some [X0, Xs0, N0] ([b] = [X0 | Xs0], ...)
   ...
10      length([b], 0 + 1)

11      [a, b]    = [a | [b]]
12      0 + 1 + 1 = 0 + 1 + 1
13      ([a, b] = [a | [b]], length([b], 0 + 1), 0 + 1 + 1 = 0 + 1 + 1)
14      some [X0, Xs0, N0] ([a, b] = [a | [b]], ...)
   ...
15      length([a, b], 0 + 1 + 1)
\end{verbatim}
and provided we accept that @0 + 1 + 1 = 2@ then our work here is done.

We haven't actually \emph{calculated} anything here; rather we've just
proved something we strongly suspected was true anyway.  What we really
want to do, as logic programmers, is find some way of working out, for
instance, what @N@ satisfies @length([a, b, c], N)@.

Just as we can take a term rewriting approach to solving problems in
propositional logic, we can use similar shortcuts in predicate logic.
In particular, this will avoid the need for us to pluck the right terms
from thin air when looking for a proof of a formula containing variables.
We use the following ``tricks'' while working backwards from the
top-level goal:
\begin{itemize}
\item existentially quantified variables are left in the 
\end{itemize}
\XXX NEED TO SORT THIS BIT OUT.





De Morgan's laws extend to cover quantification, giving us the following
identities:
\begin{align*}
\All{X}{p}
& \Eqv \Not{\Some{X}{\Not{p}}} \\
\Some{X}{p}
& \Eqv \Not{\All{X}{\Not{p}}} \\
\end{align*}

Since the names of quantified variables do not matter, we can rename
them at will.

\subsection{Using Quantifiers}

From $\All{X}{p}$ we can deduce $p[t/X]$ for any term $t$.  The
expression $p[t/X]$ denotes the application of the \emph{substitution}
$[t/X]$ to $p$ -- that is, all free occurrences of $X$ in $p$ are
replaced with $t$.  Substitution works as follows:
\begin{align*}
\text{(Over terms)} \\
X[t/X]
& = t \\
Y[t/X]
& = Y \text{ provided $X$ and $Y$ are different variables} \\
f(t_1, t_2, \ldots, t_n)[t/X]
& = f(t_1[t/X], t_2[t/X], \ldots, t_n[t/X]) \\
\text{(Over formulae)} \\
a(t_1, t_2, \ldots, t_n)[t/X]
& = a(t_1[t/X], t_2[t/X], \ldots, t_n[t/X]) \\
\Not{p}[t/X]
& = \Not{(p[t/X])} \\
(p \Conj q)[t/X]
& = (p[t/X] \Conj q[t/X]) \\
(p \Disj q)[t/X]
& = (p[t/X] \Disj q[t/X]) \\
(p \Imp q)[t/X]
& = (p[t/X] \Imp q[t/X]) \\
(\All{X}{p})[t/X]
& = \All{X}{p} \\
(\All{Y}{p})[t/X]
& = \All{Y}{p[t/X]} \text{ provided $X$ and $Y$ are different variables} \\
(\Some{X}{p})[t/X]
& = \Some{X}{p} \\
(\Some{Y}{p})[t/X]
& = \Some{Y}{p[t/X]} \text{ provided $X$ and $Y$ are different variables} \\
\end{align*}

From $p(t)$, for any term $t$, we can deduce $\Some{X}{p(X)}$.  Or, more
correctly, given $p[t/X]$ we conclude $\Some{X}{p}$.

We can simplify quantified formulae:
\begin{align*}
\All{X}{p \Conj q}
& \Eqv \All{X}{p} \Conj \All{X}{q}\\
\Some{X}{p \Conj q}
& \Eqv \Some{X}{p} \Conj \Some{X}{q}
\end{align*}

Provided $X \notin \FV(p)$ can rearrange quantifiers like so:
\begin{align*}
p \Conj \All{X}{q}
& \Eqv \All{X}{p \Conj q} \\
p \Conj \Some{X}{q}
& \Eqv \Some{X}{p \Conj q} \\
\end{align*}
The constraint is required because we do not want to inadvertently
\emph{capture} a free variable that happens to be called $X$ in $p$ in
the quantifier.  We can always move quantifiers out to the outermost
level by renaming variables as necessary.

\subsection{An Example: Schubert's Steamroller}

This rather knotty problem was devised by Mark E. Schubert \XXX{}.

We start off with a statement of the problem in plain English.
\begin{itemize}
\item Snails, caterpillars, birds, foxes and wolves are all animals.
\item Grain is a kind of plant.
\item Each species of animal eats all types of plants
or all species of smaller animals that eat some types of plants.
\item Wolves are bigger than foxes, foxes are bigger than birds, and
birds are bigger than caterpillars and snails.
\item Wolves don't eat foxes or grain.  Birds eat caterpillars, but not snails.
Caterpillars and snails like to eat plants.
\item \textbf{Is there an animal that eats a grain-eating animal?}
\end{itemize}

Now let's translate the axioms of the problem into logical formulae:
\begin{tabular}{rl}
    &  (Axioms.) \\
(1a) & $\Animal(\Snail)$ \\
(1b) & $\Animal(\Caterpillar)$ \\
(1c) & $\Animal(\Bird)$ \\
(1d) & $\Animal(\Fox)$ \\
(1e) & $\Animal(\Wolf)$ \\
\\
(2) & $\Plant(\Grain)$ \\
\\
(3) & $\All{X}{\Animal(X) \Imp \Herbivorous(X) \Disj \Carnivorous(X)}$ \\
(3a) & $\All{X}{\Herbivorous(X) \Eqv$ \\
     & $\qquad \All{Y}{\Eats(X, Y) \Bimp \Plant(Y)}}$ \\
(3b) & $\All{X}{\Carnivorous(X) \Eqv$ \\
     & $\qquad \All{Y}{\Eats(X, Y) \Bimp
                    \BiggerThan(X, Y) \Conj
                    \Some{Z}{\Plant(Z) \Conj \Eats(Y, Z)}}}$ \\
                    \\
(4a) & $\BiggerThan(\Wolf, \Fox)$ \\
(4b) & $\BiggerThan(\Fox, \Bird)$ \\
(4c) & $\BiggerThan(\Bird, \Caterpillar)$ \\
(4d) & $\BiggerThan(\Bird, \Snail)$ \\
\\
(5a) & $\Not{\Eats(\Wolf, \Fox)}$ \\
(5b) & $\Not{\Eats(\Wolf, \Grain)}$ \\
(5c) & $\Eats(\Bird, \Caterpillar)$ \\
(5d) & $\Not{\Eats(\Bird, \Snail)}$ \\
(5e) & $\Herbivorous(\Caterpillar)$ \\
(5f) & $\Herbivorous(\Snail)$ \\
\end{tabular}

The goal is then $\Some{X, Y}{\Eats(X, Y) \Conj \Eats(Y, \Grain)}$.

It turns out that the answer to the conundrum is that foxes eat birds
who eat grain (it's probably easier to write a computer program called a
\emph{theorem prover} to work this out than to do so by trying out each
combination by hand\ldots)  But how can we prove this?  Here's how:

\begin{tabular}{rl}
& (Deduce that wolves are carnivorous.) \\
(6) & $\Not{\Eats(\Wolf, \Grain)} \Conj \Plant(\Grain)$
\\ & --- by (5b) and (2) \\
(7) & $\Some{Y}{\Not{\Eats(\Wolf, Y)} \Conj \Plant(Y)}$
\\ & --- from (6) \\
(8) & $\Not{\All{Y}{\Eats(\Wolf, Y) \Disj \Not{\Plant(Y)}}}$
\\ & --- from (7) \\
(9) & $\Not{\All{Y}{\Eats(\Wolf, Y) \Bimp \Plant(Y)}}$
\\ & --- from (8) \\
(10) & $\Not{\Herbivorous(\Wolf)}$
\\ & --- from (9) and definition of $\Herbivorous$ (3a) \\
(11) & $\Carnivorous(\Wolf)$
\\ & --- by (10) and (3) via resolution \\
\\
& (Deduce therefore that foxes are carnivorous.) \\
(12) & $\All{Y}{\Eats(\Wolf, Y) \Bimp
            \BiggerThan(\Wolf, Y) \Conj
            \Some{Z}{\Plant(Z) \Conj \Eats(Y, Z)}}$
\\ & --- by (11) and (3b) via modus ponens \\
(13) & $\All{Y}{\Not{\Eats(\Wolf, Y)} \Imp
            \Not{\BiggerThan(\Wolf, Y)} \Disj
            \Not{\Some{Z}{\Plant(Z) \Conj \Eats(Y, Z)}}}$
\\ & --- contrapositive of (12) \\
(14) & $\Not{\BiggerThan(\Wolf, \Fox)} \Disj
            \Not{\Some{Z}{\Plant(Z) \Conj \Eats(\Fox, Z)}}$
\\ & --- by (13) and (5a) via modus ponens \\
(15) & $\Not{\Some{Z}{\Plant(Z) \Conj \Eats(\Fox, Z)}}$
\\ & --- by (14) and (4a) via resolution \\
(16) & $\All{Z}{\Not{\Eats(\Fox, Z)} \Bimp \Plant(Z)}$
\\ & --- from (15) \\
(17) & $\Not{\Eats(\Fox, \Grain)}$
\\ & --- by (16) and (2) via modus ponens \\
(18) & $\Not{\Eats(\Fox, \Grain)} \Conj \Plant(\Grain)$
\\ & --- by (17) and (2) \\
(19) & $\Some{Y}{\Not{\Eats(\Fox, Y)} \Conj \Plant(Y)}$
\\ & --- by (18) \\
(20) & $\Not{\All{Y}{\Eats(\Fox, Y) \Bimp \Plant(Y)}}$
\\ & --- by (19) \\
(21) & $\Not{\Herbivorous(\Fox)}$
\\ & --- by definition of $\Herbivorous$ (3a) \\
(22) & $\Carnivorous(\Fox)$
\\ & --- by (21) and (3) via resolution \\
\end{tabular}
So we've identified that the foxes eat all animals that eat some kind of
plant.  All we have to do now is show that birds eat grain, and hence
that foxes eat birds, and we have proved the goal.

\begin{tabular}{rl}
(23) & $\All{Y}{\Eats(\Snail, Y) \Bimp \Plant(Y)}$
\\ & --- by definition of $\Herbivorous(\Snail)$ (3a) \\
(24) & $\Eats(\Snail, \Grain)$
\\ & --- by (23) and (2) via modus ponens \\
(25) & $\Plant(\Grain) \Conj \Eats(\Snail, \Grain)$
\\ & --- by (24) and (2) \\
(26) & $\Some{Z}{\Plant(Z) \Conj \Eats(\Snail, Z)}$
\\ & --- by (25) \\
(27) & $\Not{\Eats(\Bird, \Snail)} \Conj
        \BiggerThan(\Bird, \Snail) \Conj
        \Some{Z}{\Plant(Z) \Conj \Eats(\Snail, Z)}$
\\ & --- by (26), (5d) and (4d) \\
(28) & $\Some{Y}{
            \Not{\Eats(\Bird, Y)} \Conj
            \BiggerThan(\Bird, Y) \Conj
            \Some{Z}{\Plant(Z) \Conj \Eats(Y, Z)}}$
\\ & --- by (27) \\
(29) & $\Not{\All{Y}{
            \Eats(\Bird, Y) \Bimp
            \BiggerThan(\Bird, Y) \Conj
            \Some{Z}{\Plant(Z) \Conj \Eats(Y, Z)}}}$
\\ & --- by (28) \\
(30) & $\Not{\Carnivorous(\Bird)}$
\\ & --- by definition of $\Not{\Carnivorous(\Bird)}$ (3b) \\
(31) & $\Herbivorous(\Bird)$
\\ & --- by (30) and (3) via resolution \\
(32) & $\All{Y}{\Eats(\Bird, Y) \Bimp \Plant(Y)}$
\\ & --- by definition of $\Herbivorous(\Bird)$ (3a) \\
(33) & $\Eats(\Bird, \Grain)$
\\ & --- by (33) and (2) via modus ponens \\
(34) & $\Plant(\Grain) \Conj \Eats(\Bird, \Grain)$
\\ & --- by (33) and (2) \\
(35) & $\Some{Z}{\Plant(Z) \Conj \Eats(\Bird, Z)}$
\\ & --- by (34) \\
(36) & $\BiggerThan(\Fox, \Bird) \Conj
        \Some{Z}{\Plant(Z) \Conj \Eats(\Bird, Z)}$
\\ & --- by (35) and (4b) \\
(37) & $\All{Y}{\Eats(\Fox, Y) \Bimp
            \BiggerThan(\Fox, Y) \Conj
            \Some{Z}{\Plant(Z) \Conj \Eats(Y, Z)}}$
\\ & --- by (22) and definition of $\Carnivorous(\Fox)$ (3b) \\
(38) & $\Eats(\Fox, \Bird)$
\\ & --- by (37) and (36) via modus ponens \\
(39) & $\Eats(\Fox, \Bird) \Conj \Eats(\Bird, \Grain)$
\\ & --- by (38) and (33) \\
(40) & $\Some{X, Y}{\Eats(X, Y) \Conj \Eats(Y, \Grain)}$
\\ & --- by (39) \\
QED \\
\end{tabular}

Well, that was hard work (in practice, when writing a paper, we would
omit many of the smaller steps in the above.)  Still, we should now have
some sort of feel for first order logic.

The point of this exercise is XXX HERE

\subsection{Aside on higher order programming?}

\XXX{We can treat closures not as higher order terms, but rather as
structures containing (amongst other things) \emph{names} of predicates
which can be interpreted by some special machinery that handles higher
order application.}

\section{Using Predicate Logic for Computation}

Horn clauses.

Proof procedures.

Clark completion.

Negation and the CWA.

Modes.




% % vim: ft=tex ff=unix ts=4 sw=4 et wm=8 tw=0

\chapter{Basic Syntax and Terminology}



\section{Introduction}

Describing Mercury takes a number of terms that are well understood
in the declarative programming community, but are rarely, if ever, used
in the more common imperative schools.  This chapter introduces the bulk
of the terms the reader may not be familiar with, but which occurr
frequently throughout the rest of this book.  The intention is to
provide a ``gloss'' to aid comprehension of what follows; more full
explanations will be given as the terms are used formally for the first
time.



\section{Declarative and Imperative}

Languages like C, C++, Java and so forth are all \emph{imperative}
programming languages.

Imperative programming considers a program to be a \emph{sequence of
instructions}.  Run-time control-flow decisions are made on the basis of
the contents of the \emph{mutable variables} that represent the
\emph{state} of the program.  Instructions fall into three categories:
those that update mutable variables with the result of some computation;
those that change the control flow of the program depending upon the
state of those variables; and those that perform input/output (IO)
operations.

Languages like Mercury and Haskell are \emph{declarative} programming
languages.

Declarative programming considers a program to be a \emph{computable
relationship} between its inputs and outputs.  There are no mutable
variables in declarative programming; the term \emph{variable} instead
refers to the name given to the result of some computation.  Some
relationships are simple enough to be computed directly, such as
@X = Y + Z@ where any two of the values are known.  Other relationships
have to be expressed by the programmer in terms of combinations of these
very basic relationships.

The ``moral'' motivation behind declarative programming is that the
resulting program is merely a refinement of the statement of the
problem, whereas it is actually very difficult to say exactly what
relationship a program is computing, making it harder to decide whether
the program contains a bug or not.

The practical motivation is that you generally get more bang for your
buck in terms of progress made per line of code written.  Moreover,
because declarative programs are essentially just statements of
mathematics, compilers can and do perform extensive checking and
verification tests on the code.  A large class of bugs that are part of
the day-to-day life of the imperative programmer simply cannot be
expressed in a declarative language; those that can be are usually
caught by the compiler, well before one needs to fire up the debugger.

\XXX{I think I'm selling this too hard.  The speech should probably go
in the introduction.}



\section{Predicates, Modes and Determinism}

In logic, a \emph{predicate} is simply the name given to a relationship
between some number of variables.  For example, @father(C, F)@ might
represent the relationship of @F@ being the father of @C@.  @C@ and @F@
are said to be the \emph{parameters} of @father@.

Given any bindings for @F@ and @C@ we can ask whether @father(C, F)@ is
true or not.  Moreover, we may be able to give the bindings for only a
subset of the variables.  Say we fix @F@, we can ask what the possible
bindings for @C@ are that would make @father(C, F)@ true.  Conversely,
we might ask what the possible values of @F@ are for a given @C@.
Indeed, we may go one step further and specify \emph{neither} @F@ nor
@C@ and simply ask for all possible pairwise bindings with the
appropriate property.

We use the term \emph{determinism} to refer to the number of possible
solutions a predicate can have in a particular \emph{mode} (i.e. given
fixed values for certain the variables.)  A predicate is
\emph{deterministic} if there is exactly one solution (set of bindings)
for the remaining variables, otherwise it is \emph{nondeterministic}.
(We will see that in Mercury it is useful to further subdivide the
determinism categories.)

Informally, we refer to the arguments with given values as inputs and
those whose value is not fixed as outputs.

For example, @father@ is deterministic if we fix only @C@ since every child
has exactly one father.  However, if we fix only @F@ we may get any
number of possible bindings for @C@ depending upon how many children @F@
has had, hence this mode of @father@ is nondeterministic.

We use the term ``procedure'' (or sometimes overload ``mode'') to refer
to a predicate restricted to a given set of argument modes.

\subsection{Syntax}

Mercury understands several determinisms: @erroneous@, @semidet@, @det@,
@nondet@, @multi@, @cc_nondet@ and @cc_multi@.  These will be explained
in detail in \XXX{}; for now it suffices to know that @semidet@ and
@nondet@ can fail (i.e. have no solutions), @semidet@ may have a single
solution, @det@ has exactly one solution (i.e. it describes a
\emph{functional} relationship -- see below), and @nondet@ and @multi@
may have more than one solution.

The two most important built-in modes are @in@ and @out@, corresponding
to input and output arguments respectively.  Modes are discussed in
\XXX{}.

The mode declarations (one per procedure) for the @father@ predicate
described above would look like this:
\begin{verbatim}
    % father(Child, Father).
    %
:- mode father(in,  out) is det.
:- mode father(out, in ) is nondet.
:- mode father(in,  in ) is semidet.
\end{verbatim}

\XXX{I'm not sure how/when/where to introduce syntax in this chapter.
It may be best to merge this chapter with the main Mercury-as-it-is-
written chapter.}



\section{Functions and Expressions}

A deterministic procedure is said to describe a \emph{functional}
relationship between its inputs and its outputs (that is, the output is
uniquely determined for each possible input.)  For instance, the
sum @X = Y + Z@ is functional given any two of @X@, @Y@ or @Z@.

Special syntax exists for functions with a single (last) output
argument.  The main reason for this is that it supports \emph{functional
composition}, which allows us to avoid naming intermediate results.  For
example, rather than writing
\begin{verbatim}
    Tmp1 = A    + B,
    Tmp2 = Tmp1 - C,
    Tmp3 = Tmp2 * D,
    X    = Tmp3 / E
\end{verbatim}
we can simply write
\begin{verbatim}
    X = (((A + B) - C) * D) / E
\end{verbatim}
(in fact the precedence and associativity of the operators @+@, @-@,
@*@, @/@ etc. has been chosen so that parentheses can often be
avoided.)

An \emph{expression} describes either a simple value or the
\emph{application} of a function to other expressions.



\section{Goals}

A goal describes either a simple \emph{predicate call} or XXX HERE!



Procedures.

Goals.

Terms.

Variables.

Quantifiers.

Conjunction, disjunction, negation, conditional.

...




% % vim: ft=tex ff=unix ts=4 sw=4 et tw=76

\chapter{Basic Types}

\XXX{ZS: explain that static typing is one end of a continuum.}

Mercury is both strongly and statically typed.  Strong typing ensures
that a program does not inadvertently compare apples with oranges, while
the purpose of static typing is to inform the programmer \emph{at
compile time} if a given program may attempt such a thing.  Types also
exist to provide a systematic way to structure data.

By way of comparison with other languages, C is not
strongly typed, since C allows unchecked \emph{casts} between values of
different types (a cast is a way of telling the compiler to
treat a particular value as if it had a different type.)

Python is strongly typed, but not statically type checked: every time a
Python program performs an addition, for instance, the arguments have to
be tested to verify that they are indeed numbers.

Java, on the other hand, \emph{is} both strongly and statically typed.
Unfortunately, however, Java's type system lacks expressive power:
\emph{checked} run time casts are frequently necessary, meaning that most
non-trivial type errors will only be spotted when a running program aborts
with an exception.

Mercury does not have these problems.  All type errors in
a Mercury program are identified at compile time, making it harder to
ship buggy programs.  An expressive type system makes Mercury programs more
efficient, since no type checks are needed at run time.  The extra
information also allows the compiler to perform optimizations that would
otherwise be impossible.  (Mercury \emph{also} makes what's known as
\emph{run-time type information} available to a program, which means it is
possible to write programs that are aware of the names and structures of
types.  The predicates @io.read@ and @io.write@ use this facility, for
example, to read and write values of arbitrary types.)

There are three different sorts of type in Mercury: discriminated
unions, equivalence types and abstract types.  A \emph{discriminated
union} is a collection of values.  An
\emph{equivalence type} is simply a different name for another type.  An
\emph{abstract type} is one whose definition is hidden from its users
(abstract types are explained fully in chapter \XXX{} on the module system.)

Every predicate and function in a Mercury program must have a \emph{type
signature} given by the corresponding @pred@ or @func@ declaration.  The
type signature specifies what type each argument has.  The Mercury
compiler uses this information to automatically infer the types of the local
variables in predicate and function definitions and to resolve potential
ambiguity when the same name has been used for more than one purpose (\eg
one might have a data constructor and a predicate with the same name, but
different argument types.)

Because Mercury's type system is rich there is a danger of confusing the
reader with too much, too soon.  This chapter concentrates on the basics,
namely the primitive (built-in) types, defining new \emph{discriminated
union} types, and equivalence types.  More complex aspects of the type
system are dealt with in later chapters \XXX{}.

\section{The Primitive Types}

Mercury's built-in types are @int@, @float@, @char@, @string@, tuples and
the predicate and function types.

\subsection{Integers}

The @int@ type represents the integers -- @123@, @-9@, @42@ and so forth --
that will fit into a machine word on the target computer (the range is from
about $-2$ billion to $+2$ billion on a 32-bit machine.)

All the core operations one can perform on @int@s, including the basic
arithmetic functions, are defined in the @int@ module in the Mercury
standard library.

Like most languages, Mercury has syntax for hexadecimal, octal, and binary
numbers, as well as for the integer codes for characters.  The interested
reader can find the details in the Mercury Reference Manual \XXX{}.

\subsection{Floating Point Numbers}

Floating point values such as @2.718@ and @3.0e8@ ($3.0\times10^8$) are
encoded by the @float@ type.  (The decimal point must always
appear in a floating point number and be preceded and followed by at
least one digit.)

The core @float@ operations are defined in the @float@ module in the
Mercury standard library.  Mathematical constants such as $\pi$ and $e$, and
things like trignometric functions, are defined in the @math@ module in the
Mercury standard library.

\subsection{Characters}

The @char@ type represents single characters.  Unfortunately, due to the
Mercury's heritage, there is no special syntax for single character
values.  Some characters, @a@, @b@, @c@, @1@, @2@, @3@ and so forth,
need no special quoting.
For others (space, tab and newline, for instance), one has to use
single quotes and write @' '@, @'\t'@ and
@'\n'@ -- single quote and backslash are @'\''@ and
@'\\'@ respectively.
Finally, some characters need to be enclosed in parentheses since
otherwise they might be confused with infix operators: @(+)@ and
@(*)@, for example.
The safest method is to use both parentheses \emph{and} quoting, as in
@('/')@.  We apologise for the inconvenience.

The core @char@ operations are defined in the @char@ module in the
Mercury standard libarary.

\XXX{ZS: Fix the language on this point.}

\XXX{ZS: Include a table of escaped characters.}

\subsection{Strings}

Mercury strings are immutable sequences of characters.  Examples include
@"bodacious"@ and @"new\nline"@.  To embed certain characters in a
string -- double quotes, backslashes, tabs and newlines
et cetera -- one has to \emph{escape} them with a preceding backslash:
@\"@, @\\@, @\t@ and @\n@.

If a string contains an explicit newline, as in
\begin{myverbatim}
    "there's a newline in here
there it is"
\end{myverbatim}
then the newline character is actually part of the string, just as
if we had written
\begin{myverbatim}
    "there's a newline in here\nthere it is"
\end{myverbatim}
However, if the \emph{last} character on a line is a non-escaped backslash
then the trailing newline is not taken to be part of the string, hence
\begin{myverbatim}
    "I see \
no newline"
\end{myverbatim}
is the same as
\begin{myverbatim}
    "I see no newline"
\end{myverbatim}

\XXX{I don't really want to mention the double-double quotes in a string
thing.}

It is even possible to include characters in a string by giving the
appropriate character code, although the reader is referred to the Mercury
Reference Manual \XXX{} for details.

The core @string@ operations are defined in the @string@ module in the
Mercury standard library.

\section{Discriminated Union Types}

One can think of a discriminated union as being a set of values, each of
which is distinguishable from the others.  At their core, virtually all
Mercury types are distributed unions.
(Equivalence types are simply a means of giving convenient names to
complicated types and abstract types are used to hide the
definition of a type.)
\XXX{Should I qualify this by saying that types implemented in foreign
code aren't strictly DUs?}

We can define a new type to represent the suits in a deck of playing
cards like this:
\begin{myverbatim}
:- type suit ---> spades ; hearts ; diamonds ; clubs.
\end{myverbatim}
The new type is called @suit@ and it has four 
\emph{data constructors}, or possible values: @spades@, @hearts@,
@diamonds@ and @clubs@.  The goal
\begin{myverbatim}
    X = spades
\end{myverbatim}
will, assuming @X@ is uninstantiated, bind @X@ to the value @spades@ which
is of type @suit@.

Data constructors can take arguments, as illustrated by the following type
for binary trees of @int@s:
\begin{myverbatim}
:- type int_tree ---> empty ; branch(int, int_tree, int_tree).
\end{myverbatim}
The code for insertion and the membership test can be written like
this:
\begin{myverbatim}
:- func insert(int, int_tree) = int_tree.

insert(X, empty          ) = branch(X, empty, empty).

insert(X, branch(Y, L, R)) =
    (      if X =< Y then branch(Y, insert(X, L), R)
      else /* X  > Y */   branch(Y, L, insert(X, R))
    ).

:- pred int_tree `contains` int.
:- mode in       `contains` in is semidet.

branch(X, L, R) `contains` Y :-
    (   Y = X
    ;   Y < X,  L `contains` Y
    ;   Y > X,  R `contains` Y
    ).
\end{myverbatim}
(Observe the C-style @/*@ comment @*/@ in the definition of @insert/2@.)

As we've seen before, a data constructor in a program denotes either a
construction or a deconstruction -- exactly which is determined by the
modes and the context.  In @insert/2@, occurrences of @branch/3@ and
@empty@ in the head are deconstructions, while occurrences in the body are
constructions.

The definition for @contains/2@ uses a plain disjunction, not a switch:
it reads ``a @branch(X, L, R)@ contains @Y@ \emph{iff} @Y = X@
\emph{or} @Y < X@ and @L@ contains @Y@ \emph{or} @Y > X@ and @R@
contains @Y@.''
If any arm of the disjunction succeeds then the predicate as a
whole succeeds.  A clause for the @empty@ case is not necessary since it
would always fail.

It is quite common for a type to have a single constructor of the same
name:
\begin{myverbatim}
:- type date ---> date(int, int, int).  % Year, month, day.
\end{myverbatim}
Mercury will never get the type and the data constructor confused since
the type name can only appear in type declarations and the constructor
name can only appear in clauses.

\subsection{Parameterised Discriminated Union Types}

So far we've only described ``concrete'' types.  It turns out to be very
useful to generalise over whole families of types.  The binary
tree \emph{structure} used in @int_tree@, for instance, would also work
fine for strings, floating
point numbers and pretty much anything else.  Rather than defining
nearly identical types (\eg @string_tree@ and @float_tree@) every time
we need them, it is far better to \emph{parameterise} our definition, thus:
\begin{verbatim}
:- type tree(T) ---> empty ; branch(T, tree(T), tree(T)).
\end{verbatim}
A value of type @tree(T)@, then, is either @empty@ or a @branch(X, Y, Z)@
where @X@ is of type @T@ and @Y@ and @Z@ are of type @tree(T)@.

Since we are now working over arbitrary types rather than plain @int@s, we
need to use \emph{generic} comparison predicates for the insert and look-up
operations:
\begin{verbatim}
:- func insert(T, tree(T)) = tree(T).

insert(X, empty          ) = branch(X, empty, empty).

insert(X, branch(Y, L, R)) =
    (      if X @=< Y then branch(Y, insert(X, L), R)
      else /* X @>  Y */   branch(Y, L, insert(X, R))
    ).

:- pred tree(T) `contains` T.
:- mode in      `contains` in is semidet.

branch(X, L, R) `contains` Y :-
    O = ordering(X, Y),
    (   O = (=)
    ;   O = (<),    R `contains` Y
    ;   O = (>),    L `contains` Y
    ).
\end{verbatim}
The predicates, \verb!@<!, \verb!@=<!, \verb!@>=!, and \verb!@>!, 
compare two terms in Mercury's \emph{standard ordering}.
It doesn't matter what that ordering is for our purposes here, although the
curious reader can find out more from the Mercury Reference Manual
\XXX{Put this documentation in the Refence Manual!}.
The function
@ordering(X, Y)@ returns @(=)@ if @X = Y@, @(<)@ if \verb!X @< Y!, and
@(>)@ if \verb!X @> Y!.

We can substitute any type we like for the parameter @T@ in @tree(T)@:
@tree(int)@ is a binary tree of integers, @tree(string)@ is a binary tree of
strings, and so on.  The implementations for @insert/2@ and @contains/2@
will work regardless of which type is substituted for @T@, something we
could not have managed if we'd stuck to using @int_tree@, @string_tree@
and so forth.
Moreover, Mercury will tell us at compile time if we ever attempt to insert
a @string@, say, into a @tree(int)@ (compare this with Java, say, where such
a problem would only become apparent when the running program threw an
exception.)

A type may have more than one parameter.  We might want to
generalise our @tree/1@ type to store arbitrary key-value mappings to form a
dictionary structure:
\begin{verbatim}
:- type dict(K, V)
    --->    empty
    ;       branch(K, V, dict(K, V), dict(K, V)).
\end{verbatim}
The insert and lookup operations are similar to their @tree/1@ counterparts:
\begin{verbatim}
:- func insert(K, V, dict(K, V)) = dict(K, V).

insert(Ka, Va, empty) =
    branch(Ka, Va, empty, empty).

insert(Ka, Va, branch(Kb, Vb, L, R)) = Tree :-
    O = ordering(Ka, Kb),
    (   O = (=),    Tree = branch(Ka, Va, L, R)
    ;   O = (<),    Tree = branch(Kb, Vb, insert(Ka, Va, L), R)
    ;   O = (>),    Tree = branch(Kb, Vb, L, insert(Ka, Va, R))
    ).

:- pred lookup(dict(K, V), K,  V).
:- mode lookup(in,         in, out) is semidet.

lookup(branch(Ka, Va, L, R), Kb, Vb) :-
    O = ordering(Ka, Kb),
    (   O = (=),    Vb = Va
    ;   O = (<),    lookup(R, Kb, Vb)
    ;   O = (>),    lookup(L, Kb, Vb)
    ).
\end{verbatim}
It is a requirement that any type variables appearing in a data constructor
in a type definition must be parameters of the type.  The following
violates the rule and would be reported as an error:
\begin{verbatim}
:- type awful ---> mistake(T).
\end{verbatim}
The justification is that Mercury has to be able to work out what type @X@
has in @mistake(X)@ in every deconstruction.
This definition,
were it legal, says that @X@ could be anything at all
-- there is, in general, simply no way for the compiler to work out the
actual type ahead of time.
(Chapter \XXX{} explains existentially quantified types which \emph{can} be
used to construct heterogeneous collections.)

\subsection{Lists}

Lists are a particular discriminated union type that are used more often
than any other.  A list is essentially a linear sequence of data.  Mercury
provides special syntax for lists, which is the topic of this subsection
(chapter \XXX{} discusses programming with lists in detail.)

We could define a list type like this:
\begin{myverbatim}
:- type list(T) ---> empty ; cons(T, list(T)).
\end{myverbatim}
where @cons/2@ stands for ``list constructor''.  In a term of the form
@cons(X, Xs)@, @X@ is referred to as the \emph{head} and @Xs@ is referred to
as the \emph{tail}.  Under this scheme, the list of the numbers @1, 2, 3@
would be represented as @cons(1, cons(2, cons(3, empty)))@.  However, this
syntax is too cumbersome for regular use.

The @list@ module in the Mercury standard library defines the type @list/1@
like this:
\begin{myverbatim}
:- type list(T) ---> [] ; [T | list(T)].
\end{myverbatim}
The empty list is represented by @[]@ and the non-empty list is represented
by @[X | Xs]@ where @X@ is the head and @Xs@ is the tail.

The term @[X | Xs]@ is syntactic sugar for @[|](X, Xs)@ -- the two forms are
completely interchangable.

The term @[1, 2, 3]@ is syntactic sugar for @[1 | [2 | [3 | []]]]@.

The term @[1, 2 | Xs]@ is syntactic sugar for @[1 | [2 | Xs]]@.

The unsugared notation is virtually never used in practice.

\subsection{Tuples}

Sometimes a small group of predicates needs to exchange parcels of data, such
as pairs of strings and numbers.  It can be annoying to have to define a new
type such as
\begin{myverbatim}
:- type string_and_int ---> string_and_int(string, int).
\end{myverbatim}
when the type is only used in a very limited scope.  One alternative is to
use a \emph{tuple}.

Tuples are the only primitive compound type in Mercury.  A tuple is any
number of comma-separated values between braces:
\begin{myverbatim}
    {"the base of natural logarithms", 'e', 2.718}
\end{myverbatim}
This tuple has type @{string, char, float}@.

There is no tuple-specific library module: one can construct them and
one can deconstruct them -- and that's it.

\section{Equivalence Types}

Imagine we are writing a personnel database relating people to lists of
various attributes and that we have already defined the types @person@ and
@attribute@.
The @dict/2@ type defined above is a likely candidate for the job.
Specifically, we'd want to use
\begin{verbatim}
    dict(person, list(attribute))
\end{verbatim}
This is something of a mouthful.  Typing this everywhere we needed it would be
tiring and, worse, a maintenance problem if we later decide that, say,
@set(attribute)@ is a better idea than @list(attribute)@.

The Mercury solution to the problem is to use an equivalence type:
\begin{verbatim}
:- type persattrs == dict(person, list(attribute)).
\end{verbatim}
and hereafter we can write @persattrs@ as shorthand for the type on the
right hand side of the @==@ symbol.  It makes no difference to Mercury
whether you use @persattrs@ or the full type name.

A common idiom is to use equivalence types to give use-specific names
to primitive types.  For instance
\begin{verbatim}
:- type name         == string. % Must be non-empty.
:- type age          == int.    % Must be non-negative.
:- type num_children == int.    % Must be non-negative.
\end{verbatim}
@type@, @pred@ and @func@ declarations that use these names
appropriately become much more readable.

Equivalence types may also be parameterised.  For example, the following
describes a suitable representation for graph structures with labelled
vertices:
\begin{verbatim}
:- type graph(T) == map(vertex(T), list(vertex(T))).
\end{verbatim}
(We assume the type @vertex/1@ has been defined elsewhere.)
\XXX{Do we need more explanation of this type?  Or a simpler example?}

Note that, as with discriminated union types, any type variable appearing on
the right hand side of the @==@ must also appear as a parameter on the left
hand side.  Thus the following is in error:
\begin{verbatim}
:- type woeful == list(T).
\end{verbatim}

\XXX{Should I mention the shadow-types ``pattern''?}

\section{Conclusion}

Some sort of conclusion.



\chapter{More About Types}

\XXX{Not sure yet quite where this chapter will appear.}

The chapter on basic types \XXX{} explained the main primitive
types and how one can define new discriminated union types and equivalence
types.  Chapter \XXX{} on modules explained abstract types.

This chapter covers several more advanced type related topics, including how
to give names to data constructor fields and how to specify your own
equality and comparison relations.

(The chapters on type classes \XXX{} and the run-time type information
system \XXX{} complete the description of the Mercury type system.)

\section{Data Constructors with Named Fields}

Like most languages, Mercury supports data structures with named fields.
Consider a type used to keep track of the tally of votes:
\begin{myverbatim}
:- type tally ---> tally(ayes :: int, nayes :: int).
\end{myverbatim}
naming the fields of the @tally/2@ data constructor @ayes@ and
@nayes@ respectively.  We can still construct and deconstruct @tally/2@
values as if it had been defined without named fields.  However, having
named fields also allows us to refer to them \emph{without} the need for an
explicit deconstruction.

\subsection{Accessing Fields}

If we have @X = tally(12, 7)@ then we can refer to the values of the fields
of the value bound to @X@ using @X ^ ayes@ and @X ^ nayes@:
\begin{myverbatim}
    X ^ ayes  = 12,
    X ^ nayes =  7
\end{myverbatim}
The expression @X ^ ayes@ is just syntactic sugar for @ayes(X)@.  The
function @ayes/1@ is automatically constructed from the field name in the
type definition and will be defined as
\begin{myverbatim}
ayes(tally(A, _)) = A.
\end{myverbatim}

\subsection{Updating Fields}

We can ``update'' a particular field like this:
\begin{myverbatim}
    Y = ( X ^ ayes := 13 )
\end{myverbatim}
giving @Y = tally(13, 7)@.  This, of course, does not affect the value
of @X@: we still have @X = tally(12, 7)@.  
@X ^ ayes := 13@ should be read as ``the value of @X@ with @13@
substituted in the @ayes@ field.''

The expression @X ^ ayes := 13@ is just syntactic sugar for
@'ayes :='(X, 13)@.  The function @'ayes :='/2@ is also automatically
constructed from the type definition.  (Anything at all can be used as a
functor name if it is enclosed in @'@single quotes@'@, which is why the
extra characters in the name do not lead to trouble.)  The definition of
the automatically created @'ayes :='/2@ would be
\begin{myverbatim}
'ayes :='(tally(_, B), A) = tally(A, B).
\end{myverbatim}

Updating several fields in one go is straightforward:
\begin{myverbatim}
    Z = (( Y ^ ayes  := Y ^ ayes  + 2 )
             ^ nayes := Y ^ nayes + 3 )
\end{myverbatim}
giving @Z = tally(15, 10)@.

\subsection{Restrictions on Field Names}

At the time of writing, Mercury requires that no field name be defined twice
in the same module in order to avoid problems with ambiguity.  The following
are therefore all errors.

This example is illegal because it uses the same field name twice in the
same data constructor:
\begin{verbatim}
:- type foo ---> foo(a :: int, a :: int).
\end{verbatim}
Because the module qualified forms of the name will be different, there is
no problem with different \emph{modules} using the same field name.

The next example uses the same field name in two different data
constructors and is therefore illegal, despite the fact that they belong to
the same type:
\begin{verbatim}
:- type foo ---> foo1(a :: int)
            ;    foo2(a :: int).
\end{verbatim}
Finally, one cannot use the same field name twice, even in different types:
\begin{verbatim}
:- type foo ---> foo(a :: int).
:- type bar ---> bar(a :: int).
\end{verbatim}

(The names @foo@, @bar@, @baz@ and @quux@ are conventionally used by
computer scientists when they can't think of anything better to use in an
example.)

\subsection{Fields Within Fields}

Consider the following:
\begin{myverbatim}
:- type car
    --->    car(
                make            :: string,
                registration    :: string,
                owner           :: person
            ).

:- type person
    --->    person(
                name            :: string,
                date_of_birth   :: date,
                address         :: string
            ).
\end{myverbatim}
(we assume @date@ is defined elsewhere) and the bindings
\begin{myverbatim}
    Fred = person(
               "Fred Bloggs",
               date(1965, 7, 4),
               "11 Strangetrousers Terrace"
           ),
    Car  = car( 
               "Ford Escort",
               "PQR 123",
               Fred
           )
\end{myverbatim}
We can obtain the @name@ of the @owner@ of @Car@ with
@Car ^ owner ^ name@:
\begin{myverbatim}
    Car ^ owner ^ name = "Fred Bloggs"
\end{myverbatim}
So @Car ^ owner ^ name@ is the same as writing @(Car ^ owner) ^ name@.

If Fred moves house and we need to update his address then we can write
\begin{myverbatim}
    Car1 = (Car ^ owner ^ address := "42 Strawberry Fields")
\end{myverbatim}
which is the same as if we'd written
\begin{myverbatim}
    Owner  = Car ^ owner,
    Owner1 = (Owner ^ address := "42 Strawberry Fields"),
    Car1   = (Car   ^ owner   := Owner1                )
\end{myverbatim}
So the parentheses matter.  If we just wanted an updated version of the
@owner@ field of @Car@, rather than an updated version of @Car@ itself,
we'd write
\begin{myverbatim}
    Owner1 = ((Car ^ owner) ^ address := "42 Strawberry Fields")
\end{myverbatim}
which this time is the same as having written
\begin{myverbatim}
    Owner  = Car ^ owner,
    Owner1 = (Owner ^ address := "42 Strawberry Fields")
\end{myverbatim}

\subsection{User-Defined Field Access Functions}

It can be useful to explicitly define functions to be used for field access,
either to provide ``virtual'' fields that are computed rather than stored
(C$\sharp$ calls such things \emph{properties}), or to provide sanity checks
on accesses and updates, or to define ``indexed fields''.

\subsubsection{Virtual Fields}

Say we want to add a ``virtual'' field to the @person@ type, computing the
number of children a particular individual has.
To do this we simply write
\begin{verbatim}
:- func person ^ num_children = int.

Person ^ num_children = list.length(Person ^ children).
\end{verbatim}
(Virtual field access functions do not represent real fields, of course, and
so don't appear as field names in the type definition.)

\subsubsection{Checked Fields}

We can add a sanity check to the @children@ field's update function
ensuring that a person never \emph{loses} children when the list is updated:
(here we are just providing our own definition for the @children/2@ field
access function; Mercury will not construct a definition for a field access
or update function if we have already given one)
\begin{myverbatim}
:- func (person ^ children := list(child)) = person.

(person(A, B, C, D0, E) ^ children := D) = Person :-
    ( if all [Child] (
            list.member(Child, D0) => list.member(Child, D)
      ) 
      then Person = person(A, B, C, D, E)
      else exception.throw("you can't unhave children!")
    ).
\end{myverbatim}
(The condition of the conditional goal reads ``for all @Child@, \emph{if}
@Child@ is a member of @D0@ \emph{then} @Child@ must be a member of @D@.'')

Say we have 
\begin{myverbatim}
    Homer = person(
               "Fred Bloggs",
               date(1966, 4, 9),
               "Evergreen Terrace",
               [lisa, bart],
               [ /* no secrets */ ]
           )
\end{myverbatim}
and there is a new addition to the family.  If we write
\begin{myverbatim}
    Homer1 = (Homer ^ children := [lisa, bart, maggie])
\end{myverbatim}
then the update proceeds as expected.  On the other hand, forgetting one of
Homer's existing children in the update, as in
\begin{myverbatim}
    Homer1 = (Homer ^ children := [bart, maggie])
\end{myverbatim}
will result in the update operation throwing an exception since @lisa@ is a
member of @Homer ^ children@, but not @[bart, maggie]@.

\subsubsection{Indexed Fields}

We can also add ``virtual, indexed'' fields.  If we wish to
access a @person@'s @children@ by number we might write
\begin{verbatim}
:- func person ^ child(int) = child.

Person ^ child(N) = list.index0_det(Person ^ children, N).
\end{verbatim}
where @list.index0_det(Children, N)@ returns the @N@th member of @Children@,
(counting from zero, in traditional computer science style) or
throws an exception if @N@ is out of range.

Under this scheme we get
\begin{myverbatim}
    Homer ^ child(0) = lisa
    Homer ^ child(1) = bart
\end{myverbatim}
but @Homer ^ child(2)@, for instance, will throw an exception.

(The expression @Person ^ child(N)@ is syntactic sugar for
@child(N, Person)@.)

\XXX{Should I mention @map.elem/2@ etc. here as examples?}

Virtul indexed fields may have any number of arguments.  For example, the
two-dimensional array module (for instance, @array2d@ in the Mercury
standard library, defines @Array2D ^ elem(I, J)@ to access the element at
row @I@, column @J@ of @Array2D@.)
array.

\subsection{Mixing Named and Unnamed Fields}

It is not necessary to name every field in a data constructor:
\begin{myverbatim}
:- type person
    --->    person(
                name            :: string,
                date_of_birth   :: date,
                address         :: string,
                children        :: list(child),
                /* secrets */      list(secret)
            ).
\end{myverbatim}
The @list(secret)@ field is not named and, unlike the named fields, can
therefore only be accessed by deconstruction and ``updated'' by
construction.

\subsection{Semideterministic Field Access}

Consider the following type:
\begin{myverbatim}
:- type data ---> empty ; datum(payload :: int, rest :: data).
\end{myverbatim}
Given an @X@ of type @data@, the expression @X ^ payload@, may
\emph{fail}.  The field access @X ^ payload@ can only succeed if @X@ is
bound to a @datum/2@ value -- if @X@ is bound to @empty@ then there is no
@payload@ field.

Field accesses for field names that do not occur in all the data
constructors of a type are therefore \emph{semideterministic}.  Be aware of
his point, as it can from time to time lead to confusing error messages from
the compiler.

\XXX{Do I need to say anything about using pattern matching to resolve the
issue?}

\section{Explicit Type Qualification}

The types of the \emph{head variables} in a clause (\ie those appearing in
the clause head) are given by the corresponding @pred@ or @func@
declaration.  The compiler deduces the types of the \emph{local} variables
via a process known as \emph{type inference}.  Very occasionally, type
inference alone is not sufficient to unambiguously decide the type of a
local variable.

\subsection{Name Ambiguity}

Sometimes these problems arise due to overloaded names \XXX{this term is
introduced in the ``Hello, World!'' chapter}.  Say we have a module
@a@ exporting a predicate @num/1@,
\begin{myverbatim}
:- module a.
:- interface.

:- pred num(int).
:- mode num(out) is multi.

:- implementation.

num(1).
num(2).
num(3).
\end{myverbatim}
and a module @b@ exporting a predicate also called @num/1@,
\begin{myverbatim}
:- module b.
:- interface.

:- pred num(float).
:- mode num(out) is multi.

:- implementation.

num(1.414).
num(2.718).
num(3.141).
\end{myverbatim}
and a third module that imports both @a@ and @b@, defining a predicate @p@
testing whether there are solutions for @num/1@ whose sum is also a
solution:
\begin{myverbatim}
:- import int, float, a, b.

:- pred p.
:- mode p is semidet.

p :-
    num(X),
    num(Y),
    num(X + Y).
\end{myverbatim}
Since the code here does not module qualify @num/1@, the compiler cannot
decide if the programmer means @a.num/1@ and @int.(+)/2@ or @b.num/1@ and
@float.(+)/2@.

There are two ways to solve the problem: \emph{either} explicitly
module qualify the names (not always convenient) \emph{or} add
explicit type qualifiers to local variables or expressions to provide
the compiler with sufficient information.  Explicit
type qualification connects an expression to a type via the @with_type@
keyword.  The type ambiguity in @p@ above can be fixed with
\begin{myverbatim}
p :-
    num(X `with_type` int),
    num(Y `with_type` int),
    num((X + Y) `with_type` int).
\end{myverbatim}
although this is overkill.  In fact only one type qualification is
necessary, for instance
\begin{myverbatim}
p :-
    num(X),
    num(Y),
    num((X + Y) `with_type` int).
\end{myverbatim}
The compiler will reason that since the result of @X + Y@ must be an
@int@, the addition is a call to @int:(+)/2@.  This means that @X@ and @Y@
must also have type @int@ (since the type of @int:(+)/2@ is
@func(int, int) = int@) and therefore that the calls to @num/1@ must refer
to @a.num/1@ (and not @b.num/1@ which has type @pred(float)@.)

\subsection{Type Ambiguity}

Here is an example of type ambiguity that arises for different reasons:
\begin{myverbatim}
main(!IO) :-
    io.read(Result, !IO),
    (
        Result = ok(X),
        io.print(X, !IO)
    ;
        Result = eof
    ;
        Result = error(_, _),
        throw(Result)
    ).
\end{myverbatim}
The predicates @io.read/3@ and @io.print/3@ read and write values of
virtually \emph{any} type.  However, the compiler needs to be able to work
out the type of @X@ before it can compile @main/2@ (chapter \XXX{} on
run-time type information explains why this is the case.)  Our program,
however, gives no clue as to what type @X@ is supposed to have and the
compiler will issue an error message of the form
\begin{myverbatim}
oops.m:017: In predicate `oops.main/2':
oops.m:017:   warning: unresolved polymorphism.
oops.m:017:   The variables with unbound types were:
oops.m:017:       X: T
oops.m:017:       Result: (io.read_result(T))
\end{myverbatim}

Let's say we intended @X@ to be an @int@.
To resolve this problem we need to add an explicit type qualifier to @X@
-- because @X@ can never change its type it doesn't matter where --
and it's usually a good idea to pick the first occurrence:
\begin{myverbatim}
main(!IO) :-
    io.read(Result, !IO),
    (
        Result = ok(X `with_type` int),
        io.print(X, !IO)
    ;
        ...
    ).
\end{myverbatim}

\section{Type Specialised Predicates and Functions}

\XXX{To go in the optimization chapter.}

Polymorphic predicates that test parameterised values for equality or
compare them against each other have to pay a small performance penalty.

The Mercury compiler transparently constructs specialised equality and
comparison predicates for every type in a program.  These predicates are
passed as ``hidden'' arguments in each call to a polymorphic predicate or
function.

Consider the following function:
\begin{myverbatim}
:- func max(T, T) = T.

max(X, Y) = ( if X @=< Y then Y else X ).
\end{myverbatim}
The call to @max/2@ in the goal @Max = max(Foo, Bar)@ includes a hidden
argument giving the equality and comparison predicates for the specific type
that @Foo@ and @Bar@ share.  This information is passed to the call to
\verb!@=</2! which makes use of the comparison predicate.

As with all higher order programming, there is a small cost associated with
passing and using these hidden arguments.  For the most part, the cost is
trivial.  However, in some very rare circumstances performance requirements
may dictate that, say, a version of @max/2@ specialised for integer
arguments be used.

Mercury lets us do this without altering the definitions in our program.  We
merely add the following line to the module defining @max/2@:
\begin{myverbatim}
:- pragma type_spec(max/2, T = int).
\end{myverbatim}
This causes Mercury to construct a specialised version of @max/2@ as if it
had been declared as
\begin{myverbatim}
:- func max(int, int) = int.
\end{myverbatim}
(obtained by applying the substitution @T = int@ to the original function
declaration)
and all calls to @max/2@ where the compiler knows that the arguments must be
@int@s will use this version rather than the polymorphic version.
Because this version isn't polymorphic, the compiler can replace the call to
\verb!@=</2! in the definition with the
@int@ version, @=</2@, and avoid the overhead of an higher order predicate
call.

The reader is referred to the Mercury Reference Manual \XXX{} for complete
information on the @type_spec@ pragma.

\section{Types with User Defined Equality}

\XXX{This could conceivably go in the chapter on advanced modes.}

\XXX{Want to add a paragraph explaining that this is black-belt stuff and
can be skimmed over by the faint of heart.}

Imagine we wish to define a type to represent sets.  For whatever reasons
(e.g. we expect most sets to be small) we decide to use lists containing no
duplicates:
\begin{myverbatim}
:- type set(T) ---> set(list(T)).   % No duplicates in list.

:- func list_to_set(list(T)) = set(T).

list_to_set(Xs) = set(remove_dups(Xs)).

:- func remove_dups(list(T)) = list(T).

remove_dups([]      ) = [].
remove_dups([X | Xs]) =
    ( if list.member(X, Xs) then remove_dups(Xs)
                            else [X | remove_dups(Xs)] ).
\end{myverbatim}
This representation raises a problem: every set with more than two members
can be represented in multiple ways.  That is, the set ${1, 2, 3}$ can be
represented in any one of six ways:
\begin{myverbatim}
    [1, 2, 3]
    [1, 3, 2]
    [2, 1, 3]
    [2, 3, 1]
    [3, 1, 2]
    [3, 2, 1]
\end{myverbatim}
This is going to cause problems.  We would like it to be true that
\begin{myverbatim}
    list_to_set([1, 2, 3]) = list_to_set([3, 2, 1])
\end{myverbatim}
Unfortunately, as things stand, this isn't true.

One way around the problem is to revisit our design and perhaps use sorted
lists without duplicates instead.  This approach not be compatible with our
reasons for choosing unsorted lists in the first place.

The alternative is to supply our own equality predicate and tell Mercury to
use that, rather than use the default:
\begin{myverbatim}
:- type set(T) ---> set(list(T))
    where equality is equals_set.

:- pred equals_set(set(T), set(T)).
:- mode equals_set(in,     in    ) is semidet.

equals_set(A, B) :-
    A `subset_of` B,
    B `subset_of` A.

:- pred set(T) `subset_of` set(T).
:- mode in     `subset_of` in      is semidet.

set(As) `subset_of` set(Bs) :-
    all [X] ( list.member(X, As) => list.member(X, Bs) ).
\end{myverbatim}
The modifier @where equality is equals_set@ on the type definition for
@set/1@ tells Mercury to use @equals_set/2@ rather than the standard
equality test when testing two @set/1@ values for equality.  It is the
responsibility of the programmer to ensure that the type specific equality
predicate really does define an equivalence relation (\ie it should be
reflexive, symmetric and transitive).

Any attempt to examine the representation of a type with user defined
equality is, conceptually, non-deterministic (because any particular value
of such a type may have multiple representations.)  This means that Mercury
has to place certain restrictions on such code to ensure that programs have
sound semantics.

The constraints are these and they apply to deconstructions of values whose
types have user defined equality:
\begin{itemize}
\item such a deconstruction that does not appear as a ``discriminator'' in a
switch must be guaranteed to succeed (\ie you can't have non-switching
deconstructions if the type has more than one data constructor);
\item a switch whose ``discriminators'' are such deconstructions must be
exhaustive (\ie there must be one arm of the switch for every data
constructor);
\item deconstructions or switches of this sort have determinism @cc_multi@
(see chapter \XXX{} on advanced modes.)
\end{itemize}
Code with determinism @cc_multi@ cannot be backtracked into (the compiler
will report an error), which makes user defined equality quite restrictive.

The following, for instance, is in error:
\begin{myverbatim}
:- pred set(T) `contains` T.
:- mode in     `contains` in is semidet.

set(Xs) `contains` X :- list.member(X, Xs).
\end{myverbatim}
@set/1@ has only the one data constructor, @set/1@, hence the deconstruction
of the left hand argument in the head is guaranteed to succeed.
Since @set/1@ has user defined equality this deconstruction has determinism
@cc_multi@ and cannot be backtracked over.
@list.member/2@ is @semidet@ and therefore \emph{may} cause the program to
backtrack over the deconstruction.  Consequently this predicate definition
is invalid.

The way around the problem is as follows:
\begin{myverbatim}
:- pred set(T) `contains` T.
:- mode in     `contains` in is semidet.

Set `contains` X :-
    promise_only_one_solution(contains_2(Set, X)) = yes.

:- pred contains_2(set(T), T,  bool).
:- mode contains_2(in,     in, out) is cc_multi.

contains_2(set(Xs), X, Result) :-
    Result = ( if list.member(X, Xs) then yes else no ).
\end{myverbatim}
The built-in function @promise_only_one_solution/1@ takes a @cc_multi@
closure with one output as its argument (the closure
@contains_2(Set, X)@ has type @pred(bool)@ since the first two arguments
have already been supplied) and returns the result as if the computation
were deterministic.  It is up to the programmer to ensure that the promise
is correct -- rest assured that strange bugs will eventually appear in your
program if you lie to the compiler.

(@contains_2@ is valid because its definition cannot fail and hence will
never backtrack over the deconstruction of its first argument.)

In summary, types with user-defined equality can be very useful.  However,
the programmer has to write in a somewhat constrained style to make it work.

\section{Types with User Defined Comparison}

\XXX{This could conceivably go in the chapter on advanced modes.}

\XXX{Want to add a paragraph explaining that this is black-belt stuff and
can be skimmed over by the faint of heart.}

Just as it is sometimes useful to specify a particular equality relation for
a type, it can also be useful to specify a particular ordering relation.

The example we will use this time is for an alternative representation for
strings that favours fast -- $O(1)$ -- concatenation:
\begin{myverbatim}
:- type str ---> string(string) ; str ++ str.

:- func to_string(str) = string.

to_string(Str) = string.append_list(to_string_list(Str, [])).

:- func to_string_list(str, list(string)) = list(string).

to_string_list(string(S),    Ss) = [S | Ss].
to_string_list(StrA ++ StrB, Ss) =
    to_string_list(StrA, to_string_list(StrB, Ss)).
\end{myverbatim}
So we use @string(S)@ to turn the string @S@ into a @str@ and
@StrA ++ StrB@ to denote the concatenation of @StrA@ and @StrB@.  The
function @to_string/1@ converts @str@ values into ordinary strings.

A natural thing to want to do is to compare (representations of) strings.
Unfortunately for our purposes, the default Mercury ordering has
\begin{myverbatim}
    s("ZYX")  @<  s("A") ++ s("BC")
\end{myverbatim}
rather than the other way around as we would prefer.

Mercury allows us to provide our own comparison predicate for @str@ values
to be used in place of the built-in ordering:
\begin{myverbatim}
:- type str ---> string(string) ; str ++ str
    where comparison is compare_str.

:- pred compare_str(comparison_result, str, str).
:- mode compare_str(out,               in,  in ) is det.

compare_str(O, StrA, StrB) :-
    O = ordering(to_string(StrA), to_string(StrB)).
\end{myverbatim}
The type @comparison_result@ is built-in and defined thus:
\begin{myverbatim}
:- type comparison_result ---> (<) ; (=) ; (>).
\end{myverbatim}
The requirement for a three-place predicate with the comparison result
argument being computed in the first argument is due to historical reasons.
The built-in function @ordering/2@ is just a more convenient way of calling
the built-in predicate @compare/3@:
\begin{myverbatim}
ordering(A, B) = O :-
    compare(O, A, B).
\end{myverbatim}
Supplying a user-defined comparison relation causes Mercury to use the
given predicate rather than the default ordering.

It is the programmer's duty to ensure that the user defined comparison
predicate does define a total ordering and agrees with the equality
relation.
That is
\begin{itemize}
\item #ordering(A, B)# is always defined,
\item it is always the case that #ordering(A, A) = (=)#,
\item if #A @< B# then #B @> A# and vice versa,
\item if #A @< B# and #B @< C# then #A @< C# (and similarly for #@>#), and
\item if #A = B# then #ordering(A, B) = (=)# and vice versa.
\end{itemize}

Finally, it is possible to specify both an equality predicate and a
comparison predicate for the same type:
\begin{myverbatim}
:- type str ---> string(string) ; str ++ str
    where equality is equals_str, ordering is compare_str.
\end{myverbatim}
This is occasionally useful when the equality test can be implemented more
cheaply than comparison.

\section{Conclusion}

\XXX{Some kind of summary.}

\section{Function and Predicate Types}

\XXX{To go in the HO chapter.}

Functions and predicates are first class citizens in the Mercury type
system: they can be passed as arguments, returned as results
and stored in other data structures.  This aspect of Mercury is
discussed in detail in chapter \XXX{} on higher order programming.

Function and predicate types look just like the corresponding @func@ and
@pred@ declarations with the name is omitted.
The @func@ declaration for @math.sqrt/1@, for example, is given as
\begin{myverbatim}
:- func sqrt(float) = float.
\end{myverbatim}
The type of @math.sqrt/1@ is therefore @func(float) = float@.

The @pred@ declaration for @set.contains/2@, which tests whether a
particular value is a member of a set, is declared as
\begin{myverbatim}
:- pred contains(set(T), T).
\end{myverbatim}
so @set.contains/2@ has the type @pred(set(T), T)@.
The @T@ is a \emph{type parameter} (about which we will say more
in a little while) meaning we can substitute any type at all for @T@ and
the predicate will still work.

\XXX{Since this will be in the HO chapter, it needs rewording.}

Although we haven't yet covered higher order programming (see chapter
\XXX{}), we should still be able to understand higher order type
signatures.  @string.words/2@, for instance, has the type
\begin{myverbatim}
    func(pred(char), string) = list(string)
\end{myverbatim}
indicating that its first argument is a predicate over characters.
(Note that the \emph{type} of a function or predicate says nothing about
its \emph{mode} -- which arguments are inputs and its determinism and so
forth.  Modes are the subject of chapter \XXX{}.)

\section{Abstract Types}

\XXX{To go in the modules chapter.}

It's not always a good to reveal the \emph{definition} of a type to its
users.  For one thing, doing so means one cannot later change the
definition of the type without the risk of breaking other programs that use
it.  It is a good engineering principle to hide 
implementation detail from users wherever possible.

The @dict/2@ dictionary data type we defined in chapter \XXX{} on basic
types is a good candidate for a library module that we can re-use over and
over again.  The implementation details of @dict/2@ are not important to the
users of the module: they will only be interested in the functionality on
offer.  Furthermore, we want the freedom to change the definition of
@dict/2@ at some later point if we find a more efficient representation --
and we want to be able to do so without requiring existing users to make
changes to their code.

The solution to the problem is to make @dict/2@ an \emph{abstract type}.
\begin{myverbatim}
:- module dict.
:- interface.


:- type dict(K, V).     % Abstract type.

:- func new_dict = dict(K, V).

:- func insert(K, V, dict(K, V)) = dict(K, V).

:- pred lookup(dict(K, V), K,  V  ).
:- mode lookup(in,         in, out) is semidet.


:- implementation.


:- type dict(K, V)      % Definition of abstract type.
    --->    empty
    ;       branch(K, V, dict(K, V), dict(K, V)).

new_dict = empty.

...
\end{myverbatim}
The line @:- type dict(K, V).@ declares @dict/2@ to be an abstract type:
an abstract type is just one whose type declaration omits the definition.
The full type definition instead appears in the implementation section.

Since the users of an abstract type have no idea how it is defined, they can
neither create values of that type directly via construction nor examine
them with pattern matching (deconstruction).  The only way to manipulate
such values is to use the operations exported by the module defining the
abstract type.

This module includes the function @new_dict@ so that users can create new,
empty @dict/2@ values in the first place (functions of arity zero are just
fixed values.)
Other than that, users can only extend @dict/2@ values with @insert/3@ and
examine them with @lookup/3@.
The following is therefore wrong:
\begin{myverbatim}
:- import_module dict.

this_wont_compile :-
    Dict0 = empty,
    Dict1 = branch("Gateau", 'x', Dict0, Dict0),
    ...
\end{myverbatim}
because the implementation section of the @dict@ module is the \emph{only}
place that knows what the @dict/2@ data constructors are.  This, however, 
is fine:
\begin{myverbatim}
:- import_module dict.

this_will_compile :-
    Dict0 = new_dict,
    Dict1 = dict.insert("Gateau", 'x', Dict0),
    ...
\end{myverbatim}

% % vim: ft=tex ff=unix ts=4 sw=4 et wm=8 tw=0

\chapter{Predicates}

The key computational unit in Mercury is the predicate.  A
predicate describes a relationship between its arguments; it is
also possible to provide a procedural view of Mercury predicates,
which is how they can be used in a programming language.

Predicates cover not only tests and functions, as found in other
languages, but also procedures with multiple return values and
even non-deterministic relationships, in which there may be
several different solutions for a given set of input arguments.

\section{Introduction}

A predicate is defined by a set of \emph{clauses}, where each
clause takes the form
\begin{verbatim}
Head :- Body.
\end{verbatim}
This should be read as saying ``@Head@ is true if @Body@ is true'' with
the set of clauses forming an exhaustive specification of the
predicate.

If the body of a predicate is empty (taken to mean just @true@), one can
just write
\begin{verbatim}
    Head.
\end{verbatim}
Clauses like this are called \emph{facts}.

The body of a predicate clause is a \emph{goal} -- a logical formula
that must be satisfied in order for the head to hold.

A very simple example of a predicate is
\begin{verbatim}
:- pred max(int, int, int).         % A, B, Max.
:- mode max(in,  in,  out) is det.

max(A, B, Max) :-
    ( if B < A then Max = A else Max = B ).
\end{verbatim}
This predicate takes two integers, @A@ and @B@, and succeeds
binding @Max@ to @A@ if @A@ is greater than @B@ or binding @Max@ to @B@
otherwise.  The part of the mode declaration @is det@ means
that this is a deterministic predicate: it will always succeed
and has just one solution for any given pair of inputs.

(This sort of deterministic predicate with a single output is
called a \emph{function}.  Functions are so common that Mercury
includes special syntax and conventions to simplify working
with them.  More of this in a later section \XXX{}.)

For a more interesting example, consider the following small
genealogical database of people defined by the @person@ type:
\begin{verbatim}
:- type person
    --->    arthur  ; bill      ; carl
    ;       alice   ; betty     ; cissy.
\end{verbatim}
We start by defining predicates that we can use to decide if a
particular person is male or female.  These predicates are
labelled semidet because they have at most one solution
(success) for a given argument, but might also fail.
\begin{verbatim}
:- pred male(person).
:- mode male(in) is semidet.

male(arthur).
male(bill).
male(carl).

:- pred female(person).
:- mode female(in) is semidet.

female(Person) :- not male(Person).
\end{verbatim}
The definition for @female/1@ states that @female(Person)@ succeeds
iff\footnote{\emph{Iff}: if and only if} @male(Person)@ fails.  (Recall
that variables in Mercury are distinguished by starting with a capital
letter.)
\begin{verbatim}
:- pred father(person, person).     % Child, Father.
:- mode father(in,     out) is semidet.

father(betty, arthur).
father(carl,  bill).
father(cissy, bill).

:- pred mother(person, person).     % Child, Mother.
:- mode mother(in,     out) is semidet.

mother(bill,  alice).
mother(carl,  betty).
mother(cissy, betty).
\end{verbatim}
The predicates @father/2@ and @mother/2@ take their first argument
as an input and return a result in the second.  The
determinism is semidet again because they are not exhaustive:
some inputs can cause them to fail (\eg there is no solution
for @father(arthur, X)@.)
\begin{verbatim}
:- pred parent(person, person).     % Child, Parent.
:- mode parent(in,     out) is nondet.

parent(Child, Parent) :- father(Child, Parent).
parent(Child, Parent) :- mother(Child, Parent).
\end{verbatim}
This simply says that the parent of a child is either the
father or the mother.  The determinism is nondet because for a
given @Child@ argument this predicate may fail or may have more
than one solution:
\begin{itemize}
\item both @father/2@ and @mother/2@ can fail
(\eg if @Child = arthur@);
\item just one of @father/2@ or @mother/2@ could succeed
(\eg if @Child = bill@);
\item both @father/2@ and @mother/2@ could succeed
(\eg if @Child = cissy@).
\end{itemize}

(How failure and multiple solutions affect the execution of a
Mercury program will be explained below.)
\begin{verbatim}
:- pred ancestor(person, person).   % Person, Ancestor.
:- mode ancestor(in,     out) is nondet.

ancestor(Person, Ancestor) :-
    parent(Person, Ancestor).

ancestor(Person, Ancestor) :-
    parent(Person, Parent),
    ancestor(Parent, Ancestor).
\end{verbatim}
We can now define an ancestor of a Person as being either
\begin{itemize}
\item a parent of that Person or
\item an ancestor of a parent of that Person.
\end{itemize}

Since @parent/2@ is nondet, so is @ancestor/2@.

This table explains the difference between the number of
solutions a predicate can have for a given determinism:

\begin{tabular}{lll}
\textbf{Determinism}       & \multicolumn{2}{l}{\textbf{Number of Solutions}} \\
            & \textbf{Min} & \textbf{Max} \\
\hline \\
@semidet@   & 0            & 1 \\
@nondet@    & 0            & 1 or more \\
@det@       & 1            & 1 \\
@multi@     & 1            & 1 or more \\
\end{tabular}

(There are a few other determinisms, but we don't need to
consider them just yet \XXX{}.)

The compiler will check that your determinism declarations are
correct.

One interesting thing about this database is that there's no
reason why it shouldn't be run in ``reverse''.  That is, for any
father or mother, we should be able to deduce who their
children are and for any ancestor we should be able to work
out who their descendants are.

To gain this extra functionality we have only to add the
required extra mode declarations; there is no need to touch
the definitions themselves!  The extra mode declarations are
\begin{verbatim}
:- mode father(out, in) is nondet.   % Infer Child from Father.
:- mode mother(out, in) is nondet.   % Infer Child from Mother.
:- mode parent(out, in) is nondet.   % Infer Child from Parent.
:- mode ancestor(out, in) is nondet. % Infer Person from Ancestor.
\end{verbatim}
Notice that @father/2@ and @mother/2@ are nondet in this mode
rather than semidet: looking at the definitions we see that
@father(Child, bill)@ has multiple solutions @Child = carl@ and
@Child = sissy@, while @mother(Child, betty)@ also has solutions
@Child = carl@ and @Child = sissy@.

There is no reason why a predicate cannot have several output
arguments or even no input arguments.  For example, we
might go on to add
\begin{verbatim}
:- pred parents(person, person, person).    % Child, Mother, Father.
:- mode parents(in,     out,    out   ) is semidet.
:- mode parents(out,    in,     in    ) is nondet.

parents(Child, Mother, Father) :-
    mother(Child, Mother),
    father(Child, Father).
\end{verbatim}
The first mode of @parents/3@ is semidet because both @mother/2@
and @father/2@ are semidet when called with @Child@ as an input
and both must succeed for @parents/3@ to succeed.

The second mode is nondet for similar reasons, but exhibits an
interesting property.  At first glance you might think that
something will go wrong here: when @parents/3@ is called in the
second mode, initially @Mother@ and @Father@ are inputs and @Child@
is an output.  However, if the call to @mother/2@ succeeds, then
@Child@ will end up being bound to the identifier for some
person.  This appears that we would then be trying to call
@father/2@ with \emph{both} arguments as inputs, but there is no such
mode declaration for @father/2@.  There is no need to worry,
however -- the compiler will recognise the situation and treat
the definition of @parents/3@ like this as far as the second
mode is concerned:\footnote{The compiler isn't doing any special analysis here;
this is just a natural consequence of conversion to
super-homogeneous normal form, which will be explained later
on \XXX{}.}
\begin{verbatim}
:- mode parents(out, in, in) is nondet.

parents(Child, Mother, Father) :-
    mother(Child, Mother),
    father(X,     Father),
    X = Child.
\end{verbatim}
So here @father/2@ is being called in mode @(out, in) is nondet@ and
@parents/3@ succeeds if the result @X@ is bound to the same value
as @Child@.

In effect the compiler has deduced that @father/2@ has the
implied mode
\begin{verbatim}
:- mode father(in, in) is semidet.
\end{verbatim}
In general, Mercury will reorder each clause body for each mode
declaration so that results are generated before they are needed -- each
mode of a predicate is referred to as a \emph{procedure}.

\section{Execution}
This section explains Mercury's execution model in more
detail.  In particular, the notions of failure, backtracking
and non-determinism are addressed.

Let's start by looking at some of the code for our
genealogical database again -- this time we're going to label
the clauses to help us see how programs execute in Mercury.
\begin{verbatim}

    :- pred father(person, person).     % Child, Father.
    :- mode father(in,     out   ) is semidet.
    :- mode father(out,    in    ) is nondet.

f1  father(betty, arthur).
f2  father(carl,  bill).
f3  father(cissy, bill).

    :- pred mother(person, person).     % Child, Mother.
    :- mode mother(in,     out   ) is semidet.
    :- mode mother(out,    in    ) is nondet.

m1  mother(bill,  alice).
m2  mother(carl,  betty).
m3  mother(cissy, betty).

    :- pred parent(person, person).     % Child, Parent.
    :- mode parent(in,     out   ) is nondet.
    :- mode parent(out,    in    ) is nondet.

p1  parent(Child, Parent) :- father(Child, Parent).
p2  parent(Child, Parent) :- mother(Child, Parent).

    :- pred ancestor(person, person).   % Person, Ancestor.
    :- mode ancestor(in,     out   ) is nondet.

a1  ancestor(Person, Ancestor) :-
        parent(Person, Ancestor).

a2  ancestor(Person, Ancestor) :-
        parent(Person, Parent),
        ancestor(Parent, Ancestor).

    :- pred parents(person, person, person).    % Child, Mother, Father.
    :- mode parents(in,     out,    out   ) is semidet.
    :- mode parents(out,    in,     in    ) is nondet.

s1  parents(Child, Mother, Father) :-
        mother(Child, Mother),
        father(Child, Father).
\end{verbatim}
First off, we'll consider the goal @parent(cissy, Parent)@.
Every time we see a goal we try expanding it according to each
clause definition in turn (this is what being referentially
transparent is all about).  If we get to a dead end we have to
\emph{backtrack} to the last point where we had a choice and try a
different clause (when we backtrack we also forget any
variable bindings we might have made one the way from the
/choice point/.)
\begin{verbatim}
        parent(cissy, Parent)
(by p1)     father(cissy, Parent)
(by f1)         false because betty \= cissy
(by f2)         false because carl  \= cissy
(by f3)         Parent = bill
\end{verbatim}
So one solution to @parent(cissy, Parent)@ is @Parent = bill@.  We
obtained this by first expanding the goal according to p1 and
then expanding the resulting goal @father(cissy, Parent)@
according to each of @f1@, @f2@ and @f3@ until we found one that
succeeded.

If the program later has to backtrack over this goal then we
have to forget the binding @Parent = bill@ and try the next
clause for @parent/2@ (since there are no more clauses for
@father/2@):
\begin{verbatim}
        parent(cissy, Parent)
(by p1)     mother(cissy, Parent)
(by m1)         false because bill \= cissy
(by m2)         false because carl \= cissy
(by m3)         Parent = betty
\end{verbatim}
Now, say we wanted to ask the question ``which ancestor of
@carl@, if any, is also a parent of @bill@?''  Here's how things
would proceed:\footnote{We have to supply fresh local vars in each
 expansion -- for instance, @Ancestor@ in the expansion of @a2@ has
 been replaced with a new variable @Z@.}
\begin{verbatim}
        ancestor(carl, X), parent(bill, Y), X = Y
(by a1)     parent(carl, X), parent(bill, Y), X = Y
(by p1)         father(carl, X), parent(bill, Y), X = Y
(by f1)             false because betty \= carl
(by f2)             X = bill, parent(bill, Y), X = Y
(  =  )             parent(bill, Y), bill = Y
(by p1)                 father(bill, Y), bill = Y
(by f1)                     false because bill \= betty
(by f2)                     false because bill \= carl
(by f3)                     false because bill \= cissy
(by p2)                 mother(bill, Y), bill = Y
(by m1)                     Y = alice, bill = Y
(  =  )                     bill = alice
(  =  )                     false because bill \= alice
(by m2)                     false because bill \= carl
(by m3)                     false because bill \= cissy
(by p2)         mother(carl, X), parent(bill, Y), X = Y
(by m1)             false because carl \= bill
(by m2)             X = betty, parent(bill, Y), X = Y
(  =  )             parent(bill, Y), betty = Y
(by p1)                 father(bill, Y), betty = Y
(by f1)                     false because bill \= betty
(by f2)                     false because bill \= carl
(by f3)                     false because bill \= cissy
(by p2)                 mother(bill, Y), betty = Y
(by m1)                     Y = alice, betty = Y
(  =  )                     betty = alice
(  =  )                     false because alice \= betty
(by m2)                     false because bill \= carl
(by m3)                     false because bill \= cissy
(by a2)     parent(carl, Z), ancestor(Z, X), parent(bill, Y), X = Y
(by p1)         father(carl, Z), ancestor(Z, X), parent(bill, Y), X = Y
(by f1)             false because carl \= betty
(by f2)             Z = bill, ancestor(Z, X), parent(bill, Y), X = Y
(  =  )             ancestor(bill, X), parent(bill, Y), X = Y
(by a1)                 parent(bill, X), parent(bill, Y), X = Y
(by p1)                     father(bill, X), parent(bill, Y), X = Y
(by f1)                         false because bill \= betty
(by f2)                         false because bill \= carl
(by f3)                         false because bill \= cissy
(by p2)                     mother(bill, X), parent(bill, Y), X = Y
(by m1)                         X = alice, parent(bill, Y), X = Y
(  =  )                         parent(bill, Y), alice = Y
(by p1)                             father(bill, Y), alice = Y
(by f1)                                 false because bill \= betty
(by f2)                                 false because bill \= carl
(by f3)                                 false because bill \= cissy
(by p2)                             mother(bill, Y), alice = Y
(by m1)                                 Y = alice, alice = Y
(  =  )                                 alice = alice
(  =  )                                 true
\end{verbatim}
So our program concludes that a solution to
\begin{verbatim}
    ancestor(carl, X), parent(bill, Y), X = Y
\end{verbatim}
is
\begin{verbatim}
    X = alice, Y = alice
\end{verbatim}
As we indicated earlier in the definition of @parents/3@, in
practice we'd be more likely to write the goal as just
\begin{verbatim}
    ancestor(carl, X), parent(bill, X)
\end{verbatim}
and let Mercury work out where the implicit unification should
go.

\Aside{The above example might give the impression that Mercury is a term
rewriting system.  This is not true (although conceivably a very
slow Mercury implementation might use such a technique\ldots)  What
happens behind the scenes is much more efficient, albeit
computationally equivalent.}

In practice, ninety per cent or so of the code one writes is
deterministic.  However, there are times (as in the example above) when
non-determinism can be used to write very clear, concise programs.
Support for backtracking and unification is what distinguishes Mercury
from purely functional languages.

\section{Recursion}

Imperative languages like C and Java provide a whole slew of
mechanisms for supporting iterative (looping) control flow --
while loops, repeat-until loops, for loops and so forth.

Declarative languages typically provide but one mechanism for
such things: \emph{recursion}.  A recursive predicate is simply one
defined in terms of itself \XXX{Don't forget to mention mutual
recursion}.  (Later on we will discover \emph{higher order} predicates
and observe that the standard library supplies predicates that stand in
for the common iterative mechanisms found in imperative languages.)

Some simple examples:
\begin{verbatim}
:- func factorial(int) = int.

factorial(N) =
    ( if N  = 0 then 1 else N * factorial(N - 1) ).


:- func fibonacci(int) = int.

fibonacci(N) =
    ( if N =< 2 then 1 else fibonacci(N - 1) + fibonacci(N - 2) ).
\end{verbatim}
These are examples of what is sometimes called ``middle
recursion'' where the recursive call is \emph{not} the last thing
that the predicate (or function) does when looping.  Here,
calls to @factorial/1@ finishes with a multiplication and calls
to @fibonacci/1@ finish with an addition.

Although middle recursion is easy to read, it incurrs some
performance penalty in that each iteration has to record
something extra on the stack in order to finish up the
computation.\footnote{That said, the compiler can in fact turn some
middle recursive predicates into equivalent tail recursive
predicates that do not incurr a stack overhead \XXX{}.}

\XXX{Include a side-bar or something explaining stack frames.}

\emph{Tail recursion} describes the situation where the last thing a
predicate does is call itself.  Since there is nothing left to
do after the call, the compiler can reuse the current call's
stack frame for the recursive call, allowing the predicate to
execute using only a fixed amount of stack space.  For
example, tail recursive implementations of the above
functions are
\begin{verbatim}
factorial(N) = fac(N, 1).

fac(N, X) =
    ( if N = 0 then X     else fac(N - 1, N * X) ).


fibonacci(N) = fib(1, 1, N).

fib(FN_2, FN_1, N) =
    ( if N =< 2 then FN_1 else fib(FN_1, FN_1 + FN_2, N - 1) ).
\end{verbatim}
Tail recursive code like this should execute just as quickly
and efficiently as a for or while loop in an imperative
language.

\section{Unifications}

Mercury has but two basic atomic (\ie indivisible) types of
goal: unifications and calls.

A unification is written @X = Y@.  A unification can fail if @X@
and @Y@ are not unifiable.

Two values are unifiable if they are ``structurally similar'' --
that is, where you see a data constructor in one, you either
see the same data constructor in the other (and the
corresponding arguments are also structurally similar) or a
variable, which will end up being bound to the corresponding
term on the other side if the unification is successful.\footnote{Unlike Prolog, Mercury forbids the aliasing of
variables whereby a partially instantiated data structure
may contain the same unbound variable in two different places.
This will be explained fully in a later chapter.  \XXX{}}

Unifications, therefore, can be used to bind variables to
values, test to see if a variable is bound to a particular
constructor, unpack the arguments of a constructor or all of
the above.

In the following examples we assume that variables are
initially unbound:

\begin{tabular}{rcll}
         @123@ & @=@ & @123@ &
                -- Succeeds \\
           @X@ & @=@ & @123@ &
                -- Binds @X = 123@ \\
         @123@ & @=@ & @234@ &
                -- Fails \\
           @X@ & @=@ & @foo(1, 2, 3)@ &
                -- Binds @X = foo(1, 2, 3)@ \\
@foo(X, Y, Z)@ & @=@ & @foo(1, 2, 3)@ &
                -- Binds @X = 1@, @Y = 2@, @Z = 3@ \\
@foo(X, Y, 4)@ & @=@ & @foo(1, 2, 3)@ &
                -- Fails (@4 \= 3@) \\
@foo(X, 2, 3)@ & @=@ & @foo(1, 2, Z)@ &
                -- Binds @X = 1@, @Z = 3@ \\
\end{tabular}

The complex unifications are most easily understood by first
converting them into \emph{super homogeneous normal form} (which is
what the compiler does.)  In SHNF, the left hand side of each
unification is a variable and the right hand side is either
another variable or a functor \XXX{have I explained this term?}, all of
whose arguments are variables.  For example

\begin{verbatim}
    foo(X, 2, 3) = foo(1, 2, Z)
\end{verbatim}
becomes
\begin{verbatim}
    V_1 = X, V_2 = 2, V_3 = 3, V_4 = foo(V_1, V_2, V_3),
    V_5 = 1, V_6 = 2, V_7 = Z, V_8 = foo(V_5, V_6, V_7),
    V_4 = V_8
\end{verbatim}
where @V_1@\ldots@V_8@ are all temporary variables
introduced in the conversion to SHNF.

\section{Calls}
The other kind of primitive goal supported by Mercury is the
predicate call, @p(X1, ..., Xn)@, which we have already seen in
the examples above.

One way to picture the evaluation of a call is to think of it
as expanding into the different clause bodies for the
predicate definition while looking for solutions.

Two relatively important built-in predicates are @true@ and
@false@ \XXX{}.  @true@ always succeeds and @false@ always fails.\footnote{@false@ is often written as @fail@, which is a hangover
from Mercury's Prolog roots where it was sometimes more useful
to think in procedural terms.}

\section{Conjunction}

A goal of the form @G1, G2, ..., Gn@ is called a \emph{conjunction}
with the separating commas read as ``and''.  A conjunction
succeeds iff a consistent solution to each of the sub-goals @G1@,
@G2@, \ldots, @Gn@ can be found by the program.

(The compiler may have to reorder the sequence of goals in a
clause in order to satisfy the mode constraints.  Although one
can set a flag to force the compiler to do no more reordering
than is necessary, in general this will mean that certain
optimizations will not be possible.  The upshot of this is
that one should avoid writing code that assumes a particular
evaluation order other than that dictated by the mode
constraints.)

A conjunction executes by trying each of the goals in order.
If a goal fails then the program backtracks to the nearest
preceding choice-point (\ie non-deterministic goal that may
have other solutions).

\section{Negation}

A goal of the form not @G@ succeeds iff @G@ has no solutions.  @G@
may be a compound goal, in which case it should be enclosed in
parentheses to avoid syntactic precedence problems.

The sub-goal @G@ is said to be in a \emph{negated context} and as
such cannot bind any variables visible outside the negation
(since the only way not @G@ can succeed is if @G@ fails, in which
case it will not produce any variable bindings.)

Note that not not @G@ is equivalent to @G@ and hence may bind
variables if it succeeds (while not not @G@ would be an odd
thing to write, some of the code transformations the compiler
performs can generate such things; getting the modes right for
such things requires that the compiler observe this
simplification rule.)

\section{If-Then-Else Goals}

Mercury's if-then-else construct looks like this:
\begin{verbatim}
    ( if ConditionGoal then YesGoal else NoGoal )
\end{verbatim}
Note that the else part is \emph{not} optional.  \XXX{Except in DCG
code...  But we probably don't need to mention this.}

You may also see if-then-elses written as
\begin{verbatim}
    ( ConditionGoal -> YesGoal ; NoGoal )
\end{verbatim}
although the author finds this style less appealing.

While the parentheses are not always required, it is a very
good idea to include them in order to avoid confusing
syntactic precedence errors.  One common exception is a chain
of if-then-elses where only the top level of parentheses are
necessary:
\begin{verbatim}
    ( if      ConditionGoal1 then YesGoal1
      else if ConditionGoal2 then YesGoal2
      else if ConditionGoal3 then YesGoal3
      ...
      else                        NoGoal
    )
\end{verbatim}
Extra parentheses are not required even if any of the @ConditionGoal@s,
@YesGoal@s or the @NoGoal@ are compound goals.

The if-then-else goal
\begin{verbatim}
    ( if ConditionGoal then YesGoal else NoGoal )
\end{verbatim}
is semantically equivalent to the disjunction
\begin{verbatim}
    ( ConditionGoal, YesGoal ; not ConditionGoal, NoGoal )
\end{verbatim}
but will be implemented more efficiently by the compiler (if
@ConditionGoal@ produces no solutions in the first disjunct then
there's no point in checking again that it has none in the
second disjunct.)

It's worth looking at a few examples to really understand how
if-then-else goals work.  Again, we assume that all variables
are initially unbound:
\begin{verbatim}
    ( if ( X = 1 ; X = 2 ) then ( X = 2 ; X = 4 ) else X = 5 )
\end{verbatim}
The above goal is @nondet@ (the condition is @multi@
and the then-goal is @nondet@, since @X@ will be bound at this point),
but has the single solution @X = 2@.
\begin{verbatim}
    ( if ( X = 1 ; X = 2 ) then ( X = 3 ; X = 4 ) else X = 5 )
\end{verbatim}
The above goal is also @nondet@ for the same reason, but in fact has no
solutions (the compiler can only be expected to perform a certain amount
of program analysis and will sometimes not be completely precise --
mode inference is actually undecidable in general, although this is not
a problem in practice.  \XXX{Check this is true!})
\begin{verbatim}
    ( if 1 = 2 then ( X = 3 ; X = 4 ) else ( X = 5 ; X = 6 ) )
\end{verbatim}
The above goal is @nondet@ because the condition is @semidet@ and the
then- and else-goals are
@multi@.  (Since the condition will fail, the
else-goal is evaluated, with solutions @X = 5@ and @X = 6@.)
\begin{verbatim}
    ( if 1 = 1 then ( X = 3 ; X = 4 ) else ( X = 5 ; X = 6 ) )
\end{verbatim}
Similary, the above goal is @nondet@ because the condition is @semidet@
and the then- and else-goals are
@multi@.  (Since the condition will succeed, the
else-goal is evaluated, with solutions @X = 3@ and @X = 4@.)

That said, in the vast majority of cases where the
condition-goal is semidet and the then- and else-goals are
deterministic, if-then-else goals will act in very much the
same way as similar structures in other programming languages.

Since the condition-goal is in a negated context in the else-arm
of the disjunctive form of an if-then-else, it cannot
produce any outputs that would be used in @NoGoal@ or anything
outside the if-then-else as a whole.  It can, however, produce
outputs that are only used in the @YesGoal@.  (The reason for
this restriction is slightly subtle and will be explained in
more detail later.  \XXX{It's to do with having mode
independent semantics.})

For example, the following somewhat contrived code violates
the constraint because @S@ is bound inside the condition and is
also visible outside the if-then-else goal.
\begin{verbatim}

:- pred prime_divisor(int, int).
:- mode prime_divisor(in,  out) is nondet.
...

:- pred prime_divisor_or_zero(int, int).
:- mode prime_divisor_or_zero(in,  out) is multi.

prime_divisor_or_zero(N, S) :-
    ( if   prime_divisor(N, S)
      then true
      else S = 0
    ).

\end{verbatim}
The correct way to write @prime_divisor_or_zero/2@ is
\begin{verbatim}
prime_divisor_or_zero(N, S) :-
    ( if   prime_divisor(N, D)
      then S = D
      else S = 0
    ).
\end{verbatim}
Note that if-then-else \emph{expressions} are slightly different;
the following is perfectly legal:
\begin{verbatim}
prime_divisor_or_zero(N, S) :-
    S = ( if prime_divisor(N, D) then D else 0 ).
\end{verbatim}

\section{Disjunction}

Just as conjunction lets you use ``and'' to construct goals,
disjunction lets you use ``or''.  A \emph{disjunctive} goal takes the
form @(G1 ; G2 ; ... ; Gn)@ with the separating semicolons being
read as ``or''.

A disjunction succeeds iff any of its disjunct sub-goals
succeeds.  A disjunction has as many solutions as all of its
disjuncts put together: if one disjunct fails or backtracking
exhausts all the solutions for one disjunct then execution
proceeds with another disjunct.  Again, the compiler is
generally free to reorder disjuncts, although this should not
have a visible impact on programs.  Disjunctions are typically
non-deterministic, although switches, mentioned earlier, are a
special case.

\XXX{Need examples?}

The clausal notation we have been using in the examples above
is in fact convenient syntactic sugar for writing top-level
disjunctions.  For example, the @ancestor/2@ predicate we
defined earlier
\begin{verbatim}

ancestor(Person, Ancestor) :-
    parent(Person, Ancestor).

ancestor(Person, Ancestor) :-
    parent(Person, Parent),
    ancestor(Parent, Ancestor).
\end{verbatim}
could equivalently be written as
\begin{verbatim}
ancestor(Person, Ancestor) :-
    (
        parent(Person, Ancestor)
    ;
        parent(Person, Parent),
        ancestor(Parent, Ancestor)
    ).
\end{verbatim}
(Indeed, this is how the compiler sees multi-clause
definitions.)

In general it is better style to use clausal form for
top-level disjunctions.

\section{Switches}
Mercury recognises particular forms of (@semi-@)@det@
disjunction which it can compile very efficiently.

The @string@ library module defines the following type:
\begin{verbatim}
:- type poly_type
    --->    f(float)
    ;       i(int)
    ;       s(string)
    ;       c(char).
\end{verbatim}
which can be used to form heterogeneous collections of the
primitive types (this is useful, amongst other things, for
supplying argument lists for formatted output.)

Say we wanted to write a predicate that would convert any
@poly_type@ value into a @string@.  Here's how we might write the
code (in practice we would use a function; here we use a
predicate for the purposes of illustration):
\begin{verbatim}
:- pred poly_type_to_string(poly_type, string).
:- mode poly_type_to_string(in, out) is det.

poly_type_to_string(f(F), S) :- float_to_string(F, S)
poly_type_to_string(i(I), S) :- int_to_string(I, S)
poly_type_to_string(s(S), S).
poly_type_to_string(c(C), S) :- char_to_string(C, S)
\end{verbatim}
this is equivalent to the single clause definition
\begin{verbatim}
poly_type_to_string(Poly, S) :-
    (   Poly = f(F), float_to_string(F, S)
    ;   Poly = i(I), int_to_string(I, S)
    ;   Poly = s(S)
    ;   Poly = c(C), char_to_string(C, S)
    ).
\end{verbatim}

The compiler knows that since @Poly@ is an input variable it
must be bound when the disjunction is evaluated.  The compiler
also sees that each arm of the disjunction unifies @Poly@
against a different data constructor.  The compiler therefore
deduces that at most one disjunct can succeed on a particular
call (and, since all @poly_type@ data constructors are tested
for, \emph{exactly} one must succeed.)  The compiler generates very
efficient code for so-called \emph{switch} constructs such as this.\footnote{The name \emph{switch} is used because of its similarity
to the C language construct of the same name.}

Switches are an elegant way of describing conditions based on
unification tests and are typically more efficient than the
equivalent if-then-else chains.

\section{Existential Quantification}

Sometimes we only need to know whether a solution exists, but
are not interested in the result.  For this we use existential
quantification, which looks like this:
\begin{verbatim}
    (some [X, Y, Z] G)
\end{verbatim}
A goal of this form will succeed iff there is a solution to @G@,
but any bindings for @X@, @Y@ and @Z@ will not be visible outside
the quantification -- it's rather like saying that @X@, @Y@ and @Z@ 
are local variables for the goal @G@.

Mercury has an rule that any variables in a clause that do not
also appear in the head are implicitly existentially
quantified, which means you never actually need to use
explicit existential quantification in your programs.

\section{Universal Quantification}

On the other hand, we may wish to know whether a particular
property holds for all solutions to a particular goal.  This
is where universal quantification is useful.

The goal
\begin{verbatim}
    (all [X, Y, Z] G)
\end{verbatim}
is equivalent to writing
\begin{verbatim}
    not (some [X, Y, Z] not G)
\end{verbatim}

\section{Implication}

Mercury has three types of goal for describing implicative
relationships between goals.\footnote{The translations are given by de Morgan's laws.}
\begin{itemize}
\item @(G1 => G2)@ is shorthand for @(not G1 ; G2)@;
\item @(G1 <= G2)@ is shorthand for @(G2 => G1)@; and
\item @(G1 <=> G2)@ is shorthand for @((G1 => G2), (G1 <= G2))@.
\end{itemize}

Note that parentheses are required around @G1@ and @G2@ if they
are not atomic goals; it is a good idea to also put
parentheses around the implication as a whole to avoid
ambiguity in the scope of the implication.)

Implication is most often used with universal quantification
to test for some general property.

\subsection{Examples}

This example uses the predicate @member(X, Xs)@ to non-deterministically
project members @X@ from the @list@ @Xs@ and the semidet predicate
@even(X)@ which succeeds iff @X@ is even.\footnote{The convention is to
name a @list@ of items, @X@, as @Xs@.}
\begin{verbatim}
:- pred all_even(list(int)).
:- mode all_even(in) is semidet.

all_even(Xs) :-
    all [X] ( member(X, Xs) => even(X) ).

\end{verbatim}

\begin{verbatim}

    % Two list can be interpreted as equivalent sets if
    % each contains the same members as the other.
    %
:- pred equivalent_sets(list(T), list(T)).
:- mode equivalent_sets(in, in) is semidet.

equivalent_sets(Xs, Ys) :-
    all [Z] ( member(Z, Xs) <=> member(Z, Ys) ).
\end{verbatim}
The auxiliary predicates @member/2@ and @even/1@ are defined as\footnote{The Mercury parser views anything in @`@backquotes@`@
as an infix operator.  This is the same as the rule used in Haskell.}
\begin{verbatim}
:- pred member(T, list(T)).
:- mode member(out, in) is nondet.

    % X is a member of a list if it is either the head of
    % that list or a member of the tail.
    %
member(X, [X | _ ]).
member(X, [_ | Xs]) :- member(X, Xs).


:- pred even(int).
:- mode even(in) is semidet.

even(X) :- X `mod` 2 = 0.
\end{verbatim}

\section{Higher Order Application}

\XXX{I'll talk about this later.}

\section{Anonymous and Singleton Variables}

Often one is not interested in a particular output variable
from a call or unification.  In these cases you can use the
special variable named @_@ (a single underscore) which stands
for a different anonymous or ``don't care'' variable every time
it appears.

Sometimes, however, it makes programs easier to read if you do
name don't care variables.  Since variables that only appear
once in a clause are usually the result of typographical
error, the compiler will issue a warning when it sees such
things.  To get around this problem, giving a variable a
name that starts with an underscore (\eg @_X@) tells the compiler that
you know this is a named don't care variable and that there's
no need to issue a warning.




% \section{Functions}

Functions are @det@ (or @semidet@) relations with (at least) one output.

Essentially, a function is any relation with a single solution for
a given set of inputs.

Functions with single output values are so common that Mercury provides
special syntax to make working with them easier.  One of the key
advantages of functions is that they can be used as parts of
expressions, rather than having to have a separate goal computing each
subexpression in turn.  That is, one can use an expression as in
\begin{verbatim}
    X = (B + sqrt(B*B - 4*A*C)) / (2*A)
\end{verbatim}
rather than the somewhat verbose
\begin{verbatim}
    square(B, BSquared),
    multiply(4, A, FourA),
    multiply(FourA, C, FourAC),
    subtract(BSquared, FourAC, BSquaredMinusFourAC),
    sqrt(BSquaredMinusFourAC, Sqrt),
    add(B, Sqrt, Numerator),
    multiply(2, A, Denominator),
    divide(Numerator, Denominator, X)
\end{verbatim}

\subsection{Definition}

This example illustrates how functions are defined:
\begin{verbatim}
:- func length(list(T)) = int.
:- mode length(in) = out is det.

    % The length of an empty list is 0.
    % The length of a non-empty list is 1 for the head
    % plus the length of the tail.
    %
length([]      ) = 0.
length([_ | Xs]) = 1 + length(Xs).
\end{verbatim}
Like predicates, functions may be defined using several clauses
and make use of pattern matching.

Functions can be computed from goals, where the head and goal
are separated by @:-@ in the definition:
\begin{verbatim}
        % take(N, Xs) is the length min(N, length(Xs))
        % prefix of Xs.
        %
    :- func take(int, list(T)) = list(T).
    :- func take(in, in) = out is det.

    take(N, Xs) = Ys :-
        split(N, Xs, Ys, _).

        % drop(N, Xs) is the length max(0, length(Xs) - N)
        % suffix of Xs.
        %
    :- func drop(int, list(T)) = list(T).
    :- func drop(in, in) = out is det.

    drop(N, Xs) = Zs :-
        split(N, Xs, _, Zs).

        % split(N, Xs, Prefix, Suffix)
        %
    :- pred split(int, list(T), list(T), list(T)).
    :- mode split(in,  in,      out,     out    ) is det.

    split(N, Xs, Ys, Zs) :-
        ( if N > 0, Xs = [X | Xs0] then
            Ys = [X | Ys0],
            split(N - 1, Xs0, Ys0, Zs)
          else
            Ys = [],
            Zs = Xs
        ).
\end{verbatim}
By far the most common mode for a function is
\begin{verbatim}
:- mode f(in, in, ..., in) = out is det.
\end{verbatim}

If a function has (just) this sort of mode, then the mode
declaration can be ommitted and the compiler will simply assume
this mode is what is intended.  Hereafter we will omit unnecessary
mode declarations for functions.

\XXX{What exactly are the constraints on function
determinisms?  Remember to point out (somewhere) that
functions may also have multiple procedures.  See the ref.
manual section on Determinism.}

\subsection{Pattern Matching}

\XXX{Dealt with above.}

\subsection{Recursion}

\XXX{Dealt with above.}

\subsection{Conditional Expressions}

Conditional (if-then-else) expressions look like this:
\begin{verbatim}
    ( if ConditionGoal then YesExpr else NoExpr )
\end{verbatim}
where the usual rules for if-then-elses apply (in particular,
@ConditionGoal@ may bind output variables that are used in
YesExpr, but not elsewhere), except that the then and else
arms are \emph{expressions}, rather than goals.

Note that as with if-then-else goals, the else clause is \emph{not}
optional in a conditional expression (it would make no sense
not to have one.)

\subsection{* Partial (Semi-Deterministic) Functions}

\XXX{Leave for later.}

\subsection{Overview of Semidet Predicates}

\XXX{Deal with this in a later section.  It's sort-of advanced
stuff.}

\subsection{Polymorphism}

\XXX{Dealt with in section on types.}

\subsection{Infix Notation}

Mercury syntax supports a number of prefix, infix and postfix
operators, including all the usual arithmetic operators.  This
is just syntactic sugar, however, and there is no difference
between @X + Y@ and @+(X, Y)@ as far as the compiler is concerned.

If you want to use another name as an infix operator, you can
simply place it in @`@backquotes@`@:
\begin{verbatim}
    X `union` Y `union` Z
\end{verbatim}
is seen by the compiler as
\begin{verbatim}
    union(X, union(Y, Z))
\end{verbatim}
Backquoted symbols bind more tightly than anything else and
associate to the right.




% \section{Input and Output}

One unfortunate consequence of being a pure declarative language
is that IO becomes somewhat more complex than is the case for
imperative languages.

\subsection{IO \emph{Is} a Side-Effect}

One problem is that performing IO necessarily has an effect on the
outside world that cannot be backtracked over or undone -- there is
no returning to an earlier state of the world!  This is in
contrast to the mathematically pure world that Mercury inhabits,
where there is no concept of a value (such as the state of the
world) ``changing'', only one of new such values being computed.

\subsection{Order is Important}

Another problem is that since logically there is no difference
between the goal @(G1, G2)@ and the goal @(G2, G1)@, we also need to
find some mechanism for ensuring that IO operations happen in the
intended order and are not mixed up as a consequence of the
compiler reordering goals for optimization purposes and so forth.

\subsection{Uniqueness}

A number of solutions to the IO problem have been adopted by
the pure, declarative languages, the main contenders being the
monadic approach (as exemplified by Haskell) and the
uniqueness approach (as exemplified by Clean and Mercury.)

The uniqueness approach works like this: we view a Mercury
program as a function from world states to world states.  The
top-level @main/2@ predicate of a Mercury program takes the
current world state as an input and computes a new world state
as its result.  Every IO operation does the same thing: takes
a world state as input and produces a new world state as
output -- notionally updated with the effects of the IO
operation (and the actions of the world at large between IO
operations).  The so-called IO state is opaque to the Mercury
program; it can only be queried via the operations defined in
the io module.\footnote{Of course, the Mercury program doesn't actually
pass the state of the world around in fact -- the IO state
abstraction serves both to ensure the properties we require
and to give a semantics to IO in Mercury.}

Example (we eschew state variable notation here for clarity of
exposition):
\begin{verbatim}
:- pred main(io, io).
:- mode main(di, uo) is det.

main(IO0, IO) :-
    io__write_string("pi = ", IO0, IO1),
    io__write_float(math__pi, IO1, IO2),
    io__nl(                   IO2, IO ).
\end{verbatim}
The type of the IO state is called just io and is defined
as an abstract type in the @io@ library module.  The
top-level predicate @main/2@ takes the initial IO state as a
@di@ mode argument (\emph{destructive input}), and produces another as a
@uo@ mode result (\emph{unique output}).  The three IO operations in the body
show how the initial IO state, @IO0@, is transformed into
the final IO state, @IO@.  So, @io__write_string/3@ destroys
@IO0@ and produces @IO1@ as a unique output.  Next,
@io__write_float/3@ destroys @IO1@ and produces @IO2@ as a
unique output.  Finally, @io__nl@ (which writes out a
newline) destroys @IO2@ and produces @IO@ as a unique output.

In order to ensure that old versions of the world state cannot
be accessed after an IO operation, the IO state is \emph{unique} --
this means that there can only ever be one live reference to
it.\footnote{A live reference is one that will be used
later on in computation.}  The old version of the IO state
is said to be clobbered by an IO operation -- the compiler will
report an error if the old version is still live after the IO
operation in question.  Similarly, it is impossible to make a
copy the IO state.\footnote{It is possible to ``fork'' the IO
state, this is necessary to support concurrency.  Concurrency
is dealt with in a later chapter. \XXX{}}

Example:
\begin{verbatim}
:- pred main(io, io).
:- mode main(di, uo) is det.

main(IO0, IO) :-
    io__write_string("Hello, ",  IO0, IO1),
    io__write_string("world!\n", IO0, IO ).
\end{verbatim}
Here @IO0@ is used twice; the compiler spots the bug and
rejects the program with
\begin{verbatim}
In clause for `main(di, uo)':
  in argument 2 of call to predicate `io:write_string/3':
  unique-mode error: the called procedure would clobber
  its argument, but variable `IO0' is still live.
\end{verbatim}
The uniqueness constraint is sufficient to ensure that IO
operations happen in a strict order, specified by the
programmer, and that it is impossible to backtrack over IO or
refer to a dead IO state.

Note that uniqueness is not a property reserved for IO states:
it is used to implement destructively updated arrays, the
store data type which allows the construction of arbitrary
pointer graphs, hash tables and so forth.  Uniqueness allows
the compiler to use a safe form of destructive update of data
structures: there is no reason why a dead unique object cannot
be reused to create a new live unique object (since the old
value can never be accessed), which is exactly what happens
for the data types just mentioned.

\subsection{Stylistic Considerations}

Since passing the IO state around everywhere is a little
cumbersome and also quite restrictive (it can only be passed
into det predicates), Mercury programmers naturally find
themselves separating applications into two parts: the part
that handles all the IO and the part that handles all the
interesting processing.  This is good style in general, and
although one might find it slightly annoying not to be able to
insert print statements willy-nilly as is the case with impure
languages, one soon finds that the discipline imposed pays
real dividends in terms of reusability, maintainability, ease
of debugging and so forth.

\subsection{* Determinism Restrictions}

Since IO operations cannot be backtracked across, the IO state
(and, indeed, unique objects in general) cannot be passed to
non-deterministic predicates -- that is, only deterministic
predicates can take unique IO states as arguments.\footnote{This is not strictly true; there is another
determinism, cc-multi, which is compatible with uniqueness.}

\subsection{* State Variable Syntax}

Having to name and pass two variables around for every IO
operation quickly becomes tiresome.  Mercury has a special
\emph{state variable} syntax for just this purpose.  The idea is to
write code that looks a little more like what one would write
in an imperative language, but which is transformed by the
compiler into pure Mercury.  A state variable argument @!X@
stands for two real arguments, @!.X@ and @!:X@, which in turn
stand for the ``current'' and ``next'' values of the state variable
@X@, respectively.  Occurrences of @!.X@ and @!:X@ are converted by
the compiler into appropriately numbered variables.

For example, the following code
\begin{verbatim}
    % Writes out a list of strings, separated by
    % commas.
    %
:- pred write_strings(list(string), io, io).
:- mode write_strings(in,           di, uo) is det.

write_strings([],            !IO).

write_strings([S1],          !IO) :-
    io__write_string(S1,     !IO).

write_strings([S1, S2 | Ss], !IO) :-
    io__write_string(S1,     !IO),
    io__write_string(", ",   !IO),
    write_strings(Ss,        !IO).
\end{verbatim}
is transformed by the compiler into\footnote{Note that the pred and mode declarations reflect
the fact that @!IO@ is actually two arguments, not one.}
\begin{verbatim}
    % Writes out a list of strings, separated by
    % commas.
    %
:- pred write_strings(list(string), io, io).
:- mode write_strings(in,           di, uo) is det.

write_strings([],            IO0, IO) :-
    IO = IO0.

write_strings([S1],          IO0, IO) :-
    io__write_string(S1,     IO0, IO).

write_strings([S1, S2 | Ss], IO0, IO) :-
    io__write_string(S1,     IO0, IO1),
    io__write_string(", ",   IO1, IO2),
    write_strings(Ss,        IO2, IO ).
\end{verbatim}
Henceforth we will use state variable syntax rather than
explicitly numbered IO states.

\subsection{Common IO Operations}

The @io@ library module defines a plethora of useful IO
operations and as usual with libraries, the reader is
encouraged to take some time to peruse the interface section.
Here we will present some basic IO operations to help get the
ball rolling.

\subsubsection{Output}

Output is generally simpler to deal with than input,
because, by and large, there are no error codes to deal
with.

The @io@ library module includes predicates for the output
of the basic types:
\begin{verbatim}
:- pred io__write_string(string, io, io).
:- mode io__write_string(in,     di, uo) is det.

:- pred io__write_char(char, io, io).
:- mode io__write_char(in,   di, uo) is det.

:- pred io__write_int(int, io, io).
:- mode io__write_int(in,  di, uo) is det.

:- pred io__write_float(float, io, io).
:- mode io__write_float(in,    di, uo) is det.
\end{verbatim}
However, it's typically easier to use the more general
formatted output predicate:
\begin{verbatim}
:- pred io__format(string, list(poly_type), io, io).
:- mode io__format(in,     in,              di, uo) is det.
\end{verbatim}
The @string@ argument describes how the output is to be formatted, very
similar in spirit and detail to what one would pass to C's @printf()@.
The @list@ argument is a type safe means of passing the parameters to be
formatted.

Using @io__format/4@ one might write
\begin{verbatim}
:- pred write_record(string, int, float, io, io).
:- mode write_record(in,     in,  in,    di, uo) is det.

write_record(Name, Age, Children, !IO) :-
    io__format("%s is %d years old and has %f children.\n",
               [s(Name), i(Age), f(Children)], !IO).
\end{verbatim}
and then we could call
\begin{verbatim}
    write_record("Joe Bloggs", 43, 2.4, !IO)
\end{verbatim}
and the program would write out
\begin{verbatim}
Joe Bloggs is 43 years old and has 2.4 children.
\end{verbatim}
The implementation of @write_record/5@ is much simpler than the
functionally equivalent
\begin{verbatim}
:- pred write_record(string, int, float, io, io).
:- mode write_record(in,     in,  in,    di, uo) is det.

write_record(Name, Age, Children,           !IO) :-
    io__write_string(Name,                  !IO),
    io__write_string(" is ",                !IO),
    io__write_int(Age,                      !IO),
    io__write_string(" years old and has ", !IO),
    io__write_float(Children,               !IO),
    io__write_string(" children.\n",        !IO).
\end{verbatim}

In fact, @io__format/4@ is quite a bit more powerful, in
that the @%@ formatting specifications can include
details as to the style of formatting, precision,
justification and so forth.

\XXX{I'm not sure I should mention @io\_\_print/3@ this early.}
Another useful predicate the @io@ library module provides is
\begin{verbatim}
:- pred io__print(T,  io, io).
:- mode io__print(in, di, uo) is det.
\end{verbatim}
@io__print/3@ is used to print a representation of arbitrary
Mercury values.  Be aware, though, that if you try to
print the results of expressions, the compiler may ask you
to supply more type information to help resolve what
exactly it is you are printing (\ie the expression or the
value of the expression.)  This is subtle stuff and will
be dealt with in a later chapter. \XXX{}

\subsubsection{Input}

Input is marginally more complex than output since
typically on of three things can happen:
\begin{enumerate}
\item we successfully read a value of the required type from
  the input stream;
\item we hit the end-of-file;
\item an error of some kind occurs (\eg the input is
  malformed or the input source has gone away unexpectedly.)
\end{enumerate}

To this end the @io@ library module defines the following
type to report the results of input operations:
\begin{verbatim}
:- type io__result(T) ---> ok(T)
                      ;    eof
                      ;    error(io__error).
\end{verbatim}
In order, @ok(X)@ is returned if the input operation
succeeded, reading @X@; @eof@ is returned if the end of file
has been reached; and @error(ErrorCode)@ is returned if
something went wrong (the function @io__error_message/1@ can
be used to turn @ErrorCode@ into a printable error message.)

This arrangement forces the programmer to handle the error
cases.  There is still lively debate over whether error
codes or throwing exceptions is the best way to handle
errors for things like this.  A genuine advantage of the
error code approach is that you have to consider the error
cases from the outset, which, while requiring a little
more initial thought from the programmer, usually pays
real dividends later on.
\XXX{is this the right place to say this?  Should I
enlarge on the debate?  Probably no and yes
respectively...}

Two very basic input predicates are
\begin{verbatim}
:- pred io__read_char(io__result(char), io, io).
:- mode io__read_char(in,               di, uo) is det.

:- pred io__read_line_as_string(io__result(string), io, io).
:- mode io__read_line_as_string(in,                 di, uo)
            is det.
\end{verbatim}

The input predicates are also less comprehensive in the
sense that there are no predicates @io__read_int/3@,
@io__read_float/3@ or @io__read_string/3@.  The problem is
that it's not clear exactly what should be allowed to
terminate the input stream for an @int@, @float@ or @string@.
Instead, the library leaves issues of parsing up to
applications (there are several programs in the Mercury
extras distribution to help, including a lexer and parser
generators.)

Here's a simple program echoes the capitalised version of
letters in the input stream:
\begin{verbatim}
:- module capitalise.
:- interface.
:- import_module io.

:- pred main(io, io).
:- mode main(di, uo) is det.

:- implementation.
:- import_module char, exception.

main(!IO) :-
    io__read_char(R, !IO),
    (
        R = ok(C),
        io__write_char(char__to_upper(C), !IO),
        main(!IO)
    ;
        R = eof
    ;
        R = error(E),
        exception__throw(E)
    ).
\end{verbatim}




% % vim: ft=tex ff=unix ts=4 sw=4 et tw=76

\chapter{Modes}

REORGANISE CHAPTER AS FOLLOWS:
\begin{itemize}
\item Split into two chapters.
\emph{Basic Modes.}
\begin{itemize}
\item in, out
\item determinisms
\item di, uo
\end{itemize}
\emph{Advanced Modes.}
\begin{itemize}
\item modes and insts
\item mode definitions
\item inst definitions
\item SHNF
\item reordering
\item determinism inference
\item committed choice
\item mostly uniqueness
\item mode-specific clauses
\end{itemize}
\item \XXX{Move higher order insts to HO chapter.}
\item \XXX{In modules chapter, mention that there are no abstract insts or
modes.}
\end{itemize}

\XXX{Add a light-and-fluffy intro.  Explain that what's important is
in/out/di/uo and a basic grasp of determinism categories.  Thereafter,
one can return to this chapter on an as-needs basis.}

Mercury's mode system performs several jobs: it tells us whether a given
predicate argument is an input or output, whether it is unique on input
or output, and how many solutions a particular predicate call can have.

\section{Predicates and Procedures}

Consider the following predicate that computes the concatenation of two
lists:
\begin{myverbatim}
:- pred append(list(T), list(T), list(T)).
:- mode append(in,      in,      out    ) is det.
:- mode append(in,      out,     in     ) is semidet.
:- mode append(out,     out,     in     ) is multi.

append([],       Bs, Bs      ).
append([A | As], Bs, [A | Cs]) :- append(As, Bs, Cs).
\end{myverbatim}
The definition reads as follows: ``appending the empty list and the list
@Bs@ is just the list @Bs@; appending the non-empty list @[A | As]@ and
the list @Bs@ is the list whose head is @A@ and whose tail @Cs@ is
formed by appending @As@ and @Bs@.

The following are all solutions of @append/3@:
\begin{myverbatim}
    append([], [1, 2, 3], [1, 2, 3])
    append([1],   [2, 3], [1, 2, 3])
    append([1, 2],   [3], [1, 2, 3])
    append([1, 2, 3], [], [1, 2, 3])
\end{myverbatim}
but these are \emph{not}:
\begin{myverbatim}
    append([],     [], [1, 2, 3])
    append([1],   [3], [1, 2, 3])
    append([], [2, 3], [1, 2, 3])
\end{myverbatim}

Four mode declarations are given for @append/3@, each one describing a
different \emph{procedure}.  A procedure is a particular combination
of inputs and outputs for a given predicate.

It's easiest to understand the different procedures of @append/3@ if we
first convert its definition into \emph{super-homogeneous normal form}.
SHNF expands out all syntactic sugar and introduces new variables and
unifications so that each atomic goal is one of the following:
\begin{description}
\item [a predicate call,] @p(X, ...)@, in which each argument is a
variable;
\item [a construction or deconstruction,] @X = a(Y, ...)@, in which @a@
is a data constructor, each of whose arguments is a variable;
\item [an equality test or assignment,] @X = Y@, between two variables.
\end{description}
In a construction, @X@ is an output and @Y, ...@ are inputs.  In a
deconstruction, @X@ is an input and @Y, ...@ are outputs.  In an
equality test, both @X@ and @Y@ are inputs, while in an assignment
either @X@ is an input and @Y@ is an output or vice versa.  Assignments
and constructions always succeed, whereas deconstructions and equality
tests may fail.

The SHNF for @append/3@ is
\begin{myverbatim}
append(H1, H2, H3) :-
    (
        H1 = [],
        H2 = Bs,
        H3 = Bs
    ;
        H1 = [X1 | X2],  X1 = A,  X2 = As,
        H2 = Bs,
        H3 = [Y1 | Y2],  Y1 = A,  Y2 = Cs,
        append(Z1, Z2, Z3),  Z1 = As,  Z2 = Bs,  Z3 = Cs
    ).
\end{myverbatim}
(Although this \emph{looks} more complicated than the original
definition, SHNF means we only have to consider very simple kinds of
goal in each instance.  Any unnecessary unifications and so forth will
be optimized away by the compiler before generating an executable.)

The cardinal rule is that a goal requiring a particular variable as an
input can only be executed \emph{after} that variable has been
\emph{bound}, either because it is an input head variable, or it
previously appeared as an output argument of a predicate call, or it was
previously bound in a unification.

In order to satisfy the rule, the Mercury compiler has to reorder the
goals in conjunctions separately for each procedure.  (In other words,
the same predicate definition will be compiled for each
procedure.)
Reordering is sound because
Mercury is a logic programming language and, declaratively speaking, it
doesn't matter if we write ``@P@ and @Q@'' or ``@Q@ and @P@'' -- they
both mean the same thing.  The order matters when we try to execute
a program, but it doesn't affect the \emph{semantics} (\ie the meaning)
of the program.

Let's see how mode reordering works for each mode of @append/3@.

\subsection{The First Mode}

The first mode of @append/3@ is
\begin{myverbatim}
:- mode append(in, in, out) is det.
\end{myverbatim}
which tells us that calls to this procedure have head variables @H1@ and
@H2@ as inputs (\ie they start off already bound to something) while
@H3@ is an output (\ie it starts off \emph{free}, but will be bound to
something by the time the procedure finishes.)

A goal @append(As, Bs, Cs)@ in this mode reads ``given @As@ and @Bs@,
compute the @Cs@ that results from appending @As@ and @Bs@''.  Thus the
goal
\begin{myverbatim}
    append([1], [2, 3], Cs)
\end{myverbatim}
will compute the solution @Cs = [1, 2, 3]@.

Since disjuncts have no effect on one another, we can consider each
disjunct separately.

The first disjunct is
\begin{myverbatim}
    H1 = [],
    H2 = Bs,
    H3 = Bs
\end{myverbatim}
Since @H1@ is an input and @[]/0@ is a functor, the first goal @H1 = []@
must be a deconstruction and can be \emph{scheduled} immediately.

In the second goal, @H2 = Bs@, @H2@ is an input, but @Bs@ has not been
bound to anything (all local variables start off as free), hence this is
an assignment and afterwards @Bs@ will also be bound.  We can schedule
this goal straight away, too.

In the third goal, @Bs@ is now an input while @H3@ is not bound,
therefore @H3 = Bs@ is an assignment to @H3@.

So in this procedure no reordering is required for the first disjunct
and @H3@ finishes up bound as required.

The second disjunct is
\begin{myverbatim}
    H1 = [X1 | X2],  X1 = A,  X2 = As,
    H2 = Bs,
    H3 = [Y1 | Y2],  Y1 = A,  Y2 = Cs
    append(Z1, Z2, Z3),  Z1 = As,  Z2 = Bs,  Z3 = Cs
\end{myverbatim}
Since @H1@ is input, but @X1@ and @X2@ are not bound, @H1 = [X1 | X2]@
is a deconstruction.  Then @X1 = A@ and @X2 = As@ are assignments to @A@
and @As@ respectively.

Similarly, @H2@ is an input and @Bs@ is not bound, so @H2 = Bs@ is an
assignment to @Bs@

The third line needs some reordering.  The goal @H3 = [Y1 | Y2]@
contains no bound variables at this point, so it will have to be
scheduled later.  The unification @Y1 = A@ is an assignment to @Y1@,
since @A@ is now bound, and can be scheduled now.  The unification
@Y2 = Cs@, on the other hand, also only contains free variables and has
to be scheduled later.

The fourth line also needs reordering.  Neither @Z1@, @Z2@ nor @Z3@ are
bound at this point, so the call to @append/3@ cannot be scheduled here.
However, if we first schedule the assignments @Z1 = As@ and
@Z2 = Bs@ then we \emph{can} subsequently call @append(Z1, Z2, Z3)@
via
\begin{myverbatim}
:- mode append(in, in, out) is det.
\end{myverbatim}
(because @Z1@ and @Z2@ are now bound) and, as @Z3@ will be bound after
the call, follow this up with the assignment @Z3 = Cs@.

Returning to our two unscheduled goals, since @Cs@ is
now bound, we can schedule the assignment @Y2 = Cs@ and then the
construction @H3 = [Y1 | Y2]@.

The mode reordered version of the second disjunct is therefore
\begin{myverbatim}
    H1 = [X1 | X2],  X1 = A,  X2 = As,
    H2 = Bs,
    Z1 = As,  Z2 = Bs,  append(Z1, Z2, Z3),  Z3 = Cs
    Y1 = A,   Y2 = Cs,  H3 = [Y1 | Y2]
\end{myverbatim}
again leaving @H3@ bound as an output as required (we've moved the goal
@Y1 = A@ a little later than its earliest possible scheduling in order to
preserve as much of the structure of the original definition as
possible; the compiler almost certainly wouldn't bother with such
niceties!)

Putting our two disjuncts together, we get
\begin{myverbatim}
:- mode append(in, in, out) is det.

append(H1, H2, H3) :-
    (
        H1 = [],
        H2 = Bs,
        H3 = Bs
    ;
        H1 = [X1 | X2],  X1 = A,  X2 = As,
        H2 = Bs,
        Z1 = As,  Z2 = Bs,  append(Z1, Z2, Z3),  Z3 = Cs
        Y1 = A,   Y2 = Cs,  H3 = [Y1 | Y2]
    ).
\end{myverbatim}
All that remains is to verify that the @is det@ determinism declaration
is satisfied.  The @pred@ declaration tells us that @H1@ has type
@list(T)@, and so must be bound to either @[]@ or @[X1 | X2]@ for some
@X1@ and @X2@.  The top-level goal is a disjunction and for each of the
two possible bindings of @H1@ there is a disjunct starting with the
corresponding deconstruction.  The disjunction is therefore an
\emph{exhaustive switch} and, since the other goals in each disjunct are
all deterministic, the switch as a whole must be deterministic.  Hence
the @is det@ determinism declaration is satisfied.

(You start to see why we get the compiler to do all this hard work for
us\ldots)

% \subsection{The Second Mode}
% 
% OMITTED BECAUSE THE MODE IS INFERRED AS NONDET, BUT IN PRACTICE IS
% SEMIDET
% 
% The second mode is
% \begin{myverbatim}
% :- mode append(out, in, in) is nondet.
% \end{myverbatim}
% A goal @append(As, Bs, Cs)@ in this mode reads ``given @Bs@ and @Cs@,
% what @As@ (if any) can be appended with @Bs@ to get @Cs@?''  Hence
% \begin{myverbatim}
%     append(As, [3], [1, 2, 3])
% \end{myverbatim}
% will compute the solution @As = [1, 2]@ whereas
% \begin{myverbatim}
%     append(As, [1], [1, 2, 3])
% \end{myverbatim}
% will just fail.
% 
% To speed up the process of explanation we'll just present the correctly
% reordered disjuncts with comments to explain as we go along.
% 
% The first disjunct
% \begin{myverbatim}
%     H1 = [],
%     H2 = Bs,
%     H3 = Bs
% \end{myverbatim}
% is ordered as follows:
% \begin{myverbatim}
%                         % Start,          binds H2, H3
%     H1 = [],            % Construction,   binds H1
%     H2 = Bs,            % Assignment,     binds Bs
%     H3 = Bs             % Equality test
% \end{myverbatim}
% 
% The second disjunct
% \begin{myverbatim}
%     H1 = [X1 | X2],  X1 = A,  X2 = As,
%     H2 = Bs,
%     H3 = [Y1 | Y2],  Y1 = A,  Y2 = Cs
%     append(Z1, Z2, Z3),  Z1 = As,  Z2 = Bs,  Z3 = Cs
% \end{myverbatim}
% is reordered thus:
% \begin{myverbatim}
%                         % Start,          binds H2, H3
%     H2 = Bs,            % Assignment,     binds Bs
%     H3 = [Y1 | Y2],     % Deconstruction, binds Y1, Y2
%     Y1 = A,             % Assignment,     binds A
%     Y2 = Cs,            % Assignment,     binds Cs
%     Z2 = Bs,            % Assignment,     binds Z2
%     Z3 = Cs,            % Assignment,     binds Z3
%     append(Z1, Z2, Z3), % Call append(out, in, in) is semidet
%                         %                 binds Z1
%     Z1 = As,            % Assignment,     binds As
%     X1 = A,             % Assignment,     binds X1
%     X2 = As,            % Assignment,     binds X2
%     H1 = [X1 | X2]      % Construction,   binds H1
% \end{myverbatim}
% 
% The @is nondet@ determinism is justified because...

\subsection{The Second Mode}

\begin{myverbatim}
:- mode append(in, out, in) is semidet.
\end{myverbatim}
A goal @append(As, Bs, Cs)@ in this mode reads ``given @As@ and @Cs@,
what @Bs@ (if any) can @As@ be appended with to get @Cs@?''  The goal
\begin{myverbatim}
    append([1], Bs, [1, 2, 3])
\end{myverbatim}
will compute the solution @Bs = [2, 3]@, but
\begin{myverbatim}
    append([2], Bs, [1, 2, 3])
\end{myverbatim}
will fail.

To speed up the process of explanation we'll just present the correctly
reordered disjuncts with comments to explain as we go along.  At each
point we try to schedule the ``earliest'' goal appearing in the original
definition that can be scheduled.

The first disjunct
\begin{myverbatim}
    H1 = [],
    H2 = Bs,
    H3 = Bs
\end{myverbatim}
is ordered as follows:
\begin{myverbatim}
                        % Start,          binds H1, H3
    H1 = [],            % Deconstruction
    H3 = Bs             % Assignment,     binds Bs
    H2 = Bs,            % Assignment,     binds H2
\end{myverbatim}

The second disjunct
\begin{myverbatim}
    H1 = [X1 | X2],  X1 = A,  X2 = As,
    H2 = Bs,
    H3 = [Y1 | Y2],  Y1 = A,  Y2 = Cs
    append(Z1, Z2, Z3),  Z1 = As,  Z2 = Bs,  Z3 = Cs
\end{myverbatim}
is reordered thus:
\begin{myverbatim}
                        % Start,          binds H1, H3
    H1 = [X1 | X2],     % Deconstruction, binds X1, X2
    X1 = A,             % Assignment,     binds A
    X2 = As,            % Assignment,     binds As
    H3 = [Y1 | Y2],     % Deconstruction, binds Y1, Y2
    Y1 = A,             % Equality test
    Y2 = Cs             % Assignment,     binds Cs
    Z1 = As,            % Assignment,     binds Z1
    Z3 = Cs             % Assignment,     binds Z3
    append(Z1, Z2, Z3), % Call append(in, out, in) is semidet
                        %                 binds Z2
    Z2 = Bs,            % Assignment,     binds Bs
    H2 = Bs             % Assignment,     binds H2
\end{myverbatim}

The @is semidet@ determinism declaration is justified by the fact that
while the disjunction is a switch on @H1@, as before, the second disjunct
can \emph{fail} because of the deconstruction of @H3@, the equality test
@Y1 = A@ or the recursive call to the @semidet@ mode of @append/3@.

\subsection{The Third Mode}

\begin{myverbatim}
:- mode append(out, out, in) is multi.
\end{myverbatim}
A goal @append(As, Bs, Cs)@ in this mode reads ``given @Cs@, what @As@
and @Bs@ can be appended to get @Cs@?''  The goal
\begin{myverbatim}
    append(As, Bs, [1, 2, 3])
\end{myverbatim}
has the following possible solutions:
\begin{myverbatim}
    As = [],        Bs = [1, 2, 3]
    As = [1],       Bs =    [2, 3]
    As = [1, 2],    Bs =       [3]
    AS = [1, 2, 3], Bs =        []
\end{myverbatim}
each of which will be computed by Mercury on backtracking.
Every list can be decomposed in at least one way, so this predicate
has determinism @multi@, meaning any given call will have at least one
solution and possibly more.

The first disjunct
\begin{myverbatim}
    H1 = [],
    H2 = Bs,
    H3 = Bs
\end{myverbatim}
is ordered like this:
\begin{myverbatim}
                        % Start,          binds H3
    H1 = [],            % Assignment,     binds H1
    H3 = Bs,            % Assignment,     binds Bs
    H2 = Bs             % Assignment,     binds H2
\end{myverbatim}

The second disjunct
\begin{myverbatim}
    H1 = [X1 | X2],  X1 = A,  X2 = As,
    H2 = Bs,
    H3 = [Y1 | Y2],  Y1 = A,  Y2 = Cs
    append(Z1, Z2, Z3),  Z1 = As,  Z2 = Bs,  Z3 = Cs
\end{myverbatim}
is reordered this way:
\begin{myverbatim}
                        % Start,          binds H3
    H3 = [Y1 | Y2],     % Deconstruction, binds Y1, Y2
    Y1 = A,             % Assignment,     binds A
    Y2 = Cs,            % Assignment,     binds Cs
    Z3 = Cs,            % Assignment,     binds Z3
    append(Z1, Z2, Z3), % Call append(out, out, in) is multi
                        %                 binds Z1, Z2
    Z1 = As,            % Assignment,     binds As
    Z2 = Bs,            % Assignment,     binds Bs
    X1 = A,             % Assignment,     binds X1
    X2 = As,            % Assignment,     binds X2
    H1 = [X1 | X2],     % Construction,   binds H1
    H2 = Bs             % Assignment,     binds H2
\end{myverbatim}

Every call to this mode of @append/3@ creates a \emph{choice
point} -- a place where different disjuncts may each lead to a solution.
The first disjunct always succeeds (it consists entirely of
assignments.)  The second, however, can fail if the deconstruction of
@H3@ fails, but otherwise may have many solutions thanks to the
recursive call to the @is multi@ mode of @append/3@.

\subsection{Nondeterminism}

Since this is such an important point, let's spend a little time seeing
how nondeterminism works in practice.  Now that we understand mode
reordering, we can dispense with unfriendly SHNF and work with the
altogether fluffier original definition.
\begin{myverbatim}
:- mode append(out, out, in) is multi.

append([],       Bs, Bs      ).
append([A | As], Bs, [A | Cs]) :- append(As, Bs, Cs).
\end{myverbatim}
Consider the goal
\begin{myverbatim}
    append(As, Bs, [1, 2, 3])
\end{myverbatim}
For each call to @append/3@, Mercury creates a choice point for the
disjunction and tries one of the disjuncts.  Failure later on will cause
Mercury to \emph{backtrack} to the most recent choice point and try the
other disjunct.

The possible paths to a solution are illustrated in figure
\XXX{append:fig}
(local variables have been renamed @As1@, @As2@, @As3@ and so on to
avoid confusion).
Each of the coloured arrows on the left hand side leads from an
@append/3@ choice point to the result of taking one of the two
clauses.
The arrows on the right hand side show the data flow
from the third argument when taking the second clause.
\begin{figure}[ht]
\flushleft{\epsfbox{append.eps}}
\caption{Choice points and nondeterminism.}[append:fig]
\end{figure}

When execution chooses the first clause for @append(As, Bs, Cs)@
we get @As = [], Bs = Cs@ as the solution.  When
execution chooses the second clause we get the deconstruction
@Cs = [A | Cs1]@, followed by the call @append(As1, Bs, Cs1)@ and the
construction @As = [A | As1]@.

Following the ``choice point'' arrows in the figure backwards, we see for
instance that the second-to-last solution generated is
\begin{myverbatim}
    Bs  = [3],
    As2 = [],
    As1 = [2 | As2],
    As  = [1 | As1]
\end{myverbatim}
which is the same as saying @As = [1, 2], Bs = [3]@.

\subsubsection{An Example of Use}

\XXX{I think this may be rather too complicated.  Any suggestions?}

So how is nondeterminism useful?  Nondeterminism is most commonly used
to succinctly describe searching problems.  Say we are asked to write a
predicate that takes a list of integers and succeeds iff the list is in
ascending order.  We could take the laborious route of writing
\begin{myverbatim}
:- pred in_ascending_order(list(int)).
:- mode in_ascending_order(in       ) is semidet.

in_ascending_order([]).
in_ascending_order([_]).
in_ascending_order([A, B | List]) :-
    A < B,
    in_ascending_order([B | List]).
\end{myverbatim}
But this involves rather too much ``how'' compared to ``what'' and
requires some analysis on the part of the reader to understand the
definition.  An alternative approach is to rephrase the specification to
say a list is in ascending order \emph{if} it contains no consecutive
members @A@ and @B@ such that @not A < B@.  We can then state the
solution directly:
\begin{myverbatim}
in_ascending_order(List) :-
    not (has_consecutive_members(List, A, B), not A < B).

:- pred has_consecutive_members(list(T), T,   T  ).
:- mode has_consecutive_members(in,      out, out) is nondet.

has_consecutive_members(List, A, B) :-
    append(Prefix, Suffix, List),
    Suffix = [A, B | _].
\end{myverbatim}
(This version is admittedly no shorter; we'll remedy this shortcoming in
a little while.)

The definition of @has_consecutive_members/3@ says that ``@List@
contains the consecutive members @A@ and @B@ \emph{if} it can be split
into @Prefix@ and @Suffix@ where @Suffix@ is a list with at least two
members, starting with @A@ and @B@.''

@has_consecutive_members/3@ works by exploiting the
@append(out, out, in)@ @is multi@ procedure.  For example, consider the
goal
\begin{myverbatim}
    has_consecutive_members([1, 2, 3, 4], A, B).
\end{myverbatim}
The possible solutions for the body of @has_consecutive_members/3@
are
\begin{myverbatim}
    Prefix = [],     Suffix = [1, 2, 3, 4], A = 1, B = 2
    Prefix = [1],    Suffix =    [2, 3, 4], A = 2, B = 3
    Prefix = [1, 2], Suffix =       [3, 4], A = 3, B = 4
\end{myverbatim}
(The solutions to the @append/3@ subgoal giving @Suffix = [4]@ and
@Suffix = []@ cause the deconstruction
@Suffix = [A, B | _]@ to fail.)

Let's return to the definition of @in_ascending_order/1@.  Since a goal
of the form @not P@ means ``@P@ has no solutions'', Mercury must try all
possible ways of finding a solution for @P@ before deciding whether
@not P@ succeeds or fails.

We'll use a couple of examples to illustrate the principle.
From the definition, the goal @in_ascending_order([1, 2, 3, 4])@ is
equivalent to
\begin{myverbatim}
    not (
        has_consecutive_members([1, 2, 3, 4], A, B),
        not A < B
    )
\end{myverbatim}
As we've just seen, the solutions for the @has_consecutive_members/3@
subgoal are
\begin{myverbatim}
    A = 1, B = 2
    A = 2, B = 3
    A = 3, B = 4
\end{myverbatim}
Since @not A < B@ is false in each case, the conjunction as a whole has
no solution and is
also false.  Therefore the \emph{negated conjunction} is \emph{true}
and @[1, 2, 3, 4]@ is in ascending order.

What about @in_ascending_order([1, 2, 4, 3])@?  This is equivalent to
\begin{myverbatim}
    not (
        has_consecutive_members([1, 2, 4, 3], A, B),
        not A < B
    )
\end{myverbatim}
The possible solutions for the @has_consecutive_members/3@ subgoal are
\begin{myverbatim}
    A = 1, B = 2
    A = 2, B = 4
    A = 4, B = 3
\end{myverbatim}
@not A < B@ is false for the first two subgoal solutions, but
\emph{true} for the third.  Since the conjunction as a whole has a
solution, its negation is \emph{false}.  Hence @[1, 2, 4, 3]@ is not in
ascending order.

(Note that for a goal @P, Q@, a Mercury program does not compute all
the solutions to @P@ \emph{before} considering @Q@.  Rather, it tests
@Q@ as soon as @P@ succeeds, only backtracking into
@P@ for a different solution if @Q@ fails.  \XXX{I don't really want to
get into a discussion of termination semantics etc. here.})

\subsubsection{Tidying Up}

The double negation in the definition of @in_ascending_order/1@ is
unfortunate.  Another way of stating the predicate specification is that
a list is in ascending order if every consecutive pair of members @A@
and @B@ satisfies @A < B@.  Mercury has syntactic sugar to allow us to
write this directly:
\begin{myverbatim}
in_ascending_order(List) :-
    all [A, B] (has_consecutive_members(List, A, B) => A < B).
\end{myverbatim}
This reads as ``@List@ is in ascending order \emph{if} for all @A, B@,
if @A, B@ are consecutive members of @List@ then @A < B@.''

The goal @all [X, Y, Z] P@ is shorthand for
@not some [X, Y, Z] (not P)@ -- that is, ``@P@ is true for all
@X, Y, Z@'' is the same as saying ``there is no @X, Y, Z@ for which @P@
is not true.''

The goal @some [X, Y, Z] Q@ can be shortened to just @Q@ since Mercury
assumes that any variables that appear just in @Q@ are existentially
quantified.  \XXX{Remember to mention quantification in some earlier
chapter.}

The goal @R => S@ is shorthand for @not (R, not S)@ -- that is,
``if @R@ is true then @S@ is true'' is the same as saying ``it is not
the case that @R@ can be true and @S@ false''.

Combining the above we get
\begin{myverbatim}
in_ascending_order(List) :-
    not not not (
        has_consecutive_members(List, A, B),
        not A < B
    ).
\end{myverbatim}
Finally, a goal @not not P@ is equivalent to just @P@ -- ``it's not the
case that P is not true'' is just the same as saying ``P is true'' -- so
we end up with our original definition,
\begin{myverbatim}
in_ascending_order(List) :-
    not (
        has_consecutive_members(List, A, B),
        not A < B
    ).
\end{myverbatim}

Next, we turn our attention to @has_consecutive_members/3@.  The
original definition was
\begin{myverbatim}
has_consecutive_members(List, A, B) :-
    append(Prefix, Suffix, List),
    Suffix = [A, B | _].
\end{myverbatim}
We can shorten this two ways.  First, because @Prefix@ appears only once,
we can replace it with the ``don't care'' variable, @_@.  Second, @=/2@ in
mercury really does denote equality which means, this being a declarative
language where we can always replace equals with equals, we can ``push''
the unification up into the call to @append/3@, giving
\begin{myverbatim}
has_consecutive_members(List, A, B) :-
    append(_, [A, B | _], List),
\end{myverbatim}

\section{Modes and Insts}

\XXX{I'm not going to mention uniqueness until a later section.}

So far we have only looked at the @in@ and @out@ modes without
really saying what they mean.  We've also been somewhat cavalier in our
use of terminology, using ``mode'' to refer to both possible modes of a
predicate (its procedures) and to refer to whether an argument in an
input or an output.  In this section, we use mode in the latter sense.

For the most part, @in@ and @out@ are all that are necessary.  However,
any program that performs IO, employs higher order programming techniques,
or uses subtyping will also need some more specialised modes.

\subsection{Modes}

A \emph{mode} describes the transition between instantiation states, or
\emph{insts}, for a variable when a unification or procedure call 
executes successfully.  (A goal that fails has no effect on anything.)
An \emph{inst} describes the possible bindings for a variable
at a particular instant in time.

The two most basic insts are @free@ and @ground@.  A free variable has
not yet been bound to anything whereas a ground variable \emph{is}
bound to something, but we don't know what, exactly.

The built-in modes @in@ and @out@ are therefore defined like this:
\begin{myverbatim}
:- mode out == (free   >> ground).
:- mode in  == (ground >> ground).
\end{myverbatim}
These are mode \emph{definitions} (as opposed to mode \emph{declarations}
which are associated with procedures.)  The left and right hand sides of
the @>>@ symbol are the ``before'' and ``after'' insts, respectively.

The mode definition for @out@ says that an @out@ mode argument must be
free before the goal, but will be ground afterwards (\ie it will
become bound \emph{if the goal succeeds.})  The definition for @in@ says
that an @in@ mode argument must be ground before the goal is executed
and will still be ground afterwards.

\subsubsection{Parametric Modes}

Just as types can have parameters, modes can be generalised over insts.
For instance, the built-in parametric modes @in/1@ and @out/1@ are defined as
\begin{myverbatim}
:- mode out(I) == (free >> I).
:- mode in(I)  == (I    >> I).
\end{myverbatim}
so we could have defined @in/0@ and @out/0@ with
\begin{myverbatim}
:- mode out == out(ground).
:- mode in  == in(ground).
\end{myverbatim}

\subsection{Refined Insts}

One can define more refined insts than @ground@.  To illustrate,
consider the following predicate:
\begin{myverbatim}
:- pred head_tail(list(T), T,   list(T)).
:- mode head_tail(in,      out, out    ) is semidet.

head_tail([X | Xs], X, Xs).
\end{myverbatim}
The goal @head_tail([1, 2, 3], Head, Tail)@ has the solution
@Head = 1,@ @Tail = [2, 3]@, whereas the goal
@head_tail([], Head, Tail)@ will fail.

Now, say that at a particular point in our program we have the goal
@head_tail(List, Head, Tail)@ and we \emph{know} that @List@ will not be
empty.  It would be horrible to have to write
\begin{myverbatim}
    ( if head_tail(List, Head, Tail) then
        ...rest of the program...
      else
        exception.throw("this wasn't supposed to happen!")
    )
\end{myverbatim}
to make this particular goal deterministic rather than
semideterministic.

A more elegant solution is to define an inst for non-empty lists:
\begin{myverbatim}
:- inst non_empty_list ---> [ground | ground].
\end{myverbatim}
This says that a variable with inst @non_empty_list@ will be bound to
a @[|]/2@ functor, both of whose arguments will be ground.

Next, we specify another procedure for @head_tail/3@:
\begin{myverbatim}
:- mode head_tail(in(non_empty_list), out, out) is det.
\end{myverbatim}

The new mode declaration for @head_tail/3@ states
that if the first argument is a non-empty list input, and the other two
arguments are outputs, then @head_tail/3@ is guaranteed to succeed.  The
compiler can verify this because it knows that if the first argument
(call it @List@) to @head_tail/3@ is non-empty then the deconstruction
@List = [X | Xs]@ \emph{must} succeed.

Now the sordid conditional goal wrapper around our
call to @head_tail/3@ is no longer necessary.  Which is super.

\subsection{Parametric Insts}

We can also have parametric insts.  Consider the following:
\begin{myverbatim}
:- inst list(I) ---> [] ; [I | list(I)].
\end{myverbatim}
This says that a variable with inst @list(I)@ (for some inst @I@) will
be bound to either @[]@ or to a @[|]/2@ functor whose first argument has
inst @I@ and whose second argument has inst @list(I)@ (notice the
similarity to the type definition for lists.)

Now we can write @list(non_empty_list)@, for instance, to refer to any
list of non-empty lists.

\section{Determinism}

We need to examine the various kinds of determinism in more detail.

Let's start by reviewing the determinism categories we've encountered so
far, plus one or two others that turn out to be useful:

\begin{tabular}{l|l|l}
Category    & Failure       & Solutions \\
\hline \hline
@failure@   & always fails  & has no solutions \\
@semidet@   & may fail      & has at most one solution \\
@det@       & cannot fail   & has exactly one solution \\
@nondet@    & may fail      & may have several solutions \\
@multi@     & cannot fail   & may have several solutions \\
@erroneous@ & \emph{abnormal termination} \\
\end{tabular}

We haven't mentioned @failure@ and @erroneous@ before.

@failure@ is the determinism category for goals that cannot succeed.
This might seem like an odd thing to have, but it's rather like having
zero in mathematics: without it there's a gaping hole in the scheme.
The built-in predicate @false/0@ has determinism @erroneous@ and is
equivalent to @not true@.

@erroneous@ is the determinism category for goals like
@exception.throw/1@ which do not exit normally.  @exception.throw/1@
neither fails nor succeeds.  Instead, an \emph{exception} occurs which
returns control to the most recent \emph{exception handler} on the call
stack.  \XXX{Do I need to explain ``call stack''?}  This is explained
fully in chapter \XXX{} on exceptions.

How do we work out the determinism category for a goal (and hence a
procedure)?  It turns out we only have to understand five cases:
atomic goals, conjunction and disjunction, negation and conditional
goals.  From these we can generalise to any goal.

\subsection{Atomic Goals}

An atomic goal is either a unification or a procedure call.  The
determinism of a procedure call is given by the corresponding predicate
mode declaration.

Unifications come in four flavours:
\begin{description}
\item [constructions and assignments] always have exactly one
solution and hence have determinism category @det@;
\item [equality tests and deconstructions] either fail or succeed once,
giving them a determinism category of @semidet@.
\end{description}

\subsection{Conjunctions}

Consider the goal @P, Q@.

If either @P@ or @Q@ has determinism @failure@ then the conjunction
as a whole has determinism @failure@.

Similarly, if either @P@ or @Q@ has determinism @erroneous@ then the
conjunction as a whole has determinism @erroneous@.

Otherwise the determinism of the conjunction is worked out as follows:
if either of @P@ or @Q@ can fail then the conjunction as a whole can
fail.  If either of @P@ or @Q@ may have several solutions then the
conjunction as a whole may have several solutions.

The following table gives the determinism category of a conjunction from
the determinisms of its conjuncts:

\begin{center}
\begin{tabular}{l|llll}
              & @semidet@   & @det@       & @nondet@    & @multi@ \\
\hline
@semidet@     & @semidet@   & @semidet@   & @nondet@    & @nondet@ \\
@det@         & @semidet@   & @det@       & @nondet@    & @multi@ \\
@nondet@      & @nondet@    & @nondet@    & @nondet@    & @nondet@ \\
@multi@       & @nondet@    & @multi@     & @nondet@    & @multi@ \\
\end{tabular}
\end{center}

\subsection{Disjunctions}

Consider the goal @(P ; Q)@.

If @P@ has determinism @failure@ then the disjunction as a whole has
the same determinsm of @Q@ -- and vice versa.

Similarly, if @P@ has determinism @erroneous@ then the disjunction as a
whole has the same determinsm of @Q@ -- and vice versa.

Otherwise, if both @P@ and @Q@ can fail then the disjunction as a whole
can fail and if either of @P@ or @Q@ may have several solutions then the
disjunction as a whole may have several solutions.

The following table gives the determinism category of a disjunction from
the determinsms of its disjuncts:

\begin{center}
\begin{tabular}{l|llll}
            & @semidet@   & @det@       & @nondet@    & @multi@ \\
\hline
@semidet@   & @semidet@   & @multi@     & @nondet@    & @nondet@ \\
@det@       & @multi@     & @det@       & @multi@     & @multi@ \\
@nondet@    & @nondet@    & @multi@     & @nondet@    & @multi@ \\
@multi@     & @nondet@    & @multi@     & @multi@     & @multi@ \\
\end{tabular}
\end{center}

\subsection{Negations}

Consider the goal @not P@.

If @P@ has determinsm @failure@ then the negation has determinsm @det@.

If @P@ has determinsm @erroneous@ then the negation has determinsm
@erroneous@ as well.

Otherwise, if @P@ cannot fail then the negation has determinsm
@failure@, otherwise it has determinsm @semidet@.

The following table gives the determinism category of a negation from
the determinism of its subgoal:

\begin{center}
\begin{tabular}{l|l}
@P@           & @not P@ \\
\hline \hline
@semidet@     & @semidet@ \\
@det@         & @failure@ \\
@nondet@      & @semidet@ \\
@multi@       & @failure@ \\
\end{tabular}
\end{center}

\subsection{Conditional Goals}

Consider the goal @( if P then Q else R )@.

This is semantically equivalent to @(P, Q ; not P, R)@.  We can use this
form to work out the determinism of the conditional goal as a whole from
the determinism of @P@, @Q@ and @R@.

\section{Polymorphic Modes}

\XXX{Do I want to mention this here?  At all?}

\section{Uniqueness}

\XXX{I think this should probably go after the insts section and before
the determinism section.}

So far, the only unique structure we've seen has been the @io.state@ used by
IO operations.  Uniqueness, however, is needed by several other structures
such as arrays and stores that we'll come to in later chapters \XXX{}.

The key characteristic of these structures is that, for reasons of either
efficiency or correctness, values of these types can only ever have one
\emph{live} reference at a time.  Liveness, in this case, means that the
referring variable in question can be used \emph{at most once} in subsequent
computation (except for one special case that will be explained shortly.) As
a consequence, a value cannot be unique at a particular point in a program
if there are two or more live variables that refer to it or if the
computation can be backtracked over.

There are three main argument modes associated with unique values:
\begin{itemize}
\item @di@ -- destructive input arguments must be unique at the start of the
call and are \emph{dead} (\ie no longer available) afterwards;
\item @ui@ -- unique input arguments must be unique at the start of the call
and are still unique afterwards;
\item @uo@ -- unique output arguments are like ordinary @out@ arguments
except that they are guaranteed to be unique.
\end{itemize}
\XXX{Fix @ui@ modes.}

These built-in modes are defined as follows:
\begin{verbatim}
:- mode di == (unique >> dead  ).
:- mode ui == (unique >> unique).
:- mode uo == (free   >> unique).
\end{verbatim}

The inst @unique@ is like @ground@, except that every part of the
corresponding value (if it has structure) is also required to be unique.

Insts describing values whose top-level functor is unique, but whose
arguments may not be, are defined as follows:
\begin{myverbatim}
:- inst unique_list(I) == unique( [] ; [I | unique_list(I)] ).
\end{myverbatim}
This defines a new inst, @unique_list/1@ whose top-level functor is unique
and bound to either a @[]/0@ or a @[|]/2@ value with arguments of inst @I@
and @unique_list(I)@ respectively.

\XXX{It's a shame that we don't have a simpler syntax for unique(...)
in the same was as we do for bound(...).}

\XXX{Add top\_unique inst which only applies to the top-level functor.}

\XXX{The ui mode doesn't yet work.}

\XXX{Nested uniqueness doesn't yet work.}

\XXX{I don't want to say too much here until we've got uniqueness sorted
out.}

\subsection{Uniqueness and Reuse}

\XXX{Reuse doesn't yet work.}

\section{Higher Order Modes}

\XXX{This will go in the HO chapter.}

\section{Committed Choice} 

\XXX{Do I want to mention this here?  At all?}

\section{Mostly Uniqueness}

\XXX{Do I want to mention this here?  At all?}

\section{Conclusion}

Modes are helpful documentation for the programmer.

Modes can be used to implement subtyping.

Modes allow the compiler to generate efficient code by reordering for
each procedure.

% vim: ft=tex ff=unix ts=4 sw=4 et wm=8 tw=0

\chapter{Modules}

The Mercury unit of compilation is the module.  Modules
are compiled separately and the resulting object files linked
together to make a program or aggregated to form libraries.

A module has two parts, the interface and the implementation sections.

The interface section describes the types, insts, preds and funcs and so
forth that are said to be exported or externally visible.

The implementation section supplies all the code that defines the
behaviour of the exported preds and funcs and defines any abstract
types or type class instances exported by the module (we will come
across type classes later on \XXX{}.)  Declarations and definitions in
the implementation section are not externally visible.

\section{Names and Namespaces}

It would be unreasonable to insist that no two preds (or types
or \ldots) share the same name in any Mercury program.  To this
end, the Mercury module system makes use of two mechanisms:
restrictions on the visiblity of names and a module qualified
naming scheme.

In this section we will use the term \emph{name} to refer to the
name of any type, func, pred, inst, mode, type class and so
forth that one might find in a Mercury program.

\section{Visibility}

In order to use a name, it must be visible at each point in
a module where it appears.

\XXX{This stuff needs illustrating with examples.}

\subsection{In the Interface Section}

A name is visible in an interface section if it is
\begin{itemize}
\item declared or defined there;
\item is exported by a module which is imported in this
  module's interface section;
\item is declared or defined in a parent module (we will come
  to submodules shortly \XXX{}.)
\end{itemize}

Any name defined or declared in the interface section is
said to be \emph{public} or \emph{exported} by that module.

\subsection{In the Implementation Section}

A name is visible in an implementation section if it is
\begin{itemize}
\item declared or defined there;
\item is exported by a module which is imported in this
  module's interface or implementation section;
\item is declared or defined in a parent module.
\end{itemize}

Any name declared or defined in the implementation section
that is not exported by the module is said to be \emph{private}
or \emph{local} to the module.

\section{Qualification}

Name qualification allows us to avoid abiguity where one
module imports two other modules which export different things
under the same ``base'' name.

A fully qualified name takes the form @<modulename>__<thingname>@
(or, for submodules, @<modulename>__<submodulename>__...__<thingname>@)
where the double-underscore @__@ is the module name separator symbol.

Using fully qualified names throughout a program quickly
becomes tedious and can distract from the readability of the
source code.  Mercury therefore allows one to omit qualifiers
(or partial prefixes in the case of submodules) where there is
no ambiguity.

The compiler will not confuse names of different sorts of
things -- for instance, there is never ambiguity between a type
and an inst with the same name since the types and insts
always appear in separate contexts.

Potential ambiguity arises with funcs, preds and data
constructors since all three can appear in unifications.
Ambiguity can still be resolved by the compiler if the
possible ways of resolving the name in question each have
different types or arities (\XXX{Need to go over some examples
a la @io\_\_print@.  And be more precise.})

A convention worth adopting is to explicitly qualify
predicates, but not types, constructors and functions.

(Mercury also provides @use_module@, an alternative to @import_module@,
which does much the same thing, except that names imported via
@use_module@ \emph{must} be fully qualified.)

\section{Submodules and Hierarchical Namespaces}

The module namespace may be structured rather than flat.  Modules can
also have submodules to form a tree-like namespace.

Submodules come in two flavours, nested and separate.

\subsection{Nested Submodules}

Nested submodules appear in the same source file as the parent module.

For example,
\begin{verbatim}
:- module vector3.
:- interface.
:- import_module float.

:- type vector3 == {float, float, float}.

:- func vector3 + vector3 = vector3.
:- func vector3 `dot` vector3 = float.

:- module scalar_vector3.
:- interface.

    :- func float * vector3 = vector3.

:- end_module scalar_vector3.

:- module vector3_scalar.
:- interface.

    :- func vector3 * scalar = vector3.

:- end_module vector3_scalar.
    
:- implementation.

{Xa, Ya, Za} + {Xb, Yb, Zb} = {Xa + Xb, Ya + Yb, Za + Zb}.

{Xa, Ya, Za} `dot` {Xb, Yb, Zb} = Xa * Xb + Ya * Yb + Za * Zb.

:- module scalar_vector3.
:- implementation.

    A * {X, Y, Z} = {A * X, A * Y, A * Z}.

:- end_module scalar_vector3.

:- module vector3_scalar.
:- implementation.

    {X, Y, Z} * A = {A * X, A * Y, A * Z}.

:- end_module vector3_scalar.
\end{verbatim}
Nested submodules are used in the example above to avoid overloading
problems for the functions such as @*@ that have both @float * vector3@
and @vector3 * float@ varieties (recall that overloaded functions with
the same arity cannot be defined directly in the same module.)

The submodules do not have to import the @float@ module separately
because all names visible in the parent module are also visible in the
submodules.

A client of this module would have to import each submodule to have
access to those particular names.  For example,
\begin{verbatim}
:- module foo.
:- interface.
:- import_module io.

:- pred main(io::di, io::uo) is det.

:- implementation.
:- import_module vector3, vector3_scalar, scalar_vector3.

main(!IO) :-
    R = {1.0, 1.5, 2.0},
    io__print(2.0 * R),
    io__nl,
    io__print(R * 3.0),
    io__nl.
\end{verbatim}
should write out
\begin{verbatim}
{2.0, 3.0, 4.0}
{3.0, 4.5, 6.0}
\end{verbatim}
when run.  The compiler understands that the multiplication @2.0 * R@
is referring to the function defined in @scalar_vector3@ and that the
multiplication @R * 3.0@ is referring to the function defined in the
@vector3_scalar@.

\subsection{Separate Submodules}

Separate submodules are defined in separate files and must be explicitly
listed in the parent module in @include_module@ declarations, but are 
otherwise indistinguishable from nested submodules as far as the
programmer is concerned.  The advantage of separate submodules is that
they can be compiled separately \XXX{verify this is correct} and, for
large submodules, splitting them into separate files can simplify code
maintenance.

\XXX{What about @import\_hierarchy@ and chums?}

\XXX{We really want to just have @.@ as the module separator by the time
this goes to print rather than the current mixture of @\_\_@ and @:@ and
@.@ in separate submodule file names.}

The vector example above could be recoded to use separate submodules as
\begin{verbatim}
:- module vector3.
:- interface.

:- import_module float.

:- include_module vector3__scalar_vector3, vector3__vector3_scalar.

:- type vector3 == {float, float, float}.

:- func vector3 + vector3 = vector3.
:- func vector3 `dot` vector3 = float.
    
:- implementation.

{Xa, Ya, Za} + {Xb, Yb, Zb} = {Xa + Xb, Ya + Yb, Za + Zb}.

{Xa, Ya, Za} `dot` {Xb, Yb, Zb} = Xa * Xb + Ya * Yb + Za * Zb.
\end{verbatim}
and in a file named @vector3.scalar_vector3.m@
\begin{verbatim}
:- module vector3__scalar_vector3.
:- interface.

:- func float * vector3 = vector3.

:- implementation.

A * {X, Y, Z} = {A * X, A * Y, A * Z}.
\end{verbatim}
and in a file named @vector3.vector3_scalar.m@
\begin{verbatim}
:- module vector3__vector3_scalar.
:- interface.

:- func vector3 * scalar = vector3.

:- implementation.

{X, Y, Z} * A = {A * X, A * Y, A * Z}.
\end{verbatim}
Note the @include_module@ declaration in the interface section for
@vector3@ (making @scalar_vector3@ and @vector3_scalar@ accessible to
clients of @vector3@\footnote{Their fully qualified names being
@vector3\_\_scalar\_vector3@ and @vector3\_\_vector3\_scalar@
respectively.}).  A declaration of this kind is required for every
separate submodule.

Use of @vector3@, @scalar_vector3@ and @vector3_scalar@ is exactly the
same as if they had been written as nested submodules.

\XXX{Is this the right place to mention the file name mapping option to
@mmake@?}

\subsection{Visibility and Submodules}

\XXX{Need to check this one out carefully.}




% % vim: ft=tex ff=unix ts=4 sw=4 et wm=8 tw=0

\chapter{Higher Order Programming}

Predicates and functions are first class values, just like values of
type @int@, @string@, @list(char)@ and so forth.  They can be
passed around and constructed just like any other kind of value.  Code
that manipulates predicate values is said to be \emph{higher order}.

\XXX{Should say somewhere that equality and ordering are undefined on
higher order values.}

What is special about predicate values is that they can be
\emph{applied} to arguments in order to carry out tests or compute other
values.

This turns out to be a surprisingly useful and flexible way to program.
Indeed, it is one of the key reasons why one should consider a typical
declarative programming language in preference to the more common
imperative languages.

\section{Example: the Map Function}

It is very common to want to apply a function to each member of a @list@
to obtain the @list@ of transformed values.  That is, given a function
@F@ and a @list@ @[X1, X2, X3, ...]@, we want to compute the @list@
@[F(X1), F(X2), F(X3), ...]@.

We can achieve this with the higher order function @map@:
\begin{verbatim}
:- func map(func(T1) = T2, list(T1)) = list(T2).

map(_, []      ) = [].
map(F, [X | Xs]) = [F(X) | map(F, Xs)].
\end{verbatim}
The first argument @F@ is declared to be a function itself computing
values of type @T2@ from values of type @T1@, where @T1@ and @T2@ can be
anything.  The second argument is declared to be a @list(T1)@ and the
result of the @map@ operation is a @list(T2)@.

The first clause states that if the second argument is the empty list
@[]@, then so is the result.

The second clause states that if the second argument is a @list@ with head
@X@ and tail @Xs@, then the result is the @list@ whose head is computed by
applying @F@ to @X@, namely as @F(X)@, and whose tail is computed by
@map(F, Xs)@.

It follows that
\begin{verbatim}
map(F, [X1, X2, X2, ...]) = [F(X1), F(X2), F(X3), ...]
\end{verbatim}
as required.  Since @map/2@ is polymorphic, it will work for any @F@
with the appropriate signature -- there is no need to recode it for each
particular function we wish to map over a @list@.

\section{Example: the Foldr Function}

\XXX{Should probably reference ``Why Functional Programming Matters''.}

Say we need to write a function @sum@ that will compute the sum of a
@list@ of @int@s.  We could reason as follows: the sum of a list whose
head is @X@ and whose tail is @Xs@ is just @X@ plus the sum of the @Xs@.
Since we would like @sum(Xs ++ Ys) = sum(Xs) + sum(Ys)@, for any @Xs@
and @Ys@, we would conclude that the sum of the empty list must be @0@.
Hence
\begin{verbatim}
:- func sum(list(int)) = int.

sum([]      ) = 0.
sum([X | Xs]) = X + sum(Xs).
\end{verbatim}
Observe that @sum([X1, X2, X3]) = X1 + (X2 + (X3 + 0))@.

By a similar process we could arrive at a function @prod@ that computed
the product of a @list@ of @int@s:
\begin{verbatim}
:- func prod(list(int)) = int.

prod([]      ) = 1.
prod([X | Xs]) = X * prod(Xs).
\end{verbatim}
Observe this time that @prod([X1, X2, X3]) = X1 * (X2 * (X3 * 1))@.

At this point we see that the definitions of @sum@ and @prod@ are almost
identical, the only difference being the result for the empty list and
the function used to combine the head of the list with the result of
processing the tail.

What would be most useful would be an higher order function that
generalised this pattern so that we only have to get it right once and
can reuse it thereafter for @list@ processing functions with a similar
pattern to @sum@ and @prod@.  We call this function @foldr@ and it takes
two arguments: @A@ is the value returned when the @list@ in question is
empty and @F@ is the function that computes the combination of the head
of the @list@ with the result of processing its tail.
\begin{verbatim}
:- func foldr(func(T1, T2) = T2, T2, list(T1)) = T2.

foldr(_, A, []      ) = A.
foldr(F, A, [X | Xs]) = F(X, foldr(F, A, Xs)).
\end{verbatim}
(The type signature for @foldr@ might look a little intimidating, but a
little careful consideration should make it obvious what's going on.)

\XXX{By the way, the standard library gets the argument order wrong as
far as Currying is concerned.  Can we fix it for v2 or are we stuck with
the current ordering?}

Now, with the aid of a couple of auxiliary functions,
\begin{verbatim}
:- func plus(int, int) = int.

plus(X, Y) = X + Y.

:- func times(int, int) = int.

times(X, Y) = X * Y.
\end{verbatim}
we can go on to define @sum@ and @prod@ in terms of @foldr@:
\begin{verbatim}
sum(Xs) = foldr(plus, 0, Xs).

prod(Xs) = foldr(times, 1, Xs).
\end{verbatim}
(The reason we had to define @plus@ is that the standard @int@ library
function @+@ has more than one mode -- given any two arguments to the
equation @X + Y = Z@ one can obtain the third, although this is not true
of integer multiplication (we define @times@ merely for symmetry) -- and
Mercury currently has no syntax for specifying a particular procedure of
a function or predicate nor the means to identify from context, in
higher order code in general, which is otherwise intended.) \XXX{This is
a complicated paragraph.}

@foldr@ is surprisingly general.  For instance, consider the definition
of the @list@ concatenation function, whose name is the infix operator
@++@:
\begin{verbatim}
:- func list(T) ++ list(T) = list(T).

[]       ++ Ys = Ys.
[X | Xs] ++ Ys = [X | Xs ++ Ys].
\end{verbatim}
Once we have @foldr@, there's no need for us to have to think about the
recursion any more.  We can instead write
\begin{verbatim}
Xs ++ Ys = foldr(cons, Ys, Xs).

:- func cons(T, list(T)) = list(T).

cons(X, Xs) = [X | Xs].
\end{verbatim}
(we have to define @cons@ because data constructors such as @[|]@ are
\emph{not} functions, not least because they can also be used as
``deconstructors'' in unifications and pattern matching.)

It is worth spending the time to become familiar with higher order
functions such as @foldr@.  Such functions enable you to solve the
general case once and then never have to expend mental or physical
effort duplicating the scheme for each specific application.

\section{Lambdas}

Sometimes it is a little painful to have to name each and every small
auxiliary function when writing higher order code.  Lambdas are
auxiliary predicate or function procedures that can be constructed on an
as-needs basis in a program and passed around just as if they had been
defined as separate predicates or functions in their own right.  The
main difference is that lambdas do not have names (they are sometimes
described as `anonymous') and therefore cannot be recursive.

We illustrate lambdas by recoding @sum@, @prod@ and @++@:
\begin{verbatim}
sum(Xs)  = foldr((func(A, B)  = A + B   ), 0, Xs).

prod(Xs) = foldr((func(A, B)  = A + B   ), 1, Xs).

Xs ++ Ys = foldr((func(A, As) = [A | As]), Ys, Xs).
\end{verbatim}
As with most coding short-cuts, lambdas can make code both more and less
legible.  As a general rule, lambdas are best kept brief and used in
situations where their purpose is obvious, or, in other words, if you
think an explanatory comment is justified, then avoid using a lambda.
Their use is justified in the above cases.

\XXX{Talk about lambdas with bodies.  They only get one clause.}

\XXX{Talk about predicate lambdas, including nondeterminism etc.}

\XXX{Say that it's fine to unify lambdas with variables etc.}

\XXX{Don't forget the scope rules.}

\section{Partial Application (Currying)}

Looking at the definition of @map@ once more,
\begin{verbatim}
map(_, []      ) = [].
map(F, [X | Xs]) = [F(X) | map(F, Xs)].
\end{verbatim}
we see exactly the same pattern of recursion captured by @foldr@, hence
we can recode it as
\begin{verbatim}
map(F, Xs) = foldr(apply_cons(F), [], Xs).

:- func apply_cons(func(T1) = T2, T1, list(T2)) = list(T2).

apply_cons(F, X, Ys) = [F(X) | Ys].
\end{verbatim}
using the auxiliary function @apply_cons@.  

At this point one may be prompted to ask, ``What is this strange
@apply_cons(F)@ appearing in the definition of @++@?  And besides,
@apply_cons@ takes three arguments.''

The expression @apply_cons(F)@ is a \emph{partial application} of
@apply_cons/3@, resulting in a \emph{closure} which is equivalent to
writing @func(A, Bs) = apply_cons(F, A, Bs)@.

\XXX{Are lambdas implemented any more efficiently than this behind the
scenese?}

(Note that in practice we would probably have just written
\begin{verbatim}
map(F, Xs) = foldr((func(X, Ys) = [F(X) | Ys]), [], Xs).
\end{verbatim}
and avoided the need for a named auxiliary function at all.)

\XXX{Watch out for procedure ambiguity using closures.}

\XXX{Mention that a fully applied func expression is applied whereas a
pred expression may not be.}

\XXX{Mention @call@ and @apply@.}

\XXX{What about restrictions on partial application?}

\section{Modes}
\section{* Monomorphism Restriction}
\section{* Monomoding Restriction}
\section{* Efficiency}




% % vim: ft=tex ff=unix ts=4 sw=4 et wm=8 tw=0

\chapter{* Type Classes}

Polymorphism allows us to define predicates that range over values of
any type.  This is very useful, but polymorphism places some
restrictions on what can be done with a polymorphically typed argument.
In fact, such arguments may only be compared, be interrogated about
their type (see the chapter on RTTI \XXX{}), and unified.

In some situations, one would like to do something different depending
upon the type of a polymorphic argument, without actually knowing what
that type is.  For instance, say we were writing a graphics package
which needed to manipulate various different representations of images,
from curves and polyhedra to textured surfaces.  Each such object may
well have a different representation in the program, but we would
like to be able to do such things as rotate and scale them regardless of
their type.  This is where type classes come in.

\section{Object Orientated Programming}

The last decade or so has seen a revolution in programming style towards
the near-universal adoption of object orientated programming languages.
The three reasons most often attributed to the success of OO techniques
are encapsulation, data hiding, and inheritance.

Encapsulation refers to the practice of keeping the representation of
data and the definitions of operations on that data together in one
place.

Data hiding refers to the fact that only some parts of the data
representation are made visible outside an object (or \emph{class
instance} in the vernacular); all other manipulation must be via the
suite of \emph{methods} (operations) that are associated with the
object.

Finally, inheritance usually includes both type inheritance and
implementation inheritance.

Type inheritance (sometimes known as \emph{interface} inheritance) is
where one \emph{class} of objects may be a \emph{subclass} of another
(the \emph{superclass}), meaning that an instance of the subclass will
also have data fields with the same names and types as those of the
superclass as well as including methods with the same names and
signatures of the superclass.  An instance of a subclass can always be
passed to an operation expecting an instance of a superclass (since all
the right bits and pieces will be in place and the operation in question
won't be any the wiser.)

Implementation inheritance is where the \emph{definitions} of methods
are also inherited from the superclass.

Of all these things, a strong case can be made for interface inheritance
being the real reason why OO techniques work \XXX{find a ref?}.
Certainly, there are many cases where implementation inheritance leads
to real maintenance problems, and the conflation of the different issues
can often compromise system design \XXX{refs}.

In Mercury, as with most declarative languages that support type
classes, we separate out these concerns.

Encapsulation is handled by using ADTs and type classes.

Data hiding is handled by using ADTs.

Interface inheritance is handled via type classes and existentially
quantified types.

Implementation inheritance is not supported as it is generally felt to
be a Bad Thing amongst the language design cognoscenti.

\section{An Example}

Say we wished to generalise over the class of numbers, so that we could
write code that would work on @int@s, @float@s, complex numbers,
rationals, arbitrary precision integers, Church numerals and so forth.
We would start by defining a type class that supported the basic set of
operations (or methods) we require:
\begin{verbatim}
:- typeclass number(T) where [
    func zero          = T,
    func one           = T,
    func from_int(int) = T,
    func    -T         = T,
    func T + T         = T,
    func T - T         = T,
    func T * T         = T,
    func T / T         = T
].
\end{verbatim}
We include the nullary methods @zero@ and @one@ because they are so
useful and will have different representations in each \emph{instance}
of @number@.

We can now define functions that will work on any kind of @number@:
\begin{verbatim}
:- func cube(T) = T <= number(T).

cube(X) = X * X * X.
\end{verbatim}
A function's type signature may have one or more \emph{type class
constraints} after a trailing @<=@.  In this case we can read the type
declaration of @cube@ as ``the function @cube@ takes an argument of type
@T@ and returns an argument of type @T@, provided @T@ is an instance of
the type class @number@.''
\XXX{This example sucks.  There must be some interesting non-trivial
mathematical function that we want for all kinds of number.  Hmm, maybe
an example using rectangular and/or polar coordinates would be better.
Or how about a hash table or pretty printer?}

The actual @*@ operation used in an application of @cube@ will, of
course, depend upon the \emph{type} of its argument.

Essentially, an instance declaration defines the \emph{method
dictionary} for a type with respect to a given type class.  A type
class constraint on a function signature says that the function also
takes a `hidden' argument which is an appropriate method dictionary.
The method invocations @*@ in the body are resolved by looking up
the corresponding method definition in the hidden method dictionary
argument.

Next, let's look at how we define @int@ as an instance of @number@:
\begin{verbatim}
:- instance number(int) where [
    zero        = 0,
    one         = 1,
    from_int(X) = X,
    -X          = 'int__-'(X),
    X + Y       = X `'int__+'` Y,
    X - Y       = X `'int__-'` Y,
    X * Y       = X `'int__*'` Y,
    X / Y       = X `'int__/'` Y
].
\end{verbatim}
(Recall that @`name`@ allows us to use @name@ as an infix operator and
that we have to @'@quote@'@ names that contain so-called graphical
characters; so @`'int__+'`@ means ``using the @+@ operator defined in
the @int@ module as an infix operator.'')

If we wanted to, we could instead have written
\begin{verbatim}
:- instance number(int) where [
    ...
    func(+) is 'int__+'
    func(-) is 'int__-'
    func(*) is 'int__*'
    func(/) is 'int__/'
].
\end{verbatim}
simply using the @int@ module implementations directly, rather than
supplying in-line definitions for the methods.  There is no problem
with mixing in-line definitions with @is@ definitions in an @instance@
declaration.

Making @float@ an instance of @number@ is very similar:
\begin{verbatim}
:- instance number(float) where [
    zero        = 0.0,
    one         = 1.0,
    from_int(X) = X,
    -X          = 'float__-'(X),
    X + Y       = X `'float__+'` Y,
    X - Y       = X `'float__-'` Y,
    X * Y       = X `'float__*'` Y,
    X / Y       = X `'float__/'` Y
].
\end{verbatim}

\subsection{Instance Definition Parameters}

Instance definition parameters must take the form @t(T1, T2, ..., Tn)@
(or just @t@ if the type @t@ is not parametric.)  Each of @T1@, @T2@,
\ldots, @Tn@ must be distinct type variables.

The reason for not allowing instance declarations on types like
@list(char)@ (as opposed to @list(T)@) is that defining instances of the
former would deny the possibility of more the latter, more general,
instance in the program.
\XXX{This sounds weak.  Is there a convincing explanation?}

Each type parameter must be distinct for much the same reason.
\XXX{Ditto.}

\subsection{Caveat OO Programmer}

\XXX{What's this in Latin?}

If you come from an OO background you may be tempted to use type
classes all over the place.  This is not, in general, a good idea.
There is a performance penalty associated with using type classes (if
the compiler cannot work out the type of an argument at a particular
call method site then it cannot optimize away the dictionary look-up
step) as well as a coding burden in that one has to supply type class
constraints wherever one wants that functionality available.

In many cases, an algebraic data type is a better option.

You have been warned.

\section{Another Example}

We want to code up an hash table implementation that will work for all
types of keys and values.  Moreover, we don't want to provide a
universal hash function (which we could do using Mercury's \XXX{RTTI}
facility) since different hash functions may be better depending upon
each application.  The solution is to use type classes.

This type class simply specifies that instances have an associated
@hash@ method which converts values into (hopefully good) @int@ hashes.
\begin{verbatim}
:- typeclass hashable(T) where [
    func hash(T) = int
].
\end{verbatim}
Now we define our hash table to be an (@int@ indexed) @array@ of
buckets:
\begin{verbatim}
:- import_module array, int.

:- type hash_table(K, V) == array(assoc_list(K, V)).
:- mode hash_table_di == array_di.
:- mode hash_table_ui == array_ui.
:- mode hash_table_uo == array_uo.

:- func set(K,  V,  hash_table(K, V)) = hash_table(K, V)
            <= hashable(K).
:- mode set(in, in, hash_table_di   ) = hash_table_uo is det.

set(K, V, HT) = ( HT ^ elem(index(K, HT)) := [K - V | HT ^ elem(H)] ).

:- func lookup(K,  hash_table(K, V)) = V
            <= hashable(K).
:- mode lookup(in, hash_table_ui   ) = out is semidet.

lookup(K, HT) = HT ^ elem(index(K, HT)) ^ elem(K).

:- func index(K,  hash_table(K, V)) = int
            <= hashable(K).
:- mode index(in, hash_table_ui   ) = out is det.

index(K, HT) = hash(K) `mod` size(HT).
\end{verbatim}
\XXX{Make sure we add @elem@ and @det\_elem@ to @assoc\_list@.}

We can construct hash functions for all our favourite types:
\begin{verbatim}
:- import_module int, char, string.

:- instance hashable(int) where [
    hash(X) = X
].

:- instance hashable(string) where [
    hash(S) =
        foldl(
            func(C, H) = (H >> 28) `xor` ((H << 5) \/ to_int(C)),
            S,
            0
        )
].
\end{verbatim}
and so forth.

Hash tables of type @hash_table(int, T)@ will use the @int@ @hash@
method and hash tables of type @hash_table(string, T)@ will use the
@string@ @hash@ method.

Note that it is not possible to have more than one instance definition
for a given type in a program.  The reason for this is that there must
be no ambiguity when working out which method implementation should be
called in any given situation.
\XXX{Well, I think this should rather be ``in the same scope.''  I think
that this will actually be the easiest thing to do anyway when we move
to an explicit dictionary passing scheme.}

\section{Type Class Constraints on Type Classes}

Just as type class constraints can be placed on function and predicate
and method declarations, they can equally be applied to type class and 
instance definition.

If we place a type class constraint on a type class definition then we
are constructing an hierachy of type classes.

As an example, for many applications, the most efficient type of hash
table uses double hashing, rather than chaining buckets off a single
hash generated index.  For double hashing we probe at indices
@h1(K) + I * h2(K)@ (modulo the size of the hash table) for
$@I@ \in \{0, 1, 2, \ldots\}$, looking for the correct entry or an empty
slot, depending upon whether we are performing an insertion or a lookup.

We want to add a new type class for types with a second hash function:
\begin{verbatim}
:- typeclass hash2able(T) <= hashable(T) where [
    func hash2(T) = int
].
\end{verbatim}
This says that to be able to declare a type $T$ as an instance of
@hash2able@, it must first also be declared as an instance of
@hashable@.  Therefore, every instance of @hash2able@ is a
perfectly good argument to anything expecting an instance of @hashable@.

This relationship is transitive: if @a@ is a sub-type class of @b@ and
@b@ is a sub-type class of @c@, then @a@ is also a sub-type class of
@c@.

Note that there can be no loops in the hierarchy: it is not acceptable
for type class @a@ to be a sub-type class of @b@ and simultaneously have
@b@ as a sub-type class of @a@.

Back to the example.
\begin{verbatim}
:- type hash2_table(K, V) == array(maybe(K, V)).
:- mode hash2_table_di == array_di.
:- mode hash2_table_ui == array_ui.
:- mode hash2_table_uo == array_uo.

:- func find_empty(int, int, hash2_table(K, V)) = int.
:- mode find_empty(in,  in,  hash2_table_ui   ) = out is det.

find_empty(I, H2, HT) =
    ( if   HT ^ elem(I `mod` size(HT)) = no
      then I
      else find_empty((I + H2) `mod` size(HT), H2, HT)
    ).

:- func find_value(K,  int, int, hash2_table(K, V)) = V.
:- mode find_value(in, in,  in,  hash2_table_ui   ) = out is det.

find_key(K, I, H2, HT) = V :-
    HT ^ elem(I `mod` size(HT)) = yes(K0, V0),
    ( if   K = K0
      then V = V0
      else find_key(K, I + H2, H2, HT)
    ).

:- func set(K,  V,  hash2_table(K, V)) = hash2_table(K, V)
            <= hash2able(K).
:- mode set(in, in, hash2_table_di   ) = hash2_table_uo is det.

set(K, V, HT) = HT ^ elem(I) := yes(K, V) :-
    I = find_empty(hash(K) `mod` size(HT), hash2(K), HT).

:- func lookup(K,  hash2_table(K, V)) = V
            <= hash2able(K).
:- mode lookup(in, hash2_table_ui   ) = out is semidet.

lookup(K, HT) = find_key(K, hash(K), hash2(K), HT).
\end{verbatim}
(Note that the above implementation omits some important details.  These
include the fact that this kind of hash table can fill up and hence may
need resizing once occupancy becomes high -- the hash table must never
become completely full -- and that the hash values should be coprime
with the size of the hash table -- something that can be arranged by,
say, multiplying the hash values by three and five respectively and
ensuring that the size of the hash table is always a power of two.)
\XXX{Should I put in all the checks and balances anyway?  I'm only using
it as an illustrative example for type classes.}

Looking at the signatures for @set@ and @lookup@, we observe that the
constraint @hash2able(T)@ implicitly includes the constraint
@hashable(T)@ since by definition the latter is a sub-type class of the
former, hence we can use the @hash@ method without having to explicitly
state that $T$ must be an instance of @hashable@.

\section{Type Class Constraints on Instance Definitions}

Sometimes it is useful to place a type class constraint on the type
variable parameters in an instance definition.  For example, to make
@list@s of @hashable@ things @hashable@, we would write
\begin{verbatim}
:- instance hashable(list(T)) <= hashable(T) where [
    hash([])       = 0,
    hash([X | Xs]) = hash(X) `xor` hash(Xs)
].
\end{verbatim}
So @hash(X)@ calls the implementation for @T@ and @hash(Xs)@ calls the
implementation for @list(T)@ -- which is what is defined here!

As an aside, it is quite all right to use methods as higher order
functions.  We could quite easily have written
\begin{verbatim}
:- instance hashable(list(T)) <= hashable(T) where [
    hash(Xs) = foldl(func(X, H) = hash(X) `xor` H, Xs, 0)
].
\end{verbatim}
and obtained the same effect.

Note that type class constraints may only range over type variables and
not structured types.  That is, @<= foo(T)@ is acceptable, but
@<= foo(list(T))@ is not.
\XXX{Check this.}

\section{Multiple Type Class Constraints}

In some cases we may want to place more than one type class constraint
on a type variable, be it on a predicate, function or type class or
instance definition.

A common idiom is to construct an ``empty'' type class with no methods
that only serves to group a number of other type classes together.  As
an example, we might describe a geometrical shape as something that has
both a list of vertices and a colour.  Each of these properties is
independent of the other, so there should be no sub-type classing
between them.  Instead, we would write:
\begin{verbatim}
:- typeclass shape(T) <= (has_vertices(T), has_colour(T)) where [].
\end{verbatim}
Note that in order to get the syntax right, we have to group multiple
type class constraints in parentheses.

There is, of course, no reason why @shape@ could not also have methods
of its own if required.

This is an example of \emph{multiple inheritance}, albeit of a set of
method signatures, rather than their implementation.

Most languages with built-in OO support do not allow multiple
inheritance of implementation because it means both complicating the
language semantics (what do you do when two methods from different
superclasses have the same name?) and the language run-time (you
generally pay a cost everywhere for such a facility, even if you don't
use it) and can lead to real code maintenance problems.
\XXX{Do I need to substantiate these claims?}

Several languages, such as Java and \Csharp, have included the notions of
\emph{interfaces}, which serve much the same purpose as type classes.
Since we are only allowed to inherit method signatures, there is no
problem (provided we insist that no method in a sub-type class may have the
same name, arity and signature as another method in a super-type class.)

\section{Abstract Instances}

It is not necessary to reveal to the outside world the details of how a
particular instance is implemented.  Instead, we can do the same thing
as we do for abstract types:
\begin{verbatim}
:- module m.
:- interface.

:- type t ...

:- typeclass c(T) where [ ... ].

:- instance c(t).           % An abstract instance declaration.

:- implementation.

:- instance c(t) where [    % Here we define the instance.
    ...
].
\end{verbatim}

\section{Instances of Abstract Types}

It is not possible to declare an abstract type that is elsewhere defined
as an equivalence type as a type class instance.  The reason for this is
probably best explained by example.  Say someone else has written module
@a@:
\begin{verbatim}
:- module a.
:- interface.

:- type t.

:- typeclass c(T) where [ ... ].
\end{verbatim}
and I write module @b@:
\begin{verbatim}
:- module b:
...
:- implementation.
:- import_module a.

:- instance c(t) where [ ... ].
\end{verbatim}
Since @t@ is abstract, we have no way of knowing whether the
implementation section of @a@ goes on to say
\begin{verbatim}
:- interface.
:- import_module int.

:- type t == int.

:- instance c(t) where [ ... ].
\end{verbatim}
thereby leaving method resolution on @int@ with respect to @c@
ambiguous.
\XXX{I don't think I have the argument right here...}

\XXX{This isn't convincing.  Have I got this right?  Either way, isn't
this just an artifact of having instances being resolved at link time
rather than as a scope issue...}

The upshot of this is that if you want to declare an abstract type to be
a type class instance, you should first wrap it up in a no-tag type:
\XXX{We should really have a section on no-tag types in the Types
chapter.}
\begin{verbatim}
    % t is an abstract type defined elsewhere.
    %
:- type my_t ---> my_t(t).

:- instance c(my_t) where [ ... ].
\end{verbatim}
Since we have \emph{wrapped up} the type @t@, there can be no confusion
between which methods for @c@ we should call.  We pay nothing (other
than a little typing) for using no-tag types since their representation
is identical to that of the type they wrap.
\XXX{Perhaps I could say this better.}

\section{Multi-Parameter Type Classes}

Type classes are not restricted to a single parameter.  It is sometimes
useful to be able to describe method suites that relate two or more
type variables (every type parameter to a type class must be a distinct
type variable.)

Consider the following simple multi-parameter type class definition used
to specify where values of one type may be ``cast'' (or coerced) into
values of another type:
\begin{verbatim}
:- typeclass castable(T1, T2) where [
    func cast(T1) = T2
].
\end{verbatim}
We can now set up instances allowing us to convert @int@s into @float@s
and @string@s and even (wrapped) @list@s of @char@:
\begin{verbatim}
:- instance castable(int, float) where [
    cast(X) = float(X)
].
:- instance castable(int, string) where [
    cast(X) = format("%d", [i(X)])
].

:- type wrapped_chars ---> chars(list(char)).

:- instance castable(int, wrapped_chars) where [
    cast(X) = chars(to_char_list(format("%d", [i(X)])))
].
\end{verbatim}
\XXX{Would it be better to use @cast(X)@ there at the end, since we
already have @castable(int, string)@?}

Note that every method signature in a type class definition must contain
\emph{every} type variable that is a parameter to the type class.
Without this constraint we could have ambiguous situations like the
following:
\begin{verbatim}
:- typeclass foo(T1, T2) where [
    func f(T1) = T2,
    func g(T1) = T1     % Error!  Doesn't mention T2.
].

:- instance foo(int, float) where [
    f(X) = 1.0,
    g(X) = X + 1
].
:- instance foo(int, string) where [
    f(X) = "bar",
    g(X) = X + 2
].
\end{verbatim}
Now there is no way of working out, \emph{in any context}, whether the
method call @g(123)@ should invoke the @foo(int, float)@ version or the
@foo(int, string)@ version.

\section{What Type Classes Can't Do (Yet)}

Work is afoot to extend the Mercury type class system to include
\emph{functional dependencies} and \emph{constructor classes}.

Functional dependencies allow one to specify that given one type
variable, another is necessarily defined.  That is, if we write
@(T1 -> T2)@ to say that ``given'' @T1@ we may always ``infer'' @T2@ for
this type class, then we could say:
\begin{verbatim}
:- typeclass contains(T1, T2) <= (T1 -> T2) where [
    pred contains(T1, T2),
    mode contains(in, in) is semidet
\end{verbatim}
and the functional dependency constraint essentially constitutes a
promise that if we, say,  declare an instance @contains(bitmap, int)@,
we will not elsewhere declare an instance @contains(bitmap, char)@.

Constructor classes refers to the ability to use only partially
specified types as type class parameters.  For instance, we might want
to say something like
\begin{verbatim}
:- typeclass mappable(T) where [
    func map(func(T1) = T2, T(T1)) = T(T2)
].
\end{verbatim}
supporting instance definitions like
\begin{verbatim}
:- instance mappable(list) where [
    map(_, []      ) = [],
    map(F, [X | Xs]) = [F(X) | map(F, Xs)]
].
:- instance mappable(array) where [
    map(F, A) = array(map(F, to_list(A)))
].
\end{verbatim}

\XXX{Should probably go through the collection type class problem in
some detail.}

\section{Existentially Quantified Types}

One thing we have yet to demonstrate is how to form heterogeneous
collections.  Say we have a typeclass
\begin{verbatim}
:- typeclass stringable(T) where [
    func string(T) = string
].
\end{verbatim}
with such things as @int@, @float@, @char@ as instances.  Our goal is to
write a predicate to print out a list a @stringable@ values, regardless
of their type.  Unfortunately we cannot simply write
\begin{verbatim}
:- pred write_strings(list(T), io, io) <= stringable(T).
:- mode write_strings(in,      di, uo) is det.

write_strings([],       !IO).
write_strings([X | Xs], !IO) :-
    io__write_string(string(X), !IO),
    write_strings(Xs, !IO).
\end{verbatim}
because this would restrict the first argument to all members of the
same type.  What we want to be able to say is, given @X = 10@,
@Y = " green bottle"@, Z = @'s'@ is to just call
@write_strings([X, Y, Z])@ and see @10 green bottles@ appear in the
output.

Existentially quantified types (sometimes abbreviated to just
existential types) to do what we want.  Here's how we could use them to
solve the problem with @write_strings@:
\begin{verbatim}
:- type some_stringable
    --->    some [T] a_stringable(T) => stringable(T).

:- pred write_strings(list(some_stringable), io, io).
:- mode write_strings(in,                    di, uo) is det.

write_strings([],                     !IO).
write_strings([a_stringable(X) | Xs], !IO) :-
    io__write_string(string(X), !IO),
    write_strings(Xs, !IO).
\end{verbatim}
We start by introducing a new type, @some_stringable@.  This type has a 
single, existentially quantified constructor called @a_stringable@ whose
single argument is constrained by the fact that whatever it is, it must
be an in an instance of @stringable@.

Note that the type variable @T@ does not appear in the head of the type
definition because it is bound to the constructor by the existential
quantifier, @some@.  This is what allows us to construct heterogeneous
collections of @stringable@ instance values.

The constraint @=> stringable(T)@ has the arrow pointing in a different
direction, because existentially quantified types place a burden on the
\emph{caller's} ability to handle a value, rather than the
\emph{callee's}.  Because of this directionality, existentially
quantified types are generally ``outputs'' since a caller has no way of
knowing which particular type is actually allowed.
\XXX{This needs clarifying.}

The compiler cannot work out without help \XXX{(Is this in general or
just hard or just a SMOP?)} whether an occurrence of an existentially
quantified data constructor is meant as a construction or
deconstruction.  To solve this problem, we insist that
\emph{constructions} be preceeded with `@new @' (note the space), while
\emph{deconstructions} are not.  Thus, to construct our heterogeneous
list of @stringable@s, we would write
\begin{verbatim}
    L = [ 'new a_stringable'(X),
          'new a_stringable'(Y),
          'new a_stringable'(Z) ]
\end{verbatim}
and we can now pass @L@ to @write_strings@ and everything will work
fine.

There is no problem with combining ordinary type class constraints with
existentially quantified type class constraints; the only rule is that
the existential quantifier and corresponding constraints must go at the
outermost level:
\begin{verbatim}
:- some [T2] pred foo(T1, T2) => bar(T1) <= baz(T2).
\end{verbatim}
where @T1@ is universally quantified and @T2@ existentially quantified.
\XXX{Check to see if I need extra parentheses anywhere here.}


% % vim: ft=tex ff=unix ts=4 sw=4 et wm=8 tw=0

\chapter{The Standard Utility Library Module}




% \section{Lists}

Lists are perhaps the single most useful data structure in the
programmer's armoury.

The @list@ module in the Mercury standard library defines lists as
follows:
\begin{verbatim}
:- type list(T) ---> []
                ;    [T | list(T)].
\end{verbatim}
The notation @[A | B]@ is special syntactic sugar recognised by the
Mercury parser for @[|](A, B)@ -- that is, @[|]/2@ is the basic list
data constructor, with @[]@ standing for the empty list.

Moreover, an expression of the form @[A, B, C]@ is syntactic sugar for
@[A | [B | [C | []]]]@ which, of course, is identical to
@[|](A, [|](B, [|](C, [])))@.

Also, an expression of the form @[A, B, C | Xs]@ is syntactic sugar for
@[A | [B | [C | Xs]]]@ which, of course, is identical to
@[|](A, [|](B, [|](C, Xs)))@.

These are obviously singly linked lists.  \XXX{Discussion of the pros
and cons of the various list ADTs out there.}

\subsection{The Main List Operations}

The higher order functions @map@, @foldl@ and @foldr@ provide easy ways
of iterating over lists, either transforming them member by member or
somehow accumulating the result of processing each member in turn.

The section on higher order programming \XXX{} has already dealt with
@map@, @foldl@ and @foldr@ in some detail, so we merely summarise that
discussion here and concentrate on operations we have not previously
examined.

\subsubsection{Miscellany}

\begin{verbatim}
    % length([X1, X2, X3, ..., XN]) = N
    %
:- func length(list(T)) = int.

length(Xs) = foldl((func(_, N) = N + 1), Xs, 0).

    % reverse([X1, X2, X3, ..., XN]) = [XN, ..., X3, X2, X1]
    %
:- func reverse(list(T)) = list(T).

reverse(Xs) = foldl((func(X, Ys) = [X | Ys]), Xs, []).
\end{verbatim}

\subsubsection{Membership}

\XXX{Should have a section on equality and @compare@ and so forth.}

@member@ is used to decide membership of a @list@ under equality and to
non-deterministically project members from a @list@ (the argument ordering
is arguably unfortunate for higher order programming, but this is
historically how things have been done):
\begin{verbatim}
    % member(X, Xs) iff X is a member of Xs.
    %
:- pred member(T,   list(T)).
:- mode member(in,  in     ) is semidet.
:- mode member(out, in     ) is nondet.

member(X, [X | _ ]).
member(X, [_ | Xs]) :- member(X, Xs).
\end{verbatim}
From time to time one wants to access members of a @list@ by their index
(\ie distance from the start of the @list@).  There are two sets of
operations for doing so, depending upon whether it is most convenient to
give the head of the list an index of 1 or 0:
\begin{verbatim}
    % index0(Xs, I, X) iff 0 =< I < length(Xs) and
    % X is the I+1th member of Xs.  That is,
    % index0([X1, X2, X3], 1, X2) while
    % index0([X1, X2, X3], 3, _ ) fails.
    %
:- pred index0(list(T), int, T  ).
:- mode index0(in,      in,  out) is semidet.

    % index1(Xs, I, X) iff 1 =< I =< length(Xs) and
    % X is the Ith member of Xs.  That is
    % index1([X1, X2, X3], 1, X1) while
    % index1([X1, X2, X3], 3, _ ) fails.
    %
:- pred index1(list(T), int, T  ).
:- mode index1(in,      in,  out) is semidet.

    % These functions correspond to the predicates above, but differ
    % in that an exception is thrown if the index is out of range.
    %
:- func index0(list(T), int) = T.
:- func index1(list(T), int) = T.
\end{verbatim}

\subsubsection{Mapping}

@map@ applies its function argument to each member of its @list@ argument
and returns the corresponding @list@ of results.
\begin{verbatim}
    % map(F, [X1, X2, X3, ...]) = [F(X1), F(X2), F(X3), ...]
    %
:- func map(func(T1) = T2, list(T1)) = list(T2).

map(_, []      ) = [].
map(F, [X | Xs]) = [F(X) | map(F, Xs)].
\end{verbatim}

A related function is @map_corresponding@ which is used to map a
function combining the corresponding members of \emph{two} lists
(@map_corresponding@ will throw an exception if the lists are of
different lengths.)
\begin{verbatim}
    % map_corresponding(F, [X1, X2, X3, ...], [Y1, Y2, Y3, ...]) =
    %       [F(X1, Y1), F(X2, Y2), F(X3, Y3), ...]
    %
:- func map_corresponding(func(T1, T2) = T3, list(T1), list(T2)) =
            list(T3).

map_corresponding(_, [],       []      ) = [].
map_corresponding(_, [],       [_ | _ ]) = <<throw exception>>.
map_corresponding(_, [_ | _ ], []      ) = <<throw exception>>.
map_corresponding(F, [X | Xs], [Y | Ys]) =
    [F(X, Y) | map_corresponding(F, Xs, Ys)].
\end{verbatim}

There is also a @map_corresponding3@ which works for functions of three
arguments:
\begin{verbatim}
:- func map_corresponding3(func(T1, T2, T3) = T4,
            list(T1), list(T2), list(T3)) = list(T4).
\end{verbatim}

\XXX{Should probably include a subsubsection on zipping and
interleaving.}

\subsubsection{Folding}

The two main folding operations are @foldl@ and @foldr@.  We have
already seen a definition of @foldr@ (the version supplied in the
Mercury standard library unfortunately uses a slightly different
argument ordering):
\begin{verbatim}
    % foldr(F, [X1, X2, X3], A) = F(X1, F(X2, F(X3, A)))
    %
:- func foldr(func(T1, T2) = T2, list(T1), T2) = T2.

foldr(_, [], A) = A.
foldr(F, [X | Xs], A) = F(X, foldr(Xs, A)).
\end{verbatim}
In many situations the function @F@ will be commutative or we will want
to process the @list@ starting with the leftmost member, in which case
it is more efficient to use the tail recursive @foldl@:
\begin{verbatim}
    % foldl(F, [X1, X2, X3], A) = F(X3, F(X2, F(X1, A)))
    %
:- func foldl(func(T1, T2) = T2, list(T1), T2) = T2.

foldl(_, [],       A) = A.
foldl(F, [X | Xs], A) = foldl(F, Xs, F(X, A))).
\end{verbatim}
As an example, here's how we could define the @reverse@ function, as
well as more efficient versions of the @sum@ and @prod@ functions
introduced in the section on higher order programming \XXX{}:
\begin{verbatim}
reverse(Xs) = foldl((func(X, Ys) = [X | Ys]), Xs, []).
sum(Xs)     = foldl((func(X, A ) = X + A   ), Xs, 0 ).
prod(Xs)    = foldl((func(X, A ) = X * A   ), Xs, 1 ).
\end{verbatim}

\subsubsection{XXX More To Come}

\subsection{General Advice}

\XXX{This probably deserves its own top-level section.}

While lists are easy to understand and work with, most programmers show
an unfortunate tendency to use lists when another data structure may be
more appropriate.  It is worth spending some time looking at the various
types provided by the Mercury standard library to see what is available.
The decision as to which data structure is best for a given situation is
one that can only be made in the light of experience, although as a rule
of thumb you cannot go far wrong by picking the data structure with the
most useful set of support functions for the problem in hand
(occasionally an @assoc_list@ may be preferable to a @map@, but in most
situations it won't be.)


% % vim: ft=tex ff=unix ts=4 sw=4 et wm=8 tw=0

\chapter{Association Lists}




% % vim: ft=tex ff=unix ts=4 sw=4 et wm=8 tw=0

\chapter{Maps}

The @map@ data type is provided by the @map@ module in the standard
Mercury library.

After @list@s, @map@s are perhaps the most widely used non-trivial
Mercury data type.  A @map@ is essentially a dictionary structure
(or \emph{associative array})
mapping \emph{keys} to \emph{values}.  The core operations involve
setting up a mapping from key to value, changing the mapping for a
given key, and looking up the value associated with a given key.

The @map@ data type is parameterised by the types of keys and values:
\begin{verbatim}
:- type map(K, V).
\end{verbatim}
Maps have efficient $O(log n)$ cost bounds on all the basic
operations for a @map@ containing $n$ mappings (the reader may be
interested to know they are currently implemented using 234-trees
\XXX{ref}).

Here is a small example of a very basic telephone directory
application using a @map@ to store the mapping from names to
phone numbers:
\begin{verbatim}
:- import_module map, list, string, std_util, assoc_list, exception.

:- pred main(io, io).
:- mode main(di, uo) is det.

main(!IO) :-
    io.read_line_as_string(Result, !IO),
    (
        Result = eof
    ;
        Result = error(_),
        throw(Result)
    ;
        Result = ok(Name0),

            % Chop off the trailing new line character and
            % convert to lower case.
            %
        Name = to_lower(substring(Name0, 0, length(Name0) - 1)),

        ( if   phone_book ^ elem(Name) = Number
          then io.format("%d\n", [s(Number)], !IO)
          else io.format("`%s' is not in the phone book.\n",
                    [s(Name0)], !IO)
        ),
        main(!IO)
    ).

    % We memoize the result of this function since we only
    % want to build it once.
    %
:- func phone_book = map(string, string).
:- pragma memo(phone_book/0).

phone_book = map.from_assoc_list([
        "ralph" -       "9873 1234",
        "tyson" -       "9873 4342",
        "fergus" -      "9873 1237",
        "zoltan" -      "9876 8754",
        ...
    ]).
\end{verbatim}
\XXX{Is this a bit too ``clever'' for a tutorial example?}
\XXX{There's probably a better example.}

\section{The Main Map Operations}

An empty map is constructed by calling the nullary @init@ function:
\begin{verbatim}
:- func init = map(K, V).
\end{verbatim}
There are predicates to decide whether a given map is empty or
contains a mapping for a particular key:
\begin{verbatim}
:- pred map.is_empty(map(K, V)).
:- mode map.is_empty(in) is semidet.

:- pred map.contains_key(map(K, V), K).
:- mode map.contains_key(in,        in) is semidet.
\end{verbatim}
We have two field-access like functions for lookup up values
associated with keys.  The first, @elem@, simply fails if there
is no mapping for the given key.  The second, @det_elem@, will
throw an exception if this is the case.
\begin{verbatim}
:- func elem(K, map(K, V)) = V is semidet.

:- func det_elem(K, map(K, V)) = V.
\end{verbatim}
So the expression @Map ^ elem(Key)@ denotes the value associated
with @key@, if any.

Mappings can be added or changed using the field assignment-like
functions @'elem :='@ and @'det_elem :='@.  The expression
@Map ^ elem(Key) := Value@ always succeeds, returning an updated
version of @Map@ overwriting the mapping for @Key@, if any, in @Map@
or adding a new mapping if @Map@ does not contain one.  The expression
@Map ^ det_elem(Key) := Value@ is similar, except that it will throw
an exception if @Map@ does not already contain a mapping for @Key@.
\begin{verbatim}
:- func 'elem :='(K, map(K, V), V) = map(K, V).

:- func 'det_elem :='(K, map(K, V), V) = map(K, V).
\end{verbatim}

\section{Bulk Initialisation and Update}

The @map@ module provides several means of initialising and updating
@map@s from other structures relating keys to values.

\subsection{Association Lists}

The simplest dictionary type is @assoc_list@.
\begin{verbatim}
:- func from_assoc_list(assoc_list(K,V)) = map(K,V).

:- func from_sorted_assoc_list(assoc_list(K,V)) = map(K,V).
\end{verbatim}
These two functions both construct a new map from the @Key - Value@
mappings in the @assoc_list@ argument.  Mappings are inserted in the
order in which the @Key - Value@ pairs appear in the @assoc_list@.

Example (the 1926 estimates of Dr Catherine Morris Cox):
\begin{verbatim}
    IQs = from_assoc_list([
        "Bobby Fischer"   - 187,
        "Galileo Galilei" - 185,
        "Rene Descartes"  - 180,
        "Immanuel Kant"   - 175,
        "Charles Darwin"  - 165,
        "Albert Einstein" - 160
    ])
\end{verbatim}
If the @assoc_list@ in question is already sorted with distinct keys in
ascending order then using @from_sorted_assoc_list@ \emph{may} be faster
(but there's certainly no point in separately sorting the @assoc_list@
just to use this function instead!)  \XXX{It's not clear to me that we
really want to mention this one at all, especially since they're
currently implemented identically.}

The following function allows one to make several updates to a @map@
from an @assoc_list@:
\begin{verbatim}
:- func set_from_assoc_list(map(K,V), assoc_list(K, V)) = map(K,V).
\end{verbatim}

Example (wild guesses):
\begin{verbatim}
    NewIQs = set_from_assoc_list(IQs, [
        "Ralph Becket"     - 253,
        "Fergus Henderson" - 211,
        "Albert Einstein"  - 161,   % Revised estimate.
        "Tyson Dowd"       -  86
    ])
\end{verbatim}

\subsection{Corresponding Pairs in Lists}

Occasionally one has keys and values in separate @list@s.  One can
construct a @map@ from them using the following function:
\begin{verbatim}
:- func from_corresponding_lists(list(K), list(V)) = map(K, V).
\end{verbatim}
The @list@s must be the same length; if they aren't then this 
function will throw an exception.

Example:
\begin{verbatim}
    QualityOfPet =
        from_corresponding_lists(
            [cat, dog, snake], [good, bad, inadvisable]
        )
\end{verbatim}
giving @QualityOfPet ^ elem(cat) = good@ and
@QualityOfPet ^ elem(dog) = bad@ and
@QualityOfPet ^ elem(snake) = inadvisable@.
\XXX{This example may also need a little work.}

\section{Set-Like Operations}

\subsection{Union}

One can use the following functions to merge @map@s together:
\begin{verbatim}
:- func merge(map(K, V), map(K, V)) = map(K, V).

:- func overlay(map(K,V), map(K,V)) = map(K,V).

:- func union(func(V, V) = V, map(K, V), map(K, V)) = map(K, V).
\end{verbatim}
The @merge@ function will throw an exception if the two maps have a key
in common.  The @overlay@ function will not, taking the values for
common keys from the second @map@ argument in the resulting @map@.

Example:
\begin{verbatim}
    RoutesA = from_assoc_list([
        adelaide - melbourne,
        perth - adelaide,
        melbourne - canberra
    ]),
    RoutesB = from_assoc_list([
        canberra - sydney,
        sydney - brisbane,
        brisbane - darwin,
        melbourne - alice_springs
    ]),
    Routes = overlay(RoutesA, RoutesB)
\end{verbatim}
will result in @Routes ^ elem(perth) = adelaide@ and
@Routes ^ elem(sydney) = brisbane@ and
@Routes ^ elem(melbourne) = alice_springs@.
\XXX{Another bad example.  Should use this one for @multi\_map@.}

The @union@ function also takes an argument that is used to decide which
value is used when the @map@ arguments have a key in common.  For
example:
\begin{verbatim}
    MapA = from_assoc_list([a - 1, b - 2, c - 3, d - 4]),
    MapB = from_assoc_list([a - 4, b - 3, c - 2, e - 1]),
    Map  = union(int.max, MapA, MapB)
\end{verbatim}
will result in
\begin{verbatim}
    to_sorted_assoc_list(Map) = [a - 4, b - 3, c - 3, d - 4, e - 1]
\end{verbatim}

\subsection{Intersection}

One can use the following to compute the ``intersection'' of two @map@s:
\begin{verbatim}
:- func intersect(func(V, V) = V, map(K, V), map(K, V)) = map(K, V).
\end{verbatim}
The function argument performs the same job as the one in @union@,
hence:
\begin{verbatim}
    MapA = from_assoc_list([a - 1, b - 2, c - 3, d - 4]),
    MapB = from_assoc_list([a - 4, b - 3, c - 2, e - 1]),
    Map  = intersect(int.max, MapA, MapB)
\end{verbatim}
will result in
\begin{verbatim}
    to_sorted_assoc_list(Map) = [a - 4, b - 3, c - 3]
\end{verbatim}

\section{Mapping and Folding}

\subsection{Mapping}

The following function will apply a transformation function to every
value in a @map@:
\begin{verbatim}
:- func map_values(func(K, V) = W, map(K, V)) = map(K, W).
\end{verbatim}
For example:
\begin{verbatim}
    CharInts    = from_assoc_list([a - 1, b - 2, c - 3]),
    CharStrings = map_values(int_to_string, CharInts)
\end{verbatim}
giving
\begin{verbatim}
    to_sorted_assoc_list(CharStrings) = [a - "1", b - "2", c - "3"]
\end{verbatim}

\subsection{Folding}

To compute a value from every key-value mapping (in ascending key order,
if that is significant), use
\begin{verbatim}
:- func foldl(func(K, V, T) = T, map(K, V), T) = T.
\end{verbatim}
For example:
\begin{verbatim}
    CharInts    = from_assoc_list([a - 1, b - 2, c - 3]),
    Sum         = foldl(func(_, X, A) = X + A, CharInts, 0)
\end{verbatim}
will compute @Sum = 6@.

\subsection{Combined Mapping and Folding}

\XXX{Do I really want to mention @map\_foldl@?}

\section{Miscellaneous Operations}

\XXX{This section needs examples throughout.}

One can delete the mappings for one or a @list@ of keys using the
following:
\begin{verbatim}
:- func delete(map(K,V), K) = map(K,V).

:- func delete_list(map(K,V), list(K)) = map(K,V).

:- pred map.remove(map(K,V), K, V, map(K,V)).
:- mode map.remove(in, in, out, out) is semidet.

:- pred map.det_remove(map(K,V), K, V, map(K,V)).
:- mode map.det_remove(in, in, out, out) is det.
\end{verbatim}
The @delete@ and @delete_list@ functions simply ignore key arguments
that do not have mappings in the @map@ argument.

The predicate @map.remove@ will fail if the given key is not present,
otherwise it returns both the value mapping of the key and a version of
the @map@ that does not contain a mapping for the key.

The present @map.det_remove@ is similar, except that it will throw an
exception if the given @map@ does not have a mapping for the key in
question.

Related functions are @select@ which allows one to specify a subset of
keys to retain the mappings for:
\begin{verbatim}
:- func select(map(K, V), set(K)) = map(K, V).
\end{verbatim}
That is, @select(Map, KeySet)@ returns the @map@ containing
only those mappings found in @Map@ whose keys are members of @KeySet@.

One can obtain a count of the number of mappings in a map with
\begin{verbatim}
:- func count(map(K, V)) = int.
\end{verbatim}

One can obtain (sorted) lists of keys and values stored in a @map@:
\begin{verbatim}
:- func keys(map(K, V)) = list(K).

:- func sorted_keys(map(K, V)) = list(K).

:- func values(map(K, V)) = list(V).
\end{verbatim}
The functions @keys@ and @sorted_keys@ are identical, except that the
latter guarantees to return the keys in ascending order.  The values
returned by @values@ are not guaranteed to be in any order.

The following convert @map@s to @assoc_list@s:
\begin{verbatim}
:- func to_assoc_list(map(K,V)) = assoc_list(K,V).
:- func to_sorted_assoc_list(map(K,V)) = assoc_list(K,V).
\end{verbatim}
@to_sorted_assoc_list@ returns the corresponding @assoc_list@ with keys
in ascending order.


% % vim: ft=tex ff=unix ts=4 sw=4 et wm=8 tw=0

\chapter{Arrays}




% \section{Compiling Programs}
\subsection{Mmake}
\subsection{Compiler Flags}
\subsection{Compilation Grades}




% % vim: ft=tex ff=unix ts=4 sw=4 et wm=8 tw=0

\chapter{* Stores}




% \section{* Exceptions}
\subsection{Throwing}
\subsection{Catching}
\subsubsection{Effect On Determinism}
\subsubsection{Restoring (Plain) Determinism (promise\_only\_solution)}




% \chapter{* Foreign Language Interface}
\section{Declarations}
\section{Data Types}




% \section{* Impure Code}
\subsection{Levels of Purity}
\subsection{Effect of Impurity Annotations}
\subsection{Promising Purity (pragma promise\_pure)}




% \section{* Pragmas}
\subsection{Inlining}
\subsection{Type Specialization}
\subsection{Obsolescence}
\subsection{Memoing}
\subsection{...Promises}




% \section{* Debugging}
\subsection{Compiling For Debugging}
\subsection{Basic Tour of the Debugger}
\subsection{Declarative Debugging}




% \section{* Optimization}
\subsection{When to Do It and When to Avoid It}
\subsection{Profiling}
\subsection{Various Considerations}
\subsection{An Overview of Contemporary Optimizer Technology}




% \include{RTTI}

\end{document}
