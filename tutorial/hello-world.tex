% vim: ft=tex ff=unix ts=4 sw=4 et tw=76

\chapter{Mercury By Example}

This chapter aims to convey through examples a basic feeling for how
Mercury works.  Pedagogical efficiency at this stage requires some glossing
over of finer details and a certain lattitude in precision; these
deficiencies are remedied in later chapters.  The approach taken here is to
start by presenting the ``obvious'' solution to each problem and then
introduce features of Mercury that allow for more elegant or efficient
programs (although we are not necessarily trying to write the most efficient
programs possible at this point!)

By convention, we use a @fixed width typeface@ for source code, code
fragments, and output from programs.  Examples of program output are
presented as if run from a standard Unix command line shell. \XXX{Is it okay
to be Unix-centric in a book?}  A @>@ character in the first column
indicates input from the user; anything else is output from the program.  It
should be obvious from context which parts of the text are program output
and which are Mercury code.



\section{Hello, World!}

It is slightly unfortunate that the ``Hello, World!'' program introduces
no less than three advanced Mercury concepts, but since tradition
dictates that any tutorial text start with ``Hello, World!'' we'll just
have to jump straight in with the knowledge that things will get easier
thereafter.

We define a \emph{predicate} called @main@ with two parameters which
we've named @IO0@ and @IO@:
\begin{myverbatim}
main(IO0, IO) :-
    io.write_string("Hello, World!\n", IO0, IO).
\end{myverbatim}
Throughout this book we will use the
notation @name/arity@ to refer to symbols, hence we will write @main/2@
when talking about this predicate.  Every Mercury program must include a
@main/2@ predicate that is called when the program is run.

Syntactically, \emph{variables} start with a capital letter, while
everything else (things that start with lower case letters, numbers or
non-alphabetic characters) is a \emph{functor}.

A predicate is defined by one or more \emph{clauses} in which the
\emph{head} is separated from the \emph{body} by a @:-@ symbol.  Here
the body is simply a call to @io.write_string/3@ -- this is a
\emph{module qualified name} and refers to @write_string/3@ as defined
in the @io@ module.  Module qualification of names is only
\emph{necessary} when there may be ambiguity, although it is often a
good idea to do so anyway for software engineering reasons.

Back to the source code: strings in Mercury are delimited by @"@double
quote marks@"@ and appear much the same as they do in C or Java
programs.  Character escapes are introduced with a preceding backslash
@\@ and have the usual meanings: @\n@, for example, denotes the newline
character.

All predicates that perform input/output (IO) operations have to include
two @io.state@ parameters denoting respectively the ``state of the
world'' before and after the IO operation.  In order to ensure that IO
operations occur in a definite order, @io.state@s are \emph{unique}:
they cannot be copied and after being used once an @io.state@ becomes
``dead'' and is no longer available.

In Mercury, every predicate must be accompanied by a @pred@ declaration,
giving the types of its parameters, and @mode@ declarations, indicating
which parameters are inputs and which are outputs.  We'll return to
these topics in later chapters, but for now we'll simply write down the
declarations for @main/2@:
\begin{myverbatim}
:- pred main(io.state, io.state).
:- mode main(di,       uo      ) is det.
\end{myverbatim}
The first declaration says that @main@ is a predicate with two
arguments, both of which have type @io.state@.  The second declaration
says that a call to @main/2@ requires that the first argument be a
``destructive input'' (@di@) which is unique on input and is dead
after the call, and that the second argument is a ``unique
output'' (\ie after the call it will be instantiated with a unique value.)
The @is det@ part of the @mode@ declaration states that @main/2@ is
\emph{deterministic} -- that is, it will always succeed and will have
only one possible \emph{solution}.  (We will see later that some
predicates can fail and some can have multiple solutions.)

Note that declarations always start with a @:-@ and that declarations
and clauses always finish with a single full stop.

Since we refer to names defined in the @io@ module (which is part of the
Mercury standard library), we must also have an @import_module@
declaration in our program:
\begin{myverbatim}
:- import_module io.
\end{myverbatim}

All that is left to do is to wrap up our clauses and declarations in a
\emph{module}, which is the unit of compilation for Mercury programs.  A
Mercury module has two parts: the \emph{interface} section containing the
declarations for the things that are \emph{exported} by the module; and the
\emph{implementation} section which contains predicate definitions and so
forth, the details of which are hidden from other users of the module.

We will call our module @hello@; the Mercury tools will expect us to put
the source code in a file called @hello.m@:
\begin{myverbatim}
:- module hello.
:- interface.
:- import_module io.

:- pred main(io.state, io.state).
:- mode main(di,       uo      ) is det.

:- implementation.

main(IO0, IO) :-
    io.write_string("Hello, World!\n", IO0, IO).
\end{myverbatim}
A Mercury module starts with a @module@ declaration giving its name,
followed immediately by an @interface@ declaration.  What follows
thereafter, up to the @implementation@ declaration, is deemed to be
\emph{exported} by the module.  We put the declarations (but not the
clauses) for @main/2@ here.  Since the declarations for @main/2@ use
names exported by the @io@ module, we have to include the appropriate
@import_module@ declaration and convention dictates we place this at the
top of the interface section.  After the @implementation@ declaration we
write our predicate definitions -- these are \emph{private} and anything
defined here, but not declared in the interface section, will not be
visible to other users of the module.

We build and run the @hello@ program like this:
\begin{myverbatim}
> mmc --make hello
Making hello.int3
Making hello.c
Making hello.o
Making hello
> ./hello
Hello, World!
\end{myverbatim}
Et voila!

\subsection{Goals and Conjunctions}

To illustrate a point, let's define @main/2@ in a slightly more
long-winded way:
\begin{myverbatim}
main(IO0, IO) :-
    io.write_string("Hello, ", IO0, IO1),
    io.write_string("World!",  IO1, IO2),
    io.nl(IO2, IO).
\end{myverbatim}
(The predicate @io.nl/2@ simply writes out a newline character.)

The body of a clause is also what's known as a \emph{goal}.  An \emph{atomic
goal} is either a call, such as @io.nl(IO2, IO)@, or a \emph{unification} of
the form @X = Y@.  A sequence of goals separated by commas (read as ``and'')
forms a compound goal called a \emph{conjunction}.

\subsection{IO and State Variables}

Since Mercury only allows a variable to be bound once, we have to use a
different variable name for each @io.state@.  As demonstrated in the
definition of @main/2@ above, this quickly becomes tedious.  So the language
provides some syntactic sugar -- a notational convenience that has nothing
to do with the language semantics -- in the guise of \emph{state variables}.
A state variable @!X@ actually stands for a \emph{pair} of ordinary
variables, renumbered as required in situations like this.  Hence 
@main/2@ above is more likely to be written as
\begin{myverbatim}
main(!IO) :-
    io.write_string("Hello, ", !IO),
    io.write_string("World!",  !IO),
    io.nl(!IO).
\end{myverbatim}
The Mercury compiler will transform this code into something just like the
version in which we numbered each @io.state@ by hand.  State variables are
explained in detail in chapter \XXX{}.



\section{Rot13}

The simplest form of Caesar cypher is @rot13@, in which each letter in the
plaintext is replaced with the letter thirteen letters ahead in the alphabet
(wrapping around from @z@ to @a@) in the cyphertext.  That is, @a@ becomes
@n@, @b@ becomes @o@, \ldots and @z@ becomes @m@.  There being twenty six
letters in the Roman alphabet, a second application of @rot13@ to the
cyphertext will reveal the original plaintext message.

A session with a @rot13@ program might look like this:
\begin{myverbatim}
> I shot an arrow in the air, she fell to Earth in Berkely Square.
V fubg na neebj va gur nve, fur sryy gb Rnegu va Orexryl Fdhner.
> V fubg na neebj va gur nve, fur sryy gb Rnegu va Orexryl Fdhner.
I shot an arrow in the air, she fell to Earth in Berkely Square.
\end{myverbatim}

It is the custom in some public fora, newsgroups for instance, to use
@rot13@ to encode things such as plot spoilers in messages discussing film
endings so as not to spoil the surprise for those who haven't yet seen the
film.  Those wishing to view the unencrypted text can simply filter the
message through a @rot13@ program.

A @rot13@ program is a simple text filter, reading in a character at a time
from the input stream and writing the converted character to the output
stream.

The predicate we will use to read in each character, @io.read_char/3@,
returns a result which is either @ok(C)@ where @C@ is the character just
read in, @eof@ for the end-of-file condition, or @error(E)@ where @E@ is a
code indicating the nature of the error.  Here is our definition for
@main/2@:
\begin{myverbatim}
main(!IO) :-
    io.read_char(Result, !IO),
    (
        Result = ok(C),
        io.write_byte(rot13(C), !IO),
        main(!IO)
    ;
        Result = eof
    ;
        Result = error(_),
        exception.throw(Result)
    ).
\end{myverbatim}
The call to @io.read_char/3@ instantiates @Result@ with one of the
possible outcomes of the input operation.  To decide what to do next we
use a \emph{disjunction}: a sequence of goals separated by
semicolons (read as ``or'') inside parentheses.  Each
\emph{arm} of the disjunction starts by attempting to unify @Result@
with one of the possible outcomes.  (This type of disjunction is called a
\emph{switch} because it's rather like the @switch@ statement found in C and
Java.)

If @Result@ matches @ok(C)@ -- unifying @C@ with the character just read
in -- then we encrypt the character using a function @rot13/1@, which we'll
come to next, write it out and make a recursive call to @main/2@ to process
the rest of the input.  If @Result@ matches @eof@ then we're finished.
Finally, if @Result@ matches @error(_)@ (the underscore @_@ stands for any
``don't care'' value) then we simply \emph{throw} it as an exception, which
will cause execution to halt with an error message.

(Note that a functor with no arguments, such as @eof@, \emph{is not}
followed by empty parentheses, as one would expect in a language like C or
Java.)

We implement @rot13/1@ as a function.  A function is just syntactic sugar
for a deterministic predicate with a single output that, unlike ordinary
predicates, can be used in expressions.  Functions are almost always the
right stylistic choice in such cases:
\begin{myverbatim}
:- func rot13(int) = int.
:- mode rot13(in ) = out is det.

rot13(X) = Z :-
    ( if   encrypt_letter(X, Y)
      then Z = Y
      else Z = X
    ).
\end{myverbatim}
(The @( if _ then  _ else )@ construct is called a \emph{conditional goal};
the @else@ arm is \emph{not} optional.)

In this case, if @encrypt_letter(X, Y)@ succeeds (unifying @Y@ with the
encrypted form of @X@) then the result, @Z@, is @Y@, otherwise @X@ can't
be a letter and we just return it unchanged in @Z@.

We're on the home strait.  All we need to do now is implement the 
translation of letters:
\begin{myverbatim}
:- pred encrypt_letter(int, int).
:- mode encrypt_letter(in,  out) is semidet.

encrypt_letter(X, Y) :-
    (   X = 'A', Y = 'N'
    ;   X = 'B', Y = 'O'
    ...
    ;   X = 'Z', Y = 'M'
    ;   X = 'a', Y = 'n'
    ;   X = 'b', Y = 'o'
    ...
    ;   X = 'z', Y = 'm'
    ).
\end{myverbatim}
Once again, we use a switch to decide what to do: if @X@ matches @'A'@
then we return @'N'@ in @Y@, and so forth.
Because this switch does not cover all possible values that @X@ might
have (the digit characters, for instance)
the switch can \emph{fail}.  Don't worry too much about what
this means for now, other than that failure of a call can be detected
using a conditional goal, as in the code for @rot13/1@ above.
Since @encrypt_letter/2@ has at most one solution for any given @X@, but can
fail, it is said to be \emph{semi-deterministic}, hence the @semidet@
determinism annotation on the mode declaration.

To turn the above code into something we can compile, we need only wrap
it up with the appropriate module declarations as before, remembering to
include
\begin{myverbatim}
:- import_module exception.
\end{myverbatim}
at the start of the implementation section since we use
@exception.throw/1@ in @main/2@.

\subsection{Functions and Conditional Expressions}

Function mode declarations like the following
common,
\begin{myverbatim}
:- mode f(in, in, ..., in) = out is det.
\end{myverbatim}
are so common that Mercury allows us to omit them altogether for functions
that have just one mode and it is of this form.

Second, just as Mercury has conditional goals, it also has \emph{conditional
expressions}.  Thus we can shorten the definition of @rot13/1@ to
\begin{myverbatim}
rot13(X) = ( if encrypt_letter(X, Y) then Y else X ) :-
    true.
\end{myverbatim}
where @true/0@ is a built-in predicate that always succeeds, but does
nothing else.  The @then@ and @else@ arms of a conditional expression
contain expressions, rather than goals.

Third, any clause in which the body is just @true@ can be written as
just the head part on its own, hence we can simplify @rot13/1@ still
further to:
\begin{myverbatim}
rot13(X) = ( if encrypt_letter(X, Y) then Y else X ).
\end{myverbatim}
This rule works for predicates as well as functions.

\subsection{Predicates and Disjunctions}

If the body of a clause is a disjunction, it can often be described more
clearly using multiple clauses.  That is, we could rewrite
@encrypt_letter/2@ as
\begin{myverbatim}
encrypt_letter(X, Y) :- X = 'A', Y = 'N'.
encrypt_letter(X, Y) :- X = 'B', Y = 'O'.
...
encrypt_letter(X, Y) :- X = 'y', Y = 'l'.
encrypt_letter(X, Y) :- X = 'z', Y = 'm'.
\end{myverbatim}
which will be transparently converted by the Mercury compiler into our
original definition using a disjunction.  So far we've actually made
life harder on our fingers\ldots but!  Mercury has some more
syntactic sugar that, together with clausal notation, really does pay
off.

If the Mercury compiler sees an expression rather than a variable in an
head argument, as in
\begin{myverbatim}
p(expr1, expr2, ...) :- ...
\end{myverbatim}
it will transparently convert this into
\begin{myverbatim}
p(X1, X2, ...) :- X1 = expr1, X2 = expr2, ...
\end{myverbatim}
where @X1@ and @X2@ and so on are chosen so as not to coincide with
anything in the original definition of the clause.

Consequently we can write @encrypt_letter/2@ in a quite perspicuous
fashion:
\begin{myverbatim}
encrypt_letter('A', 'N').
encrypt_letter('B', 'O').
...
encrypt_letter('y', 'l').
encrypt_letter('z', 'm').
\end{myverbatim}

\subsection{Using Arithmetic Instead}

If we know that our program is only going to be run on ASCII data (or
some other character set encoding in which the byte codes for the
letters of the alphabet are contiguous), we could rewrite @rot13/1@ to
do away with the need for @encrypt_letter/2@ altogether:
\begin{myverbatim}
rot13(X) = Y :-
    char.to_int(X, XCode),
    YCode = ( if      0'A =< X, X =< 0'Z
              then    ((13 + X - 0'A) `mod` 26) + 0'A
              else if 0'a =< X, X =< 0'z
              then    ((13 + X - 0'a) `mod` 26) + 0'a
              else                            X
            )
    char.to_int(Y, YCode).
\end{myverbatim}
Several new things are illustrated in this definition.  Let's start off by
looking at syntax.  The first thing to notice is that the integer code for
the character @'A'@ is written as @0'A@.  The second is that while the names
@=<@ (less than or equal to), @+@ and @-@ can all appear as infix operators
as one would expect, it is also possible to use any other name as an infix
operator simply by enclosing it in @`@backquotes@`@.  This is what we've
done with @mod/2@ above.  The Mercury Reference Manual \XXX{} contains a
list of which names can be used as prefix, infix or postfix operators.  Note
that it makes no difference to Mercury if we write @+(13, X)@ rather than
@13 + X@ or @mod(Y, 26)@ rather than @Y `mod` 26@.  The latter are simply
easier to read.

The integer operators are all defined in the @int@ module in the Mercury
standard library, which \emph{must} be imported before they can be used --
unlike most languages, basic arithmetic is not built-in to Mercury.

The other new thing here is the use of the same predicate in different
``directions''.  The predicate @char.to_int/2@ converts between characters
and integer character codes.  The first such call,
@char.to_int(X, XCode)@ uses the mode
\begin{myverbatim}
:- mode char.to_int(in, out) is det.
\end{myverbatim}
to convert @X@ into its character code, @XCode@.  The second call,
@char.to_int(Y, YCode)@ uses the mode
\begin{myverbatim}
:- mode char.to_int(out, in) is semidet.
\end{myverbatim}
to convert @YCode@ into the corresponding character, @Y@ (this ``direction''
is semi-deterministic because not all integers correspond to characters.)

\subsection{If-Then-Else Chains vs. Switches}

Where possible, it is not only clearer, on the whole, to use a switch in
preference to an @if-then-else@ chain, but the compiler can often do a
better job of optimizing switches (implementing them with jump tables or
hash
tables for instance.)



\section{A Spelling Checker}

In this section we write a simple spelling checker that will take two
files as arguments, a dictionary file and a file to be checked for spelling
mistakes, and print out any words in the latter that do not appear in the
former:
\begin{myverbatim}
> ./spell_check my_dictionary_file my_text_file
doog
kat
mowse
\end{myverbatim}
(On Unix systems one can usually find a dictionary file in
@/usr/dict/words@.)
Our first approach will simply compute the set of words in each file
and use the set difference function to decide which words in the text file
have been misspelled.

First, let's write @main/2@: it has to retrieve the file names of the
dictionary and source file from the command line arguments, read
and compute the set of words in each file, and print out any words in
the text file set that do not appear in the dictionary file set:
\begin{myverbatim}
main(!IO) :-

    io.command_line_arguments(Args, !IO),

    ( if Args = [DictFileName, TextFileName] then

            % Read in the list of words in each file.
            %
        file_to_word_set(DictFileName, DictWordSet, !IO),
        file_to_word_set(TextFileName, TextWordSet, !IO),

            % Find the list of misspelt words.
            %
        ErrorWordSet = TextWordSet `set.difference` DictWordSet,
        Errors       = set.to_sorted_list(ErrorWordSet),

            % Write out the misspelt words.
            %
        write_words_in_list(Errors, !IO)

      else

        exception.throw("usage: spell_check <dictfile> <textfile>")
    ).
\end{myverbatim}
This definition introduces several new things.  Let's start with the new
piece of syntax: comments begin with a per cent @%@ symbol and extend to the
end of the line.

@main/2@ begins by calling @io.command_line_arguments/3@ to obtain the
list of command line arguments in @Args@.  If @Args@ is a two member
list (instantiating @DictFileName@ and @TextFileName@ respectively) then we
enter the spelling checker part of the algorithm, otherwise we report a
usage error by throwing an exception.

Having obtained the dictionary and text file names, @file_to_word_set/4@
is called to get the set of words in each file.  The predicate
@file_to_word_set/4@ is described below.

We use the @set.difference/2@ function to obtain the set of words in the
text file, but not in the dictionary, and then call
@set.to_sorted_list/1@ to turn the error set into a list.

Finally, we call @write_words_in_list/3@ (also described below) to print out
each member of @ErrorWords@.

Returning to @file_to_word_set/4@, this predicate has to open the given
file, read its contents, and work out the set of words therein.  Here's
one possible implementation:
\begin{myverbatim}
:- pred file_to_word_set(string, set(string), io, io).
:- mode file_to_word_set(in,     out,         di, uo) is det.

file_to_word_set(FileName, WordSet, !IO) :-

    io.open_input(FileName, OpenResult, !IO),
    (
        OpenResult = ok(InputStream),
        io.read_file_as_string(InputStream, ReadResult, !IO),
        (
            ReadResult = ok(String),
            Words      = string.words(String),
            WordSet    = set.list_to_set(Words)
        ;
            ReadResult = error(_, _),
            exception.throw(ReadResult)
        )
    ;
        OpenResult = error(_),
        exception.throw(OpenResult)
    ).
\end{myverbatim}
The first thing to observe is that we've used @io@ rather than
@io.state@ in the @pred@ declaration; this is a common convention in
modern Mercury programs: the @io@ library defines @io.io@ as a
synonym for @io.state@ and @io@ is the unqualified form of @io.io@.  It
simply looks better on the page.

The next new thing is the type @set(string)@.  The @set/1@ type, defined
in the @set@ module in the Mercury standard library, is
\emph{parameterised}.  That is, we can have @set(int)@, @set(string)@,
@set(set(int))@ and so forth and Mercury will ensure that we never get
different types of set mixed up in our programs.  The chapter on types
\XXX{} explains parametric types in detail.

The call to @io.open_input@ tries to open an input stream for the file
named in @FileName@.  \XXX{Do I need to explain ``input stream''?}
The only possible outcomes are @ok/1@ or @error/1@, which
we test for in a switch.

If all is well, @io.read_file_as_string/4@ attempts to read all the data
from the file referred to by @InputStream@.  The result will be of the form
@ok/1@ for success or @error/2@ if there was a problem (this particular
error result includes any data that was read up to the point the error
occurred along with the error code.)  Again, we use a switch to decide what
to do.

The function @string.words/1@ is used to break the resulting string along
contiguous sequences of whitespace into a list of strings.

Finally, we turn the list of words into a @set(string)@ using
@set.list_to_set/1@.

There's only one more thing to do: define @write_words_in_list/3@.
This is easy to do: if we have the empty list then we don't write
anything at all; otherwise we write out the first string in the list
(the \emph{head}) followed by a newline, and then repeat for the rest of
the list (the \emph{tail}):
\begin{myverbatim}
:- pred write_words_in_list(list(string), io, io).
:- mode write_words_in_list(in,           di, uo) is det.

write_words_in_list([],             !IO).

write_words_in_list([Word | Words], !IO) :-
    io.write_string(Word, !IO),
    io.nl(!IO),
    write_words_in_list(Words, !IO).
\end{myverbatim}
The functor @[]/0@ matches for the empty list, while @[Word | Words]@
unifies @Word@ and @Words@ with the head and tail respectively of a
non-empty list.  Since the first argument must be one or the other, our two
clauses constitute a complete switch on the first argument.

The recursive call to @write_words_in_list/3@ handles printing the words
in the tail of the non-empty list.  Recursion is the only mechanism used for
iteration in Mercury (unlike other languages with special constructs such as
@for@ and @while@ loops.)

Now we can wrap up the code into a module as usual (this time importing
@string@, @list@, @set@ and @exception@ in the implementation section)
and\ldots \emph{hey presto!}

\subsection{Handling Punctuation}

Our spelling checker is rather literal minded.  One issue is that any
immediately adjacent punctuation is assumed to be part of a word as
well.  For example,
\begin{myverbatim}
    string.words("Panic stations!") = ["Panic", "stations!"]
\end{myverbatim}
The dictionary file is unlikely to include @stations!@ as an entry.

We can change our program to treat anything that \emph{isn't} a letter
as a word separator by replacing the following line in
@file_to_word_set/4@
\begin{myverbatim}
    Words = string.words(String),
\end{myverbatim}
with
\begin{myverbatim}
    Words = string.words(isnt_a_letter, String),
\end{myverbatim}
and adding the following predicate to the program:
\begin{myverbatim}
:- pred isnt_a_letter(char).
:- mode isnt_a_letter(in  ) is semidet.

isnt_a_letter(Char) :-
    not char.is_alpha(Char).
\end{myverbatim}
This change introduces three new points: overloading, higher order
programming and negation.

\emph{Overloading} refers to using the same name for more than one thing.
Here we use @string.words/2@ rather than @string.words/1@.  The former takes
as its first argument a predicate that decides whether a particular
character is something that separates words.  We've
used a predicate of our own, @isnt_a_letter/1@ that does the job we
want (we'll come to @isnt_a_letter/1@ in a moment.)  Passing predicates as
arguments is part of \emph{higher order programming}.

The definition of @isnt_a_letter/1@ uses a new logical connective: @not@,
the mercury symbol for \emph{logical negation}.  The goal
@not char.is_alpha(Char)@ \emph{fails} if the goal @char.is_alpha(Char)@
succeeds (which it will if @Char@ is a letter) and \emph{succeeds} if
@char.is_alpha(Char)@ fails.

(Note that since we use a predicate from the @char@ module, we have to add
this to our list of imports at the top of the implementation section.)

\subsection{Handling Capitalisation}

Another problem with our spelling checker is that two strings that are
identical, except for capitalisation, do not compare the same -- so @"Dog"@
and @"dog"@ are considered different by our program.  An easy remedy for
this is to first convert every word to lower case.  To achieve this, all we
need do is change the following lines in @file_to_word_set/4@
\begin{myverbatim}
    Words   = string.words(String),
    WordSet = set.list_to_set(Words)
\end{myverbatim}
to
\begin{myverbatim}
    Words   = string.words(String),
    LCWords = words_to_lower_case(Words),
    WordSet = set.list_to_set(LCWords)
\end{myverbatim}
and add the function
\begin{myverbatim}
:- func words_to_lower_case(list(string)) = list(string).

words_to_lower_case([]) = [].

words_to_lower_case([Word | Words]) =
    [string.to_lower(Word) | words_to_lower_case(Words)].
\end{myverbatim}
where the function @string.to_lower/1@ returns its argument string with any
upper case letters converted to lower case.

The the list constructors, @[]@ and @[|]@, \emph{deconstruct} the function
argument in the clause heads, whereas in the body they happen to
\emph{construct} the result (it all depends upon whether the term being
unified with is an input or an output, respectively.)

The pattern used in @words_to_lower_case/1@ is so common that the @list@
module abstracts it with the function @list.map/2@.  A black belt Mercury
programmer probably wouldn't bother writing @words_to_lower_case/1@ at all,
but would say @list.map(string.to_lower, Words)@ instead -- it amounts to
exactly the same thing.  Don't worry if this seems a little confusing right
now.  The mysteries of higher order programming, such as they are, will be
made clear in chapter \XXX{}.

\subsection{Writing Lists the Easy Way}

Here's another example of how higher order programming can save time and
effort: writing out each item in a list is a common task.  The @io@ library
provides the predicate @io.write_list/5@ to save you the trouble of having
to write your own routine each time.

Instead of using @write_words_in_list/3@ in @main/2@, we could have written
\begin{myverbatim}
    io.write_list(Errors, "\n", io.write_string, !IO),
    io.nl(!IO)
\end{myverbatim}
This call to @io.write_list/5@ applies @io.write_string@ to each member of
@Errors@ (along with the @!IO@ pair), and writes out @"\n"@ \emph{between}
each call (so we need to add a call to @io.nl/2@ afterwards.)



\section{A Simple Data Base}

Our final example demonstrates how some predicates can have
multiple solutions and can even be used in more than one ``direction''.

To represent the information in the data base clearly, we need to introduce
some new types.  We define the type @gender/0@ which has \emph{data
constructors} (\ie possible values) @male/0@ and @female/0@.  We also
define @date/0@ with a single data constructor, @date/3@ (another
example of overloading), with three named fields for year, month and day.
And we add the type @name/0@ as a synonym for @string@:
\begin{myverbatim}
:- type gender ---> male ; female.
:- type date   ---> date(year :: int, month :: int, day ::int).
:- type name   ==   string.
\end{myverbatim}
Next we have a @person/3@ relation (another word for predicate) which
records against each person's name their gender and date of birth:
\begin{myverbatim}
:- pred person(name, gender, date).
:- mode person(in,   out,    out ) is semidet.
:- mode person(out,  out,    out ) is multi.

    % Great-grandparents.
    %
person("Edith Sorrel",      female, date(1897,  1,  3)).

    % Grandparents.
    %
person("George Becket",     male,   date(1925,  7, 18)).
person("Faith Becket",      female, date(1923,  2, 23)).
person("William Allen",     male,   date(1917, 11, 14)).
person("Gertrude Allen",    female, date(1921,  8, 29)).

    % Parents.
    %
person("William Becket",    male,   date(1945,  6, 12)).
person("Patricia Becket",   female, date(1943,  3,  3)).
person("Tom Allen",         male,   date(1944,  4, 30)).
person("Ina Allen",         female, date(1950, 12, 12)).

    % Most recent generation.
    %
person("Ralph Becket",      male,   date(1970, 12, 10)).
person("Kieran Becket",     male,   date(1972,  7, 17)).
person("Simon Allen",       male,   date(1970, 11, 14)).
person("Suzanne Allen",     female, date(1969,  4,  1)).
\end{myverbatim}
``What's going on here?'' I hear you cry, ``@person/3@ has \emph{two} mode
declarations!''  In Mercury it is perfectly reasonable for a predicate to
have more than one mode declaration, indicating that the predicate can be
called in more than one way.  The first mode declaration,
\begin{myverbatim}
:- mode person(in, out, out) is semidet.
\end{myverbatim}
says that a goal @person(Name, Gender, Date)@ for which @Name@ is an input
(\ie has already been bound) is semi-deterministic: there is at most one
solution (\ie the goal \emph{may} fail) instantiating @Gender@ and @Date@.
On the other hand,
\begin{myverbatim}
:- mode person(out, out, out) is multi.
\end{myverbatim}
says that a goal @person(Name, Gender, Date)@ for which none of the
arguments are inputs has \emph{one or more} solutions instantiating @Name@,
@Gender@ and @Date@.  This is what the @multi@ determinism category stands
for.  We'll come to what happens with predicates that have have more than
one solution shortly.

Moving on to the other key relation in our data base, @is_child_of/2@
records which person is a child of which other person:
\begin{myverbatim}
:- pred name `is_child_of` name.
:- mode in   `is_child_of` in  is semidet.
:- mode in   `is_child_of` out is nondet.
:- mode out  `is_child_of` in  is nondet.

"Faith Becket"      `is_child_of`   "Edith Sorrel".
"William Becket"    `is_child_of`   "George Becket".
"William Becket"    `is_child_of`   "Faith Becket".
"Patricia Becket"   `is_child_of`   "William Allen".
"Patricia Becket"   `is_child_of`   "Gertrude Allen".
"Tom Allen"         `is_child_of`   "William Allen".
"Tom Allen"         `is_child_of`   "Gertrude Allen".
"Ralph Becket"      `is_child_of`   "William Becket".
"Ralph Becket"      `is_child_of`   "Patricia Becket".
"Kieran Becket"     `is_child_of`   "William Becket".
"Kieran Becket"     `is_child_of`   "Patricia Becket".
"Simon Allen"       `is_child_of`   "Tom Allen".
"Simon Allen"       `is_child_of`   "Ina Allen".
"Suzanne Allen"     `is_child_of`   "Tom Allen".
"Suzanne Allen"     `is_child_of`   "Ina Allen".
\end{myverbatim}
\XXX{Change names etc.  It's part of my family tree with wild guesses about
most of the dates.}
The first mode declaration states that given inputs @Child@ and @Parent@,
the goal @Child `is_child_of` Parent@ will either succeed or fail -- that
is, it can be used to test whether one person is the offspring of another.

The second mode declaration states that a goal
@Child `is_child_of` Parent@, for which @Child@ is an input, is
non-deterministic (hence the @nondet@ determinism category).  In other
words, the goal may fail or have one or more solutions for @Parent@.
The goal,
\begin{myverbatim}
    "Edith Sorrel" `is_child_of` Parent
\end{myverbatim}
for instance, has no solutions.  The goal
\begin{myverbatim}
    "Faith Becket" `is_child_of` Parent
\end{myverbatim}
has one solution, namely
\begin{myverbatim}
    Parent = "Edith Sorrel"
\end{myverbatim}
while
\begin{myverbatim}
    "William Becket" `is_child_of` Parent
\end{myverbatim}
has both
\begin{myverbatim}
    Parent = "George Becket"
\end{myverbatim}
and
\begin{myverbatim}
    Parent = "Faith Becket"
\end{myverbatim}
as solutions.

The third mode declaration states that a goal @Child `is_child_of` Parent@,
in the case where @Parent@ is an input, is similarly non-deterministic.

Now let's introduce some new relations (if you'll pardon the pun).  The
simplest one, @is_parent_of/2@, is just the converse of @is_child_of/2@:
\begin{myverbatim}
:- pred name `is_parent_of` name.
:- mode in   `is_parent_of` in  is semidet.
:- mode in   `is_parent_of` out is nondet.
:- mode out  `is_parent_of` in  is nondet.

Parent `is_parent_of` Child :- Child `is_child_of` Parent.
\end{myverbatim}
That was easy.  What about mothers and fathers?
\begin{myverbatim}
:- pred name `is_mother_of` name.
:- mode in   `is_mother_of` in  is semidet.
:- mode in   `is_mother_of` out is nondet.
:- mode out  `is_mother_of` in  is nondet.

Mother `is_mother_of` Child :-
    Mother `is_parent_of` Child,
    person(Mother, Gender, _),
    Gender = female.
\end{myverbatim}
(Recall that an underscore @_@ on its own means ``don't care''.) So @Mother@
is the mother of @Child@ if @Mother@ is a parent of @Child@ and the
@person/3@ record for @Mother@ lists @Mother@ as female.

The careful reader will be wondering about the third mode declaration for
@is_mother_of/2@ -- surely each @Child@ can have no more than one mother,
making the correct mode @semidet@ rather than @nondet@.  While this is true
for the genealogical domain, Mercury has no idea that this is actually a
genealogical data base and that this is therefore a necessary domain
constraint.  Rather, it reasons that if @Child@ is an input, then the call
to @is_parent_of/2@ can have more than one possible solution for @Mother@,
hence @is_mother_of/2@ may have more than one possible solution.  (There
\emph{is} a way to express this domain constraint, but doing so requires
black belt level techniques which we aren't going to get into this early in
the book.)

The code for @is_father_of/2@ is very similar:
\begin{myverbatim}
:- pred name `is_father_of` name.
:- mode in   `is_father_of` in  is semidet.
:- mode in   `is_father_of` out is nondet.
:- mode out  `is_father_of` in  is nondet.

Father `is_father_of` Child :-
    Father `is_parent_of` Child,
    person(Father, Gender, _),
    Gender = male.
\end{myverbatim}
Now, Mercury will replace a predicate call with a non-variable term as an
argument with one in which the argument \emph{is} a variable and add a
unification goal connecting the new variable and the term.  That is, a call
\begin{myverbatim}
    p(<expr1>, <expr2>, ...)
\end{myverbatim}
is transparently converted to
\begin{myverbatim}
    X1 = <expr1>,
    X2 = <expr2>,
    ...
    p(X1, X2, ...)
\end{myverbatim}
(and a similar thing happens for goals containing function applications.)
This means we can rewrite the clause for @is_father_of/2@ (and similarly for
@is_mother_of/2@) as
\begin{myverbatim}
Father `is_father_of` Child :-
    Father `is_parent_of` Child,
    person(Father, male, _).
\end{myverbatim}
which is easier to both read and maintain.

Let's return briefly to the question of what it means for a predicate
to have more than one possible solution.  Consider the goal
\begin{myverbatim}
    Mother `is_mother_of` "Ralph Becket"
\end{myverbatim}
The definition of @is_mother_of/2@ says we must therefore find a solution to
\begin{myverbatim}
    Mother `is_parent_of` "Ralph Becket",
    person(Mother, female, _)
\end{myverbatim}
and hence, from the definition of @is_parent_of/2@, a solution to
\begin{myverbatim}
    "Ralph Becket" `is_child_of` Mother,
    person(Mother, female, _)
\end{myverbatim}
Mercury works its way through the clauses of @is_child_of/2@ looking for a
solution to the first subgoal.  The first one it comes
to is
\begin{myverbatim}
    Mother = "William Becket"
\end{myverbatim}
Execution proceeds to the second subgoal,
\begin{myverbatim}
    person("William Becket", female, _)
\end{myverbatim}
Since there are no matching clauses for this goal, execution
\emph{backtracks} (or unwinds) to the
last non-deterministic goal (the call to @is_child_of/2@) to look for a
different solution.  The next solution to the @is_child_of/2@ subgoal is
\begin{myverbatim}
    Mother = "Patricia Becket"
\end{myverbatim}
As before, execution proceeds to the second subgoal
\begin{myverbatim}
    person("Patricia Becket", female, _)
\end{myverbatim}
which this time \emph{does} have a matching clause.  So now we know that
\begin{myverbatim}
    Mother = "Patricia Becket"
\end{myverbatim}
is a solution to the goal
\begin{myverbatim}
    Mother `is_mother_of` "Ralph Becket"
\end{myverbatim}
Hurrah!

To sum up, Mercury's execution model for non-deterministic
predicates is to search systematically through all possibilities while
looking for solutions, backtracking to the most recent non-exhausted
non-deterministic \emph{choice point} whenever a subgoal fails.  One final
word on the subject: Mercury is allowed to reorder the goals in a
conjunction as necessary to ensure mode correctness (\ie before a variable
can be used as an input it must be instantiated either from a unification or
as an output argument to a call.)

Now that we have a feeling for how non-determinism works, let's finish up by
writing down some more interesting relations.  To save space we won't bother
including the mode declarations (working out the possible modes for each is
left as a light exercise for the reader).

First, a grandparent is defined as the parent of a parent:
\begin{myverbatim}
Grandparent `is_grandparent_of` Child :-
    Parent `is_parent_of` Child,
    Grandparent `is_parent_of` Parent.
\end{myverbatim}
More generally, an ancestor of a person is \emph{either} a parent 
\emph{or} an ancestor of a parent of the person:
\begin{myverbatim}
Ancestor `is_ancestor_of` Person :-
    Ancestor `is_parent_of` Person.

Ancestor `is_ancestor_of` Person :-
    Parent `is_parent_of` Person,
    Ancestor `is_ancestor_of` Parent.
\end{myverbatim}
Two people are siblings if they have a parent in common and are different
(one isn't usually considered to be one's own sibling):
\begin{myverbatim}
SiblingA `is_sibling_of` SiblingB :-
    Parent `is_parent_of` SiblingA,
    Parent `is_parent_of` SiblingB,
    SiblingA \= SiblingB.
\end{myverbatim}
The goal @SiblingA \= SiblingB@ is syntactic sugar for
@not SiblingA = SiblingB@.

Two people are cousins if their parents are siblings:
\begin{myverbatim}
CousinA `is_cousin_of` CousinB :-
    ParentA `is_parent_of` CousinA,
    ParentB `is_parent_of` CousinB,
    ParentA `is_sibling_of` ParentB.
\end{myverbatim}
Here's how we define niece-or-nephewdom:
\begin{myverbatim}
NieceOrNephew `is_niece_or_nephew_of` AuntOrUncle :-
    Parent `is_sibling_of` AuntOrUncle,
    NieceOrNephew `is_child_of` Parent.
\end{myverbatim}
It's all remarkably straightforward!

\XXX{Are there too many examples here?  Do they need more explanation?
What could I do with the date information?  What about getting the solutions
to these predicates?}

\XXX{I've ditched the maze searching example as too complex to do right this
early in the book.}

\XXX{Overuse of ``simple'', ``much''...}
