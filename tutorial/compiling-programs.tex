% vim: ft=tex ff=unix ts=4 sw=4 et wm=8 tw=0

\chapter{Compiling Programs}

The Melbourne Mercury compiler comes with a suite of powerful tools to
simplify the compilation of Mercury programs for various purposes,
including building executables, shared and static libraries, linking
with foreign language object files and so forth.

This chapter aims to give a very brief introduction to the most
commonly used facilities of the compiler and its associated tools.

\section{The Mercury Directory}

But first, a useful tip.  If you create a subdirectory called @Mercury@
in the directory in which you will be running the compilation, then the
Mercury tools will assume (unless explicitly told otherwise) that all
intermediate files used in the compilation process (of which there are
many) should be placed in the Mercury directory.  This helps to avoid
cluttering up the working directory with files that should not be edited
by hand or, for the most part, be of any interest to the programmer.

The Mercury build tools are fairly smart and will go to some lengths to
avoid unnecessary recompilation.  However, there are rare occasions in
which they cannot work out that a particular intermediate file needs
updating, in which case one way of solving the problem is to simply
delete all the contents of the @Mercury@ directory, for example with 
@rm -rf Mercury/*@, and restart the build process from scratch.

\section{Building Executables}

\XXX{Make a comment somewhere at the start of the book that shell script
stuff lines will start with a @\$@ in the first column.}

The simplest way to build an executable from a Mercury program is just
this:
\begin{verbatim}
$ mmc --make foo
\end{verbatim}
where @foo@ is the name of the top-level Mercury module defining the
@main/2@ predicate.

The @--make@ option to @mmc@ will cause it to examine all the Mercury
dependencies of module @foo@ (i.e. the modules it imports, the modules
they import and so forth) and carry out any necessary compilation steps
for either @foo@ or any if its dependencies.  @mmc@ will try to do only
the minimum amount of work necessary to create an up-to-date executable
(i.e. it will avoid unnecessary recompilation of @foo@ or its
dependencies.)  There is no need to separately compile @foo@'s
dependencies -- @mmc --make@ will work it all out for you.

It can also be useful to specify @--smart-recompilation@.  With this
option turned on, @mmc@ will even avoid unnecessary
recompilation in the case where the interface of a dependency has
been changed, but not in a way that would make any difference (this
can be a \emph{real} timesaver for large projects.)

\section{Mmake}

Some projects may require various preprocessing steps or special options
to be set as part of the build process.  As a rule of thumb, any time
one needs more than @mmc --make@ then one should probably use the @mmake@
system.  If you need to use various other compilation options then @mmake@
is almost certainly the easiest tool to use.

\subsection{Mmakefiles}

@mmake@ is based upon the well known @make@ tool, common on Unix systems,
amongst others.  In order to use @mmake@, one first needs to write a
file called @Mmakefile@ with the necessary build instructions.

Here are the contents of a sample @Mmakefile@:
\begin{verbatim}
MAIN_TARGET = foo
depend: $(MAIN_TARGET).depend

GRADEFLAGS = --debug
\end{verbatim}
By and large, the order of the lines in an @Mmakefile@ doesn't make any
difference.  Each line in an @Mmakefile@ is either an assignment to an
variable or a dependency rule.

A variable assignment looks like @GRADEFLAGS = --debug@ where the variable
name appears to the left of the @=@ sign and the string to be assigned to
it appears on the right.  Any current environment variables are also
visible to an @Mmakefile@, so if you are going to run @mmake@ knowing that
an environment variable, @BAZ@, say, is going to be set, you can refer to
@BAZ@ in your @Mmakefile@ just as if it had been defined in the @Mmakefile@
itself.

Whenever @$(BAZ)@ appears in an @Mmakefile@, the string assigned to @BAZ@
is substituted instead.

If you need a @$@ sign to appear anywhere in an @Mmakefile@,
you need to write @$$@ instead: @mmake@ assumes that anything else
is introducing a variable substitution.

A simple dependency rule looks like this:
\begin{verbatim}
depend: $(MAIN_TARGET).depend
\end{verbatim}
The name of the \emph{target} appears to the left of the colon and the
names of the targets it depends on (there may be several, separated
with spaces) appear to the right.  Amongst other things, @mmake@
analyses the dependency rules and ensures the various targets are
up-to-date in bottom-up order (a target is deemed up-to-date if the
corresponding file is newer than those for its dependencies.)

Given the simple @Mmakefile@ above, if we invoke @mmake depend@ then
@mmake@ will start by building the target @foo.depend@ (since @foo@
is assigned to @MAIN_TARGET@.)  In this case @depend@ is a so-called
\emph{dummy target} since a file called @depend@ is never actually
created.  The target @foo.depend@ causes the Mercury build system to
examine the module @foo@ and compute its dependencies.  It is important
to build @foo.depend@ before attempting to compile @foo@.  Moreover,
any time the dependencies for @foo@ change (e.g. by changing the
interface of @foo@ or any of its dependencies) then @foo.depend@ must
be rebuilt for the compilation process to work properly.

Next, we can invoke just @mmake@ on its own, in which case it will
attempt to build the module named in @MAIN_TARGET@ (if it has a value)
as an executable.  Otherwise one can name an explicit target.  For
instance, one could simply do
\begin{verbatim}
$ mmake foo.depend
$ mmake foo
\end{verbatim}
without having to use an @Mmakefile@ at all!  (Of course, in this
case it's easier to just invoke @mmc --make foo@.)

Dependency rules can also include an explicit build procedure for a
target that overrides any default process that @mmake@ might have.
For example, say we needed to preprocess a given file, @bar.z@ say,
in order to obtain the requisite Mercury module, we would include
the following in the @Mmakefile@:
\begin{verbatim}
bar.m: bar.z
        cat bar.z | my_preprocessor > bar.m
\end{verbatim}
Now, when @mmake@ needs @bar.m@ and sees that either @bar.z@ is
newer than @bar.m@
or that @bar.m@ does not exist at all, it will run the commands
that appear on the lines below the dependency rule.  Commands
should be indented by a single hard tab character (ASCII code 9,
not spaces) and there should be no blank lines between commands
if there are more than one.

@mmake@ already knows how to compile ordinary Mercury code.  The main
reason for supplying an explicit build procedure is to handle foreign
code that will be linked with the output from the Mercury compiler to
construct the executable.

The main use for the @mmake@ system is to handle complex settings
for calls to the Mercury tools.  One invariably runs the
build process more than once while developing a program, so it
helps to only have to write down the various flags and settings
once in an @Mmakefile@.

\subsection{Module-Specific Options}

\subsection{Cleaning Up}

Targets clean and realclean.

\section{Compilation Grades}

Standard.

Debugging.

Profiling.

--use-grade-subdirs

\section{Linking Mercury Programs with Foreign Code}

\section{Building and Installing Libraries}

Tools know from installation where standard libs etc. live.

When installing and using libraries from elsewhere we have to pass
this information to the tools.


\section{Useful Compiler Flags and Environment Variables}
