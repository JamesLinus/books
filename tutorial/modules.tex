\section{Modules}

The Mercury unit of compilation is the module.  Modules
are compiled separately and the resulting object files linked
together to make a program or aggregated to form libraries.

A module has two parts, the interface and the implementation sections.

The interface section describes the types, insts, preds and funcs and so
forth that are said to be exported or externally visible.

The implementation section supplies all the code that defines the
behaviour of the exported preds and funcs and defines any abstract
types or type class instances exported by the module (we will come
across type classes later on \XXX{}.)  Declarations and definitions in
the implementation section are not externally visible.

\subsection{Names and Namespaces}

It would be unreasonable to insist that no two preds (or types
or \ldots) share the same name in any Mercury program.  To this
end, the Mercury module system makes use of two mechanisms:
restrictions on the visiblity of names and a module qualified
naming scheme.

In this section we will use the term \emph{name} to refer to the
name of any type, func, pred, inst, mode, type class and so
forth that one might find in a Mercury program.

\subsection{Visibility}

In order to use a name, it must be visible at each point in
a module where it appears.

\XXX{This stuff needs illustrating with examples.}

\subsubsection{In the Interface Section}

A name is visible in an interface section if it is
\begin{itemize}
\item declared or defined there;
\item is exported by a module which is imported in this
  module's interface section;
\item is declared or defined in a parent module (we will come
  to submodules shortly \XXX{}.)
\end{itemize}

Any name defined or declared in the interface section is
said to be \emph{public} or \emph{exported} by that module.

\subsubsection{In the Implementation Section}

A name is visible in an implementation section if it is
\begin{itemize}
\item declared or defined there;
\item is exported by a module which is imported in this
  module's interface or implementation section;
\item is declared or defined in a parent module.
\end{itemize}

Any name declared or defined in the implementation section
that is not exported by the module is said to be \emph{private}
or \emph{local} to the module.

\subsection{Qualification}

Name qualification allows us to avoid abiguity where one
module imports two other modules which export different things
under the same ``base'' name.

A fully qualified name takes the form @<modulename>__<thingname>@
(or, for submodules, @<modulename>__<submodulename>__...__<thingname>@)
where the double-underscore @__@ is the module name separator symbol.

Using fully qualified names throughout a program quickly
becomes tedious and can distract from the readability of the
source code.  Mercury therefore allows one to omit qualifiers
(or partial prefixes in the case of submodules) where there is
no ambiguity.

The compiler will not confuse names of different sorts of
things -- for instance, there is never ambiguity between a type
and an inst with the same name since the types and insts
always appear in separate contexts.

Potential ambiguity arises with funcs, preds and data
constructors since all three can appear in unifications.
Ambiguity can still be resolved by the compiler if the
possible ways of resolving the name in question each have
different types or arities (\XXX{Need to go over some examples
a la @io\_\_print@.  And be more precise.})

A convention worth adopting is to explicitly qualify
predicates, but not types, constructors and functions.

(Mercury also provides @use_module@, an alternative to @import_module@,
which does much the same thing, except that names imported via
@use_module@ \emph{must} be fully qualified.)

\subsection{Submodules and Hierarchical Namespaces}

The module namespace may be structured rather than flat.  Modules can
also have submodules to form a tree-like namespace.

Submodules come in two flavours, nested and separate.

\subsubsection{Nested Submodules}

Nested submodules appear in the same source file as the parent module.

For example,
\begin{verbatim}
:- module vector3.
:- interface.
:- import_module float.

:- type vector3 == {float, float, float}.

:- func vector3 + vector3 = vector3.
:- func vector3 `dot` vector3 = float.

:- module scalar_vector3.
:- interface.

    :- func float * vector3 = vector3.

:- end_module scalar_vector3.

:- module vector3_scalar.
:- interface.

    :- func vector3 * scalar = vector3.

:- end_module vector3_scalar.
    
:- implementation.

{Xa, Ya, Za} + {Xb, Yb, Zb} = {Xa + Xb, Ya + Yb, Za + Zb}.

{Xa, Ya, Za} `dot` {Xb, Yb, Zb} = Xa * Xb + Ya * Yb + Za * Zb.

:- module scalar_vector3.
:- implementation.

    A * {X, Y, Z} = {A * X, A * Y, A * Z}.

:- end_module scalar_vector3.

:- module vector3_scalar.
:- implementation.

    {X, Y, Z} * A = {A * X, A * Y, A * Z}.

:- end_module vector3_scalar.
\end{verbatim}
Nested submodules are used in the example above to avoid overloading
problems for the functions such as @*@ that have both @float * vector3@
and @vector3 * float@ varieties (recall that overloaded functions with
the same arity cannot be defined directly in the same module.)

The submodules do not have to import the @float@ module separately
because all names visible in the parent module are also visible in the
submodules.

A client of this module would have to import each submodule to have
access to those particular names.  For example,
\begin{verbatim}
:- module foo.
:- interface.
:- import_module io.

:- pred main(io::di, io::uo) is det.

:- implementation.
:- import_module vector3, vector3_scalar, scalar_vector3.

main(!IO) :-
    R = {1.0, 1.5, 2.0},
    io__print(2.0 * R),
    io__nl,
    io__print(R * 3.0),
    io__nl.
\end{verbatim}
should write out
\begin{verbatim}
{2.0, 3.0, 4.0}
{3.0, 4.5, 6.0}
\end{verbatim}
when run.  The compiler understands that the multiplication @2.0 * R@
is referring to the function defined in @scalar_vector3@ and that the
multiplication @R * 3.0@ is referring to the function defined in the
@vector3_scalar@.

\subsubsection{Separate Submodules}

Separate submodules are defined in separate files and must be explicitly
listed in the parent module in @include_module@ declarations, but are 
otherwise indistinguishable from nested submodules as far as the
programmer is concerned.  The advantage of separate submodules is that
they can be compiled separately \XXX{verify this is correct} and, for
large submodules, splitting them into separate files can simplify code
maintenance.

\XXX{What about @import\_hierarchy@ and chums?}

\XXX{We really want to just have @.@ as the module separator by the time
this goes to print rather than the current mixture of @\_\_@ and @:@ and
@.@ in separate submodule file names.}

The vector example above could be recoded to use separate submodules as
\begin{verbatim}
:- module vector3.
:- interface.

:- import_module float.

:- include_module vector3__scalar_vector3, vector3__vector3_scalar.

:- type vector3 == {float, float, float}.

:- func vector3 + vector3 = vector3.
:- func vector3 `dot` vector3 = float.
    
:- implementation.

{Xa, Ya, Za} + {Xb, Yb, Zb} = {Xa + Xb, Ya + Yb, Za + Zb}.

{Xa, Ya, Za} `dot` {Xb, Yb, Zb} = Xa * Xb + Ya * Yb + Za * Zb.
\end{verbatim}
and in a file named @vector3.scalar_vector3.m@
\begin{verbatim}
:- module vector3__scalar_vector3.
:- interface.

:- func float * vector3 = vector3.

:- implementation.

A * {X, Y, Z} = {A * X, A * Y, A * Z}.
\end{verbatim}
and in a file named @vector3.vector3_scalar.m@
\begin{verbatim}
:- module vector3__vector3_scalar.
:- interface.

:- func vector3 * scalar = vector3.

:- implementation.

{X, Y, Z} * A = {A * X, A * Y, A * Z}.
\end{verbatim}
Note the @include_module@ declaration in the interface section for
@vector3@ (making @scalar_vector3@ and @vector3_scalar@ accessible to
clients of @vector3@\footnote{Their fully qualified names being
@vector3\_\_scalar\_vector3@ and @vector3\_\_vector3\_scalar@
respectively.}).  A declaration of this kind is required for every
separate submodule.

Use of @vector3@, @scalar_vector3@ and @vector3_scalar@ is exactly the
same as if they had been written as nested submodules.

\XXX{Is this the right place to mention the file name mapping option to
@mmake@?}

\subsubsection{Visibility and Submodules}

\XXX{Need to check this one out carefully.}



