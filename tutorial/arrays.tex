% vim: ft=tex ff=unix ts=4 sw=4 et wm=8 tw=0

\chapter{Arrays}

An array in Mercury is a mutable vector of values, indexed by position
starting from zero.  Arrays are unique in that they provide $O(1)$
access and update.  In order to make this property safe in a declarative
language, arrays have to be unique: a mutable array can only ever have
one live reference at any given moment.

\section{The Array Type, Modes and Insts}

Arrays are abstract types,
\begin{verbatim}
:- type array(T).
\end{verbatim}

The three main modes for @array@s are @array_di@, @array_uo@ and
@array_ui@ with the same meaning as @di@, @uo@ and @ui@ respectively,
but specialised for @array@s.  An @array@ of values with more complex insts
(e.g. arrays of predicates) must be passed around using the parametric
insts @array(I)@ and @uniq_array(I)@, the former being used for
immutable @array@s.  \XXX{Of course, even the @array@ access preds don't use
the parametric modes, so they're rather useless except for inst
casting, in which case we don't really need them anyway...}

\section{Initialising Arrays}

Since Mercury @array@s are polymorphic and there is no equivalent notion
of C's @NULL@ and so forth, @array@s must be initialised with a
programmer-supplied default value to be be placed in each @array@
cell.  The first argument is the number of cells the @array@ is to have:
\begin{verbatim}
:- func init(int, T) = aray(T).
\end{verbatim}
The empty @array@ is a special case (@array@s can be resized) and can be
constructed without an initialiser:
\begin{verbatim}
:- func make_empty_array = array(T).
\end{verbatim}
Finally, it is possible to initialise an array from a list:
\begin{verbatim}
:- func array(list(T)) = array(T).
\end{verbatim}

\section{Array Bounds}

\XXX{The current design pays lip-service to the notion that we may one
day have offset arrays.  I don't think we should mention this here,
since I think it was a mistake to mention it and not go the whole hog.}

Array cells are indexed starting from zero.  The highest numbered cell
index in an @array@ with @N@ cells is therefore @N - 1@.  The @size@
function returns the number of cells in an @array@; the @max@ function
return the highest numbered index of the array, or @-1@ if the array is
empty \XXX{the documentation for @max/1@ should reflect this}:
\begin{verbatim}
:- func size(array(T)) = int.
:- func size(array_ui) = out is det.
:- func size(in      ) = out is det.

:- func max(array(T)) = int.
:- func max(array_ui) = out is det.
:- func max(in      ) = out is det.
\end{verbatim}
It follows that @size(A) = max(A) + 1@.

There is a predicate to check that a given index lies within a given
@array@'s bounds:
\begin{verbatim}
:- pred in_bounds(array(T), int).
:- mode in_bounds(array_ui, in ) is semidet.
:- mode in_bounds(in,       in ) is semidet.
\end{verbatim}

\section{Access and Update}

Array members are accessed via the conventionally named functions:
\begin{verbatim}
:- func array(T) ^ elem(int) = T.
:- mode array_ui ^ elem(in ) = out is det.
:- mode in       ^ elem(in ) = out is det.

:- func ( array(T) ^ elem(int) := T  ) = array(T).
:- mode ( array_di ^ elem(in ) := in ) = array_uo is det.
\end{verbatim}
These functions check that the given index is within the array bounds
and will throw an exception if not.  These lookup operations are both
$O(1)$ in space and time.  \XXX{How good are C compilers at moving
bounds checks out of inner loops?}

\XXX{Should I mention the unsafe versions?}

There are two other predicates each for access and update that are
sometimes useful.  The pair, @semidet_lookup@ and @semidet_set@, merely
fail if the index is out of bounds, rather than throwing an exception:
\begin{verbatim}
:- pred semidet_lookup(array(T), int, T  ).
:- mode semidet_lookup(array_ui, in,  out) is semidet.
:- mode semidet_lookup(in,       in,  out) is semidet.

:- pred semidet_set(array(T), int, T, array(T)).
:- mode semidet_set(array_di, in, in, array_uo) is semidet.
\end{verbatim}

The second pair, @slow_set@ and @semidet_slow_set@, do not require the
input @array@ to be unique:
\begin{verbatim}
:- func slow_set(array(T), int, T ) = array(T).
:- mode slow_set(array_ui, in,  in) = array_uo is det.
:- mode slow_set(in,       in,  in) = array_uo is det.

:- pred semidet_slow_set(array(T), int, T,  array(T)).
:- mode semidet_slow_set(array_ui, in,  in, array_uo) is semidet.
:- mode semidet_slow_set(in,       in,  in, array_uo) is semidet.
\end{verbatim}
If the input @array@ is not unique, then destructive update (i.e. $O(1)$
update) is not possible so a unique copy -- $O(n)$ -- of the input
@array@ is made made and then updated.  As before, @semidet_slow_set@
will fail rather than throw an exception if the index argument is out of
bounds.

% \XXX{Why is the unique input argument @array_ui@ rather than @array_di@ ?
% Now that we have mode-specific clauses, we should not have a problem.}

\section{Utility Functions and Predicates}

\XXX{Here!}

\section{Implementing Multi-Dimensional Arrays}

As provided, Mercury arrays are one-dimensional.  Some applications
require arrays with two or more dimensions.  These kinds of arrays come
in two flavours: rectangular and ragged.

A rectangular @array@ with dimensions @W1 * W2 * ... * Wn@ is a
one-dimensional @array@ whose cells are indexed by @(X1, X2, ..., Xn)@
where @0 =< Xi < Wi@, mapping to the index
@(W1 * X1) + (W2 * X2) + ... + (Wn * Xn)@ in the one-dimensional
representation.

A ragged @array@ is one whose cells are themselves @array@s.  Hence the
cell whose index is given by @(X1, X2, ..., Xn)@ in an $n$-dimensional
ragged array is accessed via
@A ^ elem(X1) ^ elem(X2) ^ ... ^ elem(Xn)@.  Since an @array@'s size is
not part of its type, there is no requirement that each of the elements
of @A ^ elem(X1) ^ elem(X2) ^ ... ^ elem(Xj)@ (for some @j =< n@) have
the same size.  Accessing a cell in a ragged @array@ is usually a little
slower than accessing a cell in a rectangular @array@; however, ragged
@array@s may be much more economical in terms of space for sparse data.

Representing rectangular and ragged arrays each requires a little more
work.

\subsection{Rectangular Arrays}

\XXX{I've written the @table@ library module.  Get it checked in and
talk about it here.}

It is necessary to construct a special-purpose representation for each
number of dimensions.  Here is an example of how one might code up a
two-dimensional @array@ ADT:
\begin{verbatim}

\end{verbatim}

\subsection{Ragged Arrays}

\section{Caveats}

\XXX{They're not really unique at the moment.}

\XXX{Get nested uniqueness working.}
